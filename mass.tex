\subsubsection{The Order of the Administration of the Lord's Supper, or}
\chapter{Holy Communion.}
So many as intend to be partakers of the holy Communion shall signify their names to the Curate at least some time the day before.
If a Minister be persuaded that any person who presents himself to be a partaker of the holy Communion ought not to be admitted thereunto by reason of malicious and open contention with his neighbours, or other grave and open sin without repentance, he shall give an account of the same to the Ordinary of the place, and therein obey his order and direction, but so as not to refuse the Sacrament to any person until in accordance with such order and direction he shall have called him and advertised him that in any wise he presume not to come to the Lord's Table; Provided that in case of grave and immediate scandal to the Congregation the Minister shall not admit such person, but shall give an account of the same to the Ordinary within seven days after at the latest and therein obey the order and direction given to him by the Ordinary; Provided also that before issuing his order and direction in relation to any such person the Ordinary shall afford him an opportunity for interview.

\pilcrow{The Table at the Communion time having a fair white linen cloth upon it, shall stand in the body of the Church, or in the Chancel, where Morning and Evening Prayer are appointed to be said. And the Priest standing at the north side of the Table shall say the Lord's Prayer, with the Collect following, the people kneeling.}
\drop{Our Father, which art in heaven, Hallowed be thy Name. Thy Kingdom come. Thy will be done, in earth as it is in heaven. Give us this day our daily bread. And forgive us our trespasses, As we forgive them that trespass against us. And lead us not into temptation; But deliver us from evil. Amen.}

The Collect
\drop{Almighty God, unto whom all hearts be open, all desires known, and from whom no secrets are hid; Cleanse the thoughts of our hearts by the inspiration of thy Holy Spirit, that we may perfectly love thee, and worthily magnify thy holy Name; through Christ our Lord. Amen.}

Then shall the Priest, turning to the people, rehearse distinctly all the TEN COMMANDMENTS; and the people still kneeling shall, after every Commandment, ask God mercy for their transgression thereof for the time past, and grace to keep the same for the time to come, as followeth.
Minister.
\drop{God spake these words, and said; I am the Lord thy God: Thou shalt have none other gods but me.}

People. Lord, have mercy upon us, and incline our hearts to keep this law.

Minister. Thou shalt not make to thyself any graven image, nor the likeness of any thing that is in heaven above, or in the earth beneath, or in the water under the earth. Thou shalt not bow down to them, nor worship them: for I the Lord thy God am a jealous God, and visit the sins of the fathers upon the children unto the third and fourth generation of them that hate me, and shew mercy unto thousands in them that love me, and keep my commandments.

People. Lord, have mercy upon us, and incline our hearts to keep this law.
    Minister. Thou shalt not take the Name of the Lord thy God in vain: for the Lord will not hold him guiltless, that taketh his Name in vain.
    People. Lord, have mercy upon us, and incline our hearts to keep this law.
    Minister. Remember that thou keep holy the Sabbath-day. Six days shalt thou labour, and do all that thou hast to do; but the seventh day is the Sabbath of the Lord thy God. In it thou shalt do no manner of work, thou, and thy son, and thy daughter, thy man-servant, and thy maid-servant, thy cattle, and the stranger that is within thy gates. For in six days the Lord made heaven and earth, the sea, and ail that in them is, and rested the seventh day: wherefore the Lord blessed the seventh day, and hallowed it.
    People. Lord, have mercy upon us, and incline our hearts to keep this law.
    Minister. Honour thy father and thy mother; that thy days may be long in the land which the Lord thy God giveth thee.
    People. Lord, have mercy upon us, and incline our hearts to keep this law.
    Minister. Thou shalt do no murder.
    People. Lord, have mercy upon us, and incline our hearts to keep this law.
    Minister. Thou shalt not commit adultery.
    People. Lord, have mercy upon us, and incline our hearts to keep this law.
    Minister. Thou shalt not steal.
    People. Lord, have mercy upon us, and incline our hearts to keep this law.
    Minister. Thou shalt not bear false witness against thy neighbour.
    People. Lord, have mercy upon us, and incline our hearts to keep this law.
    Minister. Thou shalt not covet thy neighbour's house, thou shalt not covet thy neighbour's wife, nor his servant, nor his maid, nor his ox, nor his ass, nor any thing that is his.
    People. Lord, have mercy upon us, and write all these thy laws in our hearts, we beseech thee.
Then shall follow one of these two Collects for the Queen, the Priest standing as before, and saying,
Let us pray.
ALMIGHTY God, whose kingdom is everlasting, and power infinite: Have mercy upon the whole Church; and so rule the heart of thy chosen servant ELIZABETH, our Queen and Governor, that she (knowing whose minister she is) may above all things seek thy honour and glory; and that we, and all her subjects (duly considering whose authority she hath) may faithfully serve, honour, and humbly obey her, in thee, and for thee, according to thy blessed Word and ordinance; through Jesus Christ our Lord, who with thee and the Holy Ghost liveth and reigneth, ever one God, world without end. Amen.
Or,
ALMIGHTY and everlasting God, we are taught by thy Holy Word, that the hearts of Kings are in thy rule and governance, and that thou dost dispose and turn them as it seemeth best to thy godly wisdom: We humbly beseech thee so to dispose and govern the heart of ELIZABETH thy Servant, our Queen and Governor, that, in all her thoughts, words, and works, she may ever seek thy honour and glory, and study to preserve thy people committed to her charge, in wealth, peace, and godliness: Grant this, O merciful Father, for thy dear Son's sake, Jesus Christ our Lord. Amen.
Then shall be said the Collect of the Day. And immediately after the Collect the Priest shall read the Epistle, saying, The Epistle [or, The portion of Scripture appointed for the Epistle] is written in the ----- Chapter of ----- beginning at the ----- Verse. And the Epistle ended, he shall say, Here endeth the Epistle. Then shall he read the Gospel (the people all standing up) saying, The holy Gospel is written in the ----- Chapter of ----- beginning at the ----- Verse. And the Gospel ended, shall be sung or said the Creed following, the people still standing, as before.

\drop{I believe in one God the Father Almighty, Maker of heaven and earth, And of all things visible and invisible:}

And in one Lord Jesus Christ, the only-begotten son of God, Begotten of his Father before all worlds, God of God, Light of Light, Very God of very God, Begotten, not made, Being of one substance with the Father, By whom all things were made: Who for us men, and for our salvation came down from heaven, And was incarnate by the Holy Ghost of the Virgin Mary, And was made man, And was crucified also for us under Pontius Pilate. He suffered and was buried, And the third day he rose again according to the Scriptures, And ascended into heaven, And sitteth on the right hand of the Father. And he shall come again with glory to judge both the quick and the dead: Whose kingdom shall have no end.

And I believe in the Holy Ghost, The Lord and giver of life, Who proceedeth from the Father and the Son, Who with the Father and the Son together is worshipped and glorified, Who spake by the Prophets. And I believe one Catholick and Apostolick Church. I acknowledge one Baptism for the remission of sins. And I look for the Resurrection of the dead, And the life of the world to come. Amen.

\pilcrow{Then the Curate shall declare unto the people what Holy-days, or Fasting-days, are in the week following to be observed. And then also (if occasion be) shall notice be given of the Communion; and Briefs, Citations, and Excommunications read. And nothing shall be proclaimed or published in the Church during the time of Divine Service, but by the Minister : nor by him any thing, but what is prescribed in the Rules of this Book, or enjoined by the Queen, or by the Ordinary of the place.}

\pilcrow{Then shall follow the Sermon, or one of the Homilies already set forth, or hereafter to be set forth, by authority.}

\pilcrow{Then shall the Priest return to the Lord's Table, and begin the Offertory, saying one or more of these Sentences following, as he thinketh most convenient in his discretion.}

LET your light so shine before men, that they may see your good works, and glorify your Father which is in heaven. St. Matth. v.
    Lay not up for yourselves treasure upon the earth; where the rust and moth doth corrupt, and where thieves break through and steal: but lay up for yourselves treasures in heaven; where neither rust nor moth doth corrupt, and where thieves do not break through and steal. St. Matth. vi.
    Whatsoever ye would that men should do unto you, even so do unto them; for this is the Law and the Prophets. St. Matth. vii.
    Not every one that saith unto me, Lord, Lord, shall enter into the kingdom of heaven; but he that doeth the will of my Father which is in heaven. St. Matth. vii
    Zacchæus stood forth, and said unto the Lord, Behold, Lord, the half of my goods I give to the poor; and if I have done any wrong to any man, I restore four-fold. St. Luke xix.
    Who goeth a warfare at any time of his own cost? Who planteth a vineyard, and eateth not of the fruit thereof? Or who feedeth a flock, and eateth not of the milk of the flock? 1 Cor. ix.
    If we have sown unto you spiritual things, is it a great matter if we shall reap your worldly things? 1 Cor. ix.
    Do ye not know, that they who minister about holy things live of the sacrifice; and they who wait at the altar are partakers with the altar? Even so hath the Lord also ordained, that they who preach the Gospel should live of the Gospel. 1 Cor. ix.
    He that soweth little shall reap little; and he that soweth plenteously shall reap plenteously. Let every man do according as he is disposed in his heart, not grudging, or of necessity; for God loveth a cheerful giver. 2 Cor. ix.
    Let him that is taught in the Word minister unto him that teacheth, in all good things. Be not deceived, God is not mocked: for whatsoever a man soweth that shall he reap. Gal. vi.
    While we have time, let us do good unto all men; and specialty unto them that are of the household of faith. Gal. vi.
    Godliness is great riches, if a man be content with that he hath: for we brought nothing into the world, neither may we carry any thing out. 1 Tim. vi.
    Charge them who are rich in this world, that they be ready to give, and glad to distribute; laying up in store for themselves a good foundation against the time to come, that they may attain eternal life. 1 Tim. vi.
    God is not unrighteous, that he will forget your works, and labour that proceedeth of love; which love ye have shewed for his Name's sake, who have ministered unto the saints, and yet do minister. Heb. vi.
    To do good, and to distribute, forget not; for with such sacrifices God is pleased. Heb. xiii.
    Whoso hath this world's good, and seeth his brother have need, and shutteth up his compassion from him, how dwelleth the love of God in him? I St. John iii.
    Give alms of thy goods, and never turn thy face from any poor man; and then the face of the Lord shall not be turned away from thee. Tobit iv.
    Be merciful after thy power. If thou hast much, give plenteously; if thou hast little, do thy diligence gladly to give of that little: for so gatherest thou thyself a good reward in the day of necessity. Tobit iv.
    He that hath pity upon the poor lendeth unto the Lord: and look, what he layeth out, it shall be paid him again. Prov. xix.
    Blessed be the man that provideth for the sick and needy: the Lord shall deliver him in the time of trouble. Psalm xli.

\pilcrow{Whilst these Sentences are in reading, the Deacons, Church-wardens, or other fit person appointed for that purpose, shall receive the Alms for the Poor, and other devotions of the people, in a decent basin to be provided by the Parish for that purpose; and reverently bring it to the Priest, who shall humbly present and place it upon the holy Table.}

\pilcrow{And when there is a Communion, the Priest shall then place upon the Table so much Bread and Wine, as he shall think sufficient. After which done, the Priest shall say,}
Let us pray for the whole state of Christ's Church militant here in earth.
\drop{Almighty and everliving God, who by thy holy Apostle hast taught us to make prayers, and supplications, and to give thanks for all men; We humbly beseech thee most mercifully [* to accept our alms and oblations, and] to receive these our prayers, which we offer unto thy Divine Majesty; beseeching thee to inspire continually the Universal Church with the spirit of truth, unity, and concord: And grant, that all they who do confess thy holy Name may agree in the truth of thy holy Word, and live in unity, and godly love. We beseech thee also to save and defend all Christian Kings, Princes, and Governours; and specially thy Servant ELIZABETH our Queen; that under her we may be godly and quietly governed: And grant unto her whole Council, and to all that are put in authority under her, that they may truly and impartially administer justice, to the punishment of wickedness and vice, and to the maintenance of thy true religion, and virtue. Give grace, O heavenly Father, to all Bishops and Curates, that they may both by their life and doctrine set forth thy true and lively Word, and rightly and duly administer thy holy Sacraments. And to all thy people give thy heavenly grace; and especially to this congregation here present; that, with meek heart and due reverence, they may hear, and receive thy holy Word; truly serving thee in holiness and righteousness all the days of their life. And we most humbly beseech thee, of thy goodness, O Lord, to comfort and succour all those who, in this transitory life, are in trouble, sorrow, need, sickness, or any other adversity. And we also bless thy holy Name for all thy servants departed this life in thy faith and fear; beseeching thee to give us grace so to follow their good examples, that with them we may be partakers of thy heavenly kingdom. Grant this, O Father, for Jesus Christ's sake, our only Mediator and Advocate. Amen.}
* If there be no alms or oblations, then the words [of accepting our alms and oblations] be left out unsaid.
When the Minister giveth warning for the celebration of the holy Communion, (which he shall always do upon the Sunday, or some Holy-day, immediately preceding,) after the Sermon or Homily ended, he shall read this Exhortation following.
DEARLY beloved, on ----- day next I purpose, through God's assistance, to administer to all such as shall be religiously and devoutly disposed the most comfortable Sacrament of the Body and Blood of Christ; to be by them received in remembrance of his meritorious Cross and Passion; whereby alone we obtain remission of our sins, and are make partakers of the Kingdom of heaven. Wherefore it is our duty to render most humble and hearty thanks to Almighty God our heavenly Father, for that he hath given his Son our Saviour Jesus Christ, not only to die for us, but also to be our spiritual food and sustenance in that holy Sacrament. Which being so divine and comfortable a thing to them who receive it worthily, and so dangerous to them that will presume to receive it unworthily; my duty is to exhort you in the mean season to consider the dignity of that holy mystery, and the great peril of the unworthy receiving thereof; and so to search and examine your own consciences, (and that nor lightly, and after the manner of dissemblers with God; but so) that ye may come holy and clean to such a heavenly Feast, in the marriage-garment required by God in holy Scripture, and be received as worthy partakers of that holy Table.
    The way and means thereto is; First, to examine your lives and conversations by the rule of God's commandments; and whereinsoever ye shall perceive yourselves to have offended, either by will, word, or deed, there to bewail your own sinfulness, and to confess yourselves to Almighty God, with full purpose of amendment of life. And if ye shall perceive your offences to be such as are not only against God, but also against your neighbours; then ye shall reconcile yourselves unto them; being ready to make restitution and satisfaction, according to the uttermost of your powers, for all injuries and wrongs done by you to any other; and being likewise ready to forgive others that have offended you, as ye would have forgiveness of your offences at God's hand: for otherwise the receiving of the holy Communion doth nothing else but increase your damnation. Therefore if any of you be a blasphemer of God, an hinderer or slanderer of his Word, an adulterer, or be in malice, or envy, or in any other grievous crime, repent you of your sins, or else come not to that holy Table; lest, after the taking of that holy Sacrament, the devil enter into you, as he entered into Judas, and fill you full of all iniquities, and bring you to destruction both of body and soul.
    And because it is requisite, that no man should come to the holy Communion, but with a full trust in God's mercy, and with a quiet conscience; therefore if there be any of you, who by this means cannot quiet his own conscience herein, but requireth further comfort or counsel, let him come to me, or to some other discreet and learned Minister of God's Word, and open his grief; that by the ministry of God's holy Word he may receive the benefit of absolution, together with ghostly counsel and advice, to the quieting of his conscience, and avoiding of all scruple and doubtfulness.

Or, in case he shall see the people negligent to come to the holy Communion, instead of the former, he shall use this Exhortation.
DEARLY beloved brethren, on ----- I intend, by God's grace, to celebrate the Lord's Supper: unto which, in God's behalf, I bid you all that are here present; and beseech you, for the Lord Jesus Christ's sake, that ye will not refuse to come thereto, being so lovingly called and bidden by God himself. Ye know how grievous and unkind a thing it is, when a man hath prepared a rich feast, decked his table with all kind of provision, so that there lacketh nothing but the guests to sit down; and yet they who are called (without any cause) most unthankfully refuse to come. Which of you in such a case would not be moved? Who would not think a great injury and wrong done unto him? Wherefore, most dearly beloved in Christ, take ye good heed, lest ye, withdrawing yourselves from this holy Supper, provoke God's indignation against you. It is an easy matter for a man to say, I will not communicate, because I am otherwise hindered with worldly business. But such excuses are not so easily accepted and allowed before God. If any man say, I am a grievous sinner, and therefore am afraid to come: wherefore then do ye not repent and amend? When God calleth you, are ye not ashamed to say ye will not come? When ye should return to God, will ye excuse yourselves, and say ye are not ready? Consider earnestly with yourselves how little such feigned excuses will avail before God. They that refused the feast in the Gospel, because they had bought a farm, or would try their yokes of oxen, or because they were married, were not so excused, but counted unworthy of the heavenly feast. 1, for my part, shall be ready; and, according to mine Office, I bid you in the Name of God, I call you in Christ's behalf, I exhort you, as ye love your own salvation, that ye will be partakers of this holy Communion. And as the Son of God did vouchsafe to yield up his soul by death upon the Cross for your salvation; so it is your duty to receive the Communion in remembrance of the sacrifice of his death, as he himself hath commanded: which if ye shall neglect to do, consider with yourselves how great injury ye do unto God, and how sore punishment hangeth over your heads for the same; when ye wilfully abstain from the Lord's Table, and separate from your brethren, who come to feed on the banquet of that most heavenly food. These things if ye earnestly consider, ye will by God's grace return to a better mind: for the obtaining whereof we shall not cease to make our humble petitions unto Almighty God our heavenly Father.

At the time of the Celebration of the Communion, the Communicants being conveniently placed for the receiving of the holy Sacrament, the Priest shall say this Exhortation.
DEARLY beloved in the Lord, ye that mind to come to the holy Communion of the Body and Blood of our Saviour Christ, must consider how Saint Paul exhorteth all persons diligently to try and examine themselves, before they presume to eat of that Bread, and drink of that Cup. For as the benefit is great, if with a true penitent heart and lively faith we receive that holy Sacrament; (for then we spiritually eat the flesh of Christ, and drink his blood; then we dwell in Christ, and Christ in us; we are one with Christ, and Christ with us;) so is the danger great, if we receive the same unworthily. For then we are guilty of the Body and Blood of Christ our Saviour; we eat and drink our own damnation, not considering the Lord's Body; we kindle God's wrath against us; we provoke him to plague us with divers diseases, and sundry kinds of death. judge therefore yourselves, brethren, that ye be not judged of the Lord; repent you truly for your sins past; have a lively and stedfast faith in Christ our Saviour; amend your lives, and be in perfect charity with all men; so shall ye be meet partakers of those holy mysteries. And above all things ye must give most humble and hearty thanks to God, the Father, the Son, and the Holy Ghost, for the redemption of the world by the death and passion of our Saviour Christ, both God and man; who did humble himself, even to the death upon the Cross, for us, miserable sinners, who lay in darkness and the shadow of death; that he might make us the children of God, and exalt us to everlasting life. And to the end that we should alway remember the exceeding great love of our Master, and only Saviour, Jesus Christ, thus dying for us, and the innumerable benefits which by his precious blood-shedding he hath obtained to us; he hath instituted and ordained holy mysteries, as pledges of his love, and for a continual remembrance of his death, to our great and endless comfort. To him therefore, with the Father and the Holy Ghost, let us give (as we are most bounden) continual thanks; submitting ourselves wholly to his holy will and pleasure, and studying to serve him in true holiness and righteousness all the days of our life. Amen.

Then shall the Priest say to them that come to receive the holy Communion,
YE that do truly and earnestly repent you of your sins, and are in love and charity with your neighbours, and intend to lead a new life, following the commandments of God, and walking from henceforth in his holy ways; Draw near with faith, and take this holy Sacrament to your comfort; and make your humble confession to Almighty God, meekly kneeling upon your knees.

Then shall this general Confession be made, in the name of all those that are minded to receive the holy Communion, by one of the Ministers; both he and all the people kneeling humbly upon their knees, and saying,
ALMIGHTY God, Father of our Lord Jesus Christ, Maker of all things, judge of all men; We acknowledge and bewail our manifold sins and wickedness, Which we, from time to time, most grievously have committed, By thought, word, and deed, Against thy Divine Majesty, Provoking most justly thy wrath and indignation against us. We do earnestly repent, And are heartily sorry for these our misdoings; The remembrance of them is grievous unto us; The burden of them is intolerable. Have mercy upon us, Have mercy upon us, most merciful Father; For thy Son our Lord Jesus Christ's sake, Forgive us all that is past; And grant that we may ever hereafter Serve and please thee In newness of life, To the honour and glory of thy Name; Through Jesus Christ our Lord. Amen.

Then shall the Priest (or the Bishop, being present,) standing up, and turning himself to the people, pronounce this Absolution.
ALMIGHTY God, our heavenly Father, who of his great mercy hath promised forgiveness of sins to all them that with hearty repentance and true faith turn unto him; Have mercy upon you; pardon and deliver you from all your sins; confirm and strengthen you in all goodness; and bring you to everlasting life; through Jesus Christ our Lord. Amen.

Then shall the Priest say,
Hear what comfortable words our Saviour Christ saith unto all that truly turn to him.
COME unto me all that travail and are heavy laden, and I will refresh you. St. Matth. xi. 28.
    So God loved the world, that he gave his only-begotten Son, to the end that all that believe in him should not perish, but have everlasting life. St. John iii. 16
Hear also what Saint Paul saith.
    This is a true saying, and worthy of all men to be received, That Christ Jesus came into the world to save sinners. 1 Tim. i. 15.
Hear also what Saint John saith.
    If any man sin, we have an Advocate with the Father, Jesus Christ the righteous; and he is the propitiation for our sins. 1 St. John ii. 1.
After which the Priest shall proceed, saying,
    Lift up your hearts.
Answer. We lift them up unto the Lord.
Priest. Let us give thanks unto our Lord God.
Answer. It is meet and right so to do.
Then shall the Priest turn to the Lord's Table, and say,
IT is very meet, right, and our bounden duty, that we should at all times, and in all places, give thanks unto thee, O Lord, *Holy Father, Almighty, Everlasting God.
* These words [Holy Father] must be omitted on Trinity Sunday.
Here shall follow the proper Preface, according to the time, if there be any specially appointed: or else immediately shall follow,
THEREFORE with Angels and Archangels, and with all the company of heaven, we laud and magnify thy glorious Name; evermore praising thee, and saying, Holy, holy, holy, Lord God of hosts, heaven and earth are full of thy glory: Glory be to thee, O Lord most High. Amen.

Proper Prefaces
Upon Christmas Day, and seven days after.
BECAUSE thou didst give Jesus Christ thine only Son to be born as at this time for us; who, by the operation of the Holy Ghost, was made very man of the substance of the Virgin Mary his mother; and that without spot of sin, to make us clean from all sin. Therefore with Angels, etc.
Upon Easter Day, and seven days after,
BUT chiefly are we bound to praise thee for the glorious Resurrection of thy Son Jesus Christ our Lord: for he is the very Paschal Lamb, which was offered for us, and hath taken away the sin of the world; who by his death hath destroyed death, and by his rising to life again hath restored to us everlasting life. Therefore with Angels, etc.
Upon Ascension Day, and seven days after.
THROUGH thy most dearly beloved Son Jesus Christ our Lord; who after his most glorious Resurrection manifestly appeared to all his Apostles, and in their sight ascended up into heaven to prepare a place for us; that where he is, thither we might also ascend, and reign with him in glory. Therefore with Angels, etc.
Upon Whitsunday, and six days after.
THROUGH Jesus Christ our Lord; according to whose most true promise, the Holy Ghost came down as at this time from heaven with a sudden great sound, as it had been a mighty wind in the likeness of fiery tongues, lighting upon the Apostles, to teach them, and to lead them to all truth; giving them both the gift of divers languages, and also boldness with fervent zeal constantly to preach the Gospel unto all nations; whereby we have been brought out of darkness and error into the clear light and true knowledge of thee, and of thy Son Jesus Christ. Therefore with Angels, etc.
Upon the Feast of Trinity only.
WHO art one God, one Lord; not one only Person, but three Persons in one Substance. For that which we believe of the glory of the Father, the same we believe of the Son, and of the Holy Ghost, without any difference or inequality. Therefore with Angels, etc.
After each of which Prefaces shall immediately be sung or said,
THEREFORE with Angels and Archangels, and with all the company of heaven, we laud and magnify thy glorious Name; evermore praising thee, and saying, Holy, holy, holy, Lord God of hosts, heaven and earth are full of thy glory: Glory be to thee, O Lord most High. Amen.

Then shall the Priest, kneeling down at the Lord's Table, say in the name of all them that shall receive the Communion this Prayer following.
WE do not presume to come to this thy Table, O merciful Lord, trusting in our own righteousness, but in thy manifold and great mercies. We are not worthy so much as to gather up the crumbs under thy Table. But thou art the same Lord, whose property is always to have mercy: Grant us therefore, gracious Lord, so to eat the flesh of thy dear Son Jesus Christ, and to drink his blood, that our sinful bodies may be made clean by his body, and our souls washed through his most precious blood, and that we may evermore dwell in him, and he in us. Amen.

\pilcrow{When the Priest, standing before the Table, hath so ordered the Bread and Wine, that he may with the more readiness and decency break the Bread before the people, and take the Cup into his hands, he shall say the Prayer of Consecration, as followeth.}
\drop{Almighty God, our heavenly Father, who of thy tender mercy didst give thine only Son Jesus Christ to suffer death upon the Cross for our redemption; who made there (by his one oblation of himself once offered) a full, perfect, and sufficient sacrifice, oblation, and satisfaction, for the sins of the whole world; and did institute, and in his holy Gospel command us to continue, a perpetual memory of that his precious death, until his coming again; Hear us, O merciful Father, we most humbly beseech thee; and grant that we receiving these thy creatures of bread and wine, according to thy Son our Saviour Jesus Christ's holy institution, in remembrance of his death and passion, may be partakers of his most blessed Body and Blood: who, in the same night that he was betrayed, (a) took Bread; and, when he had given thanks, (b) he brake it, and gave it to his disciples, saying, Take, eat, (c) this is my Body which is given for you: Do this in remembrance of me. Likewise after supper he (d) took the Cup; and, when he had given thanks, he gave it to them, saying, Drink ye all of this; for this (e) is my Blood of the New Testament, which is shed for you and for many for the remission of sins: Do this, as oft as ye shall drink it, in remembrance of me. Amen.}
(a) Here the Priest is to take the Paten unto his hands: (b) And here to break the Bread: (c) And here to lay his hand upon all the Bread. (d) Here he is to take the Cup into his hand: (e) And here to lay his hand upon every vessel (be it Chalice or Flagon) in which there is any Wine to be consecrated.
\drop{O Lord and heavenly Father, we thy humble servants entirely desire thy fatherly goodness mercifully to accept this our sacrifice of praise and thanksgiving; most humbly beseeching thee to grant, that by the merits and death of thy Son Jesus Christ, and through faith in his blood, we and all thy whole Church may obtain remission of our sins, and all other benefits of his passion. And here we offer and present unto thee, O Lord, ourselves, our souls and bodies, to be a reasonable, holy, and lively sacrifice unto thee; humbly beseeching thee, that all we, who are partakers of this holy Communion, may be fulfilled with thy grace and heavenly benediction. And although we be unworthy, through our manifold sins, to offer unto thee any sacrifice, yet we beseech thee to accept this our bounden duty and service; not weighing our merits, but pardoning our offences, through Jesus Christ our Lord; by whom, and with whom, in the unity of the Holy Ghost, all honour and glory be unto thee, O Father Almighty, world without end. Amen.}


\pilcrow{Then shall the Priest say the Lord's Prayer, the people repeating after him every Petition.}
\drop{Our Father, which art in heaven, Hallowed be thy Name. Thy kingdom come. Thy will be done, in earth as it is in heaven. Give us this day our daily bread. And forgive us our trespasses, As we forgive them that trespass against us. And lead us not into temptation; But deliver us from evil: For thine is the kingdom, The power, and the glory, For ever and ever. Amen.}


Then shall the Minister first receive the Communion in both kinds himself, and then proceed to deliver the same to the Bishops, Priests, and Deacons, in like manner, (if any be present,) and after that to the people also in order, into their hands, all meekly kneeling. And, when he delivereth the Bread to any one, he shall say,
THE Body of our Lord Jesus Christ, which was given for thee, preserve thy body and soul unto everlasting life. Take and eat this in remembrance that Christ died for thee, and feed on him in thy heart by faith with thanksgiving.

And the Minister that delivereth the Cup to any one shall say,
THE Blood of our Lord Jesus Christ, which was shed for thee, preserve thy body and soul unto everlasting life. Drink this in remembrance that Christ's Blood was shed for thee, and be thankful.

If the consecrated Bread or Wine be all spent before all have communicated, the Priest is to consecrate more cording to the Form before prescribed: Beginning at [Our Saviour Christ in the same night, etc.] for the blessing of the Bread ; and at [Likewise after Supper, etc.] for the blessing of the Cup.

When all have communicated, the Minister shall return to the Lord's Table, and reverently place upon it what remaineth of the consecrated Elements, covering the same with a fair linen cloth.

After shall be said as followeth.
\drop{Almighty and everliving God, we most heartily thank thee, for that thou dost vouchsafe to feed us, who have duly received these holy mysteries, with the spiritual food of the most precious Body and Blood of thy Son our Saviour Jesus Christ; and dost assure us thereby of thy favour and goodness towards us; and that we are very members incorporate in the mystical body of thy Son, which is the blessed company of all faithful people; and are also heirs through hope of thy everlasting kingdom, by the merits of the most precious death and passion of thy dear Son. And we most humbly beseech thee , O heavenly Father, so to assist us with thy grace, that we may continue in that holy fellowship, and do all such good works as thou hast prepared for us to walk in; through Jesus Christ our Lord, to whom, with thee and the Holy Ghost, be all honour and glory, world without end. Amen.}


Then shall be said or sung,
\drop{Glory be to God on high, and in earth peace, good will towards men. We praise thee, we bless thee, we worship thee, we glorify thee, we give thanks to thee for thy great glory, O Lord God, heavenly King, God the Father Almighty.}
    O Lord, the only begotten Son Jesu Christ; O Lord God, Lamb of God, Son of the Father, that takest away the sins of the world, have mercy upon us. Thou that takest away the sins of the world, have mercy upon us. Thou that takest away the sins of the world, receive our prayer. Thou that sittest at the right hand of God the Father, have mercy upon us.
    For thou only art holy; thou only art the Lord; thou only, O Christ, with the Holy Ghost, art most high in the glory of God the Father. Amen.
Then the Priest (or Bishop if he be present) shall let them depart with this Blessing.
THE peace of God, which passeth all understanding, keep your hearts and minds in the knowledge and love of God, and of his son Jesus Christ our Lord: and the blessing of God Almighty, the Father, the Son, and the Holy Ghost, be amongst you and remain with you always. Amen.

Collects to be said after the Offertory, when there is no Communion, every such day one or more; and the same may be said also, as often as occasion shall serve, after the Collects either of Morning or Evening Prayer, Communion, or Litany, by the discretion of the Minister.
ASSIST us mercifully, O Lord, in these our supplications and prayers, and dispose the way of thy servants towards the attainment of everlasting salvation; that, among all the changes and chances of this mortal life, they may ever be defended by thy most gracious and ready help; through Jesus Christ our Lord. Amen.

O almighty Lord, and everlasting God, vouchsafe, we beseech thee, to direct, sanctify, and govern, both our hearts and bodies, in the ways of thy laws, and in the works of thy commandments; that through thy most mighty protection, both here and ever, we may be pre- served in body and soul; through our Lord and Saviour Jesus Christ. Amen.

Grant, we beseech thee, Almighty God, that the words, which we have heard this day with our outward ears, may through thy grace be so grafted inwardly in our hearts, that they may bring forth in us the fruit of good living, to the honour and praise of thy Name; through Jesus Christ our Lord. Amen.

PREVENT us O Lord, in all our doings with thy most gracious favour, and further us with thy continual help; that in all our works begun, continued, and ended in thee, we may glorify thy holy Name, and finally by thy mercy obtain everlasting life; through Jesus Christ our Lord. Amen.

ALMIGHTY God, the fountain of all wisdom, who knowest our necessities before we ask, and our ignorance in asking; We beseech thee to have compassion upon our infirmities; and those things, which for our unworthiness we dare not, and for our blindness we cannot ask, vouchsafe to give us, for the worthiness of thy Son Jesus Christ our Lord. Amen.

ALMIGHTY God, who hast promised to hear the petitions of them that ask in thy Son's Name; We beseech thee mercifully to incline thine ears to us that have made now our prayers and supplications unto thee; and grant, that those things, which we have faithfully asked according to thy will, may effectually be obtained, to the relief of our necessity, and to the setting forth of thy glory; through Jesus Christ our Lord. Amen.

\pilcrow{Upon the Sundays and other Holy-days (if there be no Communion) shall be said all that is appointed at the Communion, until the end of the general Prayer [For the whole state of Christ's Church militant here in earth] together with one or more of these Collects last before rehearsed, concluding with the Blessing.}

And there shall be no celebration of the Lord's Supper, except there be a convenient number to communicate with the Priest, according to his discretion.

And if there be not above twenty persons in the Parish of discretion to receive the Communion: yet there shall be no Communion, except four (or three at the least) communicate with the Priest.

\pilcrow{And in Cathedral and Collegiate Churches, and Colleges, where there are many Priests and Deacons, they shall all receive the Communion with the Priest every Sunday at the least, except they have a reasonable cause to the contrary.}

\pilcrow{And to take away all occasion of dissension, and superstition, which any person hath or might have concerning the Bread and Wine, it shall suffice that the Bread be such as is usual to be eaten; but the best and purest Wheat Bread that conveniently may be gotten.}

\pilcrow{And if any of the Bread and Wine remain unconsecrated, the Curate shall have it to his own use: but if any remain of that which was consecrated, it shall not be carried out of the Church, but the Priest, and such other of the Communicants as he shall then call unto him, shall, immediately after the Blessing, reverently eat and drink the same.}

\pilcrow{The Bread and Wine for the Communion shall be provided by the Curate and the Church-wardens at the charges of the Parish.}

\pilcrow{And note, that every Parishioner shall communicate at the least three times in the year, of which Easter to be one. And yearly at Easter every Parishioner shall reckon with the Parson, Vicar, or Curate, or his or their Deputy or Deputies; and pay to them or him all Ecclesiastical Duties, accustomably due, then and at that time to be paid.}

\pilcrow{After the Divine Service ended, the money given at the Offertory shall be disposed of to such pious and charitable uses, as the Minister and Church-wardens shall think fit. Wherein if they disagree, it shall be disposed of as the Ordinary shall appoint.}

