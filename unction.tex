
\chapter[The Visitation of the Sick]{\stylechapter{The Order for}{The Visitation of the Sick}{and the Communion of the Sick}}

\section{Visitation}
%These rubrics from 1923,1928,1929
\pilcrow{When any person is sick, notice shall be given thereof to the Minister,
%  of the Parish; 
who shall minister to the sick person after the form following, or in like manner.}
\smallskip
\pilcrow{When he cometh into the sick person’s house, he shall say,}
\drop{Peace be to this house, and to all that dwell in it.}

\bigskip
{\centering\pilcrow{When he cometh into the sick person’s presence he shall say, kneeling down,}}


% \centerline{\rubric{Then the Minister shall say,}}
% Let us pray.

\centerline{Lord, have mercy upon us.}
\centerline{\emph{Christ, have mercy upon us.}}
\centerline{Lord, have mercy upon us.}

\medskip
\ourFather


\V O Lord, save thy servant; \R Which putteth \emph{his} trust in thee.

\V Send \emph{him} help from thy holy place; \R  And evermore mightily defend \emph{him}.

\V Let the enemy have no advantage of \emph{him};  \R Nor the wicked approach to hurt \emph{him}.

\V Be unto \emph{him}, O Lord, a strong tower. \R  From the face of \emph{his} enemy.

\V O Lord, hear our prayers.  \R And let our cry come unto thee.

\centerline{Let us pray.}
\drop{O Lord, look down from heaven, behold, visit, and relieve this thy servant. Look upon \emph{him} with the eyes of thy mercy, give \emph{him} comfort and sure confidence in thee, defend \emph{him} from the danger of the enemy, and keep \emph{him} in perpetual peace and safety; through Jesus Christ our Lord. \R Amen.}

\smallskip

Hear us, Almighty and most merciful God and Saviour; extend thy accustomed goodness to this thy servant who is grieved with sickness.  \R Amen. % The following divisions are from 1928 proposed.
\smallskip

% \begin{leftbar}
% restored from 1549
Visit \emph{him}, O Lord, as thou didst visit Peter’s wife’s mother, and the captain’s servant. And as thou preservedst Toby and Sarah by thy Angel from danger, so restore unto this sick person \emph{his} former health, if it be thy will.  \R Amen.
% \end{leftbar}
\smallskip

Sanctify this trial unto \emph{him}; that the sense of \emph{his} weakness may add strength to \emph{his} faith, and seriousness to \emph{his} repentance. \R Amen.

\smallskip

May it be thy good pleasure to restore \emph{him} to \emph{his} former health, that so \emph{he} may lead the residue of \emph{his} life in thy fear, and to thy glory. \R Amen.

\smallskip

And whatsoever the issue that thou shalt ordain for \emph{him}, give \emph{him} grace so to be conformed to thy will, that \emph{he} may be made meet to dwell with thee in life everlasting; through Jesus Christ our Lord. \R Amen.

\section{Exhortation to Faith and Prayer}

{\centering\pilcrow{Then shall the Minister exhort the sick person upon such subjects as the following:}}

\drop{Our Heavenly Father, in his love for all men, uses sickness as a gracious means whereby to correct his children.}

Our Lord Jesus Christ, ever present with us, is ready to impart to us spiritual strength to use sickness well to the glory of God.

Our Lord, manifested in the Gospel as the healer of disease, is still ready to minister grace for the healing of the body.

Our Lord himself, though sinless, was made perfect through sufferings; and sinful man needs discipline in order to correct and amend in him whatever is amiss in the eyes of our heavenly Father.

The aim of the Christian, whether in health or in sickness, is that God may be glorified in him through Jesus Christ.

There is great honour in suffering if our pain be conformed to the spirit of Jesus Christ; for in the bearing of pain God manifested his will to redeem the world.

In sickness as in health we are to seek constantly the inspiration of God the Holy Ghost, the Spirit of Christ.


% \medskip
% \rubric{Or the Minister may exhort the sick person after this form,}
% \drop{Dearly beloved, know this, that Almighty God is the Lord of life and death, and of all things to them pertaining, as youth, strength, health, age, weakness, and sickness. Wherefore, whatsoever your sickness is, know you certainly, that it is God’s visitation. And for what cause soever this sickness is sent unto you; whether it be to try your patience for the example of others, and that your faith may be found in the day of the Lord laudable, glorious, and honourable, to the increase of glory and endless felicity; or else it be sent unto you to correct and amend in you whatsoever doth offend the eyes of your heavenly Father; know you certainly, that if you truly repent you of your sins, and bear your sickness patiently, trusting in God’s mercy, for his dear Son Jesus Christ’s sake, and render unto him humble thanks for his fatherly visitation, submitting yourself wholly unto his will, it shall turn to your profit, and help you forward in the right way that leadeth unto everlasting life.}

% Take therefore in good part the chastisement of the Lord: For (as Saint Paul saith in the twelfth Chapter to the Hebrews) whom the Lord loveth he chasteneth, and scourgeth every son whom he receiveth. If ye endure chastening, God dealeth with you as with sons; for what son is he whom the father chasteneth not? But if ye be without chastisement, whereof all are partakers, then are ye bastards, and not sons. Furthermore, we have had fathers of our flesh, which corrected us, and we gave them reverence: shall we not much rather be in subjection unto the Father of spirits, and live? For they verily for a few days chastened us after their own pleasure; but he for our profit, that we might be partakers of his holiness. These words, good \emph{brother}, are written in holy Scripture for our comfort and instruction; that we should patiently, and with thanksgiving, bear our heavenly Father’s correction, whensoever by any manner of adversity it shall please his gracious goodness to visit us. And there should be no greater comfort to Christian persons, than to be made like unto Christ, by suffering patiently adversities, troubles, and sicknesses. For he himself went not up to joy, but first he suffered pain; he entered not into his glory before he was crucified. So truly our way to eternal joy is to suffer here with Christ; and our door to enter into eternal life is gladly to die with Christ; that we may rise again from death, and dwell with him in everlasting life. Now therefore, taking your sickness, which is thus profitable for you, patiently, I exhort you, in the Name of God, to remember the profession which you made unto God in your Baptism. And forasmuch as after this life there is an account to be given unto the righteous judge, by whom all must be judged, without respect of persons, I require you to examine yourself and your estate, both toward God and man; so that, accusing and condemning yourself for your own faults, you may find mercy at our heavenly Father’s hand for Christ’s sake, and not be accused and condemned in that fearful judgement. Therefore I shall rehearse to you the Articles of our Faith, that you may know whether you do believe as a Christian man should, or no.

\medskip
\centeredrubric{Or if need require he shall explain to him some part of the Christian faith. Which explanation ended, he shall say,}

\drop{I exhort you in the name of God to remember the profession of faith which you made unto God in your baptism, and therefore I shall rehearse to you the Articles of our Faith, that you may shew whether you do believe as a Christian man should.}
\medskip

\bigskip

{\centering\pilcrow{Here the Minister shall rehearse the Articles of the Faith, saying thus,}\par}

\drop{Dost thou believe in God the Father Almighty, Maker of heaven and earth?}


And in Jesus Christ his only-begotten Son our Lord? And that he was conceived by the Holy Ghost, born of the Virgin Mary; that he suffered under Pontius Pilate, was crucified, dead, and buried; that he went down into hell, and also did rise again the third day; that he ascended into heaven, and sitteth at the right hand of God the Father Almighty; and from thence shall come again at the end of the world, to judge the quick and the dead?

And dost thou believe in the Holy Ghost; the holy Catholick Church; the Communion of Saints; the Remission of sins; the Resurrection of the flesh; and everlasting life after death?

\centerline{\rubric{The sick person shall answer,}}
\centerline{All this I stedfastly believe.}

\medskip

%1928 proposed:
\pilcrow{Thereafter, as occasion serves, the Minister shall instruct the sick person so to order his rule of prayer, for himself and others, that his days of sickness may be a time of faithful and loving intercourse with God.}


\section{Exortation to Repentance}
\pilcrow{The Minister shall examine the sick person, whether he repent him truly of his sins, and be in charity with all the world; exhorting him to forgive, from the bottom of his heart, all persons that have offended him; and if he hath offended any other, to ask them forgiveness; and where he hath done injury or wrong to any man, that he make amends to the uttermost of his power.}

\smallskip

\pilcrow{And if he have not before disposed of his goods, let him then be admonished to make his Will, and to declare his Debts, what he oweth, and what is owing unto him; for the better discharging of his conscience, and the quietness of his Executors. But men should often be put in remembrance to take order for the settling of their temporal estates, whilst they are in health.}

% These next two omitted in 1928:
% \pilcrow{These words before rehearsed may be said before the Minister begin his Prayer, as he shall see cause.}

\pilcrow{The Minister should not omit earnestly to move such sick persons as are of ability to be liberal to the poor.}

\medskip
\centerline{\rubric{Then shall the Priest say,}}
%1928, 1929
\drop{Forasmuch as after this life there is an account to be given unto the righteous Judge, by whom all must be judged, without respect of persons, I require you to examine yourself and your state, both toward God and man; so that accusing and condemning yourself for your own faults, you may find mercy at our heavenly Father’s hand for Christ’s sake.}

\medskip

\centeredrubric{After such examination he shall say,}
\drop{Remember not, Lord, our offences, neither take thou vengeance of our sins; spare us, good Lord, spare thy people, whom thou hast redeemed with thy most precious blood, and be not angry with us for ever.}

\R Spare us, good Lord.


\medskip

\pilcrow{Here shall the sick person be moved to make a special confession of his sins, if he feel his conscience troubled with any weighty matter,
in this or other like form.%1928 proposed
}
\drop{I confess to God Almighty, the Father, the Son, and the Holy Ghost, that I have sinned in thought, word, and deed, through my own grievous fault; wherefore I pray God to have mercy on me. And especially I have sinned in these ways....}

\medskip

\centeredrubric{After which confession, the Priest shall absolve him (if he humbly and heartily desire it) after this sort.}

\drop{Our Lord Jesus Christ, who hath left power to his Church to absolve all sinners who truly repent and believe in him, of his great mercy forgive thee thine offences: And by his authority committed to me, I absolve thee from all thy sins, \grealtcross\  In the Name of the Father, and of the Son, and of the Holy Ghost. \R Amen.}

\medskip
\centerline{\rubric{And then he shall say the Collect following.}}
\centerline{Let us pray.}
\drop{O most merciful God, who, according to the multitude of thy mercies, dost so put away the sins of those who truly repent, that thou rememberest them no more: Open thine eye of mercy upon this thy servant, who most earnestly desireth pardon and forgiveness.
\vspace{-1em}
\begin{leftbar}
Renew in \emph{him}, most loving Father, whatsoever hath been decayed by the fraud and malice of the devil, or by \emph{his} own carnal will and frailness; preserve and continue this sick member in the unity of the Church; consider \emph{his} contrition, accept \emph{his} tears, asswage \emph{his} pain, as shall seem to thee most expedient for \emph{him}.
\end{leftbar}
\vspace{-1em}

And forasmuch as he putteth \emph{his} full trust only in thy mercy, impute not unto \emph{him} \emph{his} former sins, but strengthen \emph{him} with thy blessed Spirit; and, when thou art pleased to take \emph{him} hence, take \emph{him} unto
% thy favour,
thine everlasting favour;
through the merits of thy most dearly beloved Son Jesus Christ our Lord. 
\R Amen.}

%green book
\section{The Blessing and Anointing of the Sick}
% 1928 An Act of Prayer and Blessing
% 1929 ANOINTING, AND LAYING ON OF HANDS
% SA THE ANOINTING OF THE SICK 


\rubric{Anthem.} O Saviour of the world, who by thy Cross and precious Blood hast redeemed us, save us, and help us, we humbly beseech thee, O Lord.

% \subsection[{Psalm 71}]{\stylesubsec{Psalm 71.}{In te, Domine, speravi.}{}}
% \drop{In thee, O {\scshape Lord}, have I put my trust; let me never be put to confusion, \star\ but rid me and deliver me in thy righteousness; incline thine ear unto me, and save me.}

% 2\enspace Be thou my strong hold, whereunto I may alway resort: \star\ thou hast promised to help me, for thou art my house of defence and my castle.

% 3\enspace Deliver me, O my God, out of the hand of the ungodly, \star\ out of the hand of the unrighteous and cruel man.

% 4\enspace For thou, O Lord {\scshape God}, art the thing that I long for: \star\ thou art my hope, even from my youth.

% 5\enspace Through thee have I been holden up ever since I was born: \star\ thou art he that took me out of my mother’s womb: my praise shall be always of thee.

% 6\enspace I am become as it were a monster unto many, \star\ but my sure trust is in thee.

% 7\enspace O let my mouth be filled with thy praise, \star\ that I may sing of thy glory and honour all the day long.

% 8\enspace Cast me not away in the time of age; \star\ forsake me not when my strength faileth me.

% 9\enspace For mine enemies speak against me; and they that lay wait for my soul take their counsel together, saying, \star\ God hath forsaken him; persecute him, and take him, for there is none to deliver him.

% 10\enspace Go not far from me, O God; \star\ my God, haste thee to help me.

% 11\enspace Let them be confounded and perish that are against my soul; \star\ let them be covered with shame and dishonour that seek to do me evil.

% 12\enspace As for me, I will patiently abide alway, \star\ and will praise thee more and more.

% 13\enspace My mouth shall daily speak of thy righteousness and salvation; \star\ for I know no end thereof.

% 14\enspace I will go forth in the strength of the Lord {\scshape God}, \star\ and will make mention of thy righteousness only.

% 15\enspace Thou, O God, hast taught me from my youth up until now; \star\ therefore will I tell of thy wondrous works.

% 16\enspace Forsake me not, O God, in mine old age, when I am gray-headed, \star\ until I have shewed thy strength unto this generation, and thy power to all them that are yet for to come.

% 17\enspace Thy righteousness, O God, is very high, \star\ and great things are they that thou hast done: O God, who is like unto thee!

\subsection[{Psalm 121}]{\stylesubsec{Psalm 121.}{Levavi oculus.}{}}
% \drop{In thee, O {\scshape Lord}, have I put my trust; let me never be put to confusion, \star\ but rid me and deliver me in thy righteousness; incline thine ear unto me, and save me.}


\drop{I will lift up mine eyes unto the hills; \star\ from whence cometh my help?}

2\enspace My help cometh even from the {\scshape Lord}, \star\ who hath made heaven and earth.

3\enspace He will not suffer thy foot to be moved; \star\ and he that keepeth thee will not sleep.

4\enspace Behold, he that keepeth israel \star\ shall neither slumber nor sleep.

5\enspace The {\scshape Lord} himself is thy keeper; \star\ the {\scshape Lord} is thy defence upon thy right hand;

6\enspace So that the sun shall not burn thee by day, \star\ neither the moon by night.

7\enspace The {\scshape Lord} shall preserve thee from all evil; \star\ yea, it is even he that shall keep thy soul.

8\enspace The {\scshape Lord} shall preserve thy going out, and thy coming in, \star\ from this time forth for evermore.

Glory be to the Father, and to the Son, \star\  and to the Holy Ghost;

As it was in the beginning, is now, and ever shall be, \star\  world without end. Amen.

\smallskip
\centeredrubric{Or any other Psalm, such as the following: \emph{23, 27, 43, 71 (\emph{verses} 1–17), 77, 86, 91, 103, 130, 142, 146.}}

\smallskip
\rubric{Anthem.} O Saviour of the world, who by thy Cross and precious Blood hast redeemed us, save us, and help us, we humbly beseech thee, O Lord.

\medskip

\pilcrow{Then shall the Minister say (laying his hands upon the sick person if desired),} %1928
\drop{O Almighty God, who art the giver of all health, and the aid of them that seek to thee for succour: We call upon thee for thy help and goodness mercifully to be shewed upon this thy servant, that being healed of \emph{his} infirmities, \emph{he} may give thanks unto thee in thy holy Church; through Jesus Christ our Lord. \R Amen.}





\begin{leftbar}
    %nonjurors 1718
\pilcrow{If the oil is to be then hallowed, he shall say standing the following prayer.}
\drop{O Almighty Lord God, who hast taught us by thy holy Apostle Saint James to anoint the sick with oil, that they may recover their health and render thanks unto thee for the same; Bless \grealtcross\ this oil, we beseech thee, that whosoever may be anointed therewith, may be delivered from all troubles of body and mind, and from every assault of the powers of evil; through Jesus Christ our Lord. \R Amen.}
\end{leftbar}

%1929
\pilcrow{Then shall the Priest, if the sick person so desire it, proceed to anoint him with oil, saying as followeth:}

%1549
% \pilcrow{If the sick person desire to be anointed, then shall the Priest anoint him upon the forehead or breast only, making the sign of the cross, saying thus,}
\drop{N, I anoint thee with hallowed oil, \grealtcross\ In the Name of the Father, and of the Son, and of the Holy Ghost. \R Amen.}


%greenbook
\centerline{\rubric{He may add the following benediction.}}

\drop{As with this visible oil thy body outwardly is anointed, so may our heavenly Father, God Almighty, grant of his infinite goodness, that thy soul inwardly may be anointed with the Holy Ghost, who is the Spirit of all strength, comfort, relief, and gladness.}
May he % green book
%  and 1549
vouchsafe of his great mercy (if it be his blessed will) to restore unto thee thy bodily health, and strength to serve him joyfully; and send thee release of all thy pains, troubles, and diseases both in body and mind.
\vspace{-1em}
\begin{leftbar}
And howsoever his goodness, by his divine and unsearchable providence, shall dispose of thee: we, his unworthy ministers and servants, humbly beseech the eternal majesty to do with thee according to the multitude of his innumerable mercies, and to pardon thee all thy sins and offences, committted by all thy bodily senses, passions, and carnal affections.
\end{leftbar}

\vspace{-1.7em}

\begin{leftbar}
    May he % green book
% who %1549
also vouchsafe mercifully to grant unto thee ghostly strength by his Holy Spirit to withstand and overcome all temptations and assaults of thine adversary, that in no wise he prevail against thee, but that thou mayest have perfect victory and triumph against the devil, sin, and death;
\end{leftbar}
\vspace{-1em}

Through Christ our Lord, who by his death hath overcome the prince of death; and with the Father and the Holy Ghost everymore liveth and reigneth God, world without end.  \R Amen.

% Usque quo, Domine. Psalm xiii.

% How long wilt thou forget me, (O Lord,) for ever? how long wilt thou hyde thy face from me? How long shall I seke counsell in my soule? and be so vexed in myne herte? how long shall myne enemye triumph over me? Consydre, and heare me, (O lord my God): lighten myne iyes, that I slepe not in death. Leste myne enemy saye: I have prevayled against hym: for yf I be cast downe, they that trouble me will reioyce at it. But my trust is in thy mercy: and my herte is joyfull in thy salvacion. I will sing of the lord, because he hath delte so lovingly with me. Yea, I wyll prayse the name of the Lord the most highest. Glory be to the, \etc As it was in the, \etc

\medskip
\centerline{\rubric{Then shall the Minister say,}}
\drop{The Almighty Lord, who is a most strong tower to all them that put their trust in him, to whom all things in heaven, in earth, and under the earth, do bow and obey, be now and evermore thy defence; and make thee know and feel, that there is none other Name under heaven given to man, in whom, and through whom, thou mayest receive health and salvation, but only the Name of our Lord Jesus Christ. \R Amen.}

\centerline{\pilcrow{And after that he shall say,}}
\drop{Unto God’s gracious mercy and protection we commit thee. The {\scshape Lord} \cross  bless thee, and keep thee. The {\scshape Lord} make his face to shine upon thee, and be gracious unto thee. The {\scshape Lord} lift up his countenance upon thee, and give thee peace, both now and evermore. \R Amen.}


\section{The Communion of the Sick}
\pilcrow{Forasmuch as all mortal men be subject to many sudden perils, diseases, and sicknesses, and ever uncertain what time they shall depart out of this life; therefore, to the intent they may always be in a readiness to die, whensoever it shall please Almighty God to call them, the Curate shall diligently from time to time (but especially in the time of pestilence, or other infectious sickness) exhort their Parishioners to the often receiving of the Holy Communion of the Body and Blood of our Saviour Christ, when it shall be publickly administered in the church; that so doing, they may, in case of sudden visitation, have the less cause to be disquieted for lack of the same.}


% \begin{leftbar}
\pilcrow{The Curate shall also instruct the people concerning the Communion of the Sick, as occasion shall require, that they may not be in ignorance that men can receive the Holy Sacrament in their homes, if they be unable, for any just cause, to come to the church.}
% \end{leftbar}

\pilcrow{But if the sick person be not able to come to the Church, and yet is desirous to receive the Communion in his house; then he must give timely notice to the Priest, signifying also, as far as he may, whether there be some to communicate with him; as is much to be desired.}

\smallskip

\pilcrow{When the consecrated Bread and Wine are taken from the church to the sick person, before the Priest administers the Holy Sacrament, he shall use at least the parts of the \emph{Order of Communion} on \emph{pg.~\pageref{reservedSacrament}} here named: the \emph{General Confession} and \emph{Absolution}, 
% (which may be in the shorter form), 
and the prayer \emph{We do not presume, \etc}, except when extreme sickness shall otherwise require: and after the delivery of the Sacrament of Christ’s Body and Blood with the appointed words, he shall say the \emph{Lord’s Prayer} and the \emph{Blessing}. And immediately thereafter any of the consecrated Elements that remain over shall be reverently consumed, or else taken back to the church.}


% \stylesec{The Celebration}{of the}{Holy Communion for the Sick}
% \smallskip
\pilcrow{And a convenient place in the sick man’s house, together with all things necessary, having been prepared that the Curate may reverently minister, he shall there celebrate the \emph{Order of Communion}, according to the form in this Book prescribed; save only that he may, at his discretion, begin with the Collect, Epistle, and Gospel here following, or else with those of the Day.}

% \drop{O praise the {\scshape Lord}, all ye heathen; praise him, all ye nations. For his merciful kindness is ever more and more towards us; and the truth of the {\scshape Lord} endureth for ever.  Praise the {\scshape Lord}. Glory be to the Father, and to the Son, and to the Holy Ghost;  As it was in the beginning, is now, and ever shall be, world without end. Amen.}

% \medskip
% \centerline{Lord, have mercy upon us.}
% \centerline{\emph{Christ, have mercy upon us.}}
% \centerline{Lord, have mercy upon us.}
% \medskip
% \V The Lord be with you.  \R And with thy spirit.
% \centerline{Let us pray.}
\subsection{\stylesubsec{}{The Collect.}{}}


% \begin{leftbar}

\drop{Almighty and immortal God, giver of life and health: We beseech thee to hear our prayers for this thy servant, that by thy blessing upon \emph{him} and upon those who minister to \emph{him}, \emph{he} may be restored to health of body and mind, and give thanks to thee in thy holy Church; through Jesus Christ our Lord. \R Amen.}

\centerline{\rubric{Or this.}}
\drop{Almighty, everliving God, Maker of mankind, who dost correct those whom thou dost love, and chastise every one whom thou dost receive: We beseech thee to have mercy upon this thy servant visited with thine hand, and to grant that \emph{he} may take \emph{his} sickness patiently, and recover \emph{his} bodily health, (if it be thy gracious will); and whensoever \emph{his} soul shall depart from the body, it may be without spot presented unto thee; through Jesus Christ our Lord. \R Amen.}

% The following collect is one of the Postcommunions, made singular.
\centerline{\rubric{Or this.}}
\drop{Assist us mercifully, O Lord, in these our supplications and prayers, and dispose the way of thy servant towards the attainment of everlasting salvation; that among all the changes and chances of this mortal life, \emph{he} may ever be defended by thy most gracious and ready help; through Jesus Christ our Lord. \R Amen.}
% \end{leftbar}

% \medskip
% \subsection{\stylesubsec{}{The Epistle.}{Hebrews 12.~5.}}
% \drop{My son, despise not thou the chastening of the Lord, nor faint when thou art rebuked of him. For whom the Lord loveth he chasteneth; and scourgeth every son whom he receiveth.}
% % \begin{leftbar}

%     \smallskip
% \centerline{\rubric{Or this.}}
% \vspace{-10pt}
\subsection{\stylesubsec{}{The Epistle.}{2 Corinthians 1.~3.}}
\drop{Blessed be God, even the Father of our Lord Jesus Christ, the Father of mercies, and the God of all comfort; who comforteth us in all our tribulation, that we may be able to comfort them which are in any trouble, by the comfort wherewith we ourselves are comforted of God. For as the sufferings of Christ abound in us, so our consolation also aboundeth by Christ.}
% \end{leftbar}

\medskip
% \subsection{\stylesubsec{}{The Gospel.}{St.~John 5.~24.}}
% \drop{Verily, verily I say unto you, He that heareth my word, and believeth on him that sent me, hath everlasting life, and shall not come into condemnation; but is passed from death unto life.}

% % \begin{leftbar}
%     \smallskip
% \centerline{\rubric{Or this.}}
% \vspace{-10pt}
\subsection{\stylesubsec{}{The Gospel.}{St.~John 10.~14, 15; 27–30.}}
\drop{I am the good shepherd; and I know mine own, and mine own know me, even as the Father knoweth me, and I know the Father; and I lay down my life for the sheep. My sheep hear my voice, and I know them, and they follow me: and I give unto them eternal life; and they shall never perish, and no one shall pluck them out of my hand. My Father, which hath given them unto me, is greater than all; and no one is able to pluck them out of the Father’s hand. I and the Father are one.}
% \end{leftbar}

\pilcrow{After which the Priest shall proceed according to the form before prescribed for the Order of Communion.}
%, beginning at  these words \emph{Ye that do truly,} \etc}
% the {\emph Offertory}, \emph{pg.~\pageref{offertory}}}



% \drop{I will offer to thee the sacrifice of thanksgiving, and will call upon the Name of the {\scshape Lord}.}

% \V The Lord be with you. \R And with thy spirit.

% \V Lift up your hearts.  \R We lift them up unto the Lord.

% \V Let us give thanks unto our Lord God. \R It is meet and right so to do.


% \centerline{\rubric{Then shall the Priest turn to the Lord’s Table, and say,}}

% \drop{It is very meet, right, and our bounden duty, that we should at all times, and in all places, give thanks unto thee, O Lord, Holy Father, Almighty, Everlasting God.  Therefore with Angels and Archangels, and with all the company of heaven, we laud and magnify thy glorious Name; evermore praising thee, and saying,}
% \smallskip

% \drop{Holy, holy, holy, Lord God of hosts, heaven and earth are full of thy glory: Glory be to thee, O Lord most High. \grecross\ Blessed is he that cometh in the Name of the Lord;
% Hosanna in the highest.}

% \bigskip

% \pilcrow{When the Priest, standing before the Table, hath so ordered the Bread and Wine, that he may with the more readiness and decency break the Bread before the people, and take the Cup into his hands, he shall say the Prayer of Consecration, as followeth.}
% \drop{Almighty God, our heavenly Father, who of thy tender mercy didst give thine only Son Jesus Christ to suffer death upon the Cross for our redemption; who made there (by his one oblation of himself once offered) a full, perfect, and sufficient sacrifice, oblation, and satisfaction, for the sins of the whole world; and did institute, and in his holy Gospel command us to continue, a perpetual memory of that his precious death, until his coming again;

% Hear us, O merciful Father, we most humbly beseech thee; and grant that we receiving these thy creatures of bread and wine, according to thy Son our Saviour Jesus Christ’s holy institution, in remembrance of his death and passion, may be partakers of his most blessed \grealtcross\ Body and \grealtcross\ Blood: 

% Who, in the same night that he was betrayed, \footnote{\rubric{Here the Priest is to take the Paten unto his hands:}}took Bread; and, when he had given thanks, \footnote{\rubric{And here to break the Bread:}}he brake it, and gave it to his disciples, saying, Take, eat, \footnote{\rubric{And here to lay his hand upon all the Bread.}}{\scshape this is my Body which is given for you}: Do this in remembrance of me. Likewise after supper he \footnote{\rubric{Here he is to take the Cup into his hand:}}took the Cup; and, when he had given thanks, he gave it to them, saying, Drink ye all of this; \footnote{\rubric{And here to lay his hand upon every vessel (be it Chalice or Flagon) in which there is any Wine to be consecrated.}}{\scshape for this is my Blood of the New Testament, which is shed for you and for many for the remission of sins}: Do this, as oft as ye shall drink it, in remembrance of me.}

% Wherefore, O Lord and heavenly Father, we thy humble servants, having in remembrance the precious death of thy dear Son, his mighty resurrection and glorious ascension, looking also for his coming again, do render unto thee most hearty thanks for the innumerable benefits which he hath procured unto us; and we entirely desire thy fatherly goodness mercifully to accept this our sacrifice of praise and thanksgiving; most humbly beseeching thee to grant, that by the merits and death of thy Son Jesus Christ, and through faith in his blood, we and all thy whole Church may obtain remission of our sins, and all other benefits of his passion.

% And here we offer and present unto thee, O Lord, ourselves, our souls and bodies, to be a reasonable, holy, and lively sacrifice unto thee; 
% and we pray thee of thine almighty goodness to send upon us, and upon these thy gifts, thy holy and blessed Spirit, who is the Sanctifier and the Giver of life; humbly beseeching thee, that all we, who are partakers of this holy Communion, may be fulfilled with thy grace and heavenly \grecross\ benediction. 

% And although we be unworthy, through our manifold sins, to offer unto thee any sacrifice, yet we beseech thee to accept this our bounden duty and service; not weighing our merits, but pardoning our offences;

% Through Jesus Christ our Lord; by whom, and with whom, in the unity of the Holy Ghost, all honour and glory be unto thee, O Father Almighty, world without end. \R Amen.

% \smallskip
% {\centering\footnotesize\rubric{Here shall the people join with the Priest in the Lord’s Prayer, the Priest first saying,}\par}
% As our Saviour Christ hath commanded and taught us we are bold to say,
% \drop{Our Father, which art in heaven, Hallowed be thy Name. Thy kingdom come. Thy will be done, in earth as it is in heaven. Give us this day our daily bread. And forgive us our trespasses, As we forgive them that trespass against us. And lead us not into temptation; But deliver us from evil.  For thine is the kingdom, The power, and the glory, For ever and ever. Amen.}

% \bigskip
% \pilcrow{Then shall the Priest say to them that come to receive the holy Communion,}
% \drop{Draw near with faith, and take this Holy Sacrament to your comfort; and make your humble confession to Almighty God, meekly kneeling upon your knees.}

% \smallskip
% \rubric{Then shall be said by the Minister and people together,}
% \drop{We confess to God Almighty, the Father, the Son, and the Holy Ghost, that we have sinned in thought, word, and deed, through our own grievous fault.  Wherefore we pray God to have mercy upon us.}

% \medskip
% {\centering\footnotesize\rubric{Then shall the Priest standing up, and turning himself to the people, pronounce this Absolution.}\par}
% \drop{Almighty God have mercy upon you, forgive you all your sins, and deliver you from all evil, confirm and strengthen you in all goodness, and bring you to everlasting life; through Jesus Christ our Lord. \R Amen.}

% \centerline{\pilcrow{Then shall the Priest say,}}
% Hear what comfortable words our Saviour Christ saith unto all that truly turn to him.
% \drop{Come unto me all that travail and are heavy laden, and I will refresh you.}\scripture{St.~Matthew xj.~28}

% So God loved the world, that he gave his only-begotten Son, to the end that all that believe in him should not perish, but have everlasting life.\scripture{St.~John iij.~16}

% \centerline{Hear also what Saint Paul saith.}

% This is a true saying, and worthy of all men to be received, That Christ Jesus came into the world to save sinners.\scripture{1 Timothy i.~15.}

% \centerline{Hear also what Saint John saith.}

% If any man sin, we have an Advocate with the Father, Jesus Christ the righteous; and he is the propitiation for our sins.\scripture{1 St.~John ij.~1.}

% \medskip

% {\centering\footnotesize\rubric{Then shall the Priest, kneeling down at the Lord’s Table, say in the name of all them that shall receive the Communion this Prayer following.}\par}
% \drop{We do not presume to come to this thy Table, O merciful Lord, trusting in our own righteousness, but in thy manifold and great mercies. We are not worthy so much as to gather up the crumbs under thy Table. But thou art the same Lord, whose property is always to have mercy: Grant us therefore, gracious Lord, so to eat the flesh of thy dear Son Jesus Christ, and to drink his blood, that our sinful bodies may be made clean by his body, and our souls washed through his most precious blood, and that we may evermore dwell in him, and he in us. Amen.}

\medskip

\pilcrow{At the time of the distribution of the Holy Sacrament, the priest shall first receive the Communion himself, and after minister unto them that are appointed to communicate with the sick, and last of all to the sick person.}
% \medskip

% {\centering\footnotesize\rubric{And, when he delivereth the Bread to any one, he shall say,}\par}
% \drop{The Body of our Lord Jesus Christ, which was given for thee, preserve thy body and soul unto everlasting life.}

% {\centering\footnotesize\rubric{And the Minister that delivereth the Cup to any one shall say,}\par}
% \drop{The Blood of our Lord Jesus Christ, which was shed for thee, preserve thy body and soul unto everlasting life.}

% \medskip

% {\footnotesize\rubric{When all have communicated, the Minister shall return to the Lord’s Table, and reverently place upon it what remaineth of the consecrated Elements, covering the same with a fair linen cloth.}\par}


% After shall be said as followeth.

% \drop{ALMIGHTY and everliving God, we most heartily thank thee, for that thou dost vouchsafe to feed us, who have duly received these holy mysteries, with the spiritual food of the most precious Body and Blood of thy Son our Saviour Jesus Christ; and dost assure us thereby of thy favour and goodness towards us; and that we are very members incorporate in the mystical body of thy Son, which is the blessed company of all faithful people; and are also heirs through hope of thy everlasting kingdom, by the merits of the most precious death and passion of thy dear Son. And we most humbly beseech thee , O heavenly Father, so to assist us with thy grace, that we may continue in that holy fellowship, and do all such good works as thou hast prepared for us to walk in; through Jesus Christ our Lord, to whom, with thee and the Holy Ghost, be all honour and glory, world without end. Amen.}
% \medskip

% {\centering\footnotesize\rubric{Then the Priest shall let them depart with this Blessing.}\par}
% \drop{The peace of God, which passeth all understanding, keep your hearts and minds in the knowledge and love of God, and of his son Jesus Christ our Lord: and the blessing of God Almighty, the Father, \grealtcross\ the Son, and the Holy Ghost, be amongst you and remain with you always. \R Amen.}

% \bigskip

% \begin{leftbar}
% \pilcrow{In case of extreme necessity the Priest may beg in with the Consecration and, immediately after the delivery of the Holy Sacrament to the sick person, end with the Blessing.}
% \end{leftbar}


\medskip

\pilcrow{The Priest shall instruct the people that if any man, by reason of great sickness, or any other just impediment, be not able at any time to receive the Sacrament of Christ’s Body and Blood, yet if he do truly repent him of his sins, and stedfastly believe that Jesus Christ both suffered death upon the Cross for him, and shed his Blood for his redemption, earnestly remembering the benefits he hath thereby, and giving him hearty thanks therefore, he doth eat and drink the Body and Blood of our Saviour Christ profitably to his Soul’s health, although he do not receive the Sacrament with his mouth.}

% When the sick person is visited, and receiveth the holy Communion all at one time, then the Priest, for more expedition, shall cut off the form of the Visitation at the Psalm [In thee, O Lord, have I put my trust \etc] and go straight to the Communion.

% \pilcrow{In the time of the plague, sweat, or such other like contagious times of sickness or diseases, when none of the Parish or neighbors can be gotten to communicate with the sick in their houses, for fear of the infection, upon special request of the diseased, the Minister may only communicate with him.}

\bigskip


\centerline{\rule{0.5\textwidth}{0.5pt}}
\medskip
\pilcrow{When it is desirable to administer both kinds together, the words of administration shall be said thus}

\smallskip

\drop{The Body of our Lord Jesus Christ, which was given for thee, and his Blood which was shed for thee, preserve thy body and soul unto everlasting life.}

\smallskip

\centeredrubric{Take this in remembrance that Christ died for thee, and feed on him in thy heart by faith with thanksgiving.}


\medskip

\pilcrow{{\scshape Note}, that the same order shall be observed, with the permission of the Bishop, when it is deemed necessary, through grave danger of infection, to administer both kinds together to certain communicants at the open Communion.}


\section[Special Prayers]{Special Prayers to be Used as Occasion may Serve}
\subseccaption{}{A Litany for the Sick or Dying.}

\drop{O God the Father,}

\qquad\emph{Have mercy.}

O God the Son,

\qquad\emph{Have mercy.}

O God the Holy Ghost,

\qquad\emph{Have mercy.}

O Holy Trinity, one God,

\qquad\emph{Have mercy.}

Remember not, Lord, our offences.

\qquad\emph{Spare us, Good Lord.}

From all evil and sin,

\qquad\emph{Good Lord, deliver \emph{him.}}

From the assaults of the devil,

\qquad\emph{Good Lord, deliver \emph{him.}}

From thy wrath, and from everlasting damnation,

\qquad\emph{Good Lord, deliver \emph{him.}}

In the hour of death,

\qquad\emph{Good Lord, deliver \emph{him.}}

In the day of judgement,

\qquad\emph{Good Lord, deliver \emph{him.}}

By the mystery of thine Incarnation,

\qquad\emph{Save \emph{him}, O Lord.}

By thy Cross and Passion,

\qquad\emph{Save \emph{him.}, O Lord.}

By thy Resurrection and final Triumph,

\qquad\emph{Save \emph{him}, O Lord.}

That it may please thee to grant \emph{him} relief in pain;

\qquad\emph{We beseech thee to hear us.}

To give \emph{him} such health as is agreeable to thy will;

\qquad\emph{We beseech thee to hear us.}

That it may please thee to deliver \emph{his} soul;

\qquad\emph{We beseech thee to hear us.}

To cleanse \emph{him} from \emph{his} sin;

\qquad\emph{We beseech thee to hear us.}
    
That it may please thee to receive \emph{him} to thyself;

\qquad\emph{We beseech thee to hear us.}

To set \emph{him} in a place of light and peace;

\qquad\emph{We beseech thee to hear us.}

To number \emph{him} with thy saints and thine elect;

\qquad\emph{We beseech thee to hear us.}

Son of God;

\qquad\emph{We beseech thee to hear us.}

O Lamb of God;

\qquad\emph{Have mercy upon us.}

O Lamb of God;

\qquad\emph{Grant \emph{him} thy peace.}


\centerline{\rule{0.5\textwidth}{0.5pt}}


\centerline{\pilcrow{The following Prayers may be used as occasion requires.}}
\subseccaption{}{For Healing.}
\drop{O God, who by the might of thy command canst drive away from men’s bodies all sickness and infirmity: Be present in thy goodness with this thy servant, that \emph{his} weakness being banished, and \emph{his} health restored, \emph{he} may live to glorify thy holy Name; through our Lord Jesus Christ. \R Amen.}


\subseccaption{}{For a Sick Child.}
\drop{O Lord Jesus Christ, who didst with joy receive and bless the children brought to thee: Give thy blessing to this thy child; and in thine own time deliver \emph{him} from \emph{his} bodily pain, that \emph{he} may live to serve thee all \emph{his} days. \R Amen.}


\subseccaption{}{For one troubled in Conscience.}
\drop{O blessed Lord, the Father of mercies and the God of all comfort; We beseech thee, look down in pity and compassion on thy servant, whose soul is full of trouble: give \emph{him} a right understanding of \emph{himself}, and also of thy will for \emph{him}, that \emph{he} may neither cast away \emph{his} confidence in thee, nor place it anywhere but in thee; deliver \emph{him} from the fear of evil; lift up the light of thy countenance upon \emph{him}, and give \emph{him} thine everlasting peace; through the merits and mediation of Jesus Christ our Lord. \R Amen.}

%Am1928
\subseccaption{}{For a Person under Affliction.}
\drop{O merciful God, and heavenly Father, who hast taught us in thy holy Word that thou dost not willingly afflict or grieve the children of men; Look with pity, we beseech thee, upon the sorrows of thy servant for whom our prayers are offered. Remember \emph{him}, O Lord, in mercy; endue \emph{his} soul with patience; comfort \emph{him} with a sense of thy goodness; lift up thy countenance upon \emph{him}, and give \emph{him} peace; through Jesus Christ our Lord. \R Amen.}



\subseccaption{}{For a Convalescent.}
\drop{O Lord, whose compassions fail not, and whose mercies are new every morning: We give thee hearty thanks that it hath pleased thee to give to this our \emph{brother} both relief from pain and hope of renewed health; continue, we beseech thee, in \emph{him} the good work that thou hast begun; that, daily increasing in bodily strength, and humbly rejoicing in thy goodness, \emph{he} may so order \emph{his} life and conversation as always to think and do such things as shall please thee; through Jesus Christ our Lord. \R Amen.}


\subseccaption{}{For a Dying Child.}
\drop{O Lord Jesu Christ, the only-begotten Son of God, who for our sakes didst become a babe in Bethlehem: We commit unto thy loving care this child whom thou art calling to thyself. Send thy holy angel to lead \emph{him} gently to those heavenly habitations where the souls of them that sleep in thee have perpetual peace and joy, and fold \emph{him} in the everlasting arms of thine unfailing love; who livest and reignest with the Father and the Holy Ghost, one God world without end. \R Amen.}


\subseccaption{}{Commendatory Prayers.}
\drop{Thou knowest, Lord, the secrets of our hearts; shut not thy merciful ears to our prayer; but spare us, Lord most holy, O God most mighty, O holy and merciful Saviour, thou most worthy Judge eternal, suffer us not at our last hour, for any pains of death, to fall from thee. \R Amen.}

\smallskip

\drop{Unto thee, O Lord, we commend the soul of thy servant \emph{N.}, that, dying to the world, \emph{he} may live to thee; and whatsoever sins \emph{he} has committed through the frailty of earthly life, we beseech thee to do away by thy most loving and merciful forgiveness; through Jesus Christ our Lord. \R Amen.}

\smallskip

\drop{O Almighty God, with whom do live the spirits of just men made perfect, after they are delivered from their earthly prisons: We humbly commend the soul of this thy servant, our dear \emph{brother}, into thy hands, as into the hands of a faithful Creator, and most merciful Saviour; most humbly beseeching thee, that it may be precious in thy sight. Wash it, we pray thee, in the blood of that immaculate Lamb that was slain to take away the sins of the world; that whatsoever defilements it may have contracted in the midst of this miserable and naughty world, through the lusts of the flesh, or the wiles of Satan, being purged and done away, it may be presented pure and without spot before thee; through the merits of Jesus Christ thine only Son our Lord. \R Amen.}

\smallskip

\subseccaption{}{At the Point of Death.}

\drop{Go forth upon thy journey from this world, O Christian soul,}

In the Name of God the Almighty Father who created thee. \R Amen.

In the Name of Jesus Christ who suffered for thee. \R Amen.

In the Name of the Holy Ghost who strengtheneth thee. \R Amen.

In communion with the blessed Saints, and aided by Angels and Archangels,  and all the armies of the heavenly host. \R Amen.

May thy portion this day be in peace, and thy dwelling in the heavenly Jerusalem. \R Amen.

\bigskip
{\footnotesize
{\scshape Note}.— The following prayers and passages of Holy Scripture are suitable for use with the sick person: The Collect in the Communion of the Sick and the Collects appointed for the first, second and fourth Sundays in Advent, the third, fourth, and Sixth Sundays after Epiphany, Ash Wednesday, the second Sunday in Lent, the Sunday next before Easter, the fourth Sunday after Easter, Ascension Day, the Sunday after Ascension, Trinity Sunday, the fourth, sixth, seventh, twelfth, fifteenth, eighteenth, and twenty-first Sundays after Trinity, the Transfiguration, St.~Michael and All Angels, St.~Luke the Evangelist, and All Saints’ Day.

\newcommand{\numberedSuggestion}[3]{{\addfontfeatures{Numbers={Monospaced}}#1.\enspace}{\emph{#2}:\enskip}#3\par}

\numberedSuggestion{1}{Confidence in God}{Psalms 27, 46, 91, 121; Proverbs 3.~11–26; Isaiah 26.~1–9; 40.~1–11; 40.~25 to end; Lamentations 3.~22–41; St.~Matthew 6.~24 to end; Romans 8.~31 to end.}
\numberedSuggestion{2}{Answer to Prayer}{Psalms 30, 34.}
\numberedSuggestion{3}{Prayer for Divine Aid}{Psalms 43, 86, 143; St.~James 5.~10 to end.}
\numberedSuggestion{4}{Penitence}{Psalms 51, 130.}
\numberedSuggestion{5}{Praise and Thanksgiving}{Psalms 103, 146; Isaiah 12.}
\numberedSuggestion{6}{God’s dealing with Man through Affliction}{Job 33.~14–30; Hebrews 12.~1–11.}
\numberedSuggestion{7}{Christ our Example in Suffering}{Isaiah 53; St.~Matthew 26.~36–46; St.~Luke 23.~27–49.}
\numberedSuggestion{8}{God’s call to Repentance and Faith}{Isaiah 55.}
\numberedSuggestion{9}{The Beatitudes}{St.~Matthew 5.~1–12.}
\numberedSuggestion{10}{Watchfulness}{St.~Luke 12.~32–40.}
\numberedSuggestion{11}{Christ the Good Shepherd}{Psalm 23; St.~John 10.~1–18.}
\numberedSuggestion{12}{The Resurrection}{St.~John 20.~1–18; 20.~19 to end; 2 Corinthians 4.~13—5.~9.}
\numberedSuggestion{13}{Redemption}{Romans 5.~1–11; 8.~18 to end; 1 St.~John 1.~1–9.}
\numberedSuggestion{14}{Christian Love}{1 Corinthians 13.}
\numberedSuggestion{15}{Growth in Grace}{Ephesians 3.~13 to end; 6.~10–20; Philippians 3.~7–14.}
\numberedSuggestion{16}{Patience in Suffering}{St.~James 5.~10 to end.}
\numberedSuggestion{17}{God’s Love to Men}{1 St.~John 3.~1–7; 4.~9 to end.}
\numberedSuggestion{18}{The Life of the World to come}{Revelation 7.~9 to end; 21.~1–7; 21.~22 to end; 22.~1–5.}
\numberedSuggestion{19}{Our Lord’s last Discourse before his Passion}{St.~John 14, 15, 16, 17.}
\numberedSuggestion{20}{Christian Hope on the Approach of Death}{Deuteronomy 33.~27; Psalm 16.~9 to end; Psalm 23; St.~John 3.~16; 2 Corinthians 4.~16—5.~1; Revelation 21.~4–7.}
}

\fleuron

% split in two in 1928: for sick, and for dying.
% \subseccaption{}{A Prayer for a sick Child.}
% \drop{O Almighty God, and merciful Father, to whom alone belong the issues of life and death: Look down from heaven, we humbly beseech thee, with the eyes of mercy upon this child now lying upon the bed of sickness: Visit him, O Lord, with thy salvation; deliver him in thy good appointed time from his bodily pain, and save his soul for thy mercies’ sake: That, if it shall be thy pleasure to prolong his days here on earth, he may live to thee, and be an instrument of thy glory, by serving thee faithfully, and doing good in his generation; or else receive him into those heavenly habitations, where the souls of them that sleep in the Lord Jesus enjoy perpetual rest and felicity. Grant this, O Lord, for thy mercies’ sake, in the same thy Son our Lord Jesus Christ, who liveth and reigneth with thee and the Holy Ghost, ever one God, world without end. \R Amen.}

% \subseccaption{}{A Prayer for a sick person, when there appeareth small hope of recovery.}
% \drop{O Father of mercies, and God of all comfort, our only help in time of need: We fly unto thee for succour in behalf of this thy servant, here lying under thy hand in great weakness of body. Look graciously upon him, 0 Lord; and the more the outward man decayeth, strengthen him, we beseech thee, so much the more continually with thy grace and Holy Spirit in the inner man. Give him unfeigned repentance for all the errors of his life past, and stedfast faith in thy Son Jesus; that his sins may be done away by thy mercy, and his pardon sealed in heaven, before he go hence, and be no more seen. We know, 0 Lord, that there is no word impossible with thee; and that, if thou wilt, thou canst even yet raise him up, and grant him a longer continuance amongst us: Yet, forasmuch as in all appearance the time of his dissolution draweth near, so fit and prepare him, we beseech thee, against the hour of death, that after his departure hence in peace, and in thy favour, his soul may be received into thine everlasting kingdom, through the merits and mediation of Jesus Christ, thine only Son, our Lord and Saviour. \R Amen.}

%in the "Commendatory Prayers", slightly shortened.
% \subseccaption{}{A commendatory Prayer for a sick person at the point of departure.}
% \drop{O Almighty God, with whom do live the spirits of just men made perfect, after they are delivered from their earthly prisons: We humbly commend the soul of this thy servant, our dear brother, into thy hands, as into the hands of a faithful Creator, and most merciful Saviour; most humbly beseeching thee, that it may be precious in thy sight. Wash it, we pray thee, in the blood of that immaculate Lamb, that was slain to take away the sins of the world; that whatsoever defilements it may have contracted in the midst of this miserable and naughty world, through the lusts of the flesh, or the wiles of Satan, being purged and done away, it may be presented pure and without spot before thee. And teach us who survive, in this and other like daily spectacles of mortality, to see how frail and un, certain our own condition is; and so to number our days, that we may seriously apply our hearts to that holy and heavenly wisdom, whilst we live here, which may in the end bring us to life everlasting, through the merits of Jesus Christ thine only Son our Lord. \R Amen.}

% \subseccaption{}{A Prayer for persons troubled in mind or in conscience.}
% \drop{O blessed Lord, the Father of mercies, and the God of all comforts: We beseech thee, took down in pity and compassion upon this thy afflicted servant. Thou writest bitter things against him, and makest him to possess his former iniquities; thy wrath lieth hard upon him, and his soul is full of trouble: But, 0 merciful God, who hast written thy holy Word for our learning, that we, through patience and comfort of thy holy Scriptures, might have hope; give him a right understanding of himself, and of thy threats and promises; that he may neither cast away his confidence in thee, nor place it any where but in thee. Give him strength against all his temptations, and heal all his distempers. Break not the bruised reed, nor quench the smoking flax. Shut not up thy tender mercies in displeasure; but make him to hear of joy and gladness, that the bones which thou hast broken may rejoice. Deliver him from fear of the enemy, and lift up the light of thy countenance upon him, and give him peace, through the merits and mediation of Jesus Christ our Lord. \R Amen.}

% \begin{leftbar} %scottish 1912
% \subseccaption{}{A Prayer for the recovery of a sick person.}
% \drop{Almighty and immortal God, giver of life and health; We beseech thee to hear our prayers for thy servant N, for whom we implore thy mercy, that by thy blessing upon him and upon those who minister to him of thy healing gifts, he may be restored, if it be thy gracious will, to health of body and mind, and give thanks to thee in thy holy Church; through Jesus Christ our Lord. \R Amen.}
% \end{leftbar}

% If any question arise as to the manner of doing anything that is here enjoined or permitted, it shall be referred to the Bishop for his decision.
