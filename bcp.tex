\documentclass[foolscapvopaper,10pt,twoside,openany,extrafontsizes,final]{memoir}
% \documentclass[ebook,12pt,twoside,openany,extrafontsizes,final]{memoir}
\usepackage{svg}
\usepackage{fontspec}
\usepackage{lettrine}
\usepackage[normalem]{ulem} % for strike-through \sout{}
\usepackage[autocompile]{gregoriotex}
\usepackage{multicol}
\usepackage{multirow}
\usepackage{wrapfig}
\usepackage{lscape}
% \usepackage[rflt]{floatflt}
\usepackage{longtable,tabu}
\usepackage{xtab}
\usepackage{tabularx}
\usepackage{tabularray}
\usepackage[perpage]{footmisc}
% \usepackage{lua-visual-debug}
% a Font that supplies Astrological glyphs
\newfontfamily\versiculus{DejaVu Sans}

\directlua{dofile("lua-functions.lua")}
\newcommand{\drop}[1]{
    \directlua{drop("\luaescapestring{\unexpanded{#1}}")}
}


\setsecnumdepth{part} % Only number the parts.

% Chapters
\setlength{\beforechapskip}{0\baselineskip}
\setlength{\afterchapskip}{.5\baselineskip}
\renewcommand*{\chaptitlefont}{\sffamily}
\renewcommand*{\printchaptertitle}[1]{\centering\LARGE{\chaptitlefont #1}}

% Sections: Major chapter divisions; Days in the Psalter
\setbeforesecskip{1em plus 2em minus 1ex}
\setsecheadstyle{\scshape\centering}
\setaftersecskip{1pt}

%Subsection: Collects, Canticles, Psalms
%{\schshape Title}{\itshape incipit}{\normal scriptural reference.}
\setbeforesubsecskip{1em plus 0ex minus 1ex}
\setsubsecheadstyle{\centering}
\setaftersubsecskip{1pt}

\newcommand{\stylechapter}[3]{{\normalfont\scshape\normalsize#1\par}#2\par{\normalfont\scshape\normalsize#3}}
\newcommand{\stylesubsec}[3]{\small{\mbox{\scshape#1}} {\mbox{\itshape\red #2}} {\mbox{#3}}}
\newcommand{\stylesec}[3]{\section[#3]{#1\\ {\normalfont\small#2\\ } #3}}
\newcommand{\scripture}[1]{\hspace*{\fill}{\mbox{\small\itshape\red #1}}\par\smallskip}
%Subsubsection
%Descriptive names of prayers, etc.
\setbeforesubsubsecskip{1em plus 0ex minus 1ex}
\setsubsubsecheadstyle{\itshape\footnotesize\red\centering}
\setaftersubsubsecskip{1pt}

\emergencystretch 1.5em


\newcommand \fleuron{{\centering\Large\red❦\par}}

\nouppercaseheads
\makepagestyle{mystyle}
\makeevenhead{mystyle}{}{\scshape\leftmark}{}
\makeoddhead{mystyle}{}{\scshape\rightmark}{}
\makeevenfoot{mystyle}{}{\thepage}{}
\makeoddfoot{mystyle}{}{\thepage}{}
\makepsmarks{mystyle}{%
  \createmark{chapter}{left}{nonumber}{}{}}

\pagestyle{mystyle}



% For the wrapped rubric on the embertide intercession
\setlength{\intextsep}{2pt}
% Get rid of extra space around longtable
\setlength{\LTpre}{.5em}
\setlength{\LTpost}{.5em}

%s"] = "ſ
\directlua{
  fonts.handlers.otf.addfeature{
    name = "calt",
    type = "chainsubstitution",
    lookups = {
      {
        type = "substitution",
        data = {
          ["s"] = "ſ",
        },
      },
    },
    data = {
      rules = {
        {
          before = { { 0xFFFC, "-", "·", "A", "Í", "I", "P", "a", "á", "b", "c", "d", "e", "é", "g", "h", "i", "j", "k", "l", "m", "n", "o", "ó", "p", "q", "r", "s", "ſ", "t", "u", "ú", "v", "w", "x", "y", "z" } },
          after = { { "·", "-", "a", "á", "c", "d", "e", "é", "g", "h", "i", "í", "j", "l", "m", "n", "o", "ó", "p", "q", "r", "s", "t", "u", "ú", "v", "w", "x", "y", "z" } },
          current = { { "s" } },
          lookups = { 1 },
        },
      },
    },
  }
}

\setmainfont[  
    Numbers={OldStyle, Proportional},
    %Ligatures={Rare},%,Historic},
    %CharacterVariant={1:0}
    %CharacterVariant=1,
    %StylisticSet=7,
    RawFeature={+calt}
    % ItalicFeatures={Colour=990000}
]{EBGaramond}
% \setsansfont{UnifrakturMaguntia}[StylisticSet={1},CharacterVariant={11,12,4:1}]
\setsansfont{KJV1611}[RawFeature={+hist,+calt}]

\newcommand \red{\addfontfeature{Color=990000}}
\newcommand \black{\addfontfeature{Color=000000}}

\newcommand \R{{\red℟.\ }}
\newcommand \V{{\red℣.\ }}
\newcommand \ant{{\scshape\small\red Ant.\ }}
\newcommand \etc{\emph{\red\&c.}}
\newcommand \minorheading[2]{{\unexpanded\expandafter{\scshape #1}}\par{\unexpanded\expandafter{\itshape #2}}}
\newcommand \subseccaption[2]{\subsection{#1 {\itshape\small\red #2}}}
\newcommand \prefaceCaption[3]{\subsection{{\small\emph{\red #1} #2 \emph{\red #3}}}}
\newcommand \rubric[1]{\emph{\footnotesize\red #1}}
\newcommand \pilcrow[1]{\par\footnotesize\noindent\makebox[1em][l]{¶ }\hangindent1em\rubric{#1}\par\normalsize}
\newcommand \hangingrubric[1]{\par\footnotesize\noindent\hangindent1em\rubric{#1}\par\normalsize}
\newcommand \centeredrubric[1]{\par{\centering\footnotesize\rubric{#1}\par}\normalsize}

\newcommand \cross {\grecross\ }

\newcommand{\qa}[1]{{\itshape\small\red#1}}

\renewcommand{\thefootnote}{\fnsymbol{footnote}}

\medievalpage
\raggedbottom
\checkandfixthelayout



\newcounter{cnt}\setcounter{cnt}{0}
\def\t{\stepcounter{cnt}\thecnt. cat sat on the mat. }

\newdimen\tttaa
\newdimen\tttbb

\renewcommand\thepage{\the\numexpr(\value{page}+1)/2\relax}

\makeatletter
\def\merge@ps{\afterassignment\merge@ps@\tttbb}
\def\merge@ps@{\afterassignment\merge@ps@@\tttaa}

\def\merge@ps@@{%
\afterassignment\reset@WF@ps\dimen@\WF@ps\valign
%\showthe\count@
\ifnum\count@>\@ne
\advance\count@\m@ne
\expandafter\merge@ps
\fi
}


\def\reset@WF@ps{\afterassignment\reset@WF@ps@\dimen@ii}

\def\reset@WF@ps@#1\valign{%
\edef\new@wf@ps{\new@wf@ps
  \the\dimexpr\dimen@+\tttbb\relax\space
  \the\dimexpr\dimen@ii-\tttbb\relax\space}%
 \def\WF@ps{#1}}


\newcommand\wflettrine[3][]{%
  \setbox\tw@\hbox{\lettrine[#1]{#2}{#3}\global\let\gtmp\L@parshape}%
  \afterassignment\wf@getoffset\count@\gtmp\hoffset
  \setbox\WF@box\hbox{\kern-\dimen@\box\WF@box\kern\dimen@}%
  \noindent\box\tw@
    \def\new@wf@ps{}%
    \afterassignment\merge@ps\count@\gtmp
    \edef\WF@ps{\new@wf@ps\space\WF@ps}%
    \@@parshape\c@WF@wrappedlines\WF@ps\z@\columnwidth}


\def\wf@getoffset{\afterassignment\wf@get@ffset\dimen@}
\def\wf@get@ffset#1\hoffset{}

\makeatother

\makeatletter
\def\tabu@verticalmeasure{\everypar{}%
\unless\ifnum\currentgrouptype=14 \let\tabu@currentgrouptype\currentgrouptype\fi
    \ifnum \tabu@currentgrouptype>12         % 14=semi-simple, 15=math shift group
        \setbox\tabu@box =\hbox\bgroup
            \let\tabu@verticalspacing \tabu@verticalsp@lcr
            \d@llarbegin                % after \hbox ...
    \else
        \edef\tabu@temp{\ifnum\tabu@currentgrouptype=5\vtop
                        \else\ifnum\tabu@currentgrouptype=12\vcenter
                        \else\vbox\fi\fi}%
        \setbox\tabu@box \hbox\bgroup$\tabu@temp \bgroup
            \let\tabu@verticalspacing \tabu@verticalsp@pmb
    \fi
}
\makeatother



\begin{document}
\frontmatter
\title{Book of Common Prayer}
\author{Michael Shirk}
\date{MMXXI}
% \maketitle
The Book of Common Prayer

and Administration of the Sacraments 
and Other Rites and Ceremonies of the Church

{According to the (Traditional English) Use of}
The Independent Catholic Christian Church

Together with
The Psalter or Psalms of David
Pointed as they are to be Sung or Said in Churches

+

Publisher



% argent
% 	a cross gules, a paschal lamb argent shown smaller

% argent, a cross patee gules, a paschal lamb argent shown smaller
% on 
% achievement crest a mitre or shown larger
% motto "Sic, Episcope"


\chapter{Preface}
\drop{It is a most invaluable part of that blessed liberty wherewith Christ hath made us free, that in his worship different forms and usages may without offence be allowed, provided the substance of the Faith be kept entire; and that, in every Church, what cannot be clearly determined to belong to Doctrine must be referred to Discipline and therefore, by common consent and authority, may be altered, abridged, enlarged, amended, or otherwise disposed of, as may seem most convenient for the edification of the people, “according to the various exigencies of times and occasions.”}

The wisdom of our fathers under the good hand of God gave to the Church of England the Book of Common Prayer in English speech. It is, and we believe that it will always be, one of the great books of the world. Nothing save the English version of the Holy Scriptures is enwoven so closely in the language and the deepest thoughts of English speaking people.

% There was never any thing by the wit of man so well devised, or so sure established, which in continuance of time hath not been corrupted: As, among other things, it may plainly appear by the Common Prayers in the Church, commonly called Divine Service. The first original and ground whereof if a man would search out by the ancient Fathers, he shall find, that the same was not ordained but of a good purpose, and for a great advancement of godliness. 






The \emph{Church of England}, to which the Protestant Episcopal Church in these States is indebted, under God, for her first foundation, hath, in the Preface of her Book of Common Prayer, laid it down as a rule, that “The particular forms of Divine Worship, and the Rites and Ceremonies appointed to be used therein, being things in their own nature indifferent and alterable, and so acknowledged, it is but reasonable that upon weighty and important considerations, according to the various exigencies of times and occasions, such changes and alterations should be made therein; as to those who are in places of authority should, from time to time, seem either necessary or expedient.”

% The same Church hath not only in her Preface, but likewise in her Articles and Homilies declared the necessity and expediency of occasional alterations and amendments in her Forms of Public Worship; and we find accordingly, that, seeking to “keep the happy mean between too much stiffness in refusing, and too much easiness in admitting variations in things once advisedly established”, she hath, in the reign of several Princes since the first compiling of her Liturgy in the time of Edward the Sixth upon just and weighty considerations her thereunto moving yielded to make such alterations in some particulars, as in their respective times were thought convenient; yet so as that the main body and essential parts of the same (as well in the chiefest materials as in the frame and order thereof) have still been continued firm and unshaken.

% Her general aim in these different reviews and alterations hath been, as she farther declares in her said Preface "to do that which according to her best understanding, might most tend to the preservation of peace and unity in the Church; the procuring of reverence, and the exciting of piety and devotion in the worship of God; and finally the cutting off occasion, from them that seek occasion. Of cavil or quarrel against her Liturgy." And although, according to her judgment, there be not "any thing in it contrary to the Word of God or to sound doctrine, or which a godly man may not with a good conscience use and submit unto, or which is not fairly defensible, if allowed such just and favourable construction as in common equity ought to be allowed to all human writings;" yet upon the principles already laid down, it cannot but be supposed that further alterations would in time be found expedient. Accordingly, a commission for a review was issued in the year 1689: but this great and good work miscarried at that time; and the Civil Authority has not since thought proper to revive it by any new commission.

% But when in the course of Divine Providence, these American States became independent with respect to civil government, their ecclesiastical independence was necessarily included; and the different religious denominations of Christians in these States were left at full and equal liberty to model and organize their respective Churches, and forms of worship, and discipline, in such manner as they might judge most convenient for their future prosperity.%; consistently with the constitution and laws of their country.
% The attention of this Church was in the first place drawn to those alterations in the Liturgy which became necessary in the prayers for our Civil Rulers, in Consequence of the Revolution. And the principal care herein was to make them conformable to what ought to be the proper end of all such prayers, namely, that "Rulers may have grace, wisdom, and understanding to execute justice, and to maintain truth;" and that the people "may lead quiet and peaceable lives, in all godliness and honesty."

% But while these alterations were in review before the Convention, they could not but, with gratitude to God, embrace the happy occasion which was offered to them (uninfluenced and unrestrained by any worldly authority whatsoever) to take a further review of the Public Service, and to establish such other alterations and amendments therein as might be deemed expedient.

It seems unnecessary to enumerate all the different alterations and amendments. They will appear, and it is to be hoped, the reasons of them also, upon a comparison of this with the Book of Common Prayer of the Church of England. % In which it will also appear that this Church is far from intending to depart from the Church of England in any essential point of doctrine, discipline, or worship; or further than local circumstances require.








And now, this important work being brought to a conclusion, it is hoped the whole will be received and examined by every true member of our Church, and every sincere Christian, with a meek, candid, and charitable frame of mind; without prejudice or prepossessions; seriously considering what Christianity is, and what the truths of the Gospel are; and earnestly beseeching Almighty God to accompany with his blessing every endeavour for promulgating them to mankind in the clearest, plainest, most affecting and majestic manner, for the sake of Jesus Christ, our blessed Lord and Saviour.  

In all things we have set before our eyes the duty of faithfulness to the teaching of Scripture and the godly and decent order of the ancient Fathers, and we pray that by God's blessing upon our work those who use this book may be enabled to keep the unity of the Spirit in the bond of peace.



\settocdepth{chapter}
\tableofcontents*

% Of Ceremonies

The Order how the Psalter is appointed to be read.
The Order how the Rest of Holy Scripture is appointed to be read
Proper Lessons
Proper Psalms


kalendar

Tables and rules
Table to find Easter Day
Rules to Order the service
  Every service begins "in the name of ... cross"

\mainmatter

\newcommand \Sol[1]{{\versiculus☉}~in~{\versiculus#1}}
\newcommand \dub[1]{{\footnotesize\sffamily #1}}		% Double Feasts
\newcommand \mem[1]{\emph{#1}} % Memorials

%\renewcommand \dub[1]{{\sffamily #1}}		% Double Feasts
%\renewcommand \Rul[1]{{\scshape #1}}		% Ruled Feasts
%\markright{Kalendar}\part{Kalendar}

\directlua{printKalendar(true)}


\newpage

\section*{A Table to Find Easter-Day}
{\begin{multicols}{2}
\noindent\begin{tabular} { @{}c@{\hspace{.15cm}} c@{\hspace{.2cm}} r@{\hspace{.2cm}} l@{\hspace{.2cm}} }
Golden & Month & Day & Sunday \\
Number &    &   & Letter \\
\hline
xiv & March & 22 & d \\
iij & ” & 23 & e \\
    & ” & 24 & f \\
xj  & ” & 25 & g \\
    & ” & 26 & A \\
xix & ” & 27 & b \\
viij & ” & 28 & c \\
    & ” & 29 & d \\
xvj & ” & 30 & e \\
v   & ” & 31 & f \\
    & April & 1 & g \\
xiij & ” & 2 & A\\
ij & ” & 3 & b\\
  & ” & 4 & c \\
x &	”	&5&	d\\
&”	&6&	e\\
xviij	&”&	7&	f\\
vij	&”&	8&	g\\
&”	&9&	A\\
xv	&”&	10&	b\\
iv	&”&	11&	c\\
&”	&12&	d\\
xij	&”&	13&	e\\
j	&”&	14&	f\\
&”	&15&	g\\
ix	&”&	16&	A\\
xvij	&”&	17&	b\\
vj	&”&	18&	c\\
&”	&19&	d\\
&”	&20&	e\\
&”	&21&	f\\
&”	&22&	g\\
&”	&23&	A\\
&”	&24&	b\\
&”	&25&	c\\

\end{tabular}
%\vfill
%\columnbreak
\footnotesize
\vspace{6pt}

This Table contains so much of the Calendar as is necessary for the determining of Easter; to find which, look for the Golden Number of the year in the first Column of the Table, against which stands the day of the Paschal Full Moon; then look in the third Column for the Sunday Letter, next after the day of the Full Moon, and the day of the Month standing against that Sunday Letter is Easter Day. If the Full Moon happens upon a Sunday, then (according to the first rule) the next Sunday after is Easter-Day.

To find the Golden Number, or Prime, add one to the Year of our Lord, and then divide by 19; the remainder, if any, is the Golden Number; but if nothing remaineth, then 19 is the Golden Number.

To find the Dominical or Sunday Letter, according to the Calendar, until the Year 2099 inclusive, add to the Year of our Lord its Fourth Part, omitting Fractions; and also the Number 6: Divide the sum by 7; and if there is no remainder, the A is the Sunday Letter: But if any number remaineth, then the Letter standing against that number in the small annexed Table is the Sunday Letter.
\vspace{6pt}

{\centering\setlength{\extrarowheight}{2pt}\begin{tabular} { | c c c c c c c | }
\hline
0 & 1 & 2 & 3 & 4 & 5 & 6 \\
\hline
A & G & F & E & D & C & B \\
\hline
\end{tabular}\par}

\vspace{2pt}
For the next Century, that is, from the year 2100 till the year 2199 inclusive, add to the current year its fourth part, and also the number 5, and then divide by 7, and proceed as in the last Rule.

Note, that in all Bissextile or Leap-Years, the Letter found as above will be the Sunday Letter, from the intercalated day exclusive to the end of the year.

The Golden Numbers in the foregoing Calendar will point out the Days of the Paschal Full Moons from the Year 1900, to the Year 2199 inclusive.
\vspace{12pt}
\end{multicols}}

\section{How the Psalter is appointed to be read}

{\scriptsize
% American and Scottish are more similar to each other;
% Maybe better? Probably tend toward the scotteish for sundays
% and to the american for feasts (longer for the 6, and a couple
% extras.?
\drop{Psalms to be read at Morning and at Evening Prayer are appointed
for every Sunday in the year, and for certain other Holy-days.
Otherwise the Psalter will be read through in order once every
month as is appointed.}

Whensoever Proper Psalms are appointed, then the Psalms of
ordinary course for the day of the month shall be omitted.

On week days (unless Proper Psalms are provided) shall be read the
Psalms for the day of the month, as they are appointed, for
Morning and Evening Prayer.

And, whereas January, March, May, July, August, October, and
December have one-and-thirty days apiece; It is ordered, that on
the last day of any one of the said months being an ordinary week
day shall be read the Psalms assigned to the 30th day, or else
the Psalms of the monthly course omitted on one of the Sundays in
that month; So that the Psalter may begin again the first day of
the next month ensuing.

And, whereas the 119th Psalm is divided into twenty-two portions,
and is over-long to be read at one time; It is so ordered, that
at one time shall not be read above four or five of the said
portions.

And at the end of every Psalm, and of every such part of the
119th Psalm, shall be repeated this Hymn,


{\normalsize
Glory be to the Father, and to the Son : and to the Holy Ghost;

%\emph{Answer}. 
As it was in the beginning, is now, and ever shall be : world without end. Amen.
}


Note, that the Psalter followeth the Division of the Hebrews, and
the Translation of the great English Bible, set forth and used in
the time of King \emph{Henry} the Eighth, and \emph{Edward} the Sixth.

% [Psalms have also been selected for use on various occasions,
% and on such occasions one or more at the discretion of the Minister
% may be read at Morning and Evening Prayer in place of the Psalms
% of the Day.

% Upon occasions to be approved by the Bishop, other Psalms may,
% with his consent, be substituted for the Psalms of the Day or for
% those which are appointed in the Table of Proper Psalms.]
}
\medskip
\section[Proper Psalms for Certain Days]{Table of Proper Psalms for Certain Days}
{\scriptsize

\begin{longtabu} to \linewidth { @{}X | c | c@{} }
\hline 
    & {\scshape Mattins} & {\scshape Evensong}\\
\hline
\endfirsthead
%---------------------------------------------------------------%
\hline
    & {\scshape Mattins} & {\scshape Evensong}\\
\hline
\endhead

First Sunday in Advent\dotfill      & 1, 7          & 46, 48\\
Second Sunday in Advent\dotfill     & 9, 11         & 50, 67\\
Third Sunday in Advent\dotfill      & 73            & 75, 76, 82 \\
Fourth Sunday in Advent\dotfill     & 94            & 96, 97, 98 \\
Christmas Eve\dotfill               & —             & 89 (1–36) \\
Christmas Day\dotfill               & 19, 85        & 132 \\
1st Sunday after Christmas\dotfill  & 2, 8          & 45, 110, 113\\
New Year’s Eve \dotfill             & —                 & 90, 133, 134\\
Circumcision\dotfill                & 119 (1–32)        & 91, 121 \\
2nd Sunday after Christmas\dotfill  & 103            & 104 \\
Eve of Epiphany\dotfill             & —                 & 19, 87 \\
Epiphany\dotfill                    & 72            & 96, 97, 117 \\
1st Sunday after Epiphany\dotfill   & 46, 47, 67        & 18 \\
2nd Sunday after Epiphany\dotfill   & 27, 36        & 68 \\
3rd Sunday after Epiphany\dotfill   & 42, 43            & 33, 34 \\
4th Sunday after Epiphany\dotfill   & 60, 63            & 74 \\
5th Sunday after Epiphany\dotfill   & 99, 112           & 106 \\
6th Sunday after Epiphany\dotfill   & 80, 81            & 78 \\
Septuagesima\dotfill                & 104               & 147, 148 \\
Sexagesima\dotfill                  & 139               & 25, 26\\
Quinquagesima\dotfill               & 15, 20, 23        & 30, 31\\
Ash Wednesday\dotfill               & 6, 32, 38         & 102, 130, 143 \\
1st Sunday in Lent\dotfill          & 51                & 6, 32, 143 \\
2nd Sunday in Lent\dotfill          & 119 (1–32)        & 119 (33–72) \\
3rd Sunday in Lent\dotfill          & 119 (73–104)      & 119 (105–144) \\
4th Sunday in Lent\dotfill          & 119 (145–176)     & 39, 40 \\
5th Sunday in Lent\dotfill          & 22                & 51 \\
6th Sunday in Lent\dotfill          & 61, 62            & 86, 130 \\
Monday in Holy Week\dotfill         & 13, 25            & 26, 27, 28 \\
Tuesday in Holy Week\dotfill        & 31                & 88 \\
Wednesday in Holy Week\dotfill      & 41, 42, 43        & 54, 55\\
Thursday in Holy Week\dotfill       & 56, 64            & 23, 109 \\
Good Friday\dotfill                 & 22                & 40, 69\\
Easter Even\dotfill                 & 23, 30, 142       & 115, 116, 117 \\
Easter Day\dotfill                  & 2, 16, 111        & 113, 114, 118 \\
1st Sunday after Easter\dotfill     & 3, 57              & 103\\
2nd Sunday after Easter\dotfill     & 120, 121, 122, 123 & 65, 66 \\
3rd Sunday after Easter\dotfill     & 124, 125, 126, 127 & 81, 84 \\
4th Sunday after Easter\dotfill     & 128, 129, 130, 131 & 145, 146 \\
5th Sunday after Easter\dotfill     & 132, 133, 134      & 107 \\
\emph{Rogation Monday}\dotfill      & 34, 127            & 62, 63\\
\emph{Rogation Tuesday}\dotfill     & 65, 66, 67         & 102\\
\emph{Rogation Wednesday}\dotfill   & 121, 144           &\\
Eve of Ascension\dotfill            & —                  & 15, 97, 99 \\
Ascension Day\dotfill               & 8, 21              & 24, 47, 110 \\
Sunday after Ascension Day\dotfill  & 93, 96            & 148, 149, 150\\
Eve of Whitsunday\dotfill           & —                 & 48, 145 \\
Whitsunday\dotfill                  & 68                & 104\\
Trinity Sunday\dotfill              & 29, 33            & 93, 99, 115\\
1st Sunday after Trinity\dotfill    & 1, 3, 5           & 4, 7, 8\\
2nd Sunday after Trinity\dotfill    & 10, 12, 13        & 15, 16, 17\\
3rd Sunday after Trinity\dotfill    & 18                & 19, 20, 21\\
4th Sunday after Trinity\dotfill    & 24, 25            & 22, 23\\
5th Sunday after Trinity\dotfill    & 26,28             & 27, 29, 30\\
6th Sunday after Trinity\dotfill    & 31, 32            & 33, 36\\
7th Sunday after Trinity\dotfill    & 34                & 37\\
8th Sunday after Trinity\dotfill    & 39, 40           & 41, 42, 43\\
9th Sunday after Trinity\dotfill    & 46, 47, 48       & 44, 45\\
10th Sunday after Trinity\dotfill   & 50, 53           & 51, 54\\
11th Sunday after Trinity\dotfill   & 56, 57           & 61, 62, 63\\
12th Sunday after Trinity\dotfill   & 65, 66           & 68\\
13th Sunday after Trinity\dotfill   & 71                & 67, 72\\
14th Sunday after Trinity\dotfill   & 75, 76           & 73, 77\\
15th Sunday after Trinity\dotfill   & 84, 85           & 89\\
16th Sunday after Trinity\dotfill   & 86, 87           & 90, 91\\
17th Sunday after Trinity\dotfill   & 92, 93           & 100, 101, 102\\
18th Sunday after Trinity\dotfill   & 103                & 107\\
19th Sunday after Trinity\dotfill   & 111, 112, 113      & 120, 121, 122, 123\\
20th Sunday after Trinity\dotfill   & 114, 115           & 124, 125, 126, 127\\
21st Sunday after Trinity\dotfill   & 116, 117           & 128, 129, 130, 131\\
22nd Sunday after Trinity\dotfill   & 118                & 132, 133, 134\\
23rd Sunday after Trinity\dotfill   & 110, 135           & 137, 138, 139\\
24th Sunday after Trinity\dotfill   & 136                & 140, 141, 142\\
25th Sunday after Trinity\dotfill   & 49                 & 79, 83\\
26th Sunday after Trinity\dotfill   & 84, 144            & 105\\
Sunday next before Advent\dotfill   & 145, 146           & 147, 148, 149, 150\\
&&\\

Michaelmas Eve\dotfill            & —                  & 91 \\
Michaelmas\dotfill                & 34, 103            & 148 \\
All Hallows’s Eve\dotfill         & —                  & 146, 148\\
All Saints’\dotfill               & 1, 15              & 145 \\
Eve of a Greater Feast\dotfill    & —                  & 1, 30 \\
A Greater Feast\dotfill            & 111, 112           & 148, 149 \\
Eve of the Dedication\dotfill        & —                  & 84, 87\\
Feast of the Dedication\dotfill     & 132                & 122, 133, 134\\
Harvest Thanksgiving\dotfill      & 103                & 65, 67\\
                                  & 104                & 147, 150
\end{longtabu}}


\medskip
\section{Psalms for Special Occasions}
{\scriptsize
One or more of the following Psalms may be used on the occasions specified:—

\begin{hangparas}{.25in}{1}
Eves of Holy-days and Holy-days.—1, 15, 24, 30, 34, 42, 43, 84, 91, 103, 111, 112, 113, 116, portions of 119, 131, 132, 138, 145, 146, 148, 149.

Feast of Dedication or Patronal Feast.—24, 48, 84, 122, 132, 133, 134.

Thanksgiving for Harvest.—65, 67, 103, 104, 144, 145, 147, 148, 150.

For Home Missions and Missions beyond the Seas.—2, 45, 46, 47, 48, 67, 72, 85, 87, 96, 97, 100, 117.

Times of trouble or anxiety.—23, 25, 46, 77, 80, 86, 90, 130.

Occasions of thanksgiving.—30, 33, 65, 107, 111, 115, 138, 145, 146, 148, 150.

\end{hangparas}
% Propers and rogation days from "The Churchpeople's Prayer Book"(1935)
}

\fleuron
\chapter[Table of Lessons]{\stylechapter{}{Table of Lessons}{To be read in course throughout the year}}
\centerline{(The Table of 1922, as revised in 1928)}

This Table is arranged according to the weeks of the ecclesiastical year, beginning with the First Sunday in Advent.  The Lessons for the Immoveable Feasts not given in this Table are to be found in the Calendar following the Table.

Except on Septuagesima Sunday, and the Sunday next before Advent, on ever Sunday on which Lessons from the Gospels are provided both for Mattins and Evensong, one of such Lessons shall always be read.

It is convenient that, when alternative Lessons are provided, choice be exercised according to some scheme of consecutive reading.

% {\scriptsize
% \directlua{printLessons()}
% }


% \newcommand \one {{\addfontfeature{Numbers=Proportional}(1)}\ }
% \newcommand \two {{\addfontfeature{Numbers=Proportional}(2)}\ }

% {\scriptsize
% \begin{longtabu} to \linewidth { @{}X | X  | X@{} }
% \hline 
%     & {\scshape Mattins} & {\scshape Evensong}\\
% \hline
% \endfirsthead
% %---------------------------------------------------------------%
% \hline
%     & {\scshape Mattins} & {\scshape Evensong}\\
% \hline
% \endhead
% \scshape{Advent Sunday} &
%     \one Isaiah \textbf{1,} 1–20
%     \two John \textbf{3,} 1–21, \emph{or} 
%          1~Thess.~\textbf{4,} 13—\textbf{5,} 11 &
%     \one Isaiah \textbf{2,} \emph{or} Isaiah \textbf{1,} 18–end \par
%     \two Matthew \textbf{24,} 1–28, \emph{or} Revelation \textbf{14,} 13—\textbf{15,} 4\\

% \emph{Monday} &
%     \begin{tabularx}{\textwidth}{ c X }
%         \one & \textbf{3,} 1–21, \emph{or} 
%         1~Thess.~\textbf{4,} 13—\textbf{5,} 11 \\
%         \two & Mark 1, 1–20 \\
%     \end{tabularx} &
%     \one Isaiah \textbf{4,} 2–end \par \two James \textbf{1}\\
    
% \hline 
% \scshape{Sunday next before Advent} & \one Ecclesiastes \textbf{11} and \textbf{12} \par \two John \textbf{19,} 13–end, \emph{or} Hebrews \textbf{11,} 1–16 & \one Haggai \textbf{2,} 1–9, \emph{or} Malachi \textbf{3} and \textbf{4} \par \two John \textbf{20,} \emph{or} Hebrews \textbf{11,} 17—\textbf{12,} 2, \emph{or} Luke \textbf{15,} 11–end\\
% \emph{Monday} & \one Wisdom \textbf{1} \par \two Revelation \textbf{1} & \one Wisdom \textbf{2} \par \two Revelation \textbf{2}\\

% \emph{Tuesday} & \one Wisdom \textbf{3,} 1–9 \par \two Revelation \textbf{3} & \one Wisdom \textbf{4,} 7–end \par \two Revelation \textbf{4}\\

% \emph{Wednesday} & \one Wisdom \textbf{5,} 1–16 \par \two Revelation \textbf{5} & \one Wisdom \textbf{6,} 1–21 \par \two Revelation \textbf{6}\\

% \emph{Thursday} & \one Wisdom \textbf{7,} 15—\textbf{8,} 4 \par \two Revelation \textbf{7} & \one Wisdom \textbf{8,} 5–18 \par \two Revelation \textbf{10} and \textbf{11,} 1–14\\

% \emph{Friday} & \one Wisdom \textbf{8,} 21—\textbf{9} end \par \two Revelation \textbf{11,} 15—\textbf{12} end & \one Wisdom \textbf{10,} 15—\textbf{11,} 10 \par \two Revelation \textbf{14,} 1–13\\

% \emph{Saturday} & \one Wisdom \textbf{11,} 21—\textbf{12,} 2 \par \two Revelation \textbf{18} & \one Wisdom \textbf{12,} 12–21 \par \two Revelation \textbf{19,} 1–16\\


% \end{longtabu}}

Proper Lessons
Feast of Dedication or Patronal Feast

Thanksgiving for Harvest
% When this Talbe of Lessons... etcc.

% \def\tcase#1{%
%   $\left\lbrace\,
%     \begin{tabular}{@{}c @{}l@{}}
%       #1
%     \end{tabular}
%   \right.$%
% }
% \newpage
% \section{Table of Lessons}
% {\tiny
% \begin{longtabu} to \linewidth{|X[2]X[-1,r]X[2,l]|X[-1,r]l}
%     % X| l r@{} l | l r@{} l ||  l r@{} l |  l r@{} l  }
% \hline
% \multicolumn3{|c}{\scshape Mattins} & \multicolumn2{c}{\scshape Evensong} \\
% \tabuphantomline
% \hline
% Third Sunday in Advent & &
% \tcase{(1) & Isaiah {\bfseries 25,} 1–9 \\ \\
%         (2) & Luke {\bfseries 3,} 1–17, \emph{or} \\
%         & 1 Timothy {\bfseries 1,} 12—{\bfseries 2,} 7} &  &
% \tcase{(1) & Isaiah {\bfseries 26,} \emph{or} \\& Isaiah {\bfseries 28,} 1–22 \\ 
%     (2) & Matthew {\bfseries 25,} 1–30, \emph{or} \\& Revelation {\bfseries 21,} 9—{\bfseries 22,} 5}
%     \\

%     & M. & \tcase{(1) & Isaiah {\bfseries 30,} 19–end \\ (2) & Mark {\bfseries 7,} 1–23}
%     & M. & \tcase{(1) & Isaiah {\bfseries 31} \\ (2) & 1 John {\bfseries 1,} 1—{\bfseries 2,} 5} \\
%     & Tu. & \tcase{(1) & Isaiah {\bfseries 38,} 1–20 \\ (2) & Mark {\bfseries 7,} 24—{\bfseries 8,} 10} 
%     & Tu. & \tcase{(1) & Isaiah {\bfseries 40,} 1–40 \\ (2) & 1 John {\bfseries 2,} 7–end} \\
% Ember day & W. & \tcase{(1) & Isaiah {\bfseries 40,} 12–end \\ (2) & Mark {\bfseries 8,} 11—{\bfseries 9,} 1} 
%     & W. & \tcase{(1) & Isaiah {\bfseries 41} \\ (2) & 1 John {\bfseries 3}} 


% \end{longtabu}
% }







% \begin{landscape}
% \section{Table of Lessons}
% {\footnotesize
% \begin{longtabu} to \linewidth {X| l r@{} l | l r@{} l ||  l r@{} l |  l r@{} l  }
% \hline
% & \multicolumn6{c}{\scshape Mattins} & \multicolumn6{c}{\scshape Evensong} \\
% \hline
% \multirow{2}{*}{Advent Sunday\dotfill  }& \multirow{2}{*}{Isa.} & \multirow{2}{*}{\bfseries 1,} & \multirow{2}{*}{1–20} & John & {\bfseries 3,} & 1–21 & Isa. & {\bfseries 2} & & Matt. & {\bfseries 24,} & 1–28 \\
% & & & & 1 Thess. & {\bfseries 4,} & 13—{\bfseries 5,} 11 & „ & {\bfseries 1,} & 18–end & Rev. & {\bfseries 14,} & 13—{\bfseries 15,} 4 \\
% \hline

   
% \end{longtabu}

% \begin{longtabu} to \linewidth {X| l r@{} l | l r@{} l | l r@{} l }\hline
   
%  & \multicolumn3{c|}{\scshape First Evensong} & \multicolumn3{c|}{\scshape Mattins} & \multicolumn3{c}{\scshape Second Evensong} \\
%  \hline 
% St.\ Matthew\dotfill & 1 Kings  & {\bfseries 19, } & 15–end & Prov. & {\bfseries 3, } & 1–18  & 1 Chron. & {\bfseries 29, } & 9–17\\
%                      & Matt.    & {\bfseries 6, } & 19–end & Matt. & {\bfseries 19, } & 16–end & 1 Tim.   & {\bfseries 6, }  & 6–19 \\ 
% \hline

% \end{longtabu}
% }
% \end{landscape}
% \part{Service Music}

% Primarily from "A Prayer Book Revised", 1913
\chapter[Certain Notes]{\stylechapter{}{Certain Notes}{for the more Plain Explanation and Decent Ministration of things Contained in this Book}}

% ! The changeable portions of the Mass for [[Sundays]] and \Dub{Greater Feasts} will be found in the Propers through the Year.  At Mattins and Evensong, the same Collect is used as at Mass and proper Lessons are appointed, but the Invitatory and Hymns are found in the Hymnal in the Season or class of Saint.
% On \Dub{Principle Feasts}, there is furthermore a proper Preface for the Mass.

% On Lesser Feasts, the common Mass, Invitatory, and Hymns for the class of Saint will generally serve, unless propers be provided.
% The Mass of a Sunday shall serve throughout the week, unless another be appointed.

\drop{The word ‘Minister’ in this Book includes bishops, priests, and deacons. When the word ‘Bishop’ is used, none but a bishop may say the words there appointed; when the word ‘Priest’, then may none but a bishop or priest use the words; when the word ‘Deacon’ is used, then shall the words appointed to the deacon be said by one who is in that office, or by a bishop or priest executing that office for the occasion, or by the priest himself when there is no other minister.}

A Clerk is any person appointed to lead in the singing, or to serve the minister and to lead in the responses; the clerk may also read the Lessons and the Epistle.

When one service follows upon another, opportunity shall be given for people to come and go between the services, whether by the singing of a hymn or by a pause. And none shall go out of church during any service or sermon except in case of necessity.

A sermon shall be preached every Sunday at the time appointed. On Sundays and Holy-days in general, a lecture or sermon on a catechetical topic may be delivered after the Second Lesson at Evensong, or the Priest, or one chosen by him for this purpose as Catechist, may instruct the young people of his parish.

The \emph{Gloria} is always to be added to the Psalms, and to the Canticles specified in the rubric, except from Morning Prayer on Maundy Thursday until Evening Prayer on Easter Even; and also it is omitted at all Funeral and Memorial services.

When any minister or reader says a prayer or other form together with the people, he that reads shall say alone the opening words (as, \emph{Our Father}, \emph{I believe in God}, \emph{Glory be to God on high}, and in other places as far as the comma); and the clerks and people shall take up the following words with him.

The full ending of a Collect may be used on any occasion, whether it be printed or not; except that when more than two Collects are said together, without any intermediate bidding, the first and the last shall have the full ending (and the people shall say \emph{Amen}), and the intermediate Collects shall have no ending. The normal full ending is, \emph{Through Jesus Christ thy Son our Lord, who liveth and reigneth with thee in the unity of the Holy Ghost, ever one God, world without end; \emph{or, if our Lord has been already mentioned in the Collect,} Through the same thy Son Jesus Christ, \etc; \emph{or if the Holy Spirit has been already mentioned,} who liveth and reigneth with thee and the same Spirit \etc}


When Anthems are appointed, they are to be sung in full before the Psalm, and to be repeated at the end of the \emph{Gloria} (or of the Psalm itself, when there is no \emph{Gloria} said); but in any Procession the Anthem may be repeated after each verse, if necessity require.

Saying is to be taken to include singing; and words that are appointed to be sung may be said, if need be. But words which are directed to be said in a humble voice should be said without any musical note or inflection.

% When it is desired to use music composed for them, other authorized liturgical texts may be used in place of the corresponding texts in this book.

To avoid a continual repetition of rubrics, let it here also be said that a minister who is reading the service is not included in a general direction to kneel. He stands to read, unless it be expressly stated that he is to kneel down. All others present kneel during prayers, unless it be otherwise stated, except any who are helping the priest in his ministration.

Whenever any passage from the Scripture is read, he that reads shall stand and turn towards the people, who may sit; except that when the Liturgical Gospel is read, they also shall stand, and turn towards the minister who reads. And whenever the priest speaks to the people, as in absolutions and benedictions, he shall turn to them. All are to stand when Canticles are sung; but during the singing of the Psalms it is lawful to sit.

% If more specific details on the ceremonial of the services are desired, {A Directory of Ceremonial} by René Vilatte Press is recommended.

And since there must of necessity be many things not mentioned in these Notes, we may well, for the rest, observe that golden rule of the venerable Council of Nicæa, “Let ancient customs prevail,” till reason plainly requires the contrary.


\fleuron

When in time of Divine Service the Lord Jesus shall be mentioned, due and lowly reverence shall be done by all persons present, as it hath been accustomed. - Canon 18 of 1603.
Bow should also be made at all Gloria Patris.

"Reverently" or "Reverence" in a rubric generally means "bowing."

Sign of the cross:
End of the Gloria in Excelis
Gloria Tibi before the gospel, and
Benedictus qui Venit.

As touching kneeling, crossing, holding up of hands, knocking upon the breast, and other gestures, they may be used or left, as every mans devotion serveth, without blame.


1549:
IN the saying or singing of Matens and Evensong, Baptizyng and Burying, the minister, in paryshe churches and chapels annexed to the same, shall use a Surples. And in all Cathedral churches and Colledges, tharchdeacons, Deanes, Provestes, Maisters, Prebendaryes, and fellowes, being Graduates, may use in the quiere, beside the yr Surplesses, such hoodes as pertaineth to their several degrees, which they have taken in any universitie within this realme. But in all other places, every minister shall be at libertie to use any Surples or no. It is also seemely that Graduates, when they dooe preache, shoulde* use such hoodes as pertayneth to theyr severall degrees.
 
 
 
¶ And whensoever the Bishop shall celebrate the Holy communion in the church, or execute any other public ministration, he shall have upon him, besyde his rochette, a Surples or albe, and a cope or vestment, and also his pastoral staff in his hand, or else born or holden by his chaplain.

As touching kneeling, crossing, holding up of hands, knocking upon the breast, and other gestures: they may be used or left as every man’s devotion serveth without blame.

¶ Also upon Christmas day, Easter day, the Ascension day, whit-Soonday, and the feaste of the Trinitie, may bee used anye parte of holye scripture hereafter to be certaynly limited and appoynted, in the stede of the Letany.

¶ If there be a sermone, or for other greate cause, the Curate by his discretion may leave out the Letanye, Gloria in excelsis, the Crede, thomely [the homily], and the exhortation to the communion.

\chapter[An Introduction]{An Introduction to Morning or Evening Prayer}
\pilcrow{At the beginning of Morning Prayer the Minister shall read with a loud voice some one or more of these Sentences of the Scriptures that follow. And then he shall say that which is written after the said Sentences.}

\drop{When the wicked man turneth away from his wickedness that he hath committed, and doeth that which is lawful and right, he shall save his soul alive.}

\hfill\emph{Ezek.~xviii.~27.}

I acknowledge my transgressions, and my sin is ever before me.\hfill\emph{Psalm li. 3.}

Hide thy face from my sins, and blot out all mine iniquities.\hfill\emph{Psalm li. 9.}

The sacrifices of God are a broken spirit: a broken and a contrite heart, O God, thou wilt not despise.\hfill\emph{Psalm li. 17.}

Rend your heart, and not your garments, and turn unto the Lord your God: for he is gracious and merciful, slow to anger, and of great kindness, and repenteth him of the evil.

\hfill\emph{Joel ii. 13.}

To the Lord our God belong mercies and forgivenesses, though we have rebelled against him; neither have we obeyed the voice of the Lord our God, to walk in his laws which he set before us.\hfill\emph{Daniel ix. 9, 10.}

O Lord, correct me, but with judgment; not in thine anger, lest thou bring me to nothing.\hfill\emph{Jer.~x.~24.~Psalm~vi.~1.}

Repent ye; for the Kingdom of Heaven is at hand.\hfill\emph{St.\ Matt. iii. 2.}

I will arise and go to my father, and will say unto him, Father, I have sinned against heaven, and before thee, and am no more worthy to be called thy son.

\hfill\emph{St.\ Luke xv. 18, 19.}

Enter not into judgment with thy servant, O Lord; for in thy sight shall no man living be justified.\hfill\emph{Psalm cxliii. 2.}

If we say that we have no sin, we deceive ourselves, and the truth is not in us; but if we confess our sins, God is faithful and just to forgive us our sins, and to cleanse us from all unrighteousness.\hfill\emph{1 St.\ John i. 8, 9.}

\drop{Dearly beloved brethren, the Scripture moveth us, in sundry places, to acknowledge and confess our manifold sins and wickedness; and that we should not dissemble nor cloak them before the face of Almighty God our heavenly Father; but confess them with an humble, lowly, penitent, and obedient heart; to the end that we may obtain forgiveness of the same, by his infinite goodness and mercy. And although we ought, at all times, humbly to acknowledge our sins before God; yet ought we chiefly so to do, when we assemble and meet together to render thanks for the great benefits that we have received at his hands, to set forth his most worthy praise, to hear his most holy Word, and to ask those things which are requisite and necessary, as well for the body as the soul. Wherefore I pray and beseech you, as many as are here present, to accompany me with a pure heart, and humble voice, unto the throne of the heavenly grace, saying after me;}

\subsubsection{Or the following may be said instead,}
Let us humbly confess our sins to Almighty God.

\pilcrow{A general Confession to be said of the whole Congregation after the Minister, all kneeling.}

\drop{Almighty and most merciful Father; We have erred, and strayed from thy ways like lost sheep. We have followed too much the devices and desires of our own hearts. We have offended against thy holy laws. We have left undone those things which we ought to have done; And we have done those things which we ought not to have done; And there is no health in us. But thou, O Lord, have mercy upon us, miserable offenders. Spare thou them, O God, who confess their faults. Restore thou them that are penitent; According to thy promises declared unto mankind in Christ Jesu our Lord. And grant, O most merciful Father, for his sake; That we may hereafter live a godly, righteous, and sober life, To the glory of thy holy Name. Amen.}


\pilcrow{The Absolution, or Remission of sins, to be pronounced by the Priest alone, standing; the people still kneeling.}

\drop{Almighty God, the Father of our Lord Jesus Christ, who desireth not the death of a sinner, but rather that he may turn from his wickedness, and live; and hath given power, and commandment, to his Ministers, to declare and pronounce to his people, being penitent, the Absolution and Remission of their sins : He pardoneth and absolveth all them that truly repent, and unfeignedly believe his holy Gospel. Wherefore let us beseech him to grant us true repentance, and his Holy Spirit, that those things may please him, which we do at this present; and that the rest of our life hereafter may be pure, and holy; so that at the last we may come to his eternal joy; through Jesus Christ our Lord.}
\subsubsection{The people shall answer here, and at the end of all other prayers,}
\R Amen.


\pilcrow{If no priest be present the person saying the service shall read the Collect for the Twenty-First Sunday after Trinity, that person and the people still kneeling.}


\pilcrow{Then the Minister shall kneel, and say the Lord's Prayer with an audible voice; the people also kneeling, and repeating it with him, both here, and wheresoever else it is used in Divine Service.}

\drop{Our Father, which art in heaven, Hallowed be thy Name. Thy kingdom come. Thy will be done in earth, As it is in heaven. Give us this day our daily bread. And forgive us our trespasses, As we forgive them that trespass against us. And lead us not into temptation, But deliver us from evil. For thine is the kingdom, The power, and the glory, For ever and ever. Amen.}


\chapter{The Order for Mattins}
\subsubsection{Daily Throughout the Year.}


Then likewise he shall say,

    O Lord, open thou our lips.

    Answer. And our mouth shall show forth thy praise.

    Priest. O God, make speed to save us.

    Answer. O Lord, make haste to help us.

Here all standing up, the Priest shall say,

    Glory be to the Father, and to the Son, and to the Holy Ghost;

    Answer. As it was in the beginning, is now, and ever shall be, world without end. Amen.

    Priest. Praise ye the Lord.

    Answer. The Lord's Name be praised.


\pilcrow{Then shall be said or sung this Psalm following; Except on Easter Day, upon which another Anthem is appointed; and on the nineteenth day of every month it is not to be read here, but in the ordinary course of the Psalms.}

Venite, exultemus Domino.
Psalm xcv.

\drop{O come, let us sing unto the Lord : let us heartily rejoice in the strength of our salvation.}

Let us come before his presence with thanksgiving : and shew ourselves glad in him with Psalms.

For the Lord is a great God : and a great King above all gods.

In his hand are all the corners of the earth : and the strength of the hills is his also.

The sea is his, and he made it : and his hands prepared the dry land.

O come, let us worship and fall down : and kneel before the Lord our Maker.

For he is the Lord our God : and we are the people of his pasture, and the sheep of his hand.

To day if ye will hear his voice, harden not your hearts : as in the provocation, and as in the day of temptation in the wilderness;

When your fathers tempted me : proved me, and saw my works.

Forty years long was I grieved with this generation, and said : It is a people that do err in their hearts, for they have not known my ways.

Unto whom I sware in my wrath : that they should not enter into my rest.

Glory be to the Father, and to the Son : and to the Holy Ghost;

As it was in the beginning, is now, and ever shall be : world without end. Amen.

\pilcrow{Then shall follow the Psalms in order as they be appointed. And at the end of every Psalm throughout the year, and likewise at the end of \emph{Benedicite, Benedictus, Magnificat,} and \emph{Nunc dimittis}, shall be repeated,}
Glory be to the Father, and to the Son : and to the Holy Ghost;

Answer. As it was in the beginning, is now, and ever shall be : world without end. Amen.

\pilcrow{Then shall be read distinctly with an audible voice the First Lesson, taken out of the Old Testament, as is appointed in the Calendar, except there be proper Lessons assigned for that day : He that readeth so standing and turning himself, as he may best be heard of all such as are present. And after that, shall be said or sung, in English, the Hymn called Te Deum Laudamus, daily throughout the Year.}

Note, That before every Lesson the Minister shall say, Here beginneth such a Chapter, or Verse of such a Chapter, of such a Book : And after every Lesson, Here endeth the First, or the Second Lesson.

1549: After the fyrste lesson shall folowe Te Deum laudamus in Englishe, dayly throughout the yeare, excepte in Lente, all the which tyme in the place of Te Deum shalbe used Benedicite omnia Opera Domini Domino, in Englyshe as foloweth:

Te Deum Laudamus.
\drop{We praise thee, O God : we acknowledge thee to be the Lord.}

All the earth doth worship thee : the Father everlasting.

To thee all Angels cry aloud : the Heavens, and all the Powers therein.

To thee Cherubin and Seraphin : continually do cry,

Holy, Holy, Holy : Lord God of Sabaoth;

Heaven and earth are full of the Majesty : of thy glory.

The glorious company of the Apostles : praise thee.

The goodly fellowship of the Prophets : praise thee.

The noble army of Martyrs : praise thee.

The holy Church throughout all the world : doth acknowledge thee;

The Father : of an infinite Majesty;

Thine honourable, true : and only Son;

Also the Holy Ghost : the Comforter.

Thou art the King of Glory : O Christ.

Thou art the everlasting Son : of the Father.

When thou tookest upon thee to deliver man : thou didst not abhor the Virgin's womb.

When thou hadst overcome the sharpness of death : thou didst open the Kingdom of Heaven to all believers.

Thou sittest at the right hand of God : in the glory of the Father.

We believe that thou shalt come : to be our Judge.

We therefore pray thee, help thy servants : whom thou hast redeemed with thy precious blood.

Make them to be numbered with thy Saints : in glory everlasting.

O Lord, save thy people : and bless thine heritage.

Govern them : and lift them up for ever.

Day by day : we magnify thee;

And we worship thy Name : ever world without end.

Vouchsafe, O Lord : to keep us this day without sin.

O Lord, have mercy upon us : have mercy upon us.

O Lord, let thy mercy lighten upon us : as our trust is in thee.

O Lord, in thee have I trusted : let me never be confounded.

¶ Or this Canticle.

    Benedictus es Domine.
    
\drop{Blessed art thou, O Lord God of our fathers; * praised and exalted above all for ever.}

Blessed art thou for the Name of thy Majesty; * praised and exalted above all for ever.

Blessed art thou in the temple of thy holiness; * Praised and exalted above all for ever.

Blessed art thou that beholdest the depths, and dwellest between the Cherubim: * praised and exalted above all for ever.

Blessed art thou on the glorious throne of thy Kingdom: * praised and exalted above all for ever.

Blessed art thou in the firmament of heaven: * praised and exalted above all for ever.
    
    ¶ Or this,
    Benedicite, omnia opera.
\drop{O all ye Works of the Lord, bless ye the Lord : praise him, and magnify him for ever.}

O ye Angels of the Lord, bless ye the Lord : O ye Heavens, bless ye the Lord.

O ye Waters that be above the firmament, bless ye the Lord : O all ye Powers of the Lord, bless ye the Lord.

O ye Sun and Moon, bless ye the Lord : O ye Stars of heaven, bless ye the Lord.

O ye Showers and Dew, bless ye the Lord : praise him, and magnify him for ever.
O ye Winds of God, bless ye the Lord : praise him, and magnify him for ever.
O ye Fire and Heat, bless ye the Lord : praise him, and magnify him for ever.
O ye Winter and Summer, bless ye the Lord : praise him, and magnify him for ever.
O ye Dews and Frosts, bless ye the Lord : praise him, and magnify him for ever.
O ye Frost and Cold, bless ye the Lord : praise him, and magnify him for ever.
O ye Ice and Snow, bless ye the Lord : praise him, and magnify him for ever.
O ye Nights and Days, bless ye the Lord : praise him, and magnify him for ever.
O ye Light and Darkness, bless ye the Lord : praise him, and magnify him for ever.
O ye Lightnings and Clouds, bless ye the Lord : praise him, and magnify him for ever.
O let the Earth bless the Lord : yea, let it praise him, and magnify him for ever.
O ye Mountains and Hills, bless ye the Lord : praise him, and magnify him for ever.
O all ye Green Things upon the earth, bless ye the Lord : praise him, and magnify him for ever.
O ye Wells, bless ye the Lord : praise him, and magnify him for ever.
O ye Seas and Floods, bless ye the Lord : praise him, and magnify him for ever.
O ye Whales, and all that move in the waters, bless ye the Lord : praise him, and magnify him for ever.
O all ye Fowls of the air, bless ye the Lord : praise him, and magnify him for ever.
O all ye Beasts and Cattle, bless ye the Lord : praise him, and magnify him for ever.
O ye Children of Men, bless ye the Lord : praise him, and magnify him for ever.
O let Israel bless the Lord : praise him, and magnify him for ever.
O ye Priests of the Lord, bless ye the Lord : praise him, and magnify him for ever.
O ye Servants of the Lord, bless ye the Lord : praise him, and magnify him for ever.
O ye Spirits and Souls of the Righteous, bless ye the Lord : praise him, and magnify him for ever.
O ye holy and humble Men of heart, bless ye the Lord : praise him, and magnify him for ever.
O Ananias, Azarias, and Misael, bless ye the Lord : praise him, and magnify him for ever.
Glory be to the Father, and to the Son : and to the Holy Ghost;
As it was in the beginning, is now, and ever shall be : world without end. Amen.

\pilcrow{Then shall be read in like manner the Second Lesson, taken out of the New Testament. And after that, the Hymn following; except when that shall happen to be read in the Chapter for the day, or for the Gospel on Saint John Baptist's Day.}

Benedictus.
St. Luke i. 68.
\drop{Blessed be the Lord God of Israel : for he hath visited and redeemed his people;}

And hath raised up a mighty salvation for us : in the house of his servant David;

As he spake by the mouth of his holy Prophets : which have been since the world began;

That we should be saved from our enemies : and from the hand of all that hate us.

To perform the mercy promised to our forefathers : and to remember his holy Covenant;

To perform the oath which he sware to our forefather Abraham : that he would give us;

That we being delivered out of the hand of our enemies : might serve him without fear;

In holiness and righteousness before him : all the days of our life.

And thou, Child, shalt be called the Prophet of the Highest : for thou shalt go before the face of the Lord to prepare his ways;

To give knowledge of salvation unto his people : for the remission of their sins,

Through the tender mercy of our God : whereby the day-spring from on high hath visited us;

To give light to them that sit in darkness, and in the shadow of death : and to guide our feet into the way of peace.

Glory be to the Father, and to the Son : and to the Holy Ghost;

As it was in the beginning, is now, and ever shall be : world without end. Amen.

\pilcrow{Then shall be sung or said the Apostle's Creed, by the Minister and the people standing : Except only such days as the Creed of Saint Athanasius is appointed to be read.}

\drop{I believe in God the Father Almighty, Maker of heaven and earth :}

And in Jesus Christ his only Son our Lord: Who was conceived by the Holy Ghost, Born of the Virgin Mary: Suffered under Pontius Pilate, Was crucified, dead, and buried: He descended into hell; The third day he rose again from the dead: He ascended into heaven, And sitteth on the right hand of God the Father Almighty: From thence he shall come to judge the quick and the dead.

I believe in the Holy Ghost: The holy Catholick Church; The Communion of Saints: The Forgiveness of sins: The Resurrection of the body, And the Life everlasting. Amen.

\pilcrow{And after that these Prayers following, all devoutly kneeling: the Minister first pronouncing with a loud voice,}
    The Lord be with you.

    Answer. And with thy spirit.

    Minister. Let us pray.

    Lord, have mercy upon us.

       Christ, have mercy upon us.

    Lord, have mercy upon us.

[1549: Then shalbe said dailye through the yere the praiers folowing, as well at evensong as at Matins, all devoutely kneelyng.

Lorde have mercie upon us. Christe have mercie upon us. Lorde, have mercie upon us.

Then the minister shal say the Crede and the Lordes praier in englishe, with a loude voice, etc.

Answere.
But deliver us from eivill. Amen.]


Then the Minister, Clerks, and people shall say the Lord's Prayer with a loud voice.
\drop{Our Father, which art in heaven, Hallowed be thy Name. Thy kingdom come. Thy will be done in earth, As it is in heaven. Give us this day our daily bread. And forgive us our trespasses, As we forgive them that trespass against us. And lead us not into temptation, But deliver us from evil. Amen.}
Then the Priest standing up shall say,

    O Lord, shew thy mercy upon us.

    Answer. And grant us thy salvation.

    Priest. O Lord, save the Queen.

    Answer. And mercifully hear us when we call upon thee.

    Priest. Endue thy Ministers with righteousness.

    Answer. And make thy chosen people joyful.

    Priest. O Lord, save thy people.

    Answer. And bless thine inheritance.

    Priest. Give peace in our time, O Lord.

    Answer. Because there is none other that fighteth for us, but only thou, O God.

    Priest. O God, make clean our hearts within us.

    Answer. And take not thy Holy Spirit from us.

\pilcrow{Then shall follow three Collects; the first of the day, which shall be the same that is appointed at the Communion; The second for Peace; The third for Grace to live well. And the two last Collects shall never alter, but daily be said at Morning Prayer throughout all the year, as followeth, all kneeling.}

The second Collect, for Peace.
\drop{O God, who art the author of peace and lover of concord, in knowledge of whom standeth our eternal life, whose service is perfect freedom; Defend us thy humble servants in all assaults of our enemies; that we, surely trusting in thy defence, may not fear the power of any adversaries, through the might of Jesus Christ our Lord. \R Amen.}


The third Collect, for Grace.
\drop{O Lord, our heavenly Father, Almighty and everlasting God, who hast safely brought us to the beginning of this day; Defend us in the same with thy mighty power; and grant that this day we fall into no sin, neither run into any kind of danger; but that all our doings may be ordered by thy governance, to do always that is righteous in thy sight; through Jesus Christ our Lord. \R Amen.}

\pilcrow{In Quires and Places where they sing here followeth the Anthem.}

\subsubsection{Here endeth the Order of Morning Prayer throughout the Year.}

\fleuron
\chapter{The Order for Evensong}
\subsubsection{Daily Throughout the Year.}
\bigskip
\ourFather

\medskip

\pilcrow{Here all standing up, the Priest shall say,}

\drop{O God, \cross make speed to save me.  \R O Lord, make haste to help me.}

\V Glory be to the Father, and to the Son, and to the Holy Ghost;  \R As it was in the beginning, is now, and ever shall be, world without end. Amen.

\centerline{Alleluya.}

\centeredrubric{From First Evensong of Septuagesima until Easter, instead of \emph{\black Alleluya} is said:}

\V Praise ye the Lord.  \R The Lord's Name be praised.

\bigskip

\pilcrow{Then shall be said or sung the Psalms in order as they be appointed. Then a Lesson of the Old Testament, as is appointed. And after that \emph{Magnificat} (or the Song of the blessed Virgin Mary) in English, as followeth.}

\subsection{\stylesubsec{The Song of the Blessed Virgin Mary}{Magnificat.}{St.~Luke i.~46.}}

\drop{My \cross soul doth magnify the Lord, * and my spirit hath rejoiced in God my Saviour.}

2 For he hath regarded * the lowliness of his handmaiden.

3 For behold, from henceforth * all generations shall call me blessed.

4 For he that is mighty hath magnified me; * and holy is his Name.

5 And his mercy is on them that fear him * throughout all generations.

6 He hath shewed strength with his arm; * he hath scattered the proud in the imagination of their hearts.

7 He hath put down the mighty from their seat, * and hath exalted the humble and meek.

8 He hath filled the hungry with good things; * and the rich he hath sent empty away.

9 He remembering his mercy hath holpen his servant Israel; * as he promised to our forefathers, Abraham and his seed, for ever.

Glory be to the Father, and to the Son, * and to the Holy Ghost;

As it was in the beginning, is now, and ever shall be; * world without end. Amen.

\medskip

\pilcrow{Then a Lesson of the New Testament, as it is appointed. And after that \emph{Nunc dimittis} (or the Song of Symeon) in English, as followeth.}
\subsection{\stylesubsec{The Song of Symeon}{Nunc dimittis.}{St.~Luke ij.~29.}}

\drop{Lord, \cross now lettest thou thy servant depart in peace, * according to thy word.}

2 For mine eyes have seen * thy salvation,

3 Which thou hast prepared * before the face of all people;

4 To be a light to lighten the Gentiles, * and to be the glory of thy people Israel.

Glory be to the Father, and to the Son, * and to the Holy Ghost;

As it was in the beginning, is now, and ever shall be; * world without end. Amen.

\medskip

\pilcrow{Then shall be sung or said the Apostles' Creed, by the Minister and the people standing.}
\drop{I believe in God the Father Almighty, \ Maker of heaven and earth:}

And in Jesus Christ his only Son our Lord: \ Who was conceived by the Holy Ghost, \ Born of the Virgin Mary: \ Suffered under Pontius Pilate, \ Was crucified, dead, and buried: \ He descended into hell; \ The third day he rose again from the dead: \ He ascended into heaven, \ And sitteth on the right hand of God the Father Almighty: \ From thence he shall come to judge the quick and the dead.

I believe in the Holy Ghost: \ The holy Catholick Church; \ The Communion of Saints: \ The Forgiveness of sins: \ The Resurrection of the body, \ And the Life everlasting. Amen.

\medskip

\pilcrow{And after that, these Prayers following, all devoutly kneeling,}

\centerline{Lord, have mercy upon us.}
\centerline{\emph{Christ, have mercy upon us.}}
\centerline{Lord, have mercy upon us.}

\medskip

{\centering\rubric{Then the Minister, Clerks, and people shall say the Lord's Prayer with a loud voice.}\par}
\ourFather

\smallskip

\centerline{\rubric{Then the Minister standing up shall say,}}

\V O Lord, shew thy mercy upon us.  \R And grant us thy salvation.

\V O Lord, save the \emph{State}.  \R And mercifully hear us when we call upon thee.

\V Endue thy Ministers with righteousness.  \R And make thy chosen people joyful.

\V  O Lord, save thy people.  \R And bless thine inheritance.

\V Give peace in our time, O Lord.  \R Because there is none other that fighteth for us, but only thou, O God.

\V O God, make clean our hearts within us.  \R And take not thy Holy Spirit from us.

\V The Lord be with you.  \R And with thy spirit.

\centerline{Let us pray.}


\pilcrow{Then shall follow three Collects; the first of the day; The second for Peace; The third for Aid against all Perils, as hereafter followeth: which two last Collects shall be daily said at Evening Prayer without alteration.}

\subsection{\stylesubsec{The second Collect, }{for Peace.}{}}
\drop{O God, from whom all holy desires, all good counsels, and all just works do proceed: Give unto thy servants that peace which the world cannot give; that both our hearts may be set to obey thy commandments, and also that by thee we being defended from the fear of our enemies may pass our time in rest and quietness; through the merits of Jesus Christ our Saviour. \R Amen.}

\subsection{\stylesubsec{The third Collect, }{for  Aid against all Perils.}{}}
\drop{Lighten our darkness, we beseech thee, O Lord; and by thy great mercy defend us from all perils and dangers of this night; for the love of thy only Son, our Saviour, Jesus Christ. \R Amen.}

\medskip
\V The Lord be with you.  \R And with thy spirit.

\V Let us bless the Lord.  \R Thanks be to God.

\medskip

\pilcrow{In Quires and Places where they sing here followeth the Anthem.}
\smallskip
\pilcrow{Here may follow any of the Occasional Prayers and Thanksgivings, as need may require, ending with one of the Conclusions.}

\subsubsection{Here endeth the Order of Evening Prayer throughout the Year.}

\fleuron
\chapter{Quicunque vult}
\pilcrow{Upon these Feasts; Christmas Day, the Epiphany, Easter Day, Ascension Day, Whitsunday, and upon Trinity Sunday, shall be sung or said at Morning Prayer, instead of the Apostles' Creed, this Confession of our Christian Faith, commonly called the Creed of Athanasius, by the Minister and people standing.}
\pilcrow{\emph{Quicunque vult} may also be sung or said at Mattins upon these Feasts; Saint Matthias, Saint John Baptist, Saint James, Saint Bartholomew, Saint Matthew, Saint Simon and Saint Jude, and Saint Andrew}

\subsubsection{Quicunque vult.}
\drop{Whosoever will be saved : before all things it is necessary that he hold the Catholick Faith.}

2 Which Faith except every one do keep whole and undefiled : without doubt he shall perish everlastingly.

\drop{And the Catholick Faith is this: That we worship one God in Trinity, and Trinity in Unity;}

4 Neither confounding the Persons : nor dividing the Substance.

5 For there is one Person of the Father, another of the Son : and another of the Holy Ghost.

6 But the Godhead of the Father, of the Son, and of the Holy Ghost, is all one : the Glory equal, the Majesty co-eternal.

7 Such as the Father is, such is the Son : and such is the Holy Ghost.

8 The Father uncreate, the Son uncreate : and the Holy Ghost uncreate.

9 The Father incomprehensible, the Son incomprehensible : and the Holy Ghost incomprehensible.

10 The Father eternal, the Son eternal : and the Holy Ghost eternal.

11 And yet they are not three eternals : but one eternal.

12 As also there are not three incomprehensibles, nor three uncreated : but one uncreated, and one incomprehensible.

13 So likewise the Father is Almighty, the Son Almighty : and the Holy Ghost Almighty.

14 And yet they are not three Almighties : but one Almighty.

15 So the Father is God, the Son is God : and the Holy Ghost is God.

16 And yet they are not three Gods : but one God.

17 So likewise the Father is Lord, the Son Lord : and the Holy Ghost Lord.

18 And yet not three Lords : but one Lord.

19 For like as we are compelled by the Christian verity to acknowledge every Person by himself to be both God and Lord;

20 So are we forbidden by the Catholick Religion : to say, There be three Gods, or three Lords.

21 The Father is made of none : neither created, nor begotten.

22 The Son is of the Father alone : not made, nor created, but begotten.

23 The Holy Ghost is of the Father and of the Son : neither made, nor created, nor begotten, but proceeding.

24 So there is one Father, not three Fathers; one Son, not three Sons : one Holy Ghost, not three Holy Ghosts.

25 And in this Trinity none is afore, or after other : none is greater, or less than another;

26 But the whole three Persons are co-eternal together : and co-equal.

27 So that in all things, as is aforesaid : the Unity in Trinity and the Trinity in Unity is to be worshipped.

28 He therefore that will be saved : must think thus of the Trinity.

\drop{Furthermore, it is necessary to everlasting salvation : that he also believe rightly the Incarnation of our Lord Jesus Christ.}

30 For the right Faith is, that we believe and confess : that our Lord Jesus Christ, the Son of God, is God and Man;

31 God, of the substance of the Father, begotten before the worlds : and Man of the substance of his Mother, born in the world;

32 Perfect God and perfect Man : of a reasonable soul and human flesh subsisting.

33 Equal to the Father, as touching his Godhead : and inferior to the Father, as touching his manhood;

34 Who, although he be God and Man : yet he is not two, but one Christ;

35 One, not by conversion of the Godhead into flesh : but by taking of the Manhood into God;

36 One altogether; not by confusion of Substance : but by unity of Person.

37 For as the reasonable soul and flesh is one man : so God and Man is one Christ;

38 Who suffered for our salvation : descended into hell, rose again the third day from the dead.

39 He ascended into heaven, he sitteth at the right hand of the Father, God Almighty : from whence he will come to judge the quick and the dead.

40 At whose coming all men will rise again with their bodies : and shall give account for their own works.

41 And they that have done good shall go into life everlasting : and they that have done evil into everlasting fire.

\drop{This is the Catholick Faith : which except a man believe faithfully, he cannot be saved.}

Glory be to the Father, and to the Son : and to the Holy Ghost;

As it was in the beginning, is now, and ever shall be : world without end. Amen.

\fleuron


\chapter{The Litany}

\pilcrow{Here followeth the \emph{Litany}, or General Supplication, to be sung or said upon Sundays, Wednesdays, and Fridays (except on Christmas Day, Easter Day, and Whitsunday), on the Rogation Days, and at other times when it shall be commanded by the Ordinary.}

\subsubsection{The Invocations.}

% \chant{1}{}{chants/litany/oGodtheFather}
\drop{O God the Father, of héaven : have mercy upon us.}

\emph{O God the Father, of héaven : have mercy upon us.}

O God the Son, Redeemer of the wórld : have mercy upon us.

\emph{O God the Son, Redeemer of the wórld : have mercy upon us.}

O God the Holy Ghost, proceeding from the Father and the Són : have mercy upon us.

\emph{O God the Holy Ghost, proceeding from the Father and the Són : have mercy upon us.}

O holy, blessed, and glorious Trinity, three Persons and one Gód : Have mercy upon us.

\emph{O holy, blessed, and glorious Trinity, three Persons and one Gód : Have mercy upon us.}


\drop{Holy Virgin Mary, Mother of God our Saviour Jesus Chríst:}

\centerline{\emph{Pray for us.}}
% \chant{0}{}{chants/litany/oraProNobis}

All holy Angels and Archangels and all holy orders of blessed Spírits:

\centerline{\emph{Pray for us.}}

All holy Patriarchs and Prophets, Apostles, Martyrs, Confessors, and Virgins, and all the blessed company of Héaven:

\centerline{\emph{Pray for us.}}


\subsubsection{The Deprecations.}

\drop{Remember not, Lord, our offences, nor the offences of our forefathers; neither take thou vengeance of our sins: Spare us, good Lord, spare thy people, whom thou hast redeemed with thy most precious blood, and be not angry with · us for éver:}
% \chant{1}{}{chants/litany/rememberNot}

\centerline{\emph{Spare us, good Lord.}}

From all evil and mischief, from sin, from the crafts and assaults of the devil, from thy wrath, and from everlast·ing damnátion,

\centerline{\emph{Good Lord, deliver us.}}

% \chant{0}{}{chants/litany/liberaNos}
From all blindness of heart; from pride, vainglory, and hypocrisy; from envy, hatred, and malice, and · all uncháritableness,

\centerline{\emph{Good Lord, deliver us.}}

From fornication, and all other deadly sin; and from all the deceits of the world, the flesh, · and the dévil,

\centerline{\emph{Good Lord, deliver us.}}

From lightning and tempest; from earthquake, fire, and flood; from plague, pestilence, and famine; from battle and murder, and from · sudden déath,

\centerline{\emph{Good Lord, deliver us.}}

From all sedition, privy conspiracy, and rebellion; from all false doctrine, heresy, and schism; from hardness of heart, and contempt of thy Word · and Commándment,

\centerline{\emph{Good Lord, deliver us.}}



\subsubsection{The Obsecrations.}
\drop{By the mystery of thy holy Incarnation; by thy holy Nativity and Circumcision; by thy Baptism, Fasting, · and Temptátion,}

\centerline{\emph{Good Lord, deliver us.}}

By thine Agony and Bloody Sweat; by thy Cross and Passion; by thy precious Death and Burial; by thy glorious Resurrection and Ascension, and by the Coming of the · Holy Ghóst,

\centerline{\emph{Good Lord, deliver us.}}

In all time of our tribulation; in all time of our wealth; in the hour of death, and in the · day of júdgement,

\centerline{\emph{Good Lord, deliver us.}}


\subsubsection{The Intercessions.}
\drop{We sinners do beseech thee to hear us, O Lord God; and that it may please thee to rule and govern thy holy Church universal in · the right wáy;}
% \chant{0}{}{chants/litany/teRogamus}

\centerline{\emph{We beseech thee to hear us, good Lord.}}

That it may please thee so to rule the heart of thy servant, \emph{The President of the United States}, that \emph{he} may above all things seek thy ho·nour and glóry; % I had italisized "servant" - but the prayer book seems to almost always use it for both sexes.

\centerline{\emph{We beseech thee to hear us, good Lord.}}

That it may please thee to bless and preserve all Rulers and Magistrates, giving them grace to execute justice, and to · maintain trúth;

\centerline{\emph{We beseech thee to hear us, good Lord.}}

That it may please thee to illuminate all Bishops, Priests, and Deacons, with true knowledge and understanding of thy Word; and that both by their preaching and living they may set it forth, and shew · it accórdingly;

\centerline{\emph{We beseech thee to hear us, good Lord.}}

\begin{leftbar}
That it may please thee to bless thy servants at this time [to be]\begin{wrapfigure}{r}{0.46\textwidth}{\footnotesize\rubric{To be used in the Ember Weeks, and on the day of an Ordination.}\par}\end{wrapfigure} admitted to the Order of Deacons or of Priests, and to pour thy grace upon them; that they may duly execute their office to the edifying of thy Church, and to the glory of thy holy name; % see above - the prayer book uses "servants" for both sexes.

\centerline{\emph{We beseech thee to hear us, good Lord.}}
\end{leftbar}
That it may please thee to send forth labourers in·to thy hárvest;

\centerline{\emph{We beseech thee to hear us, good Lord.}}

That it may please thee to bless and keep · all thy péople;

\centerline{\emph{We beseech thee to hear us, good Lord.}}

That it may please thee to give to all nations unity, · peace, and cóncord;

\centerline{\emph{We beseech thee to hear us, good Lord.}}

That it may please thee to give us an heart to love and dread thee, and diligently to live after · thy commándments;

\centerline{\emph{We beseech thee to hear us, good Lord.}}

That it may please thee to give to all thy people increase of grace to hear meekly thy Word, and to receive it with pure affection, and to bring forth the fruits · of the Spírit;

\centerline{\emph{We beseech thee to hear us, good Lord.}}

That it may please thee to bring into the way of truth all such as have erred, and · are decéived;

\centerline{\emph{We beseech thee to hear us, good Lord.}}

That it may please thee to strengthen such as do stand; and to comfort and help the weak-hearted; and to raise up them that fall; and finally to beat down Satan un·der our féet;

\centerline{\emph{We beseech thee to hear us, good Lord.}}

That it may please thee to succour, help, and comfort, all that are in danger, necessity, and · tribulátion;

\centerline{\emph{We beseech thee to hear us, good Lord.}}

That it may please thee to preserve all that travel by land, by water, or by air; all women labouring of child, all sick persons, and young children; and to shew thy pity upon all prison·ers and cáptives; 

\centerline{\emph{We beseech thee to hear us, good Lord.}}

That it may please thee to defend, and provide for, the fatherless children, and widows, and all that are desolate · and oppréssed;

\centerline{\emph{We beseech thee to hear us, good Lord.}}

That it may please thee to have mercy · upon áll men;

\centerline{\emph{We beseech thee to hear us, good Lord.}}

That it may please thee to forgive our enemies, persecutors, and slanderers, and to · turn their héarts;

\centerline{\emph{We beseech thee to hear us, good Lord.}}

That it may please thee to give and preserve to our use the kindly fruits of the earth, so as in due time we · may enjóy them;

\centerline{\emph{We beseech thee to hear us, good Lord.}}

That it may please thee to give us true repentance; to forgive us all our sins, negligences, and ignorances; and to endue us with the grace of thy Holy Spirit to amend our lives according to thy · holy Wórd;

\centerline{\emph{We beseech thee to hear us, good Lord.}}


\subsubsection{The Conclusion.}
% \chant{1}{}{chants/litany/sonOfGod}\par%\end

\drop{Son of God : we beseech thee to hear us.}

\centerline{\emph{Son of God : we beseech thee to hear us.}}

O Lamb of God : that takest away the sins of the world;

\centerline{\emph{Grant us thy peace.}}

O Lamb of God : that takest away the sins of the world;

\centerline{\emph{Have mercy upon us.}}

\centerline{O Christ, hear us.}
\centerline{\emph{O Christ, hear us.}}

\centerline{Lord, have mercy upon us.}
\centerline{\emph{Lord, have mercy upon us.}}
\centerline{Christ, have mercy upon us.}
\centerline{\emph{Christ, have mercy upon us.}}
\centerline{Lord, have mercy upon us.}
\centerline{\emph{Lord, have mercy upon us.}}


\medskip


% \chant{1}{}{chants/litany/agnusDei2}
\noindent
\pilcrow{Then shall the Priest, and the people with him, say the Lord’s Prayer.}
\ourFather

\medskip


\pilcrow{On ordinary days, the \emph{Litany} continues at \emph{The Station} on \emph{pg.~\emph{\pageref{litanyStation}}.}}

\section{A Supplication}
\pilcrow{On Rogation Days, during penitential seasons, and in times of trouble, the Litany may continue thus,}

\V O Lord, deal not with us after our síns.  \R Neither reward us after our iníquities.

\centerline{Let us pray.}
\drop{O God, merciful Father, that despisest not the sighing of a contrite heart, nor the desire of such as be sorrowful; Mercifully assist our prayers which we make before thee in all our troubles and adversities, whensoever they oppress us; and graciously hear us, that those evils which the craft and subtilty of the devil or man worketh against us be brought to nought; and by the providence of thy goodness they may be dispersed; that we thy servants, being hurt by no persecutions, may evermore give thanks unto thee in thy holy Church; through Jesus Christ our Lord.  \R Amen.}

\medskip
% ¶ The following is then sung, usually proceeding to the Altar (when sung in procession).  When the procession is long, as on Rogation Days, other verses of {Psalm 44} may be sung as well.

% \chant{1}{ij.}{chants/litany/oLordArise}
\ant O Lord, arise, help us, and deliver us for thy name’s sake.

\V O God, we have heard with our ears, and our fathers have • declared únto us, * the noble works that thou didst in their days, and in the • old time before them.

\ant O Lord, arise, help us, and deliver us for thine honour.

\V Glory be to the Father, and to the Son, • and to the Hóly Ghost; † As it was in the beginning, is now, • and ever sháll be, * world • without end. Amen.

\ant O Lord, arise, help us, and deliver us for thy name’s sake.

\medskip

% \chant{1}{\V}{chants/litany/suffrages}

\V From our enemies defend us, O Chríst.  \R Graciously look upon our afflíctions.

\V Pitifully behold the sorrows of our héarts.  \R Mercifully forgive the sins of thy péople.

\V Favourably with mercy hear our práyers.  \R O Son of David, have mercy upon ús.

\V Both now and ever vouchsafe to hear us, O Chríst.  \R Graciously hear us, O Christ; graciously hear us, O Lord Chríst.


\section{The Station}
\label{litanyStation}

\V O Lord, let thy mercy be shewed upon ús; \R As we do put our trust in thée.

\centerline{Let us pray.}

\drop{We humbly beseech thee, O Father, mercifully to look upon our infirmities; and, for the glory of thy Name, turn from us all those evils that we most righteously have deserved; and grant, that in all our troubles we may put our whole trust and confidence in thy mercy, and evermore serve thee in holiness and pureness of living, to thy honour and glory; through our only Mediator and Advocate, Jesus Christ our Lord. \R Amen.}


\medskip


\pilcrow{The following two prayers, \emph{For the State} and \emph{For the Church}, may be replaced on Wednesdays and Fridays with \emph{Other Prayers} below.}

\subsubsection{A Prayer for \emph{The President of the United States}, and all in Civil Authority.}


\drop{O Lord, our heavenly Father, the high and mighty Ruler of the universe, who dost from thy throne behold all the dwellers upon earth; Most heartily we beseech thee with thy favour to behold and bless thy \emph{servant The President of the United States}, and all others in authority; and so replenish them with the grace of thy Holy Spirit, that they may always incline to thy will, and walk in thy way. Endue them plenteously with heavenly gifts; grant them in health and prosperity long to live; and finally, after this life, to attain everlasting joy and felicity; through Jesus Christ our Lord. \R Amen.}


\subsubsection{A Prayer for the Clergy and People.}
\drop{Almighty and everlasting God, who alone workest great marvels; Send down upon our Bishops, and Clergy, and all Congregations committed to their charge, the healthful Spirit of thy grace; and that they may truly please thee, pour upon them the continual dew of thy blessing. Grant this, O Lord, for the honour of our Advocate and Mediator, Jesus Christ. \R Amen.}


\medskip


\pilcrow{Other approved prayers may be included, \emph{ad libitum.}}


\medskip


\pilcrow{During Ember Weeks, the prayer \emph{9. For them that are to be admitted into Holy Orders} shall be said here.}


\subsubsection{A Prayer for Mercy.}
[COUDL BE MOVED TO THE END OF THE PRAYERS?]
\drop{O God, whose nature and property is ever to have mercy and to forgive, receive our humble petitions; and though we be tied and bound with the chain of our sins, yet let the pitifulness of thy great mercy loose us; for the honour of Jesus Christ, our Mediator and Advocate. \R Amen.}

\label{chrysostom}
\subsubsection{A Prayer of St. Chrysostom.}
\drop{Almighty God, who hast given us grace at this time with one accord to make our common supplications unto thee; and dost promise, that when two or three are gathered together in thy Name thou wilt grant their requests; Fulfil now, O Lord, the desires and petitions of thy servants, as may be most expedient for them; granting us in this world knowledge of thy truth, and in the world to come life everlasting. \R Amen.}

\newcommand{\theGrace}{
    \subsubsection{2 Corinthians 13.}
    \drop{The grace \cross of our Lord Jesus Christ, and the love of God, and the fellowship of the Holy Ghost, be with us all evermore. \R Amen.}
}
\theGrace


\subsubsection{Here endeth the Litany.}

\fleuron

\chapter[Prayers and Thanksgivings]{Prayers and Thanksgivings upon several Occasions}
\subsubsection{To be used before the final two prayers at the Litany, or after Mattins or Evensong; or at other times.}
\label{prayers}


\section{For the State}
%Am1928
\subseccaption{1.}{A Prayer for Congress, to be used during their Session.}
\drop{Most gracious God, we humbly beseech thee, as for the people of these United States in general, so especially for their Senate and Representatives in Congress assembled; that thou wouldest be pleased to direct and prosper all their consultations, to the advancement of thy glory, the good of thy Church, the safety, honour, and welfare of thy people; that all things may be so ordered and settled by their endeavours, upon the best and surest foundations, that peace and happiness, truth and justice, religion and piety, may be established among us for all generations. These and all other necessaries, for them, for us, and thy whole Church, we humbly beg in the Name and mediation of Jesus Christ, our most blessed Lord and Saviour. \R Amen.}


%Am1928
\subseccaption{2.}{For Courts of Justice.}
\drop{Almighty God, who sittest in the throne judging right; We humbly beseech thee to bless the courts of justice and the magistrates in all this land; and give unto them the spirit of wisdom and understanding, that they may discern the truth and impartially administer the law in the fear of thee alone; through him who shall come to be our judge, thy Son, our Saviour, Jesus Christ. \R Amen.}


%Am1928
\subseccaption{3.}{For a State Legislature.}
\drop{O God, the fountain of wisdom, whose statutes are good and gracious and whose law is truth; We beseech thee so to guide and bless the Legislature of this State, that it may ordain for our governance only such things as please thee, to the glory of thy Name and the welfare of the people; through Jesus Christ, thy Son, our Lord. \R Amen.}

%Am1928
\subseccaption{4.}{For Our Country.}
\drop{Almighty God, who hast given us this good land for our heritage; We humbly beseech thee that we may always prove ourselves a people mindful of thy favour and glad to do thy will. Bless our land with honourable industry, sound learning, and pure manners. Save us from violence, discord, and confusion; from pride and arrogancy, and from every evil way. Defend our liberties, and fashion into one united people the multitudes brought hither out of many kindreds and tongues. Endue with the spirit of wisdom those to whom in thy Name we entrust the authority of government, that there may be justice and peace at home, and that, through obedience to thy law, we may shew forth thy praise among the nations of the earth. In the time of prosperity, fill our hearts with thankfulness, and in the day of trouble, suffer not our trust in thee to fail; all which we ask through Jesus Christ our Lord. \R Amen.}


\section{For the Church}
%Am1928
\subseccaption{5.}{For the Church.}
\drop{O Gracious Father, we humbly beseech thee for thy holy Catholic Church; that thou wouldst be pleased to fill it with all truth, in all peace. Where it is corrupt, purify it; where it is in error, direct it; where in anything it is amiss, reform it. Where it is right, establish it; where it is in want, provide for it; where it is divided, reunite it; for the sake of him who died and rose again, and ever liveth to make intercession for us, Jesus Christ, thy Son, our Lord. \R Amen.}


%Eng1923
\subseccaption{6.}{For Unity.}
\drop{O Lord Jesus Christ, who didst say to thine Apostles, Peace I leave with you, my peace I give unto you: Regard not our sins but the faith of thy Church, and grant it that peace and unity which is agreeable to thy will; who livest and reignest with the Father and the Holy Spirit, one God, world without end.  \R Amen.}

%Eng1923, Am1928
\subseccaption{7.}{For the Increase of the Sacred Ministry.}
\drop{O Almighty God, look mercifully upon the world which thou hast redeemed by the blood of thy dear Son, and incline the hearts of many to the ministry of thy Church, so that by their labours thy light may shine in the darkness, and the kingdom of thy Son be hastened by the perfecting of thine elect; through Jesus Christ our Lord. \R Amen.}

% En1928 (and others, after different forms)
\subseccaption{8.}{For Missions.}
\drop{O God, who hast made of one blood all nations of men for to dwell on the face of the earth, and didst send thy blessed Son Jesus Christ to preach peace to them that are afar off, and to them that are nigh: Grant that all the peoples of the world may feel after thee and find thee; and hasten, O Lord, the fulfilment of thy promise, to pour out thy Spirit upon all flesh; through Jesus Christ our Lord.  \R Amen.}

\label{reconciliationWithTheJews}
\subseccaption{9.}{For Reconciliation with the Jews.}
\smallskip
%Prayer Book Society of Canada; Authorized for use ad libitum by the General Synod of the Anglican Church of Canada.
\drop{O God, who didst choose Israel to be thine inheritance; Have mercy upon us and forgive us for violence and wickedness against our brother Jacob; the arrogance of our hearts and minds hath deceived us, and shame hath covered our face. Take away all pride and prejudice in us, and grant that we, together with the people whom thou didst first make thine own, may attain to the fulness of redemption which thou hast promised; to the honour and glory of thy most holy Name.\R Amen.}


%Eng1923
\subseccaption{9.}{For Candidates for Confirmation.}
\drop{O God, who through the teaching of thy Son Jesus Christ didst prepare the disciples for the coming of the Comforter; Make ready, we beseech thee, the hearts and minds of thy \emph{servants} who at this time are seeking the gifts of the Holy Ghost through the laying on of hands, that, drawing near with penitent and faithful hearts, they may be filled with the power of his divine indwelling; through the same Jesus Christ our Lord. \R Amen.}


%Eng1662
\subseccaption{10.}{In the Ember Weeks, to be said every day, for them that are to be admitted into Holy Orders.}
\drop{Almighty God, our heavenly Father, who hast purchased to thyself an universal Church by the precious blood of thy dear Son; Mercifully look upon the same, and at this time so guide and govern the minds of thy servants the Bishops and Pastors of thy flock, that they may lay hands suddenly on no man, but faithfully and wisely make choice of fit persons to serve in the sacred Ministry of thy Church. And to those which shall be ordained to any holy function give thy grace and heavenly benediction; that both by their life and doctrine they may set forth thy glory, and set forward the salvation of all men; through Jesus Christ our Lord. \R Amen.}


%Sc1912
\subseccaption{11.}{For Synods and Chapters of the Church.}
\drop{O Eternal God, the fountain of all wisdom, who didst send thy Holy Spirit to lead the disciples into all the truth; Vouchsafe that he being present with thy servants and handmaidens, the Bishops [\rubric{or} Bishop] and Presbyters about to assemble [\rubric{or} now assembled] in the Synod of this jurisdiction, may so rule their hearts and guide their counsels that in all things they may seek only thy glory and the good of thy holy Church; through Jesus Christ our Lord. \R Amen.}


%En1662
\section{For the Natural Order}
\subseccaption{12.}{For Rain.}
\drop{O God, heavenly Father, who by thy Son Jesus Christ hast promised to all them that seek thy kingdom, and the righteousness thereof, all things necessary to their bodily sustenance; Send us, we beseech thee, in this our necessity, such moderate rain and showers, that we may receive the fruits of the earth to our comfort, and to thy honour; through Jesus Christ our Lord. \R Amen.}


%Sc1912
\subseccaption{13.}{For Fair Weather.}
\drop{Almighty God, our Heavenly Father, who art the author and giver of all good things; Look, we beseech thee, in thy loving-kindness upon us thine unworthy servants, and grant to us at this time such fair weather that we may receive the fruits of the earth in their season, to our comfort and the glory of thy holy Name, through Jesus Christ, our Mediator and Advocate.  \R Amen.}

%Eng1923
\subseccaption{14.}{In the time of Dearth and Famine.}
\drop{O God, our heavenly Father, who by thy blessed Son hast taught us to ask our daily bread of thee; Behold, we beseech thee, the affliction of thy people, and send us a seasonable relief in this our necessity.  Increase the fruits of the earth by thy heavenly benediction; and grant that we, receiving with thankfulness thy gracious gifts, may use the same to thy glory, the relief of those that are needy, and our own comfort; through Jesus Christ our Lord. \R Amen.}


%Eng1923
\subseccaption{15.}{In the time of any common Plague or Sickness.}
\drop{Grant, we beseech thee, merciful Lord, help and deliverance unto us, who are visited with grievous mortality and sickness. Sanctify to us this our sore distress, and prosper with thy continual blessing those who labour to devise for mankind protection against disease and pain; through him who both healed and glorified pain, thy Son Jesus Christ our Lord. \R Amen.}



%Am1928
\subseccaption{16.}{In Time of Calamity.}
\drop{O God, merciful and compassionate, who art ever ready to hear the prayers of those who put their trust in thee; Graciously hearken to us who call upon thee, and grant us thy help in this our need; through Jesus Christ our Lord. \R Amen.}

\subseccaption{17.}{On the Rogation Days.}
%En1923
\drop{Almighty God, who hast blessed the earth that it should be fruitful and bring forth abundantly whatsoever is needful for the life of man: Prosper, we beseech thee, the labours of the husbandman, and grant such seasonable weather that we may gather in the fruits of the earth and ever rejoice in thy goodness, to the praise of thy holy Name; through Jesus Christ our Lord. \R Amen.}


%En1923
\drop{O Almighty God, who hast made the sea and all that moveth therein: Bestow thy blessing on the harvest of the waters, that it may be abundant in its season, and on our fishermen and mariners, that they may be safe in every peril of the deep; so that we all with thankful hearts may acknowledge thee who art the Lord of the sea and of the dry land; through Jesus Christ our Lord. \R Amen.}


%En1923
\drop{Almighty Father, who by thy Son Jesus Christ hast sanctified labour to the welfare of mankind: Prosper, we pray thee, the industries of this land and all those who are engaged therein; that shielded in all their temptations and dangers, and receiving a rich reward of their labours, they may praise thee by living according to thy will; through Jesus Christ our Lord. \R Amen.}


\section{For the Social Order}
%Am1928
\subseccaption{18.}{For Children.}
\drop{O Lord, Jesus Christ, who dost embrace children with the arms of thy mercy, and dost make them living members of thy Church; Give them grace, we pray thee, to stand fast in thy faith, to obey thy word, and to abide in thy love; that being made strong by thy Holy Spirit they may resist temptation and overcome evil; and may rejoice in the life that now is, and dwell with thee in the life that is to come; through thy merits, O merciful Saviour, who with the Father and the Holy Ghost livest and reignest one God, world without end. \R Amen.}


%Am1928
\subseccaption{19.}{For Every Man in his Work.}
\drop{Almighty God, our heavenly Father, who declarest thy glory and shewest forth thy handiwork in the heavens and in the earth; Deliver us, we beseech thee, in our several callings, from the service of mammon, that we may do the work which thou givest us to do, in truth, in beauty, and in righteousness, with singleness of heart as thy servants, and to the benefit of our fellow men; for the sake of him who came among us as one that serveth, thy Son, Jesus Christ our Lord. \R Amen.}


%Am1928
\subseccaption{20.}{For Christian Service.}
\drop{O Lord our heavenly Father, whose blessed Son came not to be ministered unto, but to minister; We beseech thee to bless all who, following in his steps, give themselves to the service of their fellow men. Endue them with wisdom, patience, and courage, that they may strengthen the weak and raise up those who fall; and, being inspired by thy love, may worthily minister in thy Name to the suffering, the friendless, and the needy; for the sake of him who laid down his life for us, the same thy Son, our Saviour, Jesus Christ. \R Amen.}


%Am1928
\subseccaption{21.}{For Social Justice.}
\drop{Almighty God, who hast created man in thine own image; Grant us grace fearlessly to contend against evil, and to make no peace with oppression; and, that we may reverently use our freedom, help us to employ it in the maintenance of justice among men and nations, to the glory of thy holy Name; through Jesus Christ our Lord. \R Amen.}


%Am1928
\subseccaption{22.}{For the Family of Nations.}
\drop{Almighty God, our heavenly Father, guide, we beseech thee, the Nations of the world into the way of justice and truth, and establish among them that peace which is the fruit of righteousness, that they may become the Kingdom of our Lord and Saviour Jesus Christ. \R Amen.}



%Eng1923
\subseccaption{23.}{In the time of War and Tumults.}
\drop{O almighty Lord, who art a most strong tower to all them that put their trust in thee: Be now and evermore our defence: look in pity upon the wounded and the prisoners; cheer the anxious; comfort the bereaved; succour the dying; have mercy on the fallen; and hasten the time when war shall cease in all the world; through Jesus Christ our Lord. \R Amen.}

%Eng1662
\subseccaption{24.}{In the time of Insurrection.}
\drop{O Almighty God, King of all kings, and Governor of all things, whose power no creature is able to resist, to whom it belongeth justly to punish sinners, and to be merciful to them that truly repent: save and deliver us, we humbly beseech thee from the hands of our enemies; abate their pride, asswage their malice, and confound their devices, that we being armed with thy defence, may be preserved evermore from all perils, to glorify thee who art the only giver of all victory through the merits of thy only son Jesus Christ our Lord. \R Amen.}

% or
% \drop{O Lord, who discoverest the snares of death that be laid for us, and dost wonderfully deliver us from the same; Be thou still our mighty Protector, and scatter our enemies that delight in blood: Infatuate and defeat their counsels, abate their pride, asswage their malice, and confound their devices. Strengthen us with judgment and justice, and cut off all such workers of iniquity, as turn Religion into Rebellion, and Faith into Faction; that they may never prevail against us, or triumph in the ruin of thy church among us: but that we, being preserved in thy true Religion, and by thy merciful goodness protected in the same, may all duly serve thee, and give thee thanks in thy holy Congregation, through Jesus Christ our Lord. \R Amen.}



%Am1928
\subseccaption{25.}{For Soldiers.}
\drop{O Lord God of Hosts, stretch forth, we pray thee, thine almighty arm to strengthen and protect the soldiers of our country; Support them in the day of battle, and in the time of peace keep them safe from all evil; endue them with courage and loyalty; and grant that in all things they may serve without reproach; through Jesus Christ our Lord. \R Amen.}


%Am1928
\subseccaption{26.}{Memorial Days.}
\drop{Almighty God, our heavenly Father, in whose hands are the living and the dead; We give thee thanks for all those thy servants who have laid down their lives in the service of our country. Grant to them thy mercy and the light of thy presence, that the good work which thou hast begun in them may be perfected; through Jesus Christ thy Son our Lord. \R Amen.}


%Sc1919
\subseccaption{27.}{In Commemoration of the Faithful Departed.}
\drop{O Almighty God, the God of the spirits of all flesh, who by a voice from heaven didst proclaim, Blessed are the dead who die in the Lord: Multiply, we beseech thee, to those who rest in Jesus, the manifold blessings of thy love, that the good work which thou didst begin in them may be perfected unto the day of Jesus Christ. And of thy mercy, O heavenly Father, vouchsafe that we, who now serve thee here on earth, may at the last, together with them, be found meet to be partakers of the inheritance of the saints in light; for the sake of the same thy Son, Jesus Christ, our Lord and Saviour.  \R Amen.}



\subseccaption{28.}{A Collect or Prayer for all Conditions of Men, to be used at such times when the Litany is not appointed to be said.}
\lettrine{O}{ God,} the Creator and Preserver of all mankind, we humbly beseech thee for all sorts and conditions of men: that thou wouldest be pleased to make thy ways known unto them, thy saving health unto all nations. More especially, we pray for the good estate of the Catholick Church; that it may be so guided and governed by thy good Spirit, that all who profess and call themselves Christians may be led into the way of truth, and hold the faith in unity of spirit, in the bond of peace,
and in righteousness of life. Finally, we commend to thy fatherly goodness all those, who are any ways afflicted, or distressed, in
% \begin{wrapfigure}{r}[0pt]{0.45\textwidth}{\footnotesize}\par}\end{wrapfigure}\noindent
mind, body, or estate; [\footnote{\rubric{This to be said when any desire the Prayers of the Congregation.}}\emph{especially those for whom our prayers are desired;}] that it may please thee to comfort and relieve them, according to their several necessities, giving them patience under their sufferings, and a happy issue out of all their afflictions. And this we beg for Jesus Christ his sake. \R Amen.
\bigskip

\section{Thanksgivings}
\subseccaption{1.}{A General Thanksgiving.}
\lettrine{A}{lmighty} God, Father of all mercies, we thine unworthy servants do give thee most humble and hearty thanks for all thy goodness and loving-kindness to us, and to all men;
% \begin{wrapfigure}{r}[0pt]{0.45\textwidth}{\footnotesize\rubric{* This is to be said when any that have been prayed for desire to return praise.}\par}\end{wrapfigure}\noindent
[\footnote{\rubric{This is to be said when any that have been prayed for desire to return praise.}}\emph{particularly to those who desire now to offer up their praises and thanksgivings for thy late mercies vouchsafed unto them.}] We bless thee for our creation, preservation, and all the blessings of this life; but above all, for thine inestimable love in the redemption of the world by our Lord Jesus Christ; for the means of grace, and for the hope of glory. And, we beseech thee, give us that due sense of all thy mercies, that our hearts may be unfeignedly thankful, and that we shew forth thy praise, not only with our lips, but in our lives; by giving up ourselves to thy service, and by walking before thee in holiness and righteousness all our days; through Jesus Christ our Lord, to whom with thee and the Holy Ghost be all honour and glory, world without end. \R Amen.


%Am1928
\subseccaption{2.}{For Rain}
\drop{O God our heavenly Father, who by thy gracious providence dost cause the former and the latter rain to descend upon the earth, that it may bring forth fruit for the use of man; We give thee humble thanks that it hath pleased thee to send us rain to our great comfort, and to the glory of thy holy Name; through thy mercies in Jesus Christ our Lord. \R Amen.}


%Eng1923
\subseccaption{3.}{For Seasonable Weather.}
\drop{O Lord God, who hast in thy mercy relieved and comforted thy servants by this seasonable change of weather: We yield thee hearty thanks for this thy goodness towards us, beseeching thee to give us grace to use all thy mercies to the honour and glory of thy holy Name; through Jesus Christ our Lord. \R Amen.}


%Eng1923
\subseccaption{4.}{For the Blessings of Harvest.}
\drop{O Lord God Almighty, the Creator and Father of all, we yield thee hearty thanks that thou hast ordained for mankind both seedtime and harvest, and dost now bestow upon us thy children the fruits of the earth in their season. For these and all other thy mercies we laud and magnify thy glorious Name; through Jesus Christ our Lord, to whom, with thee and the Holy Ghost, be all honour and glory, now and for evermore. \R Amen.}


%Eng1662
\subseccaption{5.}{For Plenty.}
\drop{O most merciful Father, who of thy gracious goodness hast heard the devout prayers of thy Church, and turned our dearth and scarcity into cheapness and plenty; We give thee humble thanks for this thy special bounty; beseeching thee to continue thy loving-kindness unto us, that our land may yield us her fruits of increase, to thy glory and our comfort; through Jesus Christ our Lord. \R Amen.}



%Eng1662, Am1928
\subseccaption{6.}{For Peace, and Deliverance from our Enemies.}
\drop{O Almighty God, who art a strong tower of defence unto thy servants against the face of their enemies; We yield thee praise and thanksgiving for our deliverance from those great and apparent dangers wherewith we were compassed. We acknowledge it thy goodness that we were not delivered over as a prey unto them; beseeching thee still to continue such thy mercies towards us, that all the world may know that thou art our Saviour and mighty Deliverer; through Jesus Christ our Lord. \R Amen.}


%Am1928, Eng1662
\subseccaption{7.}{For restoring Publick Peace at Home.}
\drop{O eternal God, our heavenly Father, Who alone makest men to be of one mind in a house, and stillest the outrage of a violent and unruly people; We bless thy holy Name, that it hath pleased thee to appease the seditious tumults which have been lately raised up amongst us; most humbly beseeching thee to grant to all of us grace, that we may henceforth obediently walk in thy holy commandments; and, leading a quiet and peaceable life in all godliness and honesty, may continually offer unto thee our sacrifice of praise and thanksgiving for these thy mercies towards us; through Jesus Christ our Lord. \R Amen.}


\subseccaption{8.}{For Deliverance from Common Sickness.}
\drop{O Lord God, who dost not willingly afflict the children of men: We most heartily thank thee that in thy mercy thou hast delivered us from sickness and affliction, and with grateful hearts we desire to offer unto thy fatherly goodness ourselves, our souls and bodies, to be a living sacrifice unto thee, always praising and magnifying thy loving-kindness in the midst of thy Church; through Jesus Christ our Lord. \R Amen.}

\section{Conclusions}

\subseccaption{1.}{2 Corinthians xiij.}
\theGrace

\subseccaption{2.}{after Numbers vj.}
\drop{The {\scshape Lord} \cross bless us, and keep us: the {\scshape Lord} make his face to shine upon us, and be gracious unto us: the {\scshape Lord} lift up the light of his countenance upon us, and give us peace, now and for evermore. \R Amen.}

\subseccaption{3.}{1 Timothy i.}
\newcommand\nowUntoTheKing{\drop{Now unto the King eternal, immortal, invisible, the only wise God, \cross be honour and glory for ever and ever. Amen.}}
\nowUntoTheKing

\subseccaption{4.}{At Night.}
\drop{The almighty and merciful God bless \cross us and keep us this night and for evermore. \R Amen.}


\subseccaption{5.}{For the Departed.}
\drop{May the souls \cross of the faithful, through the mercy of God, rest in peace. \R Amen.}

%Am1928; conclusion Sc1912
\section{A Bidding Prayer}
\subsubsection{To be used before Sermons, or on Special Occasions.}
\smallskip
\pilcrow{And NOTE, That the Minister, in his discretion, may omit any of the clauses in this Prayer, or may add others, as occasion may require.}

\drop{Good Christian People, I bid your prayers for Christ’s holy Catholic Church, the blessed company of all faithful people; that it may please God to confirm and strengthen it in purity of faith, in holiness of life, and in perfectness of love, and to restore to it the witness of visible unity; and more especially for that branch of the same planted by God in this land, whereof we are members; that in all things it may work according to God’s will, serve him faithfully, and worship him acceptably.}

Ye shall pray for the \emph{President of these United States,} and for the \emph{Governor of this State}, and for all that are in authority; that all, and every one of them, may serve truly in their several callings to the glory of God, and the edifying and well-governing of the people, remembering the account they shall be called upon to give at the last great day.

Ye shall also pray for the ministers of God’s Holy Word and Sacraments; for Bishops (and herein more especially for our Bishop), that they may minister faithfully and wisely the discipline of Christ; likewise for all Priests and Deacons, that they may shine as lights in the world, and in all things may adorn the doctrine of God our Saviour.

And ye shall pray for a due supply of persons fitted to serve God in the Ministry and in the State; and to that end, as well as for the good education of all the youth of this land, ye shall pray for all schools, colleges, and seminaries of sound and godly learning, and for all whose hands are open for their maintenance; that whatsoever tends to the advancement of true religion and useful learning may for ever flourish and abound.

Ye shall pray for all the people of \emph{these United States}, that they may live in the true faith and fear of God, and in brotherly charity one towards another.

Ye shall pray also for all who travel by land, sea, or air; for all prisoners and captives; for all who are in sickness or in sorrow; for all who have fallen into grievous sin; for all who, through temptation, ignorance, helplessness, grief, trouble, dread, or the near approach of death, especially need our prayers.

Ye shall also praise God for rain and sunshine; for the fruits of the earth; for the products of all honest industry; and for all his good gifts, temporal and spiritual, to us and to all men.

Finally, ye shall yield unto God most high praise and hearty thanks for the wonderful grace and virtue declared in all his saints, who have been the choice vessels of his grace and the lights of the world in their several generations; and pray unto God, that we may have grace to direct our lives after their good examples; that, this life ended, we may be made partakers with them of the glorious resurrection, and the life everlasting.

All which things let us humbly ask in the words which Christ himself hath taught us, saying:

\ourFather
 
\medskip

\centerline{\rubric{Or,}}

% Adapted from the "55th Canon"
\drop{Ye shall pray for Christ’s Holy Catholic Church, that is, for the whole congregation of Christian people dispersed throughout the whole world, and more especially for that branch of the same planted by God in this land.}

Ye shall also pray for the ministers of God’s Holy Word and Sacraments; for Bishops, and likewise for all Priests and Deacons.

Ye shall pray for the \emph{President of these United States,} and for the \emph{Governor of this State}, and for all that are in authority; that all, and every one of them, may serve truly in their several callings to the glory of God, and the edifying and well-governing of the people, remembering the account they shall be called upon to give at the last great day.

Also ye shall pray for all the people of \emph{these United States}, that they may live in the true faith and fear of God and in brotherly charity one to another.

Finally let us praise God for all those which are departed out of this life in the faith of Christ,  and pray unto God, that we may have grace to direct our lives after their good examples; that, this life ended, we may be made partakers with them of the glorious resurrection, and the life everlasting.

\centerline{Our Father, \etc}




\bigskip
This Litany from the 1929 Scottish book is, in turn, from (at least)
the Salisbury form of consecrating a church, by Bp. John Wordsworth where it begins "Stretch out thy hand" and ends before the collect.

Ectene

SHORTER LITANY

II

 
\centeredrubric{A short pause for silent prayer may be made after each response.}

\drop{Let us beseech the all-holy and ever-blessed Trinity to send forth mercy and grace upon us and upon all people.}

\smallskip
O God the Father, have mercy upon us:
    
\hspace{\parindent}\emph{O God the Father, have mercy upon us.}

\smallskip
O God the Son, have mercy upon us:

\hspace{\parindent}\emph{O God the Son, have mercy upon us.}

\smallskip
O God the Holy Ghost, have mercy upon us:

\hspace{\parindent}\emph{O God the Holy Ghost, have mercy upon us.}


\smallskip
Stretch out thy hand upon us, O Lord, and save us; raise us up and defend us.

\centerline{\emph{Lord, have mercy.}}

\smallskip
Let us pray for the peace that cometh from above, and for the salvation of our souls.

\centerline{\emph{Lord, have mercy.}}

\smallskip
Let us pray for the peace of the whole world, and for the welfare and unity of the Church of God.

\centerline{\emph{Lord, have mercy.}}

\smallskip
Let us pray for the conversion of those in unbelief and error.

\centerline{\emph{Lord, have mercy.}}

\smallskip
Let us pray for our country, for this place [\rubric{or} city], for this Diocese, and for all that dwell therein.

\centerline{\emph{Lord, have mercy.}}

\smallskip
Let us pray for all Christian people throughout the world.

\centerline{\emph{Lord, have mercy.}}

\smallskip
Let us pray for all %Christian princes and governors, especially our Sovereign Lady, Queen Elizabeth.
Christian Rulers and Magistrates, especially \emph{The President of the United States.}

\centerline{\emph{Lord, have mercy.}}

\smallskip
Let us pray for all in authority in this land, especially those in this county [\rubric{or} city \rubric{or} place].

\centerline{\emph{Lord, have mercy.}}

\smallskip
Let us pray for the Bishops and Clergy of Christ's Church, especially for \emph{N.} our Bishop.

\centerline{\emph{Lord, have mercy.}}

\smallskip
Let us pray for all voyagers and travellers.

\centerline{\emph{Lord, have mercy.}}

\smallskip
Let us pray for stedfastness in the faith for our brethren beyond the seas.

\centerline{\emph{Lord, have mercy.}}

\smallskip
Let us pray for all who are sick or suffering, in mind, body, or estate.

\smallskip
\centerline{\emph{Lord, have mercy.}}

Let us pray for a holy and happy death, for rest in paradise, and for the perfect vision of the glory of God.

\centerline{\emph{Lord, have mercy.}}

\smallskip
Let us pray that we may follow the blessed saints and martyrs in bearing our cross before the world.

\centerline{\emph{Lord, have mercy.}}

\smallskip
Let us pray for the faithful departed.

\centerline{\emph{Lord, have mercy.}}

\smallskip
\centeredrubric{Minister.}

\drop{O God of unchangeable power and eternal light, look favourably on thy whole Church, that wonderful and sacred mystery; and by the tranquil operation of thy perpetual providence carry out the work of man's salvation, and let the whole world feel and see that things which were cast down are being raised up, and things which had grown old are being made new, and all things are returning to perfection through him from whom they took their origin, even Jesus Christ our Lord. \R Amen.}

\smallskip
\centeredrubric{Instead of \emph{\black Lord, have mercy} may be said,}

\V Lord, hear our prayer; \R And let our cry come unto thee.


\fleuron

\chapter{The Collects, Epistles, and Gospels}
\subsubsection{To Be Used Throughout the Year.}

\medskip

\pilcrow{Note, that the Collect appointed for every Sunday, or for any Holy-day that hath a Vigil or Eve, shall be said at the Evening Service next before.}


\section{The First Sunday of Advent.}
\subsection{\stylesubsec{}{The Collect.}{}}
\drop{Almighty God, give us grace that we may cast away the works of darkness, and put upon us the armour of light, now in the time of this mortal life in which thy Son Jesus Christ came to visit us in great humility; that in the last day, when he shall come again in his glorious majesty to judge both the quick and the dead, we may rise to the life immortal; through him who liveth and reigneth with thee and the Holy Ghost, one God, now and for ever. \R Amen.}
\pilcrow{This Collect is to be repeated every day, with the other Collects in Advent, until Christmas-Eve.}

\subsection{\stylesubsec{}{The Epistle.}{Romans 13.~8.}}
\drop{Owe no man any thing, but to love one another: for he that loveth another hath fulfilled the law. For this, Thou shalt not commit adultery, Thou shalt not kill, Thou shalt not steal, Thou shalt not bear false witness, Thou shalt not covet; and if there be any other commandment, it is briefly comprehended in this saying, namely, Thou shalt love thy neighbour as thyself. Love worketh no ill to his neighbour: therefore love is the fulfilling of the law. And that, knowing the time, that now it is high time to awake out of sleep: for now is our salvation nearer than when we believed. The night is far spent, the day is at hand: let us therefore cast off the works of darkness, and let us put on the armour of light. Let us walk honestly, as in the day; not in rioting and drunkenness, not in chambering and wantonness, not in strife and envying. But put ye on the Lord Jesus Christ, and make not provision for the flesh, to fulfil the lusts thereof.}

\subsection{\stylesubsec{}{The Gospel.}{St.~Matthew 21.~1.}}
\drop{When they drew nigh unto Jerusalem, and were come to Bethphage, unto the mount of Olives, then sent Jesus two disciples, Saying unto them, Go into the village over against you, and straightway ye shall find an ass tied, and a colt with her: loose them, and bring them unto me. And if any man say ought unto you, ye shall say, The Lord hath need of them; and straightway he will send them. All this was done, that it might be fulfilled which was spoken by the prophet, saying, Tell ye the daughter of Sion, Behold, thy King cometh unto thee, meek, and sitting upon an ass, and a colt the foal of an ass. And the disciples went, and did as Jesus commanded them, And brought the ass, and the colt, and put on them their clothes, and they set him thereon. And a very great multitude spread their garments in the way; others cut down branches from the trees, and strawed them in the way. And the multitudes that went before, and that followed, cried, saying, Hosanna to the son of David: Blessed is he that cometh in the name of the Lord; Hosanna in the highest. And when he was come into Jerusalem, all the city was moved, saying, Who is this? And the multitude said, This is Jesus the prophet of Nazareth of Galilee. And Jesus went into the temple of God, and cast out all them that sold and bought in the temple, and overthrew the tables of the moneychangers, and the seats of them that sold doves, And said unto them, It is written, My house shall be called the house of prayer; but ye have made it a den of thieves.}

\section{The Second Sunday in Advent.}
\subsection{\stylesubsec{}{The Collect.}{}}
\drop{Blessed Lord, who hast caused all holy Scriptures to be written for our learning; Grant that we may in such wise hear them, read, mark, learn, and inwardly digest them, that by patience and comfort of thy holy Word, we may embrace, and ever hold fast, the blessed hope of everlasting life, which thou hast given us in our Saviour Jesus Christ. \R Amen.}
\pilcrow{The Collect from the First Sunday in Advent is to be repeated every day, with the other Collects in Advent, until Christmas-Eve.}

\subsection{\stylesubsec{}{The Epistle.}{Romans 15.~4.}}
\drop{Whatsoever things were written aforetime were written for our learning, that we through patience and comfort of the scriptures might have hope. Now the God of patience and consolation grant you to be likeminded one toward another according to Christ Jesus: That ye may with one mind and one mouth glorify God, even the Father of our Lord Jesus Christ. Wherefore receive ye one another, as Christ also received us to the glory of God. Now I say that Jesus Christ was a minister of the circumcision for the truth of God, to confirm the promises made unto the fathers: And that the Gentiles might glorify God for his mercy; as it is written, For this cause I will confess to thee among the Gentiles, and sing unto thy name. And again he saith, Rejoice, ye Gentiles, with his people. And again, Praise the Lord, all ye Gentiles; and laud him, all ye people. And again, Esaias saith, There shall be a root of Jesse, and he that shall rise to reign over the Gentiles; in him shall the Gentiles trust. Now the God of hope fill you with all joy and peace in believing, that ye may abound in hope, through the power of the Holy Ghost.}

\subsection{\stylesubsec{}{The Gospel.}{St.~Luke 21.~25.}}
\drop{And there shall be signs in the sun, and in the moon, and in the stars; and upon the earth distress of nations, with perplexity; the sea and the waves roaring; Men's hearts failing them for fear, and for looking after those things which are coming on the earth: for the powers of heaven shall be shaken. And then shall they see the Son of man coming in a cloud with power and great glory. And when these things begin to come to pass, then look up, and lift up your heads; for your redemption draweth nigh. And he spake to them a parable; Behold the fig tree, and all the trees; When they now shoot forth, ye see and know of your own selves that summer is now nigh at hand. So likewise ye, when ye see these things come to pass, know ye that the kingdom of God is nigh at hand. Verily I say unto you, This generation shall not pass away, till all be fulfilled. Heaven and earth shall pass away: but my words shall not pass away.}


\section{The Third Sunday in Advent.}
\subsection{\stylesubsec{}{The Collect.}{}}
\drop{O Lord Jesu Christ, who at thy first coming didst send thy messenger to prepare thy way before thee; Grant that the ministers and stewards of thy mysteries may likewise so prepare and make ready thy way, by turning the hearts of the disobedient to the wisdom of the just, that at thy second coming to judge the world we may be found an acceptable people in thy sight, who livest and reignest with the Father and the Holy Spirit ever, one God, world without end. \R Amen.}

\pilcrow{The Collect from the First Sunday in Advent is to be repeated every day, with the other Collects in Advent, until Christmas-Eve.}

\subsection{\stylesubsec{}{The Epistle.}{1 Corinthians 4.~1.}}
\drop{Let a man so account of us, as of the ministers of Christ, and stewards of the mysteries of God. Moreover it is required in stewards, that a man be found faithful. But with me it is a very small thing that I should be judged of you, or of man's judgment: yea, I judge not mine own self. For I know nothing by myself; yet am I not hereby justified: but he that judgeth me is the Lord. Therefore judge nothing before the time, until the Lord come, who both will bring to light the hidden things of darkness, and will make manifest the counsels of the hearts: and then shall every man have praise of God.}

\subsection{\stylesubsec{}{The Gospel.}{St.~Matthew 11.~2.}}
\drop{Now when John had heard in the prison the works of Christ, he sent two of his disciples, And said unto him, Art thou he that should come, or do we look for another? Jesus answered and said unto them, Go and shew John again those things which ye do hear and see: The blind receive their sight, and the lame walk, the lepers are cleansed, and the deaf hear, the dead are raised up, and the poor have the gospel preached to them. And blessed is he, whosoever shall not be offended in me. And as they departed, Jesus began to say unto the multitudes concerning John, What went ye out into the wilderness to see? A reed shaken with the wind? But what went ye out for to see? A man clothed in soft raiment? behold, they that wear soft clothing are in kings' houses. But what went ye out for to see? A prophet? yea, I say unto you, and more than a prophet. For this is he, of whom it is written, Behold, I send my messenger before thy face, which shall prepare thy way before thee.}


\section{The Fourth Sunday in Advent.}
\subsection{\stylesubsec{}{The Collect.}{}}
\drop{O Lord, raise up (we pray thee) thy power, and come among us, and with great might succour us; that whereas, through our sins and wickedness, we are sore let and hindered in running the race that is set before us, thy bountiful grace and mercy may speedily help and deliver us; through the satisfaction of thy Son our Lord, to whom with thee and the Holy Ghost be honour and glory, world without end. \R Amen.}

\pilcrow{The Collect from the First Sunday in Advent is to be repeated every day, with the other Collects in Advent, until Christmas-Eve.}

\subsection{\stylesubsec{}{The Epistle.}{Philippians 4.~4.}}
\drop{Rejoice in the Lord alway: and again I say, Rejoice. Let your moderation be known unto all men. The Lord is at hand. Be careful for nothing; but in every thing by prayer and supplication with thanksgiving let your requests be made known unto God. And the peace of God, which passeth all understanding, shall keep your hearts and minds through Christ Jesus.}

\subsection{\stylesubsec{}{The Gospel.}{St.~John 1.~19.}}
\drop{This is the record of John, when the Jews sent priests and Levites from Jerusalem to ask him, Who art thou? And he confessed, and denied not; but confessed, I am not the Christ. And they asked him, What then? Art thou Elias? And he saith, I am not. Art thou that prophet? And he answered, No. Then said they unto him, Who art thou? that we may give an answer to them that sent us. What sayest thou of thyself? He said, I am the voice of one crying in the wilderness, Make straight the way of the Lord, as said the prophet Esaias. And they which were sent were of the Pharisees. And they asked him, and said unto him, Why baptizest thou then, if thou be not that Christ, nor Elias, neither that prophet? John answered them, saying, I baptize with water: but there standeth one among you, whom ye know not; He it is, who coming after me is preferred before me, whose shoe's latchet I am not worthy to unloose. These things were done in Bethabara beyond Jordan, where John was baptizing.}

\stylesec{The Nativity of our Lord, or the Birth-day of Christ,}{commonly called}{Christmas-Day}
\subsection{\stylesubsec{}{}{[December 25]}}
\subsection{\stylesubsec{}{The Collect.}{}}
\drop{Almighty God, who hast given us thy only-begotten Son to take our nature upon him, and as at this time to be born of a pure Virgin; Grant that we being regenerate, and made thy children by adoption and grace, may daily be renewed by thy Holy Spirit; through the same our Lord Jesus Christ, who liveth and reigneth with thee and the same Spirit, ever one God, world without end. \R Amen.}

\subsection{\stylesubsec{}{The Epistle.}{Hebrews 1.~1.}}
\drop{God, who at sundry times and in divers manners spake in time past unto the fathers by the prophets, Hath in these last days spoken unto us by his Son, whom he hath appointed heir of all things, by whom also he made the worlds; Who being the brightness of his glory, and the express image of his person, and upholding all things by the word of his power, when he had by himself purged our sins, sat down on the right hand of the Majesty on high: Being made so much better than the angels, as he hath by inheritance obtained a more excellent name than they. For unto which of the angels said he at any time, Thou art my Son, this day have I begotten thee? And again, I will be to him a Father, and he shall be to me a Son? And again, when he bringeth in the firstbegotten into the world, he saith, And let all the angels of God worship him. And of the angels he saith, Who maketh his angels spirits, and his ministers a flame of fire. But unto the Son he saith, Thy throne, O God, is for ever and ever: a sceptre of righteousness is the sceptre of thy kingdom. Thou hast loved righteousness, and hated iniquity; therefore God, even thy God, hath anointed thee with the oil of gladness above thy fellows. And, Thou, Lord, in the beginning hast laid the foundation of the earth; and the heavens are the works of thine hands: They shall perish; but thou remainest; and they all shall wax old as doth a garment; And as a vesture shalt thou fold them up, and they shall be changed: but thou art the same, and thy years shall not fail.}

\subsection{\stylesubsec{}{The Gospel.}{St.~John 1.~1.}}
\drop{In the beginning was the Word, and the Word was with God, and the Word was God. The same was in the beginning with God. All things were made by him; and without him was not any thing made that was made. In him was life; and the life was the light of men. And the light shineth in darkness; and the darkness comprehended it not. There was a man sent from God, whose name was John. The same came for a witness, to bear witness of the Light, that all men through him might believe. He was not that Light, but was sent to bear witness of that Light. That was the true Light, which lighteth every man that cometh into the world. He was in the world, and the world was made by him, and the world knew him not. He came unto his own, and his own received him not. But as many as received him, to them gave he power to become the sons of God, even to them that believe on his name: Which were born, not of blood, nor of the will of the flesh, nor of the will of man, but of God. And the Word was made flesh, and dwelt among us, (and we beheld his glory, the glory as of the only begotten of the Father,) full of grace and truth.}


\section{Saint Stephen’s Day.}
\subsection{\stylesubsec{}{}{[December 26]}}
\subsection{\stylesubsec{}{The Collect.}{}}
\drop{Grant, O Lord, that, in all our sufferings here upon earth for the testimony of thy truth, we may stedfastly look up to heaven, and by faith behold the glory that shall be revealed; and, being filled with the Holy Ghost, may learn to love and bless our persecutors by the example of thy first Martyr Saint Stephen, who prayed for his murderers to thee, O blessed Jesus, who standest at the right hand of God to succour all those that suffer for thee, our only Mediator and Advocate. \R Amen.}
\pilcrow{Then shall follow the Collect of the Nativity, which shall be said continually unto New-year’s Eve.}

\subsection{\stylesubsec{}{For the Epistle.}{Acts 7.~55.}}
\drop{Stephen, being full of the Holy Ghost, looked up stedfastly into heaven, and saw the glory of God, and Jesus standing on the right hand of God, And said, Behold, I see the heavens opened, and the Son of man standing on the right hand of God. Then they cried out with a loud voice, and stopped their ears, and ran upon him with one accord, And cast him out of the city, and stoned him: and the witnesses laid down their clothes at a young man's feet, whose name was Saul. And they stoned Stephen, calling upon God, and saying, Lord Jesus, receive my spirit. And he kneeled down, and cried with a loud voice, Lord, lay not this sin to their charge. And when he had said this, he fell asleep.}

\subsection{\stylesubsec{}{The Gospel.}{St.~Matthew 23.~34.}}
\drop{Behold, I send unto you prophets, and wise men, and scribes: and some of them ye shall kill and crucify; and some of them shall ye scourge in your synagogues, and persecute them from city to city: That upon you may come all the righteous blood shed upon the earth, from the blood of righteous Abel unto the blood of Zacharias son of Barachias, whom ye slew between the temple and the altar. Verily I say unto you, All these things shall come upon this generation. O Jerusalem, Jerusalem, thou that killest the prophets, and stonest them which are sent unto thee, how often would I have gathered thy children together, even as a hen gathereth her chickens under her wings, and ye would not! Behold, your house is left unto you desolate. For I say unto you, Ye shall not see me henceforth, till ye shall say, Blessed is he that cometh in the name of the Lord.}


\section{Saint John the Evangelist’s Day.}
\subsection{\stylesubsec{}{}{[December 27]}}
\subsection{\stylesubsec{}{The Collect.}{}}
\drop{Merciful Lord, we beseech thee to cast thy bright beams of light upon thy Church, that it being enlightened by the doctrine of thy blessed Apostle and Evangelist Saint John may so walk in the light of thy truth, that it may at length attain to the light of everlasting life; through Jesus Christ our Lord. \R Amen.}

\pilcrow{Then shall follow the Collect of the Nativity, which shall be said continually unto New-year’s Eve.}

\subsection{\stylesubsec{}{The Epistle.}{1 St.~John 1.~1.}}
\drop{That which was from the beginning, which we have heard, which we have seen with our eyes, which we have looked upon, and our hands have handled, of the Word of life; (For the life was manifested, and we have seen it, and bear witness, and shew unto you that eternal life, which was with the Father, and was manifested unto us;) That which we have seen and heard declare we unto you, that ye also may have fellowship with us: and truly our fellowship is with the Father, and with his Son Jesus Christ. And these things write we unto you, that your joy may be full. This then is the message which we have heard of him, and declare unto you, that God is light, and in him is no darkness at all. If we say that we have fellowship with him, and walk in darkness, we lie, and do not the truth: But if we walk in the light, as he is in the light, we have fellowship one with another, and the blood of Jesus Christ his Son cleanseth us from all sin. If we say that we have no sin, we deceive ourselves, and the truth is not in us. If we confess our sins, he is faithful and just to forgive us our sins, and to cleanse us from all unrighteousness. If we say that we have not sinned, we make him a liar, and his word is not in us.}

\subsection{\stylesubsec{}{The Gospel.}{St.~John 21.~19.}}
\drop{Jesus said unto Peter, Follow me. Then Peter, turning about, seeth the disciple whom Jesus loved following; which also leaned on his breast at supper, and said, Lord, which is he that betrayeth thee? Peter seeing him saith to Jesus, Lord, and what shall this man do? Jesus saith unto him, If I will that he tarry till I come, what is that to thee? follow thou me. Then went this saying abroad among the brethren, that that disciple should not die: yet Jesus said not unto him, He shall not die; but, If I will that he tarry till I come, what is that to thee? This is the disciple which testifieth of these things, and wrote these things: and we know that his testimony is true. And there are also many other things which Jesus did, the which, if they should be written every one, I suppose that even the world itself could not contain the books that should be written.}


\section{The Innocents’ Day.}
\subsection{\stylesubsec{}{}{[December 28]}}
\subsection{\stylesubsec{}{The Collect.}{}}
\drop{O almighty God, who out of the mouths of babes and sucklings hast ordained strength, and madest infants to glorify thee by their deaths; Mortify and kill all vices in us, and so strengthen us by thy grace, that by the innocency of our lives, and constancy of our faith even unto death, we may glorify thy holy Name; through Jesus Christ our Lord. \R Amen.}

\pilcrow{Then shall follow the Collect of the Nativity, which shall be said continually unto New-year’s Eve.}

\subsection{\stylesubsec{}{For the Epistle.}{Revelation 14.~1.}}
\drop{And I looked, and, lo, a Lamb stood on the mount Sion, and with him an hundred forty and four thousand, having his Father's name written in their foreheads. And I heard a voice from heaven, as the voice of many waters, and as the voice of a great thunder: and I heard the voice of harpers harping with their harps: And they sung as it were a new song before the throne, and before the four beasts, and the elders: and no man could learn that song but the hundred and forty and four thousand, which were redeemed from the earth. These are they which were not defiled with women; for they are virgins. These are they which follow the Lamb whithersoever he goeth. These were redeemed from among men, being the firstfruits unto God and to the Lamb. And in their mouth was found no guile: for they are without fault before the throne of God.}

\subsection{\stylesubsec{}{The Gospel.}{St.~Matthew 2.~13.}}
\drop{The angel of the Lord appeareth to Joseph in a dream, saying, Arise, and take the young child and his mother, and flee into Egypt, and be thou there until I bring thee word: for Herod will seek the young child to destroy him. When he arose, he took the young child and his mother by night, and departed into Egypt: And was there until the death of Herod: that it might be fulfilled which was spoken of the Lord by the prophet, saying, Out of Egypt have I called my son. Then Herod, when he saw that he was mocked of the wise men, was exceeding wroth, and sent forth, and slew all the children that were in Bethlehem, and in all the coasts thereof, from two years old and under, according to the time which he had diligently enquired of the wise men. Then was fulfilled that which was spoken by Jeremy the prophet, saying, In Rama was there a voice heard, lamentation, and weeping, and great mourning, Rachel weeping for her children, and would not be comforted, because they are not.}


\section{The Sunday after Christmas-Day.}
\subsection{\stylesubsec{}{The Collect.}{}}
\drop{Almighty God, who hast given us thy only-begotten Son to take our nature upon him, and as at this time to be born of a pure Virgin; Grant that we being regenerate, and made thy children by adoption and grace, may daily be renewed by thy Holy Spirit; through the same our Lord Jesus Christ, who liveth and reigneth with thee and the same Spirit, one God, world without end. \R Amen.}

\subsection{\stylesubsec{}{The Epistle.}{Galations 4.~1.}}
\drop{Now I say, that the heir, as long as he is a child, differeth nothing from a servant, though he be lord of all; But is under tutors and governors until the time appointed of the father. Even so we, when we were children, were in bondage under the elements of the world: But when the fulness of the time was come, God sent forth his Son, made of a woman, made under the law, To redeem them that were under the law, that we might receive the adoption of sons. And because ye are sons, God hath sent forth the Spirit of his Son into your hearts, crying, Abba, Father. Wherefore thou art no more a servant, but a son; and if a son, then an heir of God through Christ.}

\subsection{\stylesubsec{}{The Gospel.}{St.~Matthew 1.~18.}}
\drop{Now the birth of Jesus Christ was on this wise: When as his mother Mary was espoused to Joseph, before they came together, she was found with child of the Holy Ghost. Then Joseph her husband, being a just man, and not willing to make her a publick example, was minded to put her away privily. But while he thought on these things, behold, the angel of the Lord appeared unto him in a dream, saying, Joseph, thou son of David, fear not to take unto thee Mary thy wife: for that which is conceived in her is of the Holy Ghost. And she shall bring forth a son, and thou shalt call his name JESUS: for he shall save his people from their sins. Now all this was done, that it might be fulfilled which was spoken of the Lord by the prophet, saying, Behold, a virgin shall be with child, and shall bring forth a son, and they shall call his name Emmanuel, which being interpreted is, God with us. Then Joseph being raised from sleep did as the angel of the Lord had bidden him, and took unto him his wife: And knew her not till she had brought forth her firstborn son: and he called his name JESUS.}


\section{The Circumcision of Christ.}
\subsection{\stylesubsec{}{The Collect.}{}}
\drop{Almighty God, who madest thy blessed Son to be circumcised, and obedient to the law for man; Grant us the true Circumcision of the Spirit; that, our hearts, and all our members, being mortified from all worldly and carnal lusts, we may in all things obey thy blessed will; through the same thy Son Jesus Christ our Lord. \R Amen.}

\subsection{\stylesubsec{}{The Epistle.}{Romans 4.~8.}}
\drop{Blessed is the man to whom the Lord will not impute sin. Cometh this blessedness then upon the circumcision only, or upon the uncircumcision also? for we say that faith was reckoned to Abraham for righteousness. How was it then reckoned? when he was in circumcision, or in uncircumcision? Not in circumcision, but in uncircumcision. And he received the sign of circumcision, a seal of the righteousness of the faith which he had yet being uncircumcised: that he might be the father of all them that believe, though they be not circumcised; that righteousness might be imputed unto them also: And the father of circumcision to them who are not of the circumcision only, but who also walk in the steps of that faith of our father Abraham, which he had being yet uncircumcised. For the promise, that he should be the heir of the world, was not to Abraham, or to his seed, through the law, but through the righteousness of faith. For if they which are of the law be heirs, faith is made void, and the promise made of none effect.}

\subsection{\stylesubsec{}{The Gospel.}{St.~Luke 2.~15.}}
\drop{And it came to pass, as the angels were gone away from them into heaven, the shepherds said one to another, Let us now go even unto Bethlehem, and see this thing which is come to pass, which the Lord hath made known unto us. And they came with haste, and found Mary, and Joseph, and the babe lying in a manger. And when they had seen it, they made known abroad the saying which was told them concerning this child. And all they that heard it wondered at those things which were told them by the shepherds. But Mary kept all these things, and pondered them in her heart. And the shepherds returned, glorifying and praising God for all the things that they had heard and seen, as it was told unto them. And when eight days were accomplished for the circumcising of the child, his name was called JESUS, which was so named of the angel before he was conceived in the womb.}
\pilcrow{The same Collect, Epistle, and Gospel shall serve for every day after unto the Epiphany.}


\section{The Epiphany}
\subsubsection{Or the Manifestation of Christ to the Gentiles.}
\subsection{\stylesubsec{}{}{[January 6]}}
\subsection{\stylesubsec{}{The Collect.}{}}
\drop{O God, who by the leading of a star didst manifest thy only-begotten Son to the Gentiles: Mercifully grant, that we, which know thee now by faith, may after this life have the fruition of thy glorious Godhead; through Jesus Christ our Lord. \R Amen.}

\subsection{\stylesubsec{}{The Epistle.}{Ephesians 3.~1.}}
\drop{For this cause I Paul, the prisoner of Jesus Christ for you Gentiles, If ye have heard of the dispensation of the grace of God which is given me to you-ward: How that by revelation he made known unto me the mystery; (as I wrote afore in few words, Whereby, when ye read, ye may understand my knowledge in the mystery of Christ) Which in other ages was not made known unto the sons of men, as it is now revealed unto his holy apostles and prophets by the Spirit; That the Gentiles should be fellowheirs, and of the same body, and partakers of his promise in Christ by the gospel: Whereof I was made a minister, according to the gift of the grace of God given unto me by the effectual working of his power. Unto me, who am less than the least of all saints, is this grace given, that I should preach among the Gentiles the unsearchable riches of Christ; And to make all men see what is the fellowship of the mystery, which from the beginning of the world hath been hid in God, who created all things by Jesus Christ: To the intent that now unto the principalities and powers in heavenly places might be known by the church the manifold wisdom of God, According to the eternal purpose which he purposed in Christ Jesus our Lord: In whom we have boldness and access with confidence by the faith of him.}

\subsection{\stylesubsec{}{The Gospel.}{St.~Matthew 2.~1.}}
\drop{When Jesus was born in Bethlehem of Judaea in the days of Herod the king, behold, there came wise men from the east to Jerusalem, Saying, Where is he that is born King of the Jews? for we have seen his star in the east, and are come to worship him. When Herod the king had heard these things, he was troubled, and all Jerusalem with him. And when he had gathered all the chief priests and scribes of the people together, he demanded of them where Christ should be born. And they said unto him, In Bethlehem of Judaea: for thus it is written by the prophet, And thou Bethlehem, in the land of Juda, art not the least among the princes of Juda: for out of thee shall come a Governor, that shall rule my people Israel. Then Herod, when he had privily called the wise men, enquired of them diligently what time the star appeared. And he sent them to Bethlehem, and said, Go and search diligently for the young child; and when ye have found him, bring me word again, that I may come and worship him also. When they had heard the king, they departed; and, lo, the star, which they saw in the east, went before them, till it came and stood over where the young child was. When they saw the star, they rejoiced with exceeding great joy. And when they were come into the house, they saw the young child with Mary his mother, and fell down, and worshipped him: and when they had opened their treasures, they presented unto him gifts; gold, and frankincense and myrrh. And being warned of God in a dream that they should not return to Herod, they departed into their own country another way.}


\section{The First Sunday after the Epiphany.}
\subsection{\stylesubsec{}{The Collect.}{}}
\drop{O Lord, we beseech thee mercifully to receive the prayers of thy people which call upon thee; and grant that they may both perceive and know what things they ought to do, and also may have grace and power faithfully to fulfil the same; through Jesus Christ our Lord. \R Amen.}

\subsection{\stylesubsec{}{The Epistle.}{Romans 12.~1.}}
\drop{I beseech you therefore, brethren, by the mercies of God, that ye present your bodies a living sacrifice, holy, acceptable unto God, which is your reasonable service. And be not conformed to this world: but be ye transformed by the renewing of your mind, that ye may prove what is that good, and acceptable, and perfect, will of God. For I say, through the grace given unto me, to every man that is among you, not to think of himself more highly than he ought to think; but to think soberly, according as God hath dealt to every man the measure of faith. For as we have many members in one body, and all members have not the same office: So we, being many, are one body in Christ, and every one members one of another.}

\subsection{\stylesubsec{}{The Gospel.}{St.~Luke 2.~41.}}
\drop{Now his parents went to Jerusalem every year at the feast of the passover. And when he was twelve years old, they went up to Jerusalem after the custom of the feast. And when they had fulfilled the days, as they returned, the child Jesus tarried behind in Jerusalem; and Joseph and his mother knew not of it. But they, supposing him to have been in the company, went a day's journey; and they sought him among their kinsfolk and acquaintance. And when they found him not, they turned back again to Jerusalem, seeking him. And it came to pass, that after three days they found him in the temple, sitting in the midst of the doctors, both hearing them, and asking them questions. And all that heard him were astonished at his understanding and answers. And when they saw him, they were amazed: and his mother said unto him, Son, why hast thou thus dealt with us? behold, thy father and I have sought thee sorrowing. And he said unto them, How is it that ye sought me? wist ye not that I must be about my Father's business? And they understood not the saying which he spake unto them. And he went down with them, and came to Nazareth, and was subject unto them: but his mother kept all these sayings in her heart. And Jesus increased in wisdom and stature, and in favour with God and man.}


\section{The Second Sunday after the Epiphany.}
\subsection{\stylesubsec{}{The Collect.}{}}
\drop{Almighty and everlasting God, who dost govern all things in heaven and earth; Mercifully hear the supplications of thy people, and grant us thy peace all the days of our life; through Jesus Christ our Lord. \R Amen.}

\subsection{\stylesubsec{}{The Epistle.}{Romans 12.~6.}}
\drop{Having then gifts differing according to the grace that is given to us, whether prophecy, let us prophesy according to the proportion of faith; Or ministry, let us wait on our ministering: or he that teacheth, on teaching; Or he that exhorteth, on exhortation: he that giveth, let him do it with simplicity; he that ruleth, with diligence; he that sheweth mercy, with cheerfulness. Let love be without dissimulation. Abhor that which is evil; cleave to that which is good. Be kindly affectioned one to another with brotherly love; in honour preferring one another; Not slothful in business; fervent in spirit; serving the Lord; Rejoicing in hope; patient in tribulation; continuing instant in prayer; Distributing to the necessity of saints; given to hospitality. Bless them which persecute you: bless, and curse not. Rejoice with them that do rejoice, and weep with them that weep. Be of the same mind one toward another. Mind not high things, but condescend to men of low estate.}

\subsection{\stylesubsec{}{The Gospel.}{St.~John 2.~1.}}
\drop{And the third day there was a marriage in Cana of Galilee; and the mother of Jesus was there: And both Jesus was called, and his disciples, to the marriage. And when they wanted wine, the mother of Jesus saith unto him, They have no wine. Jesus saith unto her, Woman, what have I to do with thee? mine hour is not yet come. His mother saith unto the servants, Whatsoever he saith unto you, do it. And there were set there six waterpots of stone, after the manner of the purifying of the Jews, containing two or three firkins apiece. Jesus saith unto them, Fill the waterpots with water. And they filled them up to the brim. And he saith unto them, Draw out now, and bear unto the governor of the feast. And they bare it. When the ruler of the feast had tasted the water that was made wine, and knew not whence it was: (but the servants which drew the water knew;) the governor of the feast called the bridegroom, And saith unto him, Every man at the beginning doth set forth good wine; and when men have well drunk, then that which is worse: but thou hast kept the good wine until now. This beginning of miracles did Jesus in Cana of Galilee, and manifested forth his glory; and his disciples believed on him.}


\section{The Third Sunday after the Epiphany.}
\subsection{\stylesubsec{}{The Collect.}{}}
\drop{Almighty and everlasting God, mercifully look upon our infirmities, and in all our dangers and necessities stretch forth thy right hand to help and defend us; through Jesus Christ our Lord. \R Amen.}

\subsection{\stylesubsec{}{The Epistle.}{Romans 12.~16.}}
\drop{Be not wise in your own conceits. Recompense to no man evil for evil. Provide things honest in the sight of all men. If it be possible, as much as lieth in you, live peaceably with all men. Dearly beloved, avenge not yourselves, but rather give place unto wrath: for it is written, Vengeance is mine; I will repay, saith the Lord. Therefore if thine enemy hunger, feed him; if he thirst, give him drink: for in so doing thou shalt heap coals of fire on his head. Be not overcome of evil, but overcome evil with good.}

\subsection{\stylesubsec{}{The Gospel.}{St.~Matthew 8.~1.}}
\drop{When he was come down from the mountain, great multitudes followed him. And, behold, there came a leper and worshipped him, saying, Lord, if thou wilt, thou canst make me clean. And Jesus put forth his hand, and touched him, saying, I will; be thou clean. And immediately his leprosy was cleansed. And Jesus saith unto him, See thou tell no man; but go thy way, shew thyself to the priest, and offer the gift that Moses commanded, for a testimony unto them. And when Jesus was entered into Capernaum, there came unto him a centurion, beseeching him, And saying, Lord, my servant lieth at home sick of the palsy, grievously tormented. And Jesus saith unto him, I will come and heal him. The centurion answered and said, Lord, I am not worthy that thou shouldest come under my roof: but speak the word only, and my servant shall be healed. For I am a man under authority, having soldiers under me: and I say to this man, Go, and he goeth; and to another, Come, and he cometh; and to my servant, Do this, and he doeth it. When Jesus heard it, he marvelled, and said to them that followed, Verily I say unto you, I have not found so great faith, no, not in Israel. And I say unto you, That many shall come from the east and west, and shall sit down with Abraham, and Isaac, and Jacob, in the kingdom of heaven. But the children of the kingdom shall be cast out into outer darkness: there shall be weeping and gnashing of teeth. And Jesus said unto the centurion, Go thy way; and as thou hast believed, so be it done unto thee. And his servant was healed in the selfsame hour.}


\section{The Fourth Sunday after the Epiphany.}
\subsection{\stylesubsec{}{The Collect.}{}}
\drop{O God, who knowest us to be set in the midst of so many and great dangers, that by reason of the frailty of our nature we cannot always stand upright; Grant to us such strength and protection, as may support us in all dangers, and carry us through all temptations; through Jesus Christ our Lord. \R Amen.}

\subsection{\stylesubsec{}{The Epistle.}{Romans 13.~1.}}
\drop{Let every soul be subject unto the higher powers. For there is no power but of God: the powers that be are ordained of God. Whosoever therefore resisteth the power, resisteth the ordinance of God: and they that resist shall receive to themselves damnation. For rulers are not a terror to good works, but to the evil. Wilt thou then not be afraid of the power? do that which is good, and thou shalt have praise of the same: For he is the minister of God to thee for good. But if thou do that which is evil, be afraid; for he beareth not the sword in vain: for he is the minister of God, a revenger to execute wrath upon him that doeth evil. Wherefore ye must needs be subject, not only for wrath, but also for conscience sake. For for this cause pay ye tribute also: for they are God's ministers, attending continually upon this very thing. Render therefore to all their dues: tribute to whom tribute is due; custom to whom custom; fear to whom fear; honour to whom honour.}

\subsection{\stylesubsec{}{The Gospel.}{St.~Matthew 8.~23.}}
\drop{And when he was entered into a ship, his disciples followed him. And, behold, there arose a great tempest in the sea, insomuch that the ship was covered with the waves: but he was asleep. And his disciples came to him, and awoke him, saying, Lord, save us: we perish. And he saith unto them, Why are ye fearful, O ye of little faith? Then he arose, and rebuked the winds and the sea; and there was a great calm. But the men marvelled, saying, What manner of man is this, that even the winds and the sea obey him! And when he was come to the other side into the country of the Gergesenes, there met him two possessed with devils, coming out of the tombs, exceeding fierce, so that no man might pass by that way. And, behold, they cried out, saying, What have we to do with thee, Jesus, thou Son of God? art thou come hither to torment us before the time? And there was a good way off from them an herd of many swine feeding. So the devils besought him, saying, If thou cast us out, suffer us to go away into the herd of swine. And he said unto them, Go. And when they were come out, they went into the herd of swine: and, behold, the whole herd of swine ran violently down a steep place into the sea, and perished in the waters. And they that kept them fled, and went their ways into the city, and told every thing, and what was befallen to the possessed of the devils. And, behold, the whole city came out to meet Jesus: and when they saw him, they besought him that he would depart out of their coasts.}


\section{The Fifth Sunday after the Epiphany.}
\subsection{\stylesubsec{}{The Collect.}{}}
\drop{O Lord, we bessech thee to keep thy Church and household continually in thy true religion; that they who do lean only upon the hope of thy heavenly grace may evermore be defended by thy mighty power; through Jesus Christ our Lord. \R Amen.}

\subsection{\stylesubsec{}{The Epistle.}{Colossians 3.~12.}}
\drop{Put on therefore, as the elect of God, holy and beloved, bowels of mercies, kindness, humbleness of mind, meekness, longsuffering; Forbearing one another, and forgiving one another, if any man have a quarrel against any: even as Christ forgave you, so also do ye. And above all these things put on charity, which is the bond of perfectness. And let the peace of God rule in your hearts, to the which also ye are called in one body; and be ye thankful. Let the word of Christ dwell in you richly in all wisdom; teaching and admonishing one another in psalms and hymns and spiritual songs, singing with grace in your hearts to the Lord. And whatsoever ye do in word or deed, do all in the name of the Lord Jesus, giving thanks to God and the Father by him.}

\subsection{\stylesubsec{}{The Gospel.}{St.~Matthew 13.~24.}}
\drop{The kingdom of heaven is likened unto a man which sowed good seed in his field: But while men slept, his enemy came and sowed tares among the wheat, and went his way. But when the blade was sprung up, and brought forth fruit, then appeared the tares also. So the servants of the householder came and said unto him, Sir, didst not thou sow good seed in thy field? from whence then hath it tares? He said unto them, An enemy hath done this. The servants said unto him, Wilt thou then that we go and gather them up? But he said, Nay; lest while ye gather up the tares, ye root up also the wheat with them. Let both grow together until the harvest: and in the time of harvest I will say to the reapers, Gather ye together first the tares, and bind them in bundles to burn them: but gather the wheat into my barn.}


\section{The Sixth Sunday after the Epiphany.}
\subsection{\stylesubsec{}{The Collect.}{}}
\drop{O God, whose blessed Son was manifested that he might destroy the works of the devil, and make us the sons of God, and heirs of eternal life; Grant us, we beseech thee, that, having this hope, we may purify ourselves, even as he is pure; that, when he shall appear again with power and great glory, we may be made like unto him in his eternal and glorious kingdom; where with thee, O Father, and thee, O Holy Ghost, he liveth and reigneth, ever one God, world without end. \R Amen.}

\subsection{\stylesubsec{}{The Epistle.}{1 St.~John 3.~1.}}
\drop{Behold, what manner of love the Father hath bestowed upon us, that we should be called the sons of God: therefore the world knoweth us not, because it knew him not. Beloved, now are we the sons of God, and it doth not yet appear what we shall be: but we know that, when he shall appear, we shall be like him; for we shall see him as he is. And every man that hath this hope in him purifieth himself, even as he is pure. Whosoever committeth sin transgresseth also the law: for sin is the transgression of the law. And ye know that he was manifested to take away our sins; and in him is no sin. Whosoever abideth in him sinneth not: whosoever sinneth hath not seen him, neither known him. Little children, let no man deceive you: he that doeth righteousness is righteous, even as he is righteous. He that committeth sin is of the devil; for the devil sinneth from the beginning. For this purpose the Son of God was manifested, that he might destroy the works of the devil.}

\subsection{\stylesubsec{}{The Gospel.}{St.~Matthew 24.~23.}}
\drop{Then if any man shall say unto you, Lo, here is Christ, or there; believe it not. For there shall arise false Christs, and false prophets, and shall shew great signs and wonders; insomuch that, if it were possible, they shall deceive the very elect. Behold, I have told you before. Wherefore if they shall say unto you, Behold, he is in the desert; go not forth: behold, he is in the secret chambers; believe it not. For as the lightning cometh out of the east, and shineth even unto the west; so shall also the coming of the Son of man be. For wheresoever the carcase is, there will the eagles be gathered together. Immediately after the tribulation of those days shall the sun be darkened, and the moon shall not give her light, and the stars shall fall from heaven, and the powers of the heavens shall be shaken: And then shall appear the sign of the Son of man in heaven: and then shall all the tribes of the earth mourn, and they shall see the Son of man coming in the clouds of heaven with power and great glory. And he shall send his angels with a great sound of a trumpet, and they shall gather together his elect from the four winds, from one end of heaven to the other.}

\subsubsection{The Sunday called}
\section{Septuagesima}
\subsubsection{or the third Sunday before Lent.}

\subsection{\stylesubsec{}{The Collect.}{}}
\drop{O Lord, we beseech thee favourably to hear the prayers of thy people; that we, who are justly punished for our offences, may be mercifully delivered by thy goodness, for the glory of thy Name; through Jesus Christ our Saviour, who liveth and reigneth with thee and the Holy Ghost, ever one God, world without end. \R Amen.}

\subsection{\stylesubsec{}{The Epistle.}{1 Corinthians 9.~24.}}
\drop{Know ye not that they which run in a race run all, but one receiveth the prize? So run, that ye may obtain. And every man that striveth for the mastery is temperate in all things. Now they do it to obtain a corruptible crown; but we an incorruptible. I therefore so run, not as uncertainly; so fight I, not as one that beateth the air: But I keep under my body, and bring it into subjection: lest that by any means, when I have preached to others, I myself should be a cast-away.}

\subsection{\stylesubsec{}{The Gospel.}{St.~Matthew 20.~1.}}
\drop{The kingdom of heaven is like unto a man that is an householder, which went out early in the morning to hire labourers into his vineyard. And when he had agreed with the labourers for a penny a day, he sent them into his vineyard. And he went out about the third hour, and saw others standing idle in the marketplace, And said unto them; Go ye also into the vineyard, and whatsoever is right I will give you. And they went their way. Again he went out about the sixth and ninth hour, and did likewise. And about the eleventh hour he went out, and found others standing idle, and saith unto them, Why stand ye here all the day idle? They say unto him, Because no man hath hired us. He saith unto them, Go ye also into the vineyard; and whatsoever is right, that shall ye receive. So when even was come, the lord of the vineyard saith unto his steward, Call the labourers, and give them their hire, beginning from the last unto the first. And when they came that were hired about the eleventh hour, they received every man a penny. But when the first came, they supposed that they should have received more; and they likewise received every man a penny. And when they had received it, they murmured against the goodman of the house, Saying, These last have wrought but one hour, and thou hast made them equal unto us, which have borne the burden and heat of the day. But he answered one of them, and said, Friend, I do thee no wrong: didst not thou agree with me for a penny? Take that thine is, and go thy way: I will give unto this last, even as unto thee. Is it not lawful for me to do what I will with mine own? Is thine eye evil, because I am good? So the last shall be first, and the first last: for many be called, but few chosen.}

\subsubsection{The Sunday called}
\section{Sexagesima}
\subsubsection{or the second Sunday before Lent.}

\subsection{\stylesubsec{}{The Collect.}{}}
\drop{O Lord God, who seest that we put not our trust in any thing that we do; Mercifully grant that by thy power we may be defended against all adversity; through Jesus Christ our Lord. \R Amen.}

\subsection{\stylesubsec{}{The Epistle.}{2 Corinthians 11.~19.}}
\drop{Ye suffer fools gladly, seeing ye yourselves are wise. For ye suffer, if a man bring you into bondage, if a man devour you, if a man take of you, if a man exalt himself, if a man smite you on the face. I speak as concerning reproach, as though we had been weak. Howbeit whereinsoever any is bold, (I speak foolishly,) I am bold also. Are they Hebrews? so am I. Are they Israelites? so am I. Are they the seed of Abraham? so am I. Are they ministers of Christ? (I speak as a fool) I am more; in labours more abundant, in stripes above measure, in prisons more frequent, in deaths oft. Of the Jews five times received I forty stripes save one. Thrice was I beaten with rods, once was I stoned, thrice I suffered shipwreck, a night and a day I have been in the deep; In journeyings often, in perils of waters, in perils of robbers, in perils by mine own countrymen, in perils by the heathen, in perils in the city, in perils in the wilderness, in perils in the sea, in perils among false brethren; In weariness and painfulness, in watchings often, in hunger and thirst, in fastings often, in cold and nakedness. Beside those things that are without, that which cometh upon me daily, the care of all the churches. Who is weak, and I am not weak? who is offended, and I burn not? If I must needs glory, I will glory of the things which concern mine infirmities. The God and Father of our Lord Jesus Christ, which is blessed for evermore, knoweth that I lie not.}

\subsection{\stylesubsec{}{The Gospel.}{St.~Luke 8.~4.}}
\drop{When much people were gathered together, and were come to him out of every city, he spake by a parable: A sower went out to sow his seed: and as he sowed, some fell by the way side; and it was trodden down, and the fowls of the air devoured it. And some fell upon a rock; and as soon as it was sprung up, it withered away, because it lacked moisture. And some fell among thorns; and the thorns sprang up with it, and choked it. And other fell on good ground, and sprang up, and bare fruit an hundredfold. And when he had said these things, he cried, He that hath ears to hear, let him hear. And his disciples asked him, saying, What might this parable be? And he said, Unto you it is given to know the mysteries of the kingdom of God: but to others in parables; that seeing they might not see, and hearing they might not understand. Now the parable is this: The seed is the word of God. Those by the way side are they that hear; then cometh the devil, and taketh away the word out of their hearts, lest they should believe and be saved. They on the rock are they, which, when they hear, receive the word with joy; and these have no root, which for a while believe, and in time of temptation fall away. And that which fell among thorns are they, which, when they have heard, go forth, and are choked with cares and riches and pleasures of this life, and bring no fruit to perfection. But that on the good ground are they, which in an honest and good heart, having heard the word, keep it, and bring forth fruit with patience.}

\subsubsection{The Sunday called}
\section{Quinquagesima}
\subsubsection{or the next Sunday before Lent.}

\subsection{\stylesubsec{}{The Collect.}{}}
\drop{O Lord, who hast taught us that all our doings without charity are nothing worth; Send thy Holy Ghost and pour into our hearts that most excellent gift of charity, the very bond of peace and of all virtues, without which whosoever liveth is counted dead before thee; Grant this for thine only Son Jesus Christ's sake. \R Amen.}

\subsection{\stylesubsec{}{The Epistle.}{1 Corinthians 13.~1.}}
\drop{Though I speak with the tongues of men and of angels, and have not charity, I am become as sounding brass, or a tinkling cymbal. And though I have the gift of prophecy, and understand all mysteries, and all knowledge; and though I have all faith, so that I could remove mountains, and have not charity, I am nothing. And though I bestow all my goods to feed the poor, and though I give my body to be burned, and have not charity, it profiteth me nothing. Charity suffereth long, and is kind; charity envieth not; charity vaunteth not itself, is not puffed up, Doth not behave itself unseemly, seeketh not her own, is not easily provoked, thinketh no evil; Rejoiceth not in iniquity, but rejoiceth in the truth; Beareth all things, believeth all things, hopeth all things, endureth all things. Charity never faileth: but whether there be prophecies, they shall fail; whether there be tongues, they shall cease; whether there be knowledge, it shall vanish away. For we know in part, and we prophesy in part. But when that which is perfect is come, then that which is in part shall be done away. When I was a child, I spake as a child, I understood as a child, I thought as a child: but when I became a man, I put away childish things. For now we see through a glass, darkly; but then face to face: now I know in part; but then shall I know even as also I am known. And now abideth faith, hope, charity, these three; but the greatest of these is charity.}

\subsection{\stylesubsec{}{The Gospel.}{St.~Luke 18.~31.}}
\drop{Then Jesus took unto him the twelve, and said unto them, Behold, we go up to Jerusalem, and all things that are written by the prophets concerning the Son of man shall be accomplished. For he shall be delivered unto the Gentiles, and shall be mocked, and spitefully entreated, and spitted on: And they shall scourge him, and put him to death: and the third day he shall rise again. And they understood none of these things: and this saying was hid from them, neither knew they the things which were spoken. And it came to pass, that as he was come nigh unto Jericho, a certain blind man sat by the way side begging: And hearing the multitude pass by, he asked what it meant. And they told him, that Jesus of Nazareth passeth by. And he cried, saying, Jesus, thou son of David, have mercy on me. And they which went before rebuked him, that he should hold his peace: but he cried so much the more, Thou son of David, have mercy on me. And Jesus stood, and commanded him to be brought unto him: and when he was come near, he asked him, Saying, What wilt thou that I shall do unto thee? And he said, Lord, that I may receive my sight. And Jesus said unto him, Receive thy sight: thy faith hath saved thee. And immediately he received his sight, and followed him, glorifying God: and all the people, when they saw it, gave praise unto God.}


\subsubsection{The First day of Lent, commonly called}
\section{Ash Wednesday}
\subsection{\stylesubsec{}{The Collect.}{}}
\drop{Almighty and everlasting God, who hatest nothing that thou hast made and dost forgive the sins of all them that are penitent; Create and make in us new and contrite hearts, that we, worthily lamenting our sins, and acknowledging our wretchedness, may obtain of thee, the God of all mercy, perfect remission and forgiveness; through Jesus Christ our Lord. \R Amen.}

\pilcrow{This Collect is to be read every day in Lent after the Collect appointed for the Day.}

\subsection{\stylesubsec{}{For the Epistle.}{Joel 2.~12.}}
\drop{Turn ye even to me with all your heart, and with fasting, and with weeping, and with mourning: And rend your heart, and not your garments, and turn unto the Lord your God: for he is gracious and merciful, slow to anger, and of great kindness, and repenteth him of the evil. Who knoweth if he will return and repent, and leave a blessing behind him; even a meat offering and a drink offering unto the Lord your God? Blow the trumpet in Zion, sanctify a fast, call a solemn assembly: Gather the people, sanctify the congregation, assemble the elders, gather the children, and those that suck the breasts: let the bridegroom go forth of his chamber, and the bride out of her closet. Let the priests, the ministers of the Lord, weep between the porch and the altar, and let them say, Spare thy people, O Lord, and give not thine heritage to reproach, that the heathen should rule over them: wherefore should they say among the people, Where is their God?}


\subsection{\stylesubsec{}{The Gospel.}{St.~Matthew 6.~16.}}
\drop{When ye fast, be not, as the hypocrites, of a sad countenance: for they disfigure their faces, that they may appear unto men to fast. Verily I say unto you, They have their reward. But thou, when thou fastest, anoint thine head, and wash thy face; That thou appear not unto men to fast, but unto thy Father which is in secret: and thy Father, which seeth in secret, shall reward thee openly. Lay not up for yourselves treasures upon earth, where moth and rust doth corrupt, and where thieves break through and steal: But lay up for yourselves treasures in heaven, where neither moth nor rust doth corrupt, and where thieves do not break through nor steal: For where your treasure is, there will your heart be also.}

\section{The First Sunday in Lent}
\subsection{\stylesubsec{}{The Collect.}{}}
\drop{O Lord, who for our sake didst fast forty days and forty nights; Give us grace to use such abstinence, that, our flesh being subdued to the Spirit, we may ever obey thy godly motions in righteousness, and true holiness, to thy honour and glory, who livest and reignest with the Father and the Holy Ghost, one God, world without end. \R Amen.}

\pilcrow{The Collect from the First Day of Lent is to be read every day in Lent after the Collect appointed for the Day.}

\subsection{\stylesubsec{}{The Epistle.}{2 Corinthians 6.~1.}}
\drop{We then, as workers together with him, beseech you also that ye receive not the grace of God in vain. (For he saith, I have heard thee in a time accepted, and in the day of salvation have I succoured thee: behold, now is the accepted time; behold, now is the day of salvation.) Giving no offence in any thing, that the ministry be not blamed: But in all things approving ourselves as the ministers of God, in much patience, in afflictions, in necessities, in distresses, In stripes, in imprisonments, in tumults, in labours, in watchings, in fastings; By pureness, by knowledge, by long suffering, by kindness, by the Holy Ghost, by love unfeigned, By the word of truth, by the power of God, by the armour of righteousness on the right hand and on the left, By honour and dishonour, by evil report and good report: as deceivers, and yet true; As unknown, and yet well known; as dying, and, behold, we live; as chastened, and not killed; As sorrowful, yet alway rejoicing; as poor, yet making many rich; as having nothing, and yet possessing all things.}


\subsection{\stylesubsec{}{The Gospel.}{St.~Matthew 4.~1.}}
\drop{Then was Jesus led up of the spirit into the wilderness to be tempted of the devil. And when he had fasted forty days and forty nights, he was afterward an hungered. And when the tempter came to him, he said, If thou be the Son of God, command that these stones be made bread. But he answered and said, It is written, Man shall not live by bread alone, but by every word that proceedeth out of the mouth of God. Then the devil taketh him up into the holy city, and setteth him on a pinnacle of the temple, And saith unto him, If thou be the Son of God, cast thyself down: for it is written, He shall give his angels charge concerning thee: and in their hands they shall bear thee up, lest at any time thou dash thy foot against a stone. Jesus said unto him, It is written again, Thou shalt not tempt the Lord thy God. Again, the devil taketh him up into an exceeding high mountain, and sheweth him all the kingdoms of the world, and the glory of them; And saith unto him, All these things will I give thee, if thou wilt fall down and worship me. Then saith Jesus unto him, Get thee hence, Satan: for it is written, Thou shalt worship the Lord thy God, and him only shalt thou serve. Then the devil leaveth him, and, behold, angels came and ministered unto him.}

\section{The Second Sunday in Lent}
\subsection{\stylesubsec{}{The Collect.}{}}
\drop{Almighty God, who seest that we have no power of ourselves to help ourselves; Keep us both outwardly in our bodies, and inwardly in our souls; that we may be defended from all adversities which may happen to the body, and from all evil thoughts which may assault and hurt the soul; through Jesus Christ our Lord. \R Amen.}

\pilcrow{The Collect from the First Day of Lent is to be read every day in Lent after the Collect appointed for the Day.}

\subsection{\stylesubsec{}{The Epistle.}{1 Thessalonians 4.~1.}}
\drop{We beseech you, brethren, and exhort you by the Lord Jesus, that as ye have received of us how ye ought to walk and to please God, so ye would abound more and more. For ye know what commandments we gave you by the Lord Jesus. For this is the will of God, even your sanctification, that ye should abstain from fornication: That every one of you should know how to possess his vessel in sanctification and honour; Not in the lust of concupiscence, even as the Gentiles which know not God: That no man go beyond and defraud his brother in any matter: because that the Lord is the avenger of all such, as we also have forewarned you and testified. For God hath not called us unto uncleanness, but unto holiness. He therefore that despiseth, despiseth not man, but God, who hath also given unto us his Holy Spirit.}


\subsection{\stylesubsec{}{The Gospel.}{St.~Matthew 15.~21.}}
\drop{Jesus went thence, and departed into the coasts of Tyre and Sidon. And, behold, a woman of Canaan came out of the same coasts, and cried unto him, saying, Have mercy on me, O Lord, thou son of David; my daughter is grievously vexed with a devil. But he answered her not a word. And his disciples came and besought him, saying, Send her away; for she crieth after us. But he answered and said, I am not sent but unto the lost sheep of the house of Israel. Then came she and worshipped him, saying, Lord, help me. But he answered and said, It is not meet to take the children's bread, and to cast it to dogs. And she said, Truth, Lord: yet the dogs eat of the crumbs which fall from their masters' table. Then Jesus answered and said unto her, O woman, great is thy faith: be it unto thee even as thou wilt. And her daughter was made whole from that very hour.}

\section{The Third Sunday in Lent}
\subsection{\stylesubsec{}{The Collect.}{}}
\drop{We beseech thee, Almighty God, look upon the hearty desires of thy humble servants, and stretch forth the right hand of thy Majesty, to be our defence against all our enemies; through Jesus Christ our Lord. \R Amen.}

\pilcrow{The Collect from the First Day of Lent is to be read every day in Lent after the Collect appointed for the Day.}

\subsection{\stylesubsec{}{The Epistle.}{Ephesians 5.~1.}}
\drop{Be ye therefore followers of God, as dear children; And walk in love, as Christ also hath loved us, and hath given himself for us an offering and a sacrifice to God for a sweetsmelling savour. But fornication, and all uncleanness, or covetousness, let it not be once named among you, as becometh saints; Neither filthiness, nor foolish talking, nor jesting, which are not convenient: but rather giving of thanks. For this ye know, that no whoremonger, nor unclean person, nor covetous man, who is an idolater, hath any inheritance in the kingdom of Christ and of God. Let no man deceive you with vain words: for because of these things cometh the wrath of God upon the children of disobedience. Be not ye therefore partakers with them. For ye were sometimes darkness, but now are ye light in the Lord: walk as children of light: (For the fruit of the Spirit is in all goodness and righteousness and truth;) Proving what is acceptable unto the Lord. And have no fellowship with the unfruitful works of darkness, but rather reprove them. For it is a shame even to speak of those things which are done of them in secret. But all things that are reproved are made manifest by the light: for whatsoever doth make manifest is light. Wherefore he saith, Awake thou that sleepest, and arise from the dead, and Christ shall give thee light.}


\subsection{\stylesubsec{}{The Gospel.}{St.~Luke 11.~14.}}
\drop{Jesus was casting out a devil, and it was dumb. And it came to pass, when the devil was gone out, the dumb spake; and the people wondered. But some of them said, He casteth out devils through Beelzebub the chief of the devils. And others, tempting him, sought of him a sign from heaven. But he, knowing their thoughts, said unto them, Every kingdom divided against itself is brought to desolation; and a house divided against a house falleth. If Satan also be divided against himself, how shall his kingdom stand? because ye say that I cast out devils through Beelzebub. And if I by Beelzebub cast out devils, by whom do your sons cast them out? therefore shall they be your judges. But if I with the finger of God cast out devils, no doubt the kingdom of God is come upon you. When a strong man armed keepeth his palace, his goods are in peace: But when a stronger than he shall come upon him, and overcome him, he taketh from him all his armour wherein he trusted, and divideth his spoils. He that is not with me is against me: and he that gathereth not with me scattereth. When the unclean spirit is gone out of a man, he walketh through dry places, seeking rest; and finding none, he saith, I will return unto my house whence I came out. And when he cometh, he findeth it swept and garnished. Then goeth he, and taketh to him seven other spirits more wicked than himself; and they enter in, and dwell there: and the last state of that man is worse than the first. And it came to pass, as he spake these things, a certain woman of the company lifted up her voice, and said unto him, Blessed is the womb that bare thee, and the paps which thou hast sucked. But he said, Yea rather, blessed are they that hear the word of God, and keep it.}

\section{The Fourth Sunday in Lent}
\subsection{\stylesubsec{}{The Collect.}{}}
\drop{Grant, we beseech thee, Almighty God, that we, who for our evil deeds do worthily deserve to be punished, by the comfort of thy grace may mercifully be relieved; through our Lord and Saviour Jesus Christ. \R Amen.}

\pilcrow{The Collect from the First Day of Lent is to be read every day in Lent after the Collect appointed for the Day.}

\subsection{\stylesubsec{}{The Epistle.}{Galatians 4.~21.}}
\drop{Tell me, ye that desire to be under the law, do ye not hear the law? For it is written, that Abraham had two sons, the one by a bondmaid, the other by a freewoman. But he who was of the bondwoman was born after the flesh; but he of the freewoman was by promise. Which things are an allegory: for these are the two covenants; the one from the mount Sinai, which gendereth to bondage, which is Agar. For this Agar is mount Sinai in Arabia, and answereth to Jerusalem which now is, and is in bondage with her children. But Jerusalem which is above is free, which is the mother of us all. For it is written, Rejoice, thou barren that bearest not; break forth and cry, thou that travailest not: for the desolate hath many more children than she which hath an husband. Now we, brethren, as Isaac was, are the children of promise. But as then he that was born after the flesh persecuted him that was born after the Spirit, even so it is now. Nevertheless what saith the scripture? Cast out the bondwoman and her son: for the son of the bondwoman shall not be heir with the son of the freewoman. So then, brethren, we are not children of the bondwoman, but of the free.}


\subsection{\stylesubsec{}{The Gospel.}{St.~John 6.~1.}}
\drop{Jesus went over the sea of Galilee, which is the sea of Tiberias. And a great multitude followed him, because they saw his miracles which he did on them that were diseased. And Jesus went up into a mountain, and there he sat with his disciples. And the passover, a feast of the Jews, was nigh. When Jesus then lifted up his eyes, and saw a great company come unto him, he saith unto Philip, Whence shall we buy bread, that these may eat? And this he said to prove him: for he himself knew what he would do. Philip answered him, Two hundred pennyworth of bread is not sufficient for them, that every one of them may take a little. One of his disciples, Andrew, Simon Peter's brother, saith unto him, There is a lad here, which hath five barley loaves, and two small fishes: but what are they among so many? And Jesus said, Make the men sit down. Now there was much grass in the place. So the men sat down, in number about five thousand. And Jesus took the loaves; and when he had given thanks, he distributed to the disciples, and the disciples to them that were set down; and likewise of the fishes as much as they would. When they were filled, he said unto his disciples, Gather up the fragments that remain, that nothing be lost. Therefore they gathered them together, and filled twelve baskets with the fragments of the five barley loaves, which remained over and above unto them that had eaten. Then those men, when they had seen the miracle that Jesus did, said, This is of a truth that Prophet that should come into the world.}

\section{The Fifth Sunday in Lent}
\subsection{\stylesubsec{}{The Collect.}{}}
\drop{We beseech thee, Almighty God, mercifully to look upon thy people; that by thy great goodness they may be governed and preserved evermore, both in body and soul; through Jesus Christ our Lord. \R Amen.}

\pilcrow{The Collect from the First Day of Lent is to be read every day in Lent after the Collect appointed for the Day.}

\subsection{\stylesubsec{}{The Epistle.}{Hebrews 9.~11.}}
\drop{Christ being come an high priest of good things to come, by a greater and more perfect tabernacle, not made with hands, that is to say, not of this building; Neither by the blood of goats and calves, but by his own blood he entered in once into the holy place, having obtained eternal redemption for us. For if the blood of bulls and of goats, and the ashes of an heifer sprinkling the unclean, sanctifieth to the purifying of the flesh: How much more shall the blood of Christ, who through the eternal Spirit offered himself without spot to God, purge your conscience from dead works to serve the living God? And for this cause he is the mediator of the new testament, that by means of death, for the redemption of the transgressions that were under the first testament, they which are called might receive the promise of eternal inheritance.}


\subsection{\stylesubsec{}{The Gospel.}{St.~John 8.~46.}}
\drop{Jesus said, Which of you convinceth me of sin? And if I say the truth, why do ye not believe me? He that is of God heareth God's words: ye therefore hear them not, because ye are not of God. Then answered the Jews, and said unto him, Say we not well that thou art a Samaritan, and hast a devil? Jesus answered, I have not a devil; but I honour my Father, and ye do dishonour me. And I seek not mine own glory: there is one that seeketh and judgeth. Verily, verily, I say unto you, If a man keep my saying, he shall never see death. Then said the Jews unto him, Now we know that thou hast a devil. Abraham is dead, and the prophets; and thou sayest, If a man keep my saying, he shall never taste of death. Art thou greater than our father Abraham, which is dead? and the prophets are dead: whom makest thou thyself? Jesus answered, If I honour myself, my honour is nothing: it is my Father that honoureth me; of whom ye say, that he is your God: Yet ye have not known him; but I know him: and if I should say, I know him not, I shall be a liar like unto you: but I know him, and keep his saying. Your father Abraham rejoiced to see my day: and he saw it, and was glad. Then said the Jews unto him, Thou art not yet fifty years old, and hast thou seen Abraham? Jesus said unto them, Verily, verily, I say unto you, Before Abraham was, I am. Then took they up stones to cast at him: but Jesus hid himself, and went out of the temple.}

\stylesec{The Sunday next before Easter}{commonly called}{Palm Sunday}
\subsection{\stylesubsec{}{The Collect.}{}}
\drop{Almighty and everlasting God, who, of thy tender love towards mankind, hast sent thy Son, our Savior Jesus Christ, to take upon him our flesh, and to suffer death upon the cross, that all mankind should follow the example of his great humility; Mercifully grant, that we may both follow the example of his patience, and also be made partakers of his resurrection; through the same Jesus Christ our Lord. \R Amen.}

\pilcrow{The Collect from the First Day of Lent is to be read every day in Lent after the Collect appointed for the Day.}

\subsection{\stylesubsec{}{The Epistle.}{Philippians 2.~5.}}
\drop{Let this mind be in you, which was also in Christ Jesus: Who, being in the form of God, thought it not robbery to be equal with God: But made himself of no reputation, and took upon him the form of a servant, and was made in the likeness of men: And being found in fashion as a man, he humbled himself, and became obedient unto death, even the death of the cross. Wherefore God also hath highly exalted him, and given him a name which is above every name: That at the name of Jesus every knee should bow, of things in heaven, and things in earth, and things under the earth; And that every tongue should confess that Jesus Christ is Lord, to the glory of God the Father.}


\subsection{\stylesubsec{}{The Gospel.}{St.~Matthew 27.~1.}}
\drop{When the morning was come, all the chief priests and elders of the people took counsel against Jesus to put him to death: And when they had bound him, they led him away, and delivered him to Pontius Pilate the governor. Then Judas, which had betrayed him, when he saw that he was condemned, repented himself, and brought again the thirty pieces of silver to the chief priests and elders, Saying, I have sinned in that I have betrayed the innocent blood. And they said, What is that to us? see thou to that. And he cast down the pieces of silver in the temple, and departed, and went and hanged himself. And the chief priests took the silver pieces, and said, It is not lawful for to put them into the treasury, because it is the price of blood. And they took counsel, and bought with them the potter's field, to bury strangers in. Wherefore that field was called, The field of blood, unto this day. Then was fulfilled that which was spoken by Jeremy the prophet, saying, And they took the thirty pieces of silver, the price of him that was valued, whom they of the children of Israel did value; And gave them for the potter's field, as the Lord appointed me. And Jesus stood before the governor: and the governor asked him, saying, Art thou the King of the Jews? And Jesus said unto him, Thou sayest. And when he was accused of the chief priests and elders, he answered nothing. Then said Pilate unto him, Hearest thou not how many things they witness against thee? And he answered him to never a word; insomuch that the governor marvelled greatly. Now at that feast the governor was wont to release unto the people a prisoner, whom they would. And they had then a notable prisoner, called Barabbas. Therefore when they were gathered together, Pilate said unto them, Whom will ye that I release unto you? Barabbas, or Jesus which is called Christ? For he knew that for envy they had delivered him. When he was set down on the judgment seat, his wife sent unto him, saying, Have thou nothing to do with that just man: for I have suffered many things this day in a dream because of him. But the chief priests and elders persuaded the multitude that they should ask Barabbas, and destroy Jesus. The governor answered and said unto them, Whether of the twain will ye that I release unto you? They said, Barabbas. Pilate saith unto them, What shall I do then with Jesus which is called Christ? They all say unto him, Let him be crucified. And the governor said, Why, what evil hath he done? But they cried out the more, saying, Let him be crucified. When Pilate saw that he could prevail nothing, but that rather a tumult was made, he took water, and washed his hands before the multitude, saying, I am innocent of the blood of this just person: see ye to it. Then answered all the people, and said, His blood be on us, and on our children. Then released he Barabbas unto them: and when he had scourged Jesus, he delivered him to be crucified. Then the soldiers of the governor took Jesus into the common hall, and gathered unto him the whole band of soldiers. And they stripped him, and put on him a scarlet robe. And when they had platted a crown of thorns, they put it upon his head, and a reed in his right hand: and they bowed the knee before him, and mocked him, saying, Hail, King of the Jews! And they spit upon him, and took the reed, and smote him on the head. And after that they had mocked him, they took the robe off from him, and put his own raiment on him, and led him away to crucify him. And as they came out, they found a man of Cyrene, Simon by name: him they compelled to bear his cross. And when they were come unto a place called Golgotha, that is to say, a place of a skull, They gave him vinegar to drink mingled with gall: and when he had tasted thereof, he would not drink. And they crucified him, and parted his garments, casting lots: that it might be fulfilled which was spoken by the prophet, They parted my garments among them, and upon my vesture did they cast lots. And sitting down they watched him there; And set up over his head his accusation written, THIS IS JESUS THE KING OF THE JEWS. Then were there two thieves crucified with him, one on the right hand, and another on the left. And they that passed by reviled him, wagging their heads, And saying, Thou that destroyest the temple, and buildest it in three days, save thyself. If thou be the Son of God, come down from the cross. Likewise also the chief priests mocking him, with the scribes and elders, said, He saved others; himself he cannot save. If he be the King of Israel, let him now come down from the cross, and we will believe him. He trusted in God; let him deliver him now, if he will have him: for he said, I am the Son of God. The thieves also, which were crucified with him, cast the same in his teeth. Now from the sixth hour there was darkness over all the land unto the ninth hour. And about the ninth hour Jesus cried with a loud voice, saying, Eli, Eli, lama sabachthani? that is to say, My God, my God, why hast thou forsaken me? Some of them that stood there, when they heard that, said, This man calleth for Elias. And straightway one of them ran, and took a sponge, and filled it with vinegar, and put it on a reed, and gave him to drink. The rest said, Let be, let us see whether Elias will come to save him. Jesus, when he had cried again with a loud voice, yielded up the ghost. And, behold, the veil of the temple was rent in twain from the top to the bottom; and the earth did quake, and the rocks rent; And the graves were opened; and many bodies of the saints which slept arose, And came out of the graves after his resurrection, and went into the holy city, and appeared unto many. Now when the centurion, and they that were with him, watching Jesus, saw the earthquake, and those things that were done, they feared greatly, saying, Truly this was the Son of God.}

\section{Monday before Easter}
\subsection{\stylesubsec{}{For the Epistle.}{Isaiah 63.~1.}}
\drop{Who is this that cometh from Edom, with dyed garments from Bozrah? this that is glorious in his apparel, travelling in the greatness of his strength? I that speak in righteousness, mighty to save. Wherefore art thou red in thine apparel, and thy garments like him that treadeth in the winefat? I have trodden the winepress alone; and of the people there was none with me: for I will tread them in mine anger, and trample them in my fury; and their blood shall be sprinkled upon my garments, and I will stain all my raiment. For the day of vengeance is in mine heart, and the year of my redeemed is come. And I looked, and there was none to help; and I wondered that there was none to uphold: therefore mine own arm brought salvation unto me; and my fury, it upheld me. And I will tread down the people in mine anger, and make them drunk in my fury, and I will bring down their strength to the earth. I will mention the lovingkindnesses of the Lord, and the praises of the Lord, according to all that the Lord hath bestowed on us, and the great goodness toward the house of Israel, which he hath bestowed on them according to his mercies, and according to the multitude of his lovingkindnesses. For he said, Surely they are my people, children that will not lie: so he was their Saviour. In all their affliction he was afflicted, and the angel of his presence saved them: in his love and in his pity he redeemed them; and he bare them, and carried them all the days of old. But they rebelled, and vexed his holy Spirit: therefore he was turned to be their enemy, and he fought against them. Then he remembered the days of old, Moses, and his people, saying, Where is he that brought them up out of the sea with the shepherd of his flock? where is he that put his holy Spirit within him? That led them by the right hand of Moses with his glorious arm, dividing the water before them, to make himself an everlasting name? That led them through the deep, as an horse in the wilderness, that they should not stumble? As a beast goeth down into the valley, the Spirit of the Lord caused him to rest: so didst thou lead thy people, to make thyself a glorious name. Look down from heaven, and behold from the habitation of thy holiness and of thy glory: where is thy zeal and thy strength, the sounding of thy bowels and of thy mercies toward me? are they restrained? Doubtless thou art our father, though Abraham be ignorant of us, and Israel acknowledge us not: thou, O Lord, art our father, our redeemer; thy name is from everlasting. O Lord, why hast thou made us to err from thy ways, and hardened our heart from thy fear? Return for thy servants' sake, the tribes of thine inheritance. The people of thy holiness have possessed it but a little while: our adversaries have trodden down thy sanctuary. We are thine: thou never barest rule over them; they were not called by thy Name.}


\subsection{\stylesubsec{}{The Gospel.}{St.~Mark 14.~1.}}
\drop{After two days was the feast of the passover, and of unleavened bread: and the chief priests and the scribes sought how they might take him by craft, and put him to death. But they said, Not on the feast day, lest there be an uproar of the people. And being in Bethany in the house of Simon the leper, as he sat at meat, there came a woman having an alabaster box of ointment of spikenard very precious; and she brake the box, and poured it on his head. And there were some that had indignation within themselves, and said, Why was this waste of the ointment made? For it might have been sold for more than three hundred pence, and have been given to the poor. And they murmured against her. And Jesus said, Let her alone; why trouble ye her? she hath wrought a good work on me. For ye have the poor with you always, and whensoever ye will ye may do them good: but me ye have not always. She hath done what she could: she is come aforehand to anoint my body to the burying. Verily I say unto you, Wheresoever this gospel shall be preached throughout the whole world, this also that she hath done shall be spoken of for a memorial of her. And Judas Iscariot, one of the twelve, went unto the chief priests, to betray him unto them. And when they heard it, they were glad, and promised to give him money. And he sought how he might conveniently betray him. And the first day of unleavened bread, when they killed the passover, his disciples said unto him, Where wilt thou that we go and prepare that thou mayest eat the passover? And he sendeth forth two of his disciples, and saith unto them, Go ye into the city, and there shall meet you a man bearing a pitcher of water: follow him. And wheresoever he shall go in, say ye to the goodman of the house, The Master saith, Where is the guestchamber, where I shall eat the passover with my disciples? And he will shew you a large upper room furnished and prepared: there make ready for us. And his disciples went forth, and came into the city, and found as he had said unto them: and they made ready the passover. And in the evening he cometh with the twelve. And as they sat and did eat, Jesus said, Verily I say unto you, One of you which eateth with me shall betray me. And they began to be sorrowful, and to say unto him one by one, Is it I? and another said, Is it I? And he answered and said unto them, It is one of the twelve, that dippeth with me in the dish. The Son of man indeed goeth, as it is written of him: but woe to that man by whom the Son of man is betrayed! good were it for that man if he had never been born. And as they did eat, Jesus took bread, and blessed, and brake it, and gave to them, and said, Take, eat: this is my body. And he took the cup, and when he had given thanks, he gave it to them: and they all drank of it. And he said unto them, This is my blood of the new testament, which is shed for many. Verily I say unto you, I will drink no more of the fruit of the vine, until that day that I drink it new in the kingdom of God. And when they had sung an hymn, they went out into the mount of Olives. And Jesus saith unto them, All ye shall be offended because of me this night: for it is written, I will smite the shepherd, and the sheep shall be scattered. But after that I am risen, I will go before you into Galilee. But Peter said unto him, Although all shall be offended, yet will not I. And Jesus saith unto him, Verily I say unto thee, That this day, even in this night, before the cock crow twice, thou shalt deny me thrice. But he spake the more vehemently, If I should die with thee, I will not deny thee in any wise. Likewise also said they all. And they came to a place which was named Gethsemane: and he saith to his disciples, Sit ye here, while I shall pray. And he taketh with him Peter and James and John, and began to be sore amazed, and to be very heavy; And saith unto them, My soul is exceeding sorrowful unto death: tarry ye here, and watch. And he went forward a little, and fell on the ground, and prayed that, if it were possible, the hour might pass from him. And he said, Abba, Father, all things are possible unto thee; take away this cup from me: nevertheless not what I will, but what thou wilt. And he cometh, and findeth them sleeping, and saith unto Peter, Simon, sleepest thou? couldest not thou watch one hour? Watch ye and pray, lest ye enter into temptation. The spirit truly is ready, but the flesh is weak. And again he went away, and prayed, and spake the same words. And when he returned, he found them asleep again, (for their eyes were heavy,) neither wist they what to answer him. And he cometh the third time, and saith unto them, Sleep on now, and take your rest: it is enough, the hour is come; behold, the Son of man is betrayed into the hands of sinners. Rise up, let us go; lo, he that betrayeth me is at hand. And immediately, while he yet spake, cometh Judas, one of the twelve, and with him a great multitude with swords and staves, from the chief priests and the scribes and the elders. And he that betrayed him had given them a token, saying, Whomsoever I shall kiss, that same is he; take him, and lead him away safely. And as soon as he was come, he goeth straightway to him, and saith, Master, master; and kissed him. And they laid their hands on him, and took him. And one of them that stood by drew a sword, and smote a servant of the high priest, and cut off his ear. And Jesus answered and said unto them, Are ye come out, as against a thief, with swords and with staves to take me? I was daily with you in the temple teaching, and ye took me not: but the scriptures must be fulfilled. And they all forsook him, and fled. And there followed him a certain young man, having a linen cloth cast about his naked body; and the young men laid hold on him: And he left the linen cloth, and fled from them naked. And they led Jesus away to the high priest: and with him were assembled all the chief priests and the elders and the scribes. And Peter followed him afar off, even into the palace of the high priest: and he sat with the servants, and warmed himself at the fire. And the chief priests and all the council sought for witness against Jesus to put him to death; and found none. For many bare false witness against him, but their witness agreed not together. And there arose certain, and bare false witness against him, saying, We heard him say, I will destroy this temple that is made with hands, and within three days I will build another made without hands. But neither so did their witness agree together. And the high priest stood up in the midst, and asked Jesus, saying, Answerest thou nothing? what is it which these witness against thee? But he held his peace, and answered nothing. Again the high priest asked him, and said unto him, Art thou the Christ, the Son of the Blessed? And Jesus said, I am: and ye shall see the Son of man sitting on the right hand of power, and coming in the clouds of heaven. Then the high priest rent his clothes, and saith, What need we any further witnesses? Ye have heard the blasphemy: what think ye? And they all condemned him to be guilty of death. And some began to spit on him, and to cover his face, and to buffet him, and to say unto him, Prophesy: and the servants did strike him with the palms of their hands. And as Peter was beneath in the palace, there cometh one of the maids of the high priest: And when she saw Peter warming himself, she looked upon him, and said, And thou also wast with Jesus of Nazareth. But he denied, saying, I know not, neither understand I what thou sayest. And he went out into the porch; and the cock crew. And a maid saw him again, and began to say to them that stood by, This is one of them. And he denied it again. And a little after, they that stood by said again to Peter, Surely thou art one of them: for thou art a Galilaean, and thy speech agreeth thereto. But he began to curse and to swear, saying, I know not this man of whom ye speak. And the second time the cock crew. And Peter called to mind the word that Jesus said unto him, Before the cock crow twice, thou shalt deny me thrice. And when he thought thereon, he wept.}


\section{Tuesday before Easter}
\subsection{\stylesubsec{}{For the Epistle.}{Isaiah 50.~5.}}
\drop{The Lord God hath opened mine ear, and I was not rebellious, neither turned away back. I gave my back to the smiters, and my cheeks to them that plucked off the hair: I hid not my face from shame and spitting. For the Lord God will help me; therefore shall I not be confounded: therefore have I set my face like a flint, and I know that I shall not be ashamed. He is near that justifieth me; who will contend with me? let us stand together: who is mine adversary? let him come near to me. Behold, the Lord God will help me; who is he that shall condemn me? lo, they all shall wax old as a garment; the moth shall eat them up. Who is among you that feareth the Lord, that obeyeth the voice of his servant, that walketh in darkness, and hath no light? let him trust in the name of the Lord, and stay upon his God. Behold, all ye that kindle a fire, that compass yourselves about with sparks: walk in the light of your fire, and in the sparks that ye have kindled. This shall ye have of mine hand; ye shall lie down in sorrow.}


\subsection{\stylesubsec{}{The Gospel.}{St.~Mark 15.~1.}}
\drop{And straightway in the morning the chief priests held a consultation with the elders and scribes and the whole council, and bound Jesus, and carried him away, and delivered him to Pilate. And Pilate asked him, Art thou the King of the Jews? And he answering said unto them, Thou sayest it. And the chief priests accused him of many things: but he answered nothing. And Pilate asked him again, saying, Answerest thou nothing? behold how many things they witness against thee. But Jesus yet answered nothing; so that Pilate marvelled. Now at that feast he released unto them one prisoner, whomsoever they desired. And there was one named Barabbas, which lay bound with them that had made insurrection with him, who had committed murder in the insurrection. And the multitude crying aloud began to desire him to do as he had ever done unto them. But Pilate answered them, saying, Will ye that I release unto you the King of the Jews? For he knew that the chief priests had delivered him for envy. But the chief priests moved the people, that he should rather release Barabbas unto them. And Pilate answered and said again unto them, What will ye then that I shall do unto him whom ye call the King of the Jews? And they cried out again, Crucify him. Then Pilate said unto them, Why, what evil hath he done? And they cried out the more exceedingly, Crucify him. And so Pilate, willing to content the people, released Barabbas unto them, and delivered Jesus, when he had scourged him, to be crucified. And the soldiers led him away into the hall, called Praetorium; and they call together the whole band. And they clothed him with purple, and platted a crown of thorns, and put it about his head, And began to salute him, Hail, King of the Jews! And they smote him on the head with a reed, and did spit upon him, and bowing their knees worshipped him. And when they had mocked him, they took off the purple from him, and put his own clothes on him, and led him out to crucify him. And they compel one Simon a Cyrenian, who passed by, coming out of the country, the father of Alexander and Rufus, to bear his cross. And they bring him unto the place Golgotha, which is, being interpreted, The place of a skull. And they gave him to drink wine mingled with myrrh: but he received it not. And when they had crucified him, they parted his garments, casting lots upon them, what every man should take. And it was the third hour, and they crucified him. And the superscription of his accusation was written over, THE KING OF THE JEWS. And with him they crucify two thieves; the one on his right hand, and the other on his left. And the scripture was fulfilled, which saith, And he was numbered with the transgressors. And they that passed by railed on him, wagging their heads, and saying, Ah, thou that destroyest the temple, and buildest it in three days, Save thyself, and come down from the cross. Likewise also the chief priests mocking said among themselves with the scribes, He saved others; himself he cannot save. Let Christ the King of Israel descend now from the cross, that we may see and believe. And they that were crucified with him reviled him. And when the sixth hour was come, there was darkness over the whole land until the ninth hour. And at the ninth hour Jesus cried with a loud voice, saying, Eloi, Eloi, lama sabachthani? which is, being interpreted, My God, my God, why hast thou forsaken me? And some of them that stood by, when they heard it, said, Behold, he calleth Elias. And one ran and filled a sponge full of vinegar, and put it on a reed, and gave him to drink, saying, Let alone; let us see whether Elias will come to take him down. And Jesus cried with a loud voice, and gave up the ghost. And the veil of the temple was rent in twain from the top to the bottom. And when the centurion, which stood over against him, saw that he so cried out, and gave up the ghost, he said, Truly this man was the Son of God.}

\section{Wednesday before Easter}

\subsection{\stylesubsec{}{The Epistle.}{Hebrews 9.~16.}}
\drop{Where a testament is, there must also of necessity be the death of the testator. For a testament is of force after men are dead: otherwise it is of no strength at all while the testator liveth. Whereupon neither the first testament was dedicated without blood. For when Moses had spoken every precept to all the people according to the law, he took the blood of calves and of goats, with water, and scarlet wool, and hyssop, and sprinkled both the book, and all the people, Saying, This is the blood of the testament which God hath enjoined unto you. Moreover he sprinkled with blood both the tabernacle, and all the vessels of the ministry. And almost all things are by the law purged with blood; and without shedding of blood is no remission. It was therefore necessary that the patterns of things in the heavens should be purified with these; but the heavenly things themselves with better sacrifices than these. For Christ is not entered into the holy places made with hands, which are the figures of the true; but into heaven itself, now to appear in the presence of God for us: Nor yet that he should offer himself often, as the high priest entereth into the holy place every year with blood of others; For then must he often have suffered since the foundation of the world: but now once in the end of the world hath he appeared to put away sin by the sacrifice of himself. And as it is appointed unto men once to die, but after this the judgment: So Christ was once offered to bear the sins of many; and unto them that look for him shall he appear the second time without sin unto salvation.}


\subsection{\stylesubsec{}{The Gospel.}{St.~Luke 22.~1.}}
\drop{Now the feast of unleavened bread drew nigh, which is called the Passover. And the chief priests and scribes sought how they might kill him; for they feared the people. Then entered Satan into Judas surnamed Iscariot, being of the number of the twelve. And he went his way, and communed with the chief priests and captains, how he might betray him unto them. And they were glad, and covenanted to give him money. And he promised, and sought opportunity to betray him unto them in the absence of the multitude. Then came the day of unleavened bread, when the passover must be killed. And he sent Peter and John, saying, Go and prepare us the passover, that we may eat. And they said unto him, Where wilt thou that we prepare? And he said unto them, Behold, when ye are entered into the city, there shall a man meet you, bearing a pitcher of water; follow him into the house where he entereth in. And ye shall say unto the goodman of the house, The Master saith unto thee, Where is the guestchamber, where I shall eat the passover with my disciples? And he shall shew you a large upper room furnished: there make ready. And they went, and found as he had said unto them: and they made ready the passover. And when the hour was come, he sat down, and the twelve apostles with him. And he said unto them, With desire I have desired to eat this passover with you before I suffer: For I say unto you, I will not any more eat thereof, until it be fulfilled in the kingdom of God. And he took the cup, and gave thanks, and said, Take this, and divide it among yourselves: For I say unto you, I will not drink of the fruit of the vine, until the kingdom of God shall come. And he took bread, and gave thanks, and brake it, and gave unto them, saying, This is my body which is given for you: this do in remembrance of me. Likewise also the cup after supper, saying, This cup is the new testament in my blood, which is shed for you. But, behold, the hand of him that betrayeth me is with me on the table. And truly the Son of man goeth, as it was determined: but woe unto that man by whom he is betrayed! And they began to enquire among themselves, which of them it was that should do this thing. And there was also a strife among them, which of them should be accounted the greatest. And he said unto them, The kings of the Gentiles exercise Lordship over them; and they that exercise authority upon them are called benefactors. But ye shall not be so: but he that is greatest among you, let him be as the younger; and he that is chief, as he that doth serve. For whether is greater, he that sitteth at meat, or he that serveth? is not he that sitteth at meat? but I am among you as he that serveth. Ye are they which have continued with me in my temptations. And I appoint unto you a kingdom, as my Father hath appointed unto me; That ye may eat and drink at my table in my kingdom, and sit on thrones judging the twelve tribes of Israel. And the Lord said, Simon, Simon, behold, Satan hath desired to have you, that he may sift you as wheat: But I have prayed for thee, that thy faith fail not: and when thou art converted, strengthen thy brethren. And he said unto him, Lord, I am ready to go with thee, both into prison, and to death. And he said, I tell thee, Peter, the cock shall not crow this day, before that thou shalt thrice deny that thou knowest me. And he said unto them, When I sent you without purse, and scrip, and shoes, lacked ye any thing? And they said, Nothing. Then said he unto them, But now, he that hath a purse, let him take it, and likewise his scrip: and he that hath no sword, let him sell his garment, and buy one. For I say unto you, that this that is written must yet be accomplished in me, And he was reckoned among the transgressors: for the things concerning me have an end. And they said, Lord, behold, here are two swords. And he said unto them, It is enough. And he came out, and went, as he was wont, to the mount of Olives; and his disciples also followed him. And when he was at the place, he said unto them, Pray that ye enter not into temptation. And he was withdrawn from them about a stone's cast, and kneeled down, and prayed, Saying, Father, if thou be willing, remove this cup from me: nevertheless not my will, but thine, be done. And there appeared an angel unto him from heaven, strengthening him. And being in an agony he prayed more earnestly: and his sweat was as it were great drops of blood falling down to the ground. And when he rose up from prayer, and was come to his disciples, he found them sleeping for sorrow, And said unto them, Why sleep ye? rise and pray, lest ye enter into temptation. And while he yet spake, behold a multitude, and he that was called Judas, one of the twelve, went before them, and drew near unto Jesus to kiss him. But Jesus said unto him, Judas, betrayest thou the Son of man with a kiss? When they which were about him saw what would follow, they said unto him, Lord, shall we smite with the sword? And one of them smote the servant of the high priest, and cut off his right ear. And Jesus answered and said, Suffer ye thus far. And he touched his ear, and healed him. Then Jesus said unto the chief priests, and captains of the temple, and the elders, which were come to him, Be ye come out, as against a thief, with swords and staves? When I was daily with you in the temple, ye stretched forth no hands against me: but this is your hour, and the power of darkness. Then took they him, and led him, and brought him into the high priest's house. And Peter followed afar off. And when they had kindled a fire in the midst of the hall, and were set down together, Peter sat down among them. But a certain maid beheld him as he sat by the fire, and earnestly looked upon him, and said, This man was also with him. And he denied him, saying, Woman, I know him not. And after a little while another saw him, and said, Thou art also of them. And Peter said, Man, I am not. And about the space of one hour after another confidently affirmed, saying, Of a truth this fellow also was with him: for he is a Galilaean. And Peter said, Man, I know not what thou sayest. And immediately, while he yet spake, the cock crew. And the Lord turned, and looked upon Peter. And Peter remembered the word of the Lord, how he had said unto him, Before the cock crow, thou shalt deny me thrice. And Peter went out, and wept bitterly. And the men that held Jesus mocked him, and smote him. And when they had blindfolded him, they struck him on the face, and asked him, saying, Prophesy, who is it that smote thee? And many other things blasphemously spake they against him. And as soon as it was day, the elders of the people and the chief priests and the scribes came together, and led him into their council, saying, Art thou the Christ? tell us. And he said unto them, If I tell you, ye will not believe: And if I also ask you, ye will not answer me, nor let me go. Hereafter shall the Son of man sit on the right hand of the power of God. Then said they all, Art thou then the Son of God? And he said unto them, Ye say that I am. And they said, What need we any further witness? for we ourselves have heard of his own mouth.}

\section{Thursday before Easter}
\subsection{\stylesubsec{}{The Epistle.}{1 Corinthians 11.~17.}}
\drop{In this that I declare unto you I praise you not, that ye come together not for the better, but for the worse. For first of all, when ye come together in the church, I hear that there be divisions among you; and I partly believe it. For there must be also heresies among you, that they which are approved may be made manifest among you. When ye come together therefore into one place, this is not to eat the Lord's supper. For in eating every one taketh before other his own supper: and one is hungry, and another is drunken. What? have ye not houses to eat and to drink in? or despise ye the church of God, and shame them that have not? what shall I say to you? shall I praise you in this? I praise you not. For I have received of the Lord that which also I delivered unto you, that the Lord Jesus the same night in which he was betrayed took bread: And when he had given thanks, he brake it, and said, Take, eat: this is my body, which is broken for you: this do in remembrance of me. After the same manner also he took the cup, when he had supped, saying, this cup is the new testament in my blood: this do ye, as oft as ye drink it, in remembrance of me. For as often as ye eat this bread, and drink this cup, ye do shew the Lord's death till he come. Wherefore whosoever shall eat this bread, and drink this cup of the Lord, unworthily, shall be guilty of the body and blood of the Lord. But let a man examine himself, and so let him eat of that bread, and drink of that cup. For he that eateth and drinketh unworthily, eateth and drinketh damnation to himself, not discerning the Lord's body. For this cause many are weak and sickly among you, and many sleep. For if we would judge ourselves, we should not be judged. But when we are judged, we are chastened of the Lord, that we should not be condemned with the world. Wherefore, my brethren, when ye come together to eat, tarry one for another. And if any man hunger, let him eat at home; that ye come not together unto condemnation. And the rest will I set in order when I come.}


\subsection{\stylesubsec{}{The Gospel.}{St.~Luke 23.~1.}}
\drop{The whole multitude of them arose, and led him unto Pilate. And they began to accuse him, saying, We found this fellow perverting the nation, and forbidding to give tribute to Caesar, saying that he himself is Christ a King. And Pilate asked him, saying, Art thou the King of the Jews? And he answered him and said, Thou sayest it. Then said Pilate to the chief priests and to the people, I find no fault in this man. And they were the more fierce, saying, He stirreth up the people, teaching throughout all Jewry, beginning from Galilee to this place. When Pilate heard of Galilee, he asked whether the man were a Galilaean. And as soon as he knew that he belonged unto Herod's jurisdiction, he sent him to Herod, who himself also was at Jerusalem at that time. And when Herod saw Jesus, he was exceeding glad: for he was desirous to see him of a long season, because he had heard many things of him; and he hoped to have seen some miracle done by him. Then he questioned with him in many words; but he answered him nothing. And the chief priests and scribes stood and vehemently accused him. And Herod with his men of war set him at nought, and mocked him, and arrayed him in a gorgeous robe, and sent him again to Pilate. And the same day Pilate and Herod were made friends together: for before they were at enmity between themselves. And Pilate, when he had called together the chief priests and the rulers and the people, Said unto them, Ye have brought this man unto me, as one that perverteth the people: and, behold, I, having examined him before you, have found no fault in this man touching those things whereof ye accuse him: No, nor yet Herod: for I sent you to him; and, lo, nothing worthy of death is done unto him. I will therefore chastise him, and release him. (For of necessity he must release one unto them at the feast.) And they cried out all at once, saying, Away with this man, and release unto us Barabbas: (Who for a certain sedition made in the city, and for murder, was cast into prison.) Pilate therefore, willing to release Jesus, spake again to them. But they cried, saying, Crucify him, crucify him. And he said unto them the third time, Why, what evil hath he done? I have found no cause of death in him: I will therefore chastise him, and let him go. And they were instant with loud voices, requiring that he might be crucified. And the voices of them and of the chief priests prevailed. And Pilate gave sentence that it should be as they required. And he released unto them him that for sedition and murder was cast into prison, whom they had desired; but he delivered Jesus to their will. And as they led him away, they laid hold upon one Simon, a Cyrenian, coming out of the country, and on him they laid the cross, that he might bear it after Jesus. And there followed him a great company of people, and of women, which also bewailed and lamented him. But Jesus turning unto them said, Daughters of Jerusalem, weep not for me, but weep for yourselves, and for your children. For, behold, the days are coming, in the which they shall say, Blessed are the barren, and the wombs that never bare, and the paps which never gave suck. Then shall they begin to say to the mountains, Fall on us; and to the hills, Cover us. For if they do these things in a green tree, what shall be done in the dry? And there were also two other, malefactors, led with him to be put to death. And when they were come to the place, which is called Calvary, there they crucified him, and the malefactors, one on the right hand, and the other on the left. Then said Jesus, Father, forgive them; for they know not what they do. And they parted his raiment, and cast lots. And the people stood beholding. And the rulers also with them derided him, saying, He saved others; let him save himself, if he be Christ, the chosen of God. And the soldiers also mocked him, coming to him, and offering him vinegar, And saying, If thou be the king of the Jews, save thyself. And a superscription also was written over him in letters of Greek, and Latin, and Hebrew, THIS IS THE KING OF THE JEWS. And one of the malefactors which were hanged railed on him, saying, If thou be Christ, save thyself and us. But the other answering rebuked him, saying, Dost not thou fear God, seeing thou art in the same condemnation? And we indeed justly; for we receive the due reward of our deeds: but this man hath done nothing amiss. And he said unto Jesus, Lord, remember me when thou comest into thy kingdom. And Jesus said unto him, Verily I say unto thee, To day shalt thou be with me in paradise. And it was about the sixth hour, and there was a darkness over all the earth until the ninth hour. And the sun was darkened, and the veil of the temple was rent in the midst. And when Jesus had cried with a loud voice, he said, Father, into thy hands I commend my spirit: and having said thus, he gave up the ghost. Now when the centurion saw what was done, he glorified God, saying, Certainly this was a righteous man. And all the people that came together to that sight, beholding the things which were done, smote their breasts, and returned. And all his acquaintance, and the women that followed him from Galilee, stood afar off, beholding these things.}

\section{Good Friday}
\subsection{\stylesubsec{}{The Collects.}{}}
\drop{Almighty God, we beseech thee graciously to behold this thy family, for whom our Lord Jesus Christ was contented to be betrayed, and given up into the hands of wicked men, and to suffer death upon the cross, who now liveth and reigneth with thee and the Holy Ghost, ever one God, world without end. \R Amen.}
\drop{Almighty and everlasting God, by whose Spirit the whole body of the Church is governed and sanctified; Receive our supplications and prayers, which we offer before thee for all estates of men in thy holy Church, that every member of the same, in his vocation and ministry may truly and godly serve thee; through our Lord and Savior Jesus Christ. \R Amen.}

\drop{O merciful God, who hast made all men, and hatest nothing that thou hast made, nor wouldest the death of a sinner, but rather that he should be converted and live; Have mercy upon all Jews, Turks, Infidels, and Hereticks, and take from them all ignorance, hardness of heart, and contempt of thy Word; and so fetch them home, blessed Lord, to thy flock, that they may be saved among the remnant of the true Israelites, and be made one fold under one shepherd, Jesus Christ our Lord, who liveth and reigneth with thee and the Holy Spirit, one God, world without end. \R Amen.}

\pilcrow{The Collect from the First Day of Lent is to be read every day in Lent after the Collect appointed for the Day.}

\subsection{\stylesubsec{}{The Epistle.}{Hebrews 10.~1.}}
\drop{The law having a shadow of good things to come, and not the very image of the things, can never with those sacrifices which they offered year by year continually make the comers thereunto perfect. For then would they not have ceased to be offered? because that the worshippers once purged should have had no more conscience of sins. But in those sacrifices there is a remembrance again made of sins every year. For it is not possible that the blood of bulls and of goats should take away sins. Wherefore when he cometh into the world, he saith, Sacrifice and offering thou wouldest not, but a body hast thou prepared me: In burnt offerings and sacrifices for sin thou hast had no pleasure. Then said I, Lo, I come (in the volume of the book it is written of me,) to do thy will, O God. Above when he said, Sacrifice and offering and burnt offerings and offering for sin thou wouldest not, neither hadst pleasure therein; which are offered by the law; Then said he, Lo, I come to do thy will, O God. He taketh away the first, that he may establish the second. By the which will we are sanctified through the offering of the body of Jesus Christ once for all. And every priest standeth daily ministering and offering oftentimes the same sacrifices, which can never take away sins: But this man, after he had offered one sacrifice for sins for ever, sat down on the right hand of God; From henceforth expecting till his enemies be made his footstool. For by one offering he hath perfected for ever them that are sanctified. Whereof the Holy Ghost also is a witness to us: for after that he had said before, This is the covenant that I will make with them after those days, saith the Lord, I will put my laws into their hearts, and in their minds will I write them; And their sins and iniquities will I remember no more. Now where remission of these is, there is no more offering for sin. Having therefore, brethren, boldness to enter into the holiest by the blood of Jesus, By a new and living way, which he hath consecrated for us, through the veil, that is to say, his flesh; And having an high priest over the house of God; Let us draw near with a true heart in full assurance of faith, having our hearts sprinkled from an evil conscience, and our bodies washed with pure water. Let us hold fast the profession of our faith without wavering; (for he is faithful that promised;) And let us consider one another to provoke unto love and to good works: Not forsaking the assembling of ourselves together, as the manner of some is; but exhorting one another: and so much the more, as ye see the day approaching.}


\subsection{\stylesubsec{}{The Gospel.}{St.~John 19.~1.}}
\drop{Then Pilate therefore took Jesus, and scourged him. And the soldiers platted a crown of thorns, and put it on his head, and they put on him a purple robe, And said, Hail, King of the Jews! and they smote him with their hands. Pilate therefore went forth again, and saith unto them, Behold, I bring him forth to you, that ye may know that I find no fault in him. Then came Jesus forth, wearing the crown of thorns, and the purple robe. And Pilate saith unto them, Behold the man! When the chief priests therefore and officers saw him, they cried out, saying, Crucify him, crucify him. Pilate saith unto them, Take ye him, and crucify him: for I find no fault in him. The Jews answered him, We have a law, and by our law he ought to die, because he made himself the Son of God. When Pilate therefore heard that saying, he was the more afraid; And went again into the judgment hall, and saith unto Jesus, Whence art thou? But Jesus gave him no answer. Then saith Pilate unto him, Speakest thou not unto me? knowest thou not that I have power to crucify thee, and have power to release thee? Jesus answered, Thou couldest have no power at all against me, except it were given thee from above: therefore he that delivered me unto thee hath the greater sin. And from thenceforth Pilate sought to release him: but the Jews cried out, saying, If thou let this man go, thou art not Caesar's friend: whosoever maketh himself a king speaketh against Caesar. When Pilate therefore heard that saying, he brought Jesus forth, and sat down in the judgment seat in a place that is called the Pavement, but in the Hebrew, Gabbatha. And it was the preparation of the passover, and about the sixth hour: and he saith unto the Jews, Behold your King! But they cried out, Away with him, away with him, crucify him. Pilate saith unto them, Shall I crucify your King? The chief priests answered, We have no king but Caesar. Then delivered he him therefore unto them to be crucified. And they took Jesus, and led him away. And he bearing his cross went forth into a place called the place of a skull, which is called in the Hebrew Golgotha: Where they crucified him, and two other with him, on either side one, and Jesus in the midst. And Pilate wrote a title, and put it on the cross. And the writing was JESUS OF NAZARETH THE KING OF THE JEWS. This title then read many of the Jews: for the place where Jesus was crucified was nigh to the city: and it was written in Hebrew, and Greek, and Latin. Then said the chief priests of the Jews to Pilate, Write not, The King of the Jews; but that he said, I am King of the Jews. Pilate answered, What I have written I have written. Then the soldiers, when they had crucified Jesus, took his garments, and made four parts, to every soldier a part; and also his coat: now the coat was without seam, woven from the top throughout. They said therefore among themselves, Let us not rend it, but cast lots for it, whose it shall be: that the scripture might be fulfilled, which saith, They parted my raiment among them, and for my vesture they did cast lots. These things therefore the soldiers did. Now there stood by the cross of Jesus his mother, and his mother's sister, Mary the wife of Cleophas, and Mary Magdalene. When Jesus therefore saw his mother, and the disciple standing by, whom he loved, he saith unto his mother, Woman, behold thy son! Then saith he to the disciple, Behold thy mother! And from that hour that disciple took her unto his own home. After this, Jesus knowing that all things were now accomplished, that the scripture might be fulfilled, saith, I thirst. Now there was set a vessel full of vinegar: and they filled a sponge with vinegar, and put it upon hyssop, and put it to his mouth. When Jesus therefore had received the vinegar, he said, It is finished: and he bowed his head, and gave up the ghost. The Jews therefore, because it was the preparation, that the bodies should not remain upon the cross on the sabbath day, (for that sabbath day was an high day,) besought Pilate that their legs might be broken, and that they might be taken away. Then came the soldiers, and brake the legs of the first, and of the other which was crucified with him. But when they came to Jesus, and saw that he was dead already, they brake not his legs: But one of the soldiers with a spear pierced his side, and forthwith came there out blood and water. And he that saw it bare record, and his record is true: and he knoweth that he saith true, that ye might believe. For these things were done, that the scripture should be fulfilled, A bone of him shall not be broken. And again another scripture saith, They shall look on him whom they pierced.}

\section{Easter-Even}
\subsection{\stylesubsec{}{The Collect.}{}}
\drop{Grant, O Lord, that as we are baptized into the death of thy blessed Son our Saviour Jesus Christ, so by continual mortifying our corrupt affections we may be buried with him; and that through the grave, and gate of death, we may pass to our joyful resurrection ; for his merits, who died, and was buried, and rose again for us, thy Son Jesus Christ our Lord. \R Amen.}

\pilcrow{The Collect from the First Day of Lent is to be read every day in Lent after the Collect appointed for the Day.}

\subsection{\stylesubsec{}{The Epistle.}{1 St.~Peter 3.~17.}}
\drop{It is better, if the will of God be so, that ye suffer for well doing, than for evil doing. For Christ also hath once suffered for sins, the just for the unjust, that he might bring us to God, being put to death in the flesh, but quickened by the Spirit: By which also he went and preached unto the spirits in prison; Which sometime were disobedient, when once the longsuffering of God waited in the days of Noah, while the ark was a preparing, wherein few, that is, eight souls were saved by water. The like figure whereunto even baptism doth also now save us (not the putting away of the filth of the flesh, but the answer of a good conscience toward God,) by the resurrection of Jesus Christ: Who is gone into heaven, and is on the right hand of God; angels and authorities and powers being made subject unto him.}


\subsection{\stylesubsec{}{The Gospel.}{St.~Matthew 27.~57.}}
\drop{When the even was come, there came a rich man of Arimathaea, named Joseph, who also himself was Jesus' disciple: He went to Pilate, and begged the body of Jesus. Then Pilate commanded the body to be delivered. And when Joseph had taken the body, he wrapped it in a clean linen cloth, And laid it in his own new tomb, which he had hewn out in the rock: and he rolled a great stone to the door of the sepulchre, and departed. And there was Mary Magdalene, and the other Mary, sitting over against the sepulchre. Now the next day, that followed the day of the preparation, the chief priests and Pharisees came together unto Pilate, Saying, Sir, we remember that that deceiver said, while he was yet alive, After three days I will rise again. Command therefore that the sepulchre be made sure until the third day, lest his disciples come by night, and steal him away, and say unto the people, He is risen from the dead: so the last error shall be worse than the first. Pilate said unto them, Ye have a watch: go your way, make it as sure as ye can. So they went, and made the sepulchre sure, sealing the stone, and setting a watch.}


\section{Easter Day}
\subsubsection{At Morning Prayer, instead of the Psalm, \emph{O come, let us sing,} \&c. these Anthems shall be sung or said.}
\drop{Christ our passover is sacrificed for us : therefore let us keep the feast;}

Not with the old leaven, nor with the leaven of malice and wickedness : but with the unleavened bread of sincerity and truth.\scripture{1 Corinthians v.~7}
\drop{Christ being raised from the dead dieth no more : death hath no more dominion over him.}

For in that he died, he died unto sin once : but in that he liveth, he liveth unto God.

Likewise reckon ye also yourselves to be dead indeed unto sin : but alive unto God through Jesus Christ our Lord.\scripture{Romans vj.~9.}

\drop{Christ is risen from the dead : and become the first-fruits of them that slept.}

For since by man came death : by man came also the resurrection of the dead.

For as in Adam all die : even so in Christ shall all be made alive.\scripture{1 Corinthians xv.~20.}

Glory be to the Father, and to the Son : and to the Holy Ghost;

As it was in the beginning, is now, and ever shall be : world without end. Amen.

\subsection{\stylesubsec{}{The Collect.}{}}
\drop{Almighty God, who through thine only-begotten Son Jesus Christ hast overcome death, and opened unto us the gate of everlasting life; We humbly beseech thee, that, as by thy special grace preventing us thou dost put into our minds good desires, so by thy continual help we may bring the same to good effect; through Jesus Christ our Lord, who liveth and reigneth with thee and the Holy Ghost, ever one God, world without end. Amen.}

\subsection{\stylesubsec{}{The Epistle.}{Colossians 3.~1.}}
\drop{If ye then be risen with Christ, seek those things which are above, where Christ sitteth on the right hand of God. Set your affection on things above, not on things on the earth. For ye are dead, and your life is hid with Christ in God. When Christ, who is our life, shall appear, then shall ye also appear with him in glory. Mortify therefore your members which are upon the earth; fornication, uncleanness, inordinate affection, evil concupiscence, and covetousness, which is idolatry: For which things' sake the wrath of God cometh on the children of disobedience: In the which ye also walked some time, when ye lived in them.}

\subsection{\stylesubsec{}{The Gospel.}{St.~John 20.~1.}}
\drop{The first day of the week cometh Mary Magdalene early, when it was yet dark, unto the sepulchre, and seeth the stone taken away from the sepulchre. Then she runneth, and cometh to Simon Peter, and to the other disciple, whom Jesus loved, and saith unto them, They have taken away the Lord out of the sepulchre, and we know not where they have laid him. Peter therefore went forth, and that other disciple, and came to the sepulchre. So they ran both together: and the other disciple did outrun Peter, and came first to the sepulchre. And he stooping down, and looking in, saw the linen clothes lying; yet went he not in. Then cometh Simon Peter following him, and went into the sepulchre, and seeth the linen clothes lie, And the napkin, that was about his head, not lying with the linen clothes, but wrapped together in a place by itself. Then went in also that other disciple, which came first to the sepulchre, and he saw, and believed. For as yet they knew not the scripture, that he must rise again from the dead. Then the disciples went away again unto their own home.}

\section{Monday in Easter Week}
\subsection{\stylesubsec{}{The Collect.}{}}
\drop{Almighty God, who through thy only-begotten Son Jesus Christ hast overcome death, and opened unto us the gate of everlasting life; We humbly beseech thee, that, as by thy special grace preventing us thou dost put into our minds good desires, so by thy continual help we may bring the same to good effect; through Jesus Christ our Lord, who liveth and reigneth with thee and the Holy Ghost, ever one God, world without end. \R Amen.}

\subsection{\stylesubsec{}{For the Epistle.}{Acts 10.~34.}}
\drop{Peter opened his mouth, and said, Of a truth I perceive that God is no respecter of persons: But in every nation he that feareth him, and worketh righteousness, is accepted with him. The word which God sent unto the children of Israel, preaching peace by Jesus Christ: (he is Lord of all:) That word, I say, ye know, which was published throughout all Judaea, and began from Galilee, after the baptism which John preached; How God anointed Jesus of Nazareth with the Holy Ghost and with power: who went about doing good, and healing all that were oppressed of the devil; for God was with him. And we are witnesses of all things which he did both in the land of the Jews, and in Jerusalem; whom they slew and hanged on a tree: Him God raised up the third day, and shewed him openly; Not to all the people, but unto witnesses chosen before God, even to us, who did eat and drink with him after he rose from the dead. And he commanded us to preach unto the people, and to testify that it is he which was ordained of God to be the Judge of quick and dead. To him give all the prophets witness, that through his name whosoever believeth in him shall receive remission of sins.}

\subsection{\stylesubsec{}{The Gospel.}{St.~Luke 24.~13.}}
\drop{Behold, two of them went that same day to a village called Emmaus, which was from Jerusalem about threescore furlongs. And they talked together of all these things which had happened. And it came to pass, that, while they communed together and reasoned, Jesus himself drew near, and went with them. But their eyes were holden that they should not know him. And he said unto them, What manner of communications are these that ye have one to another, as ye walk, and are sad? And the one of them, whose name was Cleopas, answering said unto him, Art thou only a stranger in Jerusalem, and hast not known the things which are come to pass there in these days? And he said unto them, What things? And they said unto him, Concerning Jesus of Nazareth, which was a prophet mighty in deed and word before God and all the people: And how the chief priests and our rulers delivered him to be condemned to death, and have crucified him. But we trusted that it had been he which should have redeemed Israel: and beside all this, to day is the third day since these things were done. Yea, and certain women also of our company made us astonished, which were early at the sepulchre; And when they found not his body, they came, saying, that they had also seen a vision of angels, which said that he was alive. And certain of them which were with us went to the sepulchre, and found it even so as the women had said: but him they saw not. Then he said unto them, O fools, and slow of heart to believe all that the prophets have spoken: Ought not Christ to have suffered these things, and to enter into his glory? And beginning at Moses and all the prophets, he expounded unto them in all the scriptures the things concerning himself. And they drew nigh unto the village, whither they went: and he made as though he would have gone further. But they constrained him, saying, Abide with us: for it is toward evening, and the day is far spent. And he went in to tarry with them. And it came to pass, as he sat at meat with them, he took bread, and blessed it, and brake, and gave to them. And their eyes were opened, and they knew him; and he vanished out of their sight. And they said one to another, Did not our heart burn within us, while he talked with us by the way, and while he opened to us the scriptures? And they rose up the same hour, and returned to Jerusalem, and found the eleven gathered together, and them that were with them, Saying, The Lord is risen indeed, and hath appeared to Simon. And they told what things were done in the way, and how he was known of them in breaking of bread.}

\section{Tuesday in Easter Week}
\subsection{\stylesubsec{}{The Collect.}{}}
\drop{Almighty God, who through thine only-begotten Son Jesus Christ hast overcome death, and opened unto us the gate of everlasting life; We humbly beseech thee, that, as by thy special grace preventing us thou dost put into our minds good desires, so by thy continual help we may bring the same to good effect; through Jesus Christ our Lord, who liveth and reigneth with thee and the Holy Ghost, ever one God, world without end. \R Amen.}

\subsection{\stylesubsec{}{For the Epistle.}{Acts 13.~26.}}
\drop{Men and brethren, children of the stock of Abraham, and whosoever among you feareth God, to you is the word of this salvation sent. For they that dwell at Jerusalem, and their rulers, because they knew him not, nor yet the voices of the prophets which are read every sabbath day, they have fulfilled them in condemning him. And though they found no cause of death in him, yet desired they Pilate that he should be slain. And when they had fulfilled all that was written of him, they took him down from the tree, and laid him in a sepulchre. But God raised him from the dead: And he was seen many days of them which came up with him from Galilee to Jerusalem, who are his witnesses unto the people. And we declare unto you glad tidings, how that the promise which was made unto the fathers, God hath fulfilled the same unto us their children, in that he hath raised up Jesus again; as it is also written in the second psalm, Thou art my Son, this day have I begotten thee. And as concerning that he raised him up from the dead, now no more to return to corruption, he said on this wise, I will give you the sure mercies of David. Wherefore he saith also in another psalm, Thou shalt not suffer thine Holy One to see corruption. For David, after he had served his own generation by the will of God, fell on sleep, and was laid unto his fathers, and saw corruption: But he, whom God raised again, saw no corruption. Be it known unto you therefore, men and brethren, that through this man is preached unto you the forgiveness of sins: And by him all that believe are justified from all things, from which ye could not be justified by the law of Moses. Beware therefore, lest that come upon you, which is spoken of in the prophets; Behold, ye despisers, and wonder, and perish: for I work a work in your days, a work which ye shall in no wise believe, though a man declare it unto you.}

\subsection{\stylesubsec{}{The Gospel.}{St.~Luke 24.~36.}}
\drop{Jesus himself stood in the midst of them, and saith unto them, Peace be unto you. But they were terrified and affrighted, and supposed that they had seen a spirit. And he said unto them, Why are ye troubled? and why do thoughts arise in your hearts? Behold my hands and my feet, that it is I myself: handle me, and see; for a spirit hath not flesh and bones, as ye see me have. And when he had thus spoken, he shewed them his hands and his feet. And while they yet believed not for joy, and wondered, he said unto them, Have ye here any meat? And they gave him a piece of a broiled fish, and of an honeycomb. And he took it, and did eat before them. And he said unto them, These are the words which I spake unto you, while I was yet with you, that all things must be fulfilled, which were written in the law of Moses, and in the prophets, and in the psalms, concerning me. Then opened he their understanding, that they might understand the scriptures, And said unto them, Thus it is written, and thus it behoved Christ to suffer, and to rise from the dead the third day: And that repentance and remission of sins should be preached in his name among all nations, beginning at Jerusalem. And ye are witnesses of these things.}


\section{The First Sunday after Easter}
\subsection{\stylesubsec{}{The Collect.}{}}
\drop{Almighty Father, who has given thine only Son to die for our sins, and to rise again for our justification; Grant us so to put away the leaven of malice and wickedness, that we may alway serve thee in pureness of living and truth; through the merits of the same thy Son Jesus Christ our Lord. \R Amen.}

\subsection{\stylesubsec{}{The Epistle.}{1 St.~John 5.~4.}}
\drop{Whatsoever is born of God overcometh the world: and this is the victory that overcometh the world, even our faith. Who is he that overcometh the world, but he that believeth that Jesus is the Son of God? This is he that came by water and blood, even Jesus Christ; not by water only, but by water and blood. And it is the Spirit that beareth witness, because the Spirit is truth. For there are three that bear record in heaven, the Father, the Word, and the Holy Ghost: and these three are one. And there are three that bear witness in earth, the Spirit, and the water, and the blood: and these three agree in one. If we receive the witness of men, the witness of God is greater: for this is the witness of God which he hath testified of his Son. He that believeth on the Son of God hath the witness in himself: he that believeth not God hath made him a liar; because he believeth not the record that God gave of his Son. And this is the record, that God hath given to us eternal life, and this life is in his Son. He that hath the Son hath life; and he that hath not the Son of God hath not life.}

\subsection{\stylesubsec{}{The Gospel.}{St.~John 20.~19.}}
\drop{The same day at evening, being the first day of the week, when the doors were shut where the disciples were assembled for fear of the Jews, came Jesus and stood in the midst, and saith unto them, Peace be unto you. And when he had so said, he shewed unto them his hands and his side. Then were the disciples glad, when they saw the Lord. Then said Jesus to them again, Peace be unto you: as my Father hath sent me, even so send I you. And when he had said this, he breathed on them, and saith unto them, Receive ye the Holy Ghost: Whosesoever sins ye remit, they are remitted unto them; and whosesoever sins ye retain, they are retained.}

\section{The Second Sunday after Easter}
\subsection{\stylesubsec{}{The Collect.}{}}
\drop{Almighty God, who has given thine only Son to be unto us both a sacrifice for sin, and also an ensample of godly life; Give us grace that we may always most thankfully receive that his inestimable benefit, and also daily endeavour ourselves to follow the blessed steps of his most holy life; through the same Jesus Christ our Lord. \R Amen.}

\subsection{\stylesubsec{}{The Epistle.}{1 St.~Peter 2.~19.}}
\drop{This is thankworthy, if a man for conscience toward God endure grief, suffering wrongfully. For what glory is it, if, when ye be buffeted for your faults, ye shall take it patiently? but if, when ye do well, and suffer for it, ye take it patiently, this is acceptable with God. For even hereunto were ye called: because Christ also suffered for us, leaving us an example, that ye should follow his steps: Who did no sin, neither was guile found in his mouth: Who, when he was reviled, reviled not again; when he suffered, he threatened not; but committed himself to him that judgeth righteously: Who his own self bare our sins in his own body on the tree, that we, being dead to sins, should live unto righteousness: by whose stripes ye were healed. For ye were as sheep going astray; but are now returned unto the Shepherd and Bishop of your souls.}

\subsection{\stylesubsec{}{The Gospel.}{St.~John 10.~11.}}
\drop{Jesus said, I am the good shepherd: the good shepherd giveth his life for the sheep. But he that is an hireling, and not the shepherd, whose own the sheep are not, seeth the wolf coming, and leaveth the sheep, and fleeth: and the wolf catcheth them, and scattereth the sheep. The hireling fleeth, because he is an hireling, and careth not for the sheep. I am the good shepherd, and know my sheep, and am known of mine. As the Father knoweth me, even so know I the Father: and I lay down my life for the sheep. And other sheep I have, which are not of this fold: them also I must bring, and they shall hear my voice; and there shall be one fold, and one shepherd.}

\section{The Third Sunday after Easter}
\subsection{\stylesubsec{}{The Collect.}{}}
\drop{Almighty God, who shewest to them that be in error the light of thy truth, to the intent that they may return into the way of righteousness; Grant unto all them that are admitted into the fellowship of Christ's Religion, that they may eschew those things that are contrary to their profession, and follow all such things as are agreeable to the same; through our Lord Jesus Christ. \R Amen.}

\subsection{\stylesubsec{}{The Epistle.}{1 St.~Peter 2.~11.}}
\drop{Dearly beloved, I beseech you as strangers and pilgrims, abstain from fleshly lusts, which war against the soul; Having your conversation honest among the Gentiles: that, whereas they speak against you as evildoers, they may by your good works, which they shall behold, glorify God in the day of visitation. Submit yourselves to every ordinance of man for the Lord's sake: whether it be to the king, as supreme; Or unto governors, as unto them that are sent by him for the punishment of evildoers, and for the praise of them that do well. For so is the will of God, that with well doing ye may put to silence the ignorance of foolish men: As free, and not using your liberty for a cloak of maliciousness, but as the servants of God. Honour all men. Love the brotherhood. Fear God. Honour the King.}

\subsection{\stylesubsec{}{The Gospel.}{St.~John 16.~16.}}
\drop{Jesus said to his disciples, A little while, and ye shall not see me: and again, a little while, and ye shall see me, because I go to the Father. Then said some of his disciples among themselves, What is this that he saith unto us, A little while, and ye shall not see me: and again, a little while, and ye shall see me: and, Because I go to the Father? They said therefore, What is this that he saith, A little while? we cannot tell what he saith. Now Jesus knew that they were desirous to ask him, and said unto them, Do ye enquire among yourselves of that I said, A little while, and ye shall not see me: and again, a little while, and ye shall see me? Verily, verily, I say unto you, That ye shall weep and lament, but the world shall rejoice: and ye shall be sorrowful, but your sorrow shall be turned into joy. A woman when she is in travail hath sorrow, because her hour is come: but as soon as she is delivered of the child, she remembereth no more the anguish, for joy that a man is born into the world. And ye now therefore have sorrow: but I will see you again, and your heart shall rejoice, and your joy no man taketh from you.}

\section{The Fourth Sunday after Easter}
\subsection{\stylesubsec{}{The Collect.}{}}
\drop{O almighty God, who alone canst order the unruly wills and affections of sinful men; Grant unto thy people, that they may love the thing which thou commandest, and desire that which thou dost promise; that so, among the sundry and manifold changes of the world, our hearts may surely there be fixed, where true joys are to be found; through Jesus Christ our Lord. \R Amen.}

\subsection{\stylesubsec{}{The Epistle.}{St.~James 1.~17.}}
\drop{Every good gift and every perfect gift is from above, and cometh down from the Father of lights, with whom is no variableness, neither shadow of turning. Of his own will begat he us with the word of truth, that we should be a kind of firstfruits of his creatures. Wherefore, my beloved brethren, let every man be swift to hear, slow to speak, slow to wrath: For the wrath of man worketh not the righteousness of God. Wherefore lay apart all filthiness and superfluity of naughtiness, and receive with meekness the engrafted word, which is able to save your souls.}

\subsection{\stylesubsec{}{The Gospel.}{St.~John 16.~5.}}
\drop{Jesus said unto his disciples, Now I go my way to him that sent me; and none of you asketh me, Whither goest thou? But because I have said these things unto you, sorrow hath filled your heart. Nevertheless I tell you the truth; It is expedient for you that I go away: for if I go not away, the Comforter will not come unto you; but if I depart, I will send him unto you. And when he is come, he will reprove the world of sin, and of righteousness, and of judgment: Of sin, because they believe not on me; Of righteousness, because I go to my Father, and ye see me no more; Of judgment, because the prince of this world is judged. I have yet many things to say unto you, but ye cannot bear them now. Howbeit when he, the Spirit of truth, is come, he will guide you into all truth: for he shall not speak of himself; but whatsoever he shall hear, that shall he speak: and he will shew you things to come. He shall glorify me: for he shall receive of mine, and shall shew it unto you.}


\section{The Fifth Sunday after Easter}
\subsection{\stylesubsec{}{The Collect.}{}}
\drop{O Lord, from whom all good things do come; Grant to us thy humble servants, that by thy holy inspiration we may think those things that be good, and by thy merciful guiding may perform the same; through our Lord Jesus Christ. \R Amen.}

\subsection{\stylesubsec{}{The Epistle.}{St.~James 1.~22.}}
\drop{Be ye doers of the word, and not hearers only, deceiving your own selves. For if any be a hearer of the word, and not a doer, he is like unto a man beholding his natural face in a glass: For he beholdeth himself, and goeth his way, and straightway forgetteth what manner of man he was. But whoso looketh into the perfect law of liberty, and continueth therein, he being not a forgetful hearer, but a doer of the work, this man shall be blessed in his deed. If any man among you seem to be religious, and bridleth not his tongue, but deceiveth his own heart, this man's religion is vain. Pure religion and undefiled before God and the Father is this, To visit the fatherless and widows in their affliction, and to keep himself unspotted from the world.}

\subsection{\stylesubsec{}{The Gospel.}{St.~John 16.~23.}}
\drop{Verily, verily, I say unto you, Whatsoever ye shall ask the Father in my name, he will give it you. Hitherto have ye asked nothing in my name: ask, and ye shall receive, that your joy may be full. These things have I spoken unto you in proverbs: but the time cometh, when I shall no more speak unto you in proverbs, but I shall shew you plainly of the Father. At that day ye shall ask in my name: and I say not unto you, that I will pray the Father for you: For the Father himself loveth you, because ye have loved me, and have believed that I came out from God. I came forth from the Father, and am come into the world: again, I leave the world, and go to the Father. His disciples said unto him, Lo, now speakest thou plainly, and speakest no proverb. Now are we sure that thou knowest all things, and needest not that any man should ask thee: by this we believe that thou camest forth from God. Jesus answered them, Do ye now believe? Behold, the hour cometh, yea, is now come, that ye shall be scattered, every man to his own, and shall leave me alone: and yet I am not alone, because the Father is with me. These things I have spoken unto you, that in me ye might have peace. In the world ye shall have tribulation: but be of good cheer; I have overcome the world.}

\section{The Ascension Day}
\subsection{\stylesubsec{}{The Collect.}{}}
\drop{Grant, we beseech thee, Almighty God, that like as we do believe thy only-begotten Son our Lord Jesus Christ to have ascended into the heavens; so we may also in heart and mind thither ascend, and with him continually dwell, who liveth and reigneth with thee and the Holy Ghost, one God, world without end. \R Amen.}

\subsection{\stylesubsec{}{For the Epistle.}{Acts 1.~1.}}
\drop{The former treatise have I made, O Theophilus, of all that Jesus began both to do and teach, Until the day in which he was taken up, after that he through the Holy Ghost had given commandments unto the apostles whom he had chosen: To whom also he shewed himself alive after his passion by many infallible proofs, being seen of them forty days, and speaking of the things pertaining to the kingdom of God: And, being assembled together with them, commanded them that they should not depart from Jerusalem, but wait for the promise of the Father, which, saith he, ye have heard of me. For John truly baptized with water; but ye shall be baptized with the Holy Ghost not many days hence. When they therefore were come together, they asked of him, saying, Lord, wilt thou at this time restore again the kingdom to Israel? And he said unto them, It is not for you to know the times or the seasons, which the Father hath put in his own power. But ye shall receive power, after that the Holy Ghost is come upon you: and ye shall be witnesses unto me both in Jerusalem, and in all Judaea, and in Samaria, and unto the uttermost part of the earth. And when he had spoken these things, while they beheld, he was taken up; and a cloud received him out of their sight. And while they looked stedfastly toward heaven as he went up, behold, two men stood by them in white apparel; Which also said, Ye men of Galilee, why stand ye gazing up into heaven? this same Jesus, which is taken up from you into heaven, shall so come in like manner as ye have seen him go into heaven.}

\subsection{\stylesubsec{}{The Gospel.}{St.~Mark 16.~14.}}
\drop{Jesus appeared unto the eleven as they sat at meat, and upbraided them with their unbelief and hardness of heart, because they believed not them which had seen him after he was risen. And he said unto them, Go ye into all the world, and preach the Gospel to every creature. He that believeth and is baptized shall be saved; but he that believeth not shall be damned. And these signs shall follow them that believe; In my name shall they cast out devils; they shall speak with new tongues; They shall take up serpents; and if they drink any deadly thing, it shall not hurt them; they shall lay hands on the sick, and they shall recover. So then after the Lord had spoken unto them, he was received up into heaven, and sat on the right hand of God. And they went forth, and preached every where, the Lord working with them, and confirming the word with signs following.}

\section{Sunday after Ascension Day}
\subsection{\stylesubsec{}{The Collect.}{}}
\drop{O God the King of glory, who hast exalted thine only Son Jesus Christ with great triumph unto thy kingdom in heaven; We beseech thee, leave us not comfortless; but send to us thine Holy Ghost to comfort us, and exalt us unto the same place whither our Saviour Christ is gone before, who liveth and reigneth with thee and the Holy Ghost, one God, world without end. \R Amen.}

\subsection{\stylesubsec{}{The Epistle.}{1 St.~Peter 4.~7.}}
\drop{The end of all things is at hand: be ye therefore sober, and watch unto prayer. And above all things have fervent charity among yourselves: for charity shall cover the multitude of sins. Use hospitality one to another without grudging. As every man hath received the gift, even so minister the same one to another, as good stewards of the manifold grace of God. If any man speak, let him speak as the oracles of God; if any man minister, let him do it as of the ability which God giveth: that God in all things may be glorified through Jesus Christ, to whom be praise and dominion for ever and ever. Amen.}

\subsection{\stylesubsec{}{The Gospel.}{St.~John 15.~26, and part of Chapter 16.}}
\drop{When the Comforter is come, whom I will send unto you from the Father, even the Spirit of truth, which proceedeth from the Father, he shall testify of me: And ye also shall bear witness, because ye have been with me from the beginning. These things have I spoken unto you, that ye should not be offended. They shall put you out of the synagogues: yea, the time cometh, that whosoever killeth you will think that he doeth God service. And these things will they do unto you, because they have not known the Father, nor me. But these things have I told you, that when the time shall come, ye may remember that I told you of them.}

\section{Whitsunday}
\subsection{\stylesubsec{}{The Collect.}{}}
\drop{God, who as at this time didst teach the hearts of thy faithful people, by the sending to them the light of thy Holy Spirit; Grant us by the same Spirit to have a right judgement in all things, and evermore to rejoice in his holy comfort; through the merits of Christ Jesus our Saviour, who liveth and reigneth with thee, in the unity of the same Spirit, one God, world without end. \R Amen.}

\subsection{\stylesubsec{}{For the Epistle.}{Acts 2.~1.}}
\drop{When the day of Pentecost was fully come, they were all with one accord in one place. And suddenly there came a sound from heaven as of a rushing mighty wind, and it filled all the house where they were sitting. And there appeared unto them cloven tongues like as of fire, and it sat upon each of them. And they were all filled with the Holy Ghost, and began to speak with other tongues, as the Spirit gave them utterance. And there were dwelling at Jerusalem Jews, devout men, out of every nation under heaven. Now when this was noised abroad, the multitude came together, and were confounded, because that every man heard them speak in his own language. And they were all amazed and marvelled, saying one to another, Behold, are not all these which speak Galilaeans? And how hear we every man in our own tongue, wherein we were born? Parthians, and Medes, and Elamites, and the dwellers in Mesopotamia, and in Judaea, and Cappadocia, in Pontus, and Asia, Phrygia, and Pamphylia, in Egypt, and in the parts of Libya about Cyrene, and strangers of Rome, Jews and proselytes, Cretes and Arabians, we do hear them speak in our tongues the wonderful works of God.}

\subsection{\stylesubsec{}{The Gospel.}{St.~John 14.~15.}}
\drop{Jesus said unto his disciples, If ye love me, keep my commandments. And I will pray the Father, and he shall give you another Comforter, that he may abide with you for ever; Even the Spirit of truth; whom the world cannot receive, because it seeth him not, neither knoweth him: but ye know him; for he dwelleth with you, and shall be in you. I will not leave you comfortless: I will come to you. Yet a little while, and the world seeth me no more; but ye see me: because I live, ye shall live also. At that day ye shall know that I am in my Father, and ye in me, and I in you. He that hath my commandments, and keepeth them, he it is that loveth me: and he that loveth me shall be loved of my Father, and I will love him, and will manifest myself to him. Judas saith unto him, not Iscariot, Lord, how is it that thou wilt manifest thyself unto us, and not unto the world? Jesus answered and said unto him, If a man love me, he will keep my words: and my Father will love him, and we will come unto him, and make our abode with him. He that loveth me not keepeth not my sayings: and the word which ye hear is not mine, but the Father's which sent me. These things have I spoken unto you, being yet present with you. But the Comforter, which is the Holy Ghost, whom the Father will send in my name, he shall teach you all things, and bring all things to your remembrance, whatsoever I have said unto you. Peace I leave with you, my peace I give unto you: not as the world giveth, give I unto you. Let not your heart be troubled, neither let it be afraid. Ye have heard how I said unto you, I go away, and come again unto you. If ye loved me, ye would rejoice, because I said, I go unto the Father: for my Father is greater than I. And now I have told you before it come to pass, that, when it is come to pass, ye might believe. Hereafter I will not talk much with you: for the prince of this world cometh, and hath nothing in me. But that the world may know that I love the Father; and as the Father gave me commandment, even so I do.}


\section{Monday in Whitsun Week}
\subsection{\stylesubsec{}{The Collect.}{}}
OGOD, who as at this time didst teach the hearts of thy faithful people, by sending to them the light of thy Holy Spirit; Grant us by the same Spirit to have a right judgement in all things, and evermore to rejoice in his holy comfort; through the merits of Christ Jesus our Savior, who liveth and reigneth with thee, in the unity of the same Spirit, one God, world without end. Amen.

For the Epistle. Acts 10. 34.
\drop{Then Peter opened his mouth, and said, Of a truth I perceive that God is no respecter of persons: But in every nation he that feareth him, and worketh righteousness, is accepted with him. The word which God sent unto the children of Israel, preaching peace by Jesus Christ: (he is Lord of all:) That word, I say, ye know, which was published throughout all Judaea, and began from Galilee, after the baptism which John preached; How God anointed Jesus of Nazareth with the Holy Ghost and with power: who went about doing good, and healing all that were oppressed of the devil; for God was with him. And we are witnesses of all things which he did both in the land of the Jews, and in Jerusalem; whom they slew and hanged on a tree: Him God raised up the third day, and shewed him openly; Not to all the people, but unto witnesses chosen before God, even to us, who did eat and drink with him after he rose from the dead. And he commanded us to preach unto the people, and to testify that it is he which was ordained of God to be the Judge of quick and dead. To him give all the prophets witness, that through his name whosoever believeth in him shall receive remission of sins. While Peter yet spake these words, the Holy Ghost fell on all them which heard the word. And they of the circumcision which believed were astonished, as many as came with Peter, because that on the Gentiles also was poured out the gift of the Holy Ghost. For they heard them speak with tongues, and magnify God. Then answered Peter, Can any man forbid water, that these should not be baptized, which have received the Holy Ghost as well as we? And he commanded them to be baptized in the name of the Lord. Then prayed they him to tarry certain days.}

The Gospel. St. John 3. 16.
\drop{God so loved the world, that he gave his only begotten Son, that whosoever believeth in him should not perish, but have everlasting life. For God sent not his Son into the world to condemn the world; but that the world through him might be saved. He that believeth on him is not condemned: but he that believeth not is condemned already, because he hath not believed in the name of the only begotten Son of God. And this is the condemnation, that light is come into the world, and men loved darkness rather than light, because their deeds were evil. For every one that doeth evil hateth the light, neither cometh to the light, lest his deeds should be reproved. But he that doeth truth cometh to the light, that his deeds may be made manifest, that they are wrought in God.}

\section{Tuesday in Whitsun Week}
\subsection{\stylesubsec{}{The Collect.}{}}
\drop{O God, who as at this time didst teach the hearts of thy faithful people, by sending to them the light of thy Holy Spirit; Grant us by the same Spirit to have a right judgement in all things, and evermore to rejoice in his holy comfort; through the merits of Christ Jesus our Savior, who liveth and reigneth with thee, in the unity of the same Spirit, one God, world without end. \R Amen.}

For the Epistle. Acts 8. 14.
\drop{When the apostles which were at Jerusalem heard that Samaria had received the word of God, they sent unto them Peter and John: Who, when they were come down, prayed for them, that they might receive the Holy Ghost: (For as yet he was fallen upon none of them: only they were baptized in the name of the Lord Jesus.) Then laid they their hands on them, and they received the Holy Ghost. And when Simon saw that through laying on of the apostles' hands the Holy Ghost was given, he offered them money, Saying, Give me also this power, that on whomsoever I lay hands, he may receive the Holy Ghost.}

The Gospel. St. John 10. 1.
\drop{Verily, verily, I say unto you, He that entereth not by the door into the sheepfold, but climbeth up some other way, the same is a thief and a robber. But he that entereth in by the door is the shepherd of the sheep. To him the porter openeth; and the sheep hear his voice: and he calleth his own sheep by name, and leadeth them out. And when he putteth forth his own sheep, he goeth before them, and the sheep follow him: for they know his voice. And a stranger will they not follow, but will flee from him: for they know not the voice of strangers. This parable spake Jesus unto them: but they understood not what things they were which he spake unto them. Then said Jesus unto them again, Verily, verily, I say unto you, I am the door of the sheep. All that ever came before me are thieves and robbers: but the sheep did not hear them. I am the door: by me if any man enter in, he shall be saved, and shall go in and out, and find pasture. The thief cometh not, but for to steal, and to kill, and to destroy: I am come that they might have life, and that they might have it more abundantly.}

Trinity-Sunday.
\subsection{\stylesubsec{}{The Collect.}{}}
\drop{Almighty and everlasting God, who hast given unto us thy servants grace, by the confession of a true faith to acknowledge the glory of the eternal Trinity, and in the power of thy Divine Majesty to worship the Unity; We beseech thee, that thou wouldst keep us steadfast in this faith, and evermore defend us from all adversities, who livest and reignest, one God, world without end. \R Amen.}

For the Epistle. Rev. 4. 1.
\drop{After this I looked, and, behold, a door was opened in heaven: and the first voice which I heard was as it were of a trumpet talking with me; which said, Come up hither, and I will shew thee things which must be hereafter. And immediately I was in the spirit: and, behold, a throne was set in heaven, and one sat on the throne. And he that sat was to look upon like a jasper and a sardine stone: and there was a rainbow round about the throne, in sight like unto an emerald. And round about the throne were four and twenty seats: and upon the seats I saw four and twenty elders sitting, clothed in white raiment; and they had on their heads crowns of gold. And out of the throne proceeded lightnings and thunderings and voices: and there were seven lamps of fire burning before the throne, which are the seven Spirits of God. And before the throne there was a sea of glass like unto crystal: and in the midst of the throne, and round about the throne, were four beasts full of eyes before and behind. And the first beast was like a lion, and the second beast like a calf, and the third beast had a face as a man, and the fourth beast was like a flying eagle. And the four beasts had each of them six wings about him; and they were full of eyes within: and they rest not day and night, saying, Holy, holy, holy, Lord God Almighty, which was, and is, and is to come. And when those beasts give glory and honour and thanks to him that sat on the throne, who liveth for ever and ever, The four and twenty elders fall down before him that sat on the throne, and worship him that liveth for ever and ever, and cast their crowns before the throne, saying, Thou art worthy, O Lord, to receive glory and honour and power: for thou hast created all things, and for thy pleasure they are and were created.}

The Gospel. St. John 3. 1.
\drop{There was a man of the Pharisees, named Nicodemus, a ruler of the Jews: The same came to Jesus by night, and said unto him, Rabbi, we know that thou art a teacher come from God: for no man can do these miracles that thou doest, except God be with him. Jesus answered and said unto him, Verily, verily, I say unto thee, Except a man be born again, he cannot see the kingdom of God. Nicodemus saith unto him, How can a man be born when he is old? can he enter the second time into his mother's womb, and be born? Jesus answered, Verily, verily, I say unto thee, Except a man be born of water and of the Spirit, he cannot enter into the kingdom of God. That which is born of the flesh is flesh; and that which is born of the Spirit is spirit. Marvel not that I said unto thee, Ye must be born again. The wind bloweth where it listeth, and thou hearest the sound thereof, but canst not tell whence it cometh, and whither it goeth: so is every one that is born of the Spirit. Nicodemus answered and said unto him, How can these things be? Jesus answered and said unto him, Art thou a master of Israel, and knowest not these things? Verily, verily, I say unto thee, We speak that we do know, and testify that we have seen; and ye receive not our witness. If I have told you earthly things, and ye believe not, how shall ye believe, if I tell you of heavenly things? And no man hath ascended up to heaven, but he that came down from heaven, even the Son of man which is in heaven. And as Moses lifted up the serpent in the wilderness, even so must the Son of man be lifted up: That whosoever believeth in him should not perish, but have eternal life.}

The First Sunday after Trinity.
\subsection{\stylesubsec{}{The Collect.}{}}
OGOD, the strength of all them that put their trust in thee, mercifully accept our prayers; and because through the weakness of our mortal nature we can do no good thing without thee, grant us the help of thy grace, that in keeping of thy commandments we may please thee, both in will and deed; through Jesus Christ our Lord. Amen.
The Epistle. 1 St. John 4. 7.
BELOVED, let us love one another: for love is of God; and every one that loveth is born of God, and knoweth God. He that loveth not knoweth not God; for God is love. In this was manifested the love of God toward us, because that God sent his only begotten Son into the world, that we might live through him. Herein is love, not that we loved God, but that he loved us, and sent his Son to be the propitiation for our sins. Beloved, if God so loved us, we ought also to love one another. No man hath seen God at any time. If we love one another, God dwelleth in us, and his love is perfected in us. Hereby know we that we dwell in him, and he in us, because he hath given us of his Spirit. And we have seen and do testify that the Father sent the Son to be the Saviour of the world. Whosoever shall confess that Jesus is the Son of God, God dwelleth in him, and he in God. And we have known and believed the love that God hath to us. God is love; and he that dwelleth in love dwelleth in God, and God in him. Herein is our love made perfect, that we may have boldness in the day of judgment: because as he is, so are we in this world. There is no fear in love; but perfect love casteth out fear: because fear hath torment. He that feareth is not made perfect in love. We love him, because he first loved us. If a man say, I love God, and hateth his brother, he is a liar: for he that loveth not his brother whom he hath seen, how can he love God whom he hath not seen? And this commandment have we from him, That he who loveth God love his brother also.
The Gospel. St. Luke 16. 19.
THERE was a certain rich man, which was clothed in purple and fine linen, and fared sumptuously every day: And there was a certain beggar named Lazarus, which was laid at his gate, full of sores, And desiring to be fed with the crumbs which fell from the rich man's table: moreover the dogs came and licked his sores. And it came to pass, that the beggar died, and was carried by the angels into Abraham's bosom: the rich man also died, and was buried; And in hell he lift up his eyes, being in torments, and seeth Abraham afar off, and Lazarus in his bosom. And he cried and said, Father Abraham, have mercy on me, and send Lazarus, that he may dip the tip of his finger in water, and cool my tongue; for I am tormented in this flame. But Abraham said, Son, remember that thou in thy lifetime receivedst thy good things, and likewise Lazarus evil things: but now he is comforted, and thou art tormented. And beside all this, between us and you there is a great gulf fixed: so that they which would pass from hence to you cannot; neither can they pass to us, that would come from thence. Then he said, I pray thee therefore, father, that thou wouldest send him to my father's house: For I have five brethren; that he may testify unto them, lest they also come into this place of torment. Abraham saith unto him, They have Moses and the prophets; let them hear them. And he said, Nay, father Abraham: but if one went unto them from the dead, they will repent. And he said unto him, If they hear not Moses and the prophets, neither will they be persuaded, though one rose from the dead.
The Second Sunday after Trinity.
\subsection{\stylesubsec{}{The Collect.}{}}
OLORD, who never failest to help and govern them who thou dost bring up in thy stedfast fear and love; Keep us, we beseech thee, under the protection of thy good providence, and make us to have a perpetual fear and love of thy holy Name; through Jesus Christ our Lord. Amen.
The Epistle. 1 St. John 3. 13.
MARVEL not, my brethren, if the world hate you. We know that we have passed from death unto life, because we love the brethren. He that loveth not his brother abideth in death. Whosoever hateth his brother is a murderer: and ye know that no murderer hath eternal life abiding in him. Hereby perceive we the love of God, because he laid down his life for us: and we ought to lay down our lives for the brethren. But whoso hath this world's good, and seeth his brother have need, and shutteth up his bowels of compassion from him, how dwelleth the love of God in him? My little children, let us not love in word, neither in tongue; but in deed and in truth. And hereby we know that we are of the truth, and shall assure our hearts before him. For if our heart condemn us, God is greater than our heart, and knoweth all things. Beloved, if our heart condemn us not, then have we confidence toward God. And whatsoever we ask, we receive of him, because we keep his commandments, and do those things that are pleasing in his sight. And this is his commandment, That we should believe on the name of his Son Jesus Christ, and love one another, as he gave us commandment. And he that keepeth his commandments dwelleth in him, and he in him. And hereby we know that he abideth in us, by the Spirit which he hath given us.

The Gospel. St. Luke 14. 16.
\drop{A certain man made a great supper, and bade many: And sent his servant at supper time to say to them that were bidden, Come; for all things are now ready. And they all with one consent began to make excuse. The first said unto him, I have bought a piece of ground, and I must needs go and see it: I pray thee have me excused. And another said, I have bought five yoke of oxen, and I go to prove them: I pray thee have me excused. And another said, I have married a wife, and therefore I cannot come. So that servant came, and shewed his Lord these things. Then the master of the house being angry said to his servant, Go out quickly into the streets and lanes of the city, and bring in hither the poor, and the maimed, and the halt, and the blind. And the servant said, Lord, it is done as thou hast commanded, and yet there is room. And the Lord said unto the servant, Go out into the highways and hedges, and compel them to come in, that my house may be filled. For I say unto you, That none of those men which were bidden shall taste of my supper.}


The Third Sunday after Trinity.
\subsection{\stylesubsec{}{The Collect.}{}}
OLORD, we beseech thee mercifully to hear us; and grant that we, to whom thou hast given an hearty desire to pray, may by thy mighty aid be defended and comforted in all dangers and adversities; through Jesus Christ our Lord. Amen.
The Epistle. 1 St. Peter 5. 5.
ALL of you be subject one to another, and be clothed with humility: for God resisteth the proud, and giveth grace to the humble. Humble yourselves therefore under the mighty hand of God, that he may exalt you in due time: Casting all your care upon him; for he careth for you. Be sober, be vigilant; because your adversary the devil, as a roaring lion, walketh about, seeking whom he may devour: Whom resist stedfast in the faith, knowing that the same afflictions are accomplished in your brethren that are in the world. But the God of all grace, who hath called us unto his eternal glory by Christ Jesus, after that ye have suffered a while, make you perfect, stablish, strengthen, settle you. To him be glory and dominion for ever and ever. Amen.
The Gospel. St. Luke 15. 1.
THEN drew near unto him all the publicans and sinners for to hear him. And the Pharisees and scribes murmured, saying, This man receiveth sinners, and eateth with them. And he spake this parable unto them, saying, What man of you, having an hundred sheep, if he lose one of them, doth not leave the ninety and nine in the wilderness, and go after that which is lost, until he find it? And when he hath found it, he layeth it on his shoulders, rejoicing. And when he cometh home, he calleth together his friends and neighbours, saying unto them, Rejoice with me; for I have found my sheep which was lost. I say unto you, that likewise joy shall be in heaven over one sinner that repenteth, more than over ninety and nine just persons, which need no repentance. Either what woman having ten pieces of silver, if she lose one piece, doth not light a candle, and sweep the house, and seek diligently till she find it? And when she hath found it, she calleth her friends and her neighbours together, saying, Rejoice with me; for I have found the piece which I had lost. Likewise, I say unto you, there is joy in the presence of the angels of God over one sinner that repenteth.
The Fourth Sunday after Trinity.
\subsection{\stylesubsec{}{The Collect.}{}}
OGOD, the protector of all that trust in thee, without whom nothing is strong, nothing is holy; Increase and multiply upon us thy mercy; that, thou being our ruler and guide, we may so pass through things temporal, that we finally lose not the things eternal: Grant this, O heavenly Father, for Jesus Christ's sake our Lord. Amen.
The Epistle. Rom. 8. 18.
IRECKON that the sufferings of this present time are not worthy to be compared with the glory which shall be revealed in us. For the earnest expectation of the creature waiteth for the manifestation of the sons of God. For the creature was made subject to vanity, not willingly, but by reason of him who hath subjected the same in hope, Because the creature itself also shall be delivered from the bondage of corruption into the glorious liberty of the children of God. For we know that the whole creation groaneth and travaileth in pain together until now. And not only they, but ourselves also, which have the firstfruits of the Spirit, even we ourselves groan within ourselves, waiting for the adoption, to wit, the redemption of our body.
The Gospel. St. Luke 6. 36.
BE ye therefore merciful, as your Father also is merciful. Judge not, and ye shall not be judged: condemn not, and ye shall not be condemned: forgive, and ye shall be forgiven: Give, and it shall be given unto you; good measure, pressed down, and shaken together, and running over, shall men give into your bosom. For with the same measure that ye mete withal it shall be measured to you again. And he spake a parable unto them, Can the blind lead the blind? shall they not both fall into the ditch? The disciple is not above his master: but every one that is perfect shall be as his master. And why beholdest thou the mote that is in thy brother's eye, but perceivest not the beam that is in thine own eye? Either how canst thou say to thy brother, Brother, let me pull out the mote that is in thine eye, when thou thyself beholdest not the beam that is in thine own eye? Thou hypocrite, cast out first the beam out of thine own eye, and then shalt thou see clearly to pull out the mote that is in thy brother's eye.
The Fifth Sunday after Trinity.
\subsection{\stylesubsec{}{The Collect.}{}}
GRANT, O Lord, we beseech thee, that the course of this world may be so peaceably ordered by thy governance, that thy Church may joyfully serve thee in all godly quietness; through Jesus Christ our Lord. Amen.
The Epistle. 1 St. Peter 3. 8.
BE ye all of one mind, having compassion one of another, love as brethren, be pitiful, be courteous: Not rendering evil for evil, or railing for railing: but contrariwise blessing; knowing that ye are thereunto called, that ye should inherit a blessing. For he that will love life, and see good days, let him refrain his tongue from evil, and his lips that they speak no guile: Let him eschew evil, and do good; let him seek peace, and ensue it. For the eyes of the Lord are over the righteous, and his ears are open unto their prayers: but the face of the Lord is against them that do evil. And who is he that will harm you, if ye be followers of that which is good? But and if ye suffer for righteousness' sake, happy are ye: and be not afraid of their terror, neither be troubled; But sanctify the Lord God in your hearts.
The Gospel. St. Luke 5. 1.
IT came to pass, that, as the people pressed upon him to hear the word of God, he stood by the lake of Gennesaret, And saw two ships standing by the lake: but the fishermen were gone out of them, and were washing their nets. And he entered into one of the ships, which was Simon's, and prayed him that he would thrust out a little from the land. And he sat down, and taught the people out of the ship. Now when he had left speaking, he said unto Simon, Launch out into the deep, and let down your nets for a draught. And Simon answering said unto him, Master, we have toiled all the night, and have taken nothing: nevertheless at thy word I will let down the net. And when they had this done, they inclosed a great multitude of fishes: and their net brake. And they beckoned unto their partners, which were in the other ship, that they should come and help them. And they came, and filled both the ships, so that they began to sink. When Simon Peter saw it, he fell down at Jesus' knees, saying, Depart from me; for I am a sinful man, O Lord. For he was astonished, and all that were with him, at the draught of the fishes which they had taken: And so was also James, and John, the sons of Zebedee, which were partners with Simon. And Jesus said unto Simon, Fear not; from henceforth thou shalt catch men. And when they had brought their ships to land, they forsook all, and followed him.
The Sixth Sunday after Trinity.
\subsection{\stylesubsec{}{The Collect.}{}}
OGOD, who hast prepared for them that love thee such good things as pass man's understanding; Pour into our hearts such love toward thee, that wee, loving thee above all things, may obtain thy promises, which exceed all that we can desire; through Jesus Christ our Lord. Amen.

The Epistle. Rom. 6. 3.
\drop{Know ye not, that so many of us as were baptized into Jesus Christ were baptized into his death? Therefore we are buried with him by baptism into death: that like as Christ was raised up from the dead by the glory of the Father, even so we also should walk in newness of life. For if we have been planted together in the likeness of his death, we shall be also in the likeness of his resurrection: Knowing this, that our old man is crucified with him, that the body of sin might be destroyed, that henceforth we should not serve sin. For he that is dead is freed from sin. Now if we be dead with Christ, we believe that we shall also live with him: Knowing that Christ being raised from the dead dieth no more; death hath no more dominion over him. For in that he died, he died unto sin once: but in that he liveth, he liveth unto God. Likewise reckon ye also yourselves to be dead indeed unto sin, but alive unto God through Jesus Christ our Lord.}

The Gospel. St. Matth. 5. 20.
\drop{Jesus said unto his disciples, Except your righteousness shall exceed the righteousness of the scribes and Pharisees, ye shall in no case enter into the kingdom of heaven. Ye have heard that it was said of them of old time, Thou shalt not kill; and whosoever shall kill shall be in danger of the judgment: But I say unto you, That whosoever is angry with his brother without a cause shall be in danger of the judgment: and whosoever shall say to his brother, Raca, shall be in danger of the council: but whosoever shall say, Thou fool, shall be in danger of hell fire. Therefore if thou bring thy gift to the altar, and there rememberest that thy brother hath ought against thee; Leave there thy gift before the altar, and go thy way; first be reconciled to thy brother, and then come and offer thy gift. Agree with thine adversary quickly, whiles thou art in the way with him; lest at any time the adversary deliver thee to the judge, and the judge deliver thee to the officer, and thou be cast into prison. Verily I say unto thee, Thou shalt by no means come out thence, till thou hast paid the uttermost farthing.}


The Seventh Sunday after Trinity.
\subsection{\stylesubsec{}{The Collect.}{}}
LORD of all power and might, who art the author and giver of all good things; Graft in our hearts the love of thy Name, increase in us true religion, nourish us with all goodness, and of thy great mercy keep us in the same; through Jesus Christ our Lord. Amen.
The Epistle. Rom. 6. 19.
ISPEAK after the manner of men because of the infirmity of your flesh: for as ye have yielded your members servants to uncleanness and to iniquity unto iniquity; even so now yield your members servants to righteousness unto holiness. For when ye were the servants of sin, ye were free from righteousness. What fruit had ye then in those things whereof ye are now ashamed? for the end of those things is death. But now being made free from sin, and become servants to God, ye have your fruit unto holiness, and the end everlasting life. For the wages of sin is death; but the gift of God is eternal life through Jesus Christ our Lord.
The Gospel. St. Mark 8. 1.
IN those days the multitude being very great, and having nothing to eat, Jesus called his disciples unto him, and saith unto them, I have compassion on the multitude, because they have now been with me three days, and have nothing to eat: And if I send them away fasting to their own houses, they will faint by the way: for divers of them came from far. And his disciples answered him, From whence can a man satisfy these men with bread here in the wilderness? And he asked them, How many loaves have ye? And they said, Seven. And he commanded the people to sit down on the ground: and he took the seven loaves, and gave thanks, and brake, and gave to his disciples to set before them; and they did set them before the people. And they had a few small fishes: and he blessed, and commanded to set them also before them. So they did eat, and were filled: and they took up of the broken meat that was left seven baskets. And they that had eaten were about four thousand: and he sent them away.
The Eighth Sunday after Trinity.
\subsection{\stylesubsec{}{The Collect.}{}}
OGOD, whose never-failing providence ordereth all things both in heaven and earth; We humbly beseech thee to put away from us all hurtful things, and to give us those things which be profitable for us; through Jesus Christ our Lord. Amen.
The Epistle. Rom. 8. 12.
BRETHREN, we are debtors, not to the flesh, to live after the flesh. For if ye live after the flesh, ye shall die: but if ye through the Spirit do mortify the deeds of the body, ye shall live. For as many as are led by the Spirit of God, they are the sons of God. For ye have not received the spirit of bondage again to fear; but ye have received the Spirit of adoption, whereby we cry, Abba, Father. The Spirit itself beareth witness with our spirit, that we are the children of God: And if children, then heirs; heirs of God, and joint-heirs with Christ; if so be that we suffer with him, that we may be also glorified together.
The Gospel. St. Matth. 7. 15.
BEWARE of false prophets, which come to you in sheep's clothing, but inwardly they are ravening wolves. Ye shall know them by their fruits. Do men gather grapes of thorns, or figs of thistles? Even so every good tree bringeth forth good fruit; but a corrupt tree bringeth forth evil fruit. A good tree cannot bring forth evil fruit, neither can a corrupt tree bring forth good fruit. Every tree that bringeth not forth good fruit is hewn down, and cast into the fire. Wherefore by their fruits ye shall know them. Not every one that saith unto me, Lord, Lord, shall enter into the kingdom of heaven; but he that doeth the will of my Father which is in heaven.
The Ninth Sunday after Trinity.
The Collect.
GRANT to us, Lord, we beseech thee, the spirit to think and do always such things as be rightful; that we, who cannot do any thing that is good without thee, may by thee be enabled to live according to thy will; through Jesus Christ our Lord. Amen.
The Epistle. 1 Cor. 10. 1.
BRETHREN, I would not that ye should be ignorant, how that all our fathers were under the cloud, and all passed through the sea; And were all baptized unto Moses in the cloud and in the sea; And did all eat the same spiritual meat; And did all drink the same spiritual drink: for they drank of that spiritual Rock that followed them: and that Rock was Christ. But with many of them God was not well pleased: for they were overthrown in the wilderness. Now these things were our examples, to the intent we should not lust after evil things, as they also lusted. Neither be ye idolaters, as were some of them; as it is written, The people sat down to eat and drink, and rose up to play. Neither let us commit fornication, as some of them committed, and fell in one day three and twenty thousand. Neither let us tempt Christ, as some of them also tempted, and were destroyed of serpents. Neither murmur ye, as some of them also murmured, and were destroyed of the destroyer. Now all these things happened unto them for ensamples: and they are written for our admonition, upon whom the ends of the world are come. Wherefore let him that thinketh he standeth take heed lest he fall. There hath no temptation taken you but such as is common to man: but God is faithful, who will not suffer you to be tempted above that ye are able; but will with the temptation also make a way to escape, that ye may be able to bear it.
The Gospel. St. Luke 16. 1.
JESUS said unto his disciples, There was a certain rich man, which had a steward; and the same was accused unto him that he had wasted his goods. And he called him, and said unto him, How is it that I hear this of thee? give an account of thy stewardship; for thou mayest be no longer steward. Then the steward said within himself, What shall I do? for my Lord taketh away from me the stewardship: I cannot dig; to beg I am ashamed. I am resolved what to do, that, when I am put out of the stewardship, they may receive me into their houses. So he called every one of his Lord's debtors unto him, and said unto the first, How much owest thou unto my Lord? And he said, An hundred measures of oil. And he said unto him, Take thy bill, and sit down quickly, and write fifty. Then said he to another, And how much owest thou? And he said, An hundred measures of wheat. And he said unto him, Take thy bill, and write fourscore. And the Lord commended the unjust steward, because he had done wisely: for the children of this world are in their generation wiser than the children of light. And I say unto you, Make to yourselves friends of the mammon of unrighteousness; that, when ye fail, they may receive you into everlasting habitations.
The Tenth Sunday after Trinity.
\subsection{\stylesubsec{}{The Collect.}{}}
LET thy merciful ears, O Lord, be open to the prayers of thy humble servants; and that they may obtain their petitions make them to ask such things as shall please thee; through Jesus Christ our Lord. Amen.
The Epistle. 1 Cor. 12. 1.
CONCERNING spiritual gifts, brethren, I would not have you ignorant. Ye know that ye were Gentiles, carried away unto these dumb idols, even as ye were led. Wherefore I give you to understand, that no man speaking by the Spirit of God calleth Jesus accursed: and that no man can say that Jesus is the Lord, but by the Holy Ghost. Now there are diversities of gifts, but the same Spirit. And there are differences of administrations, but the same Lord. And there are diversities of operations, but it is the same God which worketh all in all. But the manifestation of the Spirit is given to every man to profit withal. For to one is given by the Spirit the word of wisdom; to another the word of knowledge by the same Spirit; To another faith by the same Spirit; to another the gifts of healing by the same Spirit; To another the working of miracles; to another prophecy; to another discerning of spirits; to another divers kinds of tongues; to another the interpretation of tongues: But all these worketh that one and the selfsame Spirit, dividing to every man severally as he will.
The Gospel. St. Luke 19. 41.
AND when he was come near, he beheld the city, and wept over it, Saying, If thou hadst known, even thou, at least in this thy day, the things which belong unto thy peace! but now they are hid from thine eyes. For the days shall come upon thee, that thine enemies shall cast a trench about thee, and compass thee round, and keep thee in on every side, And shall lay thee even with the ground, and thy children within thee; and they shall not leave in thee one stone upon another; because thou knewest not the time of thy visitation. And he went into the temple, and began to cast out them that sold therein, and them that bought; Saying unto them, It is written, My house is the house of prayer: but ye have made it a den of thieves. And he taught daily in the temple.
The Eleventh Sunday after Trinity.
\subsection{\stylesubsec{}{The Collect.}{}}
OGOD, who declarest thy almighty power most chiefly in shewing mercy and pity; Mercifully grant unto us such a measure of thy grace, that we, running the way of thy commandments, may obtain thy gracious promises, and be made partakers of thy heavenly treasure; through Jesus Christ our Lord. Amen.
The Epistle. 1 Cor. 15. 1.
BRETHREN, I declare unto you the Gospel which I preached unto you, which also ye have received, and wherein ye stand; By which also ye are saved, if ye keep in memory what I preached unto you, unless ye have believed in vain. For I delivered unto you first of all that which I also received, how that Christ died for our sins according to the scriptures; And that he was buried, and that he rose again the third day according to the scriptures: And that he was seen of Cephas, then of the twelve: After that, he was seen of above five hundred brethren at once; of whom the greater part remain unto this present, but some are fallen asleep. After that, he was seen of James; then of all the apostles. And last of all he was seen of me also, as of one born out of due time. For I am the least of the apostles, that am not meet to be called an apostle, because I persecuted the church of God. But by the grace of God I am what I am: and his grace which was bestowed upon me was not in vain; but I laboured more abundantly than they all: yet not I, but the grace of God which was with me. Therefore whether it were I or they, so we preach, and so ye believed.
The Gospel. St. Luke 18. 9.
JESUS spake this parable unto certain which trusted in themselves that they were righteous, and despised others: Two men went up into the temple to pray; the one a Pharisee, and the other a publican. The Pharisee stood and prayed thus with himself, God, I thank thee, that I am not as other men are, extortioners, unjust, adulterers, or even as this publican. I fast twice in the week, I give tithes of all that I possess. And the publican, standing afar off, would not lift up so much as his eyes unto heaven, but smote upon his breast, saying, God be merciful to me a sinner. I tell you, this man went down to his house justified rather than the other: for every one that exalteth himself shall be abased; and he that humbleth himself shall be exalted.
The Twelfth Sunday after Trinity.
\subsection{\stylesubsec{}{The Collect.}{}}
ALMIGHTY and everlasting God, who art always more ready to hear than we are to pray, and art wont to give more than either we desire, or deserve; Pour down upon us the abundance of thy mercy; forgiving us those things whereof our conscience is afraid, and giving us those good things which we are not worthy to ask, but through the merits and mediation of Jesus Christ, thy Son, our Lord. Amen.
The Epistle. 2 Cor. 3. 4.
SUCH trust have we through Christ to God-ward: Not that we are sufficient of ourselves to think any thing as of ourselves; but our sufficiency is of God; Who also hath made us able ministers of the new testament; not of the letter, but of the spirit: for the letter killeth, but the spirit giveth life. But if the ministration of death, written and engraven in stones, was glorious, so that the children of Israel could not stedfastly behold the face of Moses for the glory of his countenance; which glory was to be done away: How shall not the ministration of the spirit be rather glorious? For if the ministration of condemnation be glory, much more doth the ministration of righteousness exceed in glory.
The Gospel. St. Mark 7. 31.
JESUS, departing from the coasts of Tyre and Sidon, he came unto the sea of Galilee, through the midst of the coasts of Decapolis. And they bring unto him one that was deaf, and had an impediment in his speech; and they beseech him to put his hand upon him. And he took him aside from the multitude, and put his fingers into his ears, and he spit, and touched his tongue; And looking up to heaven, he sighed, and saith unto him, Ephphatha, that is, Be opened. And straightway his ears were opened, and the string of his tongue was loosed, and he spake plain. And he charged them that they should tell no man: but the more he charged them, so much the more a great deal they published it; And were beyond measure astonished, saying, He hath done all things well: he maketh both the deaf to hear, and the dumb to speak.
The Thirteenth Sunday after Trinity.
\subsection{\stylesubsec{}{The Collect.}{}}
ALMIGHTY and merciful God, of whose only gift it cometh that thy faithful people do unto thee true and laudable service; Grant, we beseech thee, that we may so faithfully serve thee in this life, that we fail not finally to attain thy heavenly promises; through the merits of Jesus Christ our Lord. Amen.
The Epistle. Gal. 3. 16.
TO Abraham and his seed were the promises made. He saith not, And to seeds, as of many; but as of one, And to thy seed, which is Christ. And this I say, that the covenant, that was confirmed before of God in Christ, the law, which was four hundred and thirty years after, cannot disannul, that it should make the promise of none effect. For if the inheritance be of the law, it is no more of promise: but God gave it to Abraham by promise. Wherefore then serveth the law? It was added because of transgressions, till the seed should come to whom the promise was made; and it was ordained by angels in the hand of a mediator. Now a mediator is not a mediator of one, but God is one. Is the law then against the promises of God? God forbid: for if there had been a law given which could have given life, verily righteousness should have been by the law. But the scripture hath concluded all under sin, that the promise by faith of Jesus Christ might be given to them that believe.
The Gospel. St. Luke 10. 23.
BLESSED are the eyes which see the things that ye see: For I tell you, that many prophets and kings have desired to see those things which ye see, and have not seen them; and to hear those things which ye hear, and have not heard them. And, behold, a certain lawyer stood up, and tempted him, saying, Master, what shall I do to inherit eternal life? He said unto him, What is written in the law? how readest thou? And he answering said, Thou shalt love the Lord thy God with all thy heart, and with all thy soul, and with all thy strength, and with all thy mind; and thy neighbour as thyself. And he said unto him, Thou hast answered right: this do, and thou shalt live. But he, willing to justify himself, said unto Jesus, And who is my neighbour? And Jesus answering said, A certain man went down from Jerusalem to Jericho, and fell among thieves, which stripped him of his raiment, and wounded him, and departed, leaving him half dead. And by chance there came down a certain priest that way: and when he saw him, he passed by on the other side. And likewise a Levite, when he was at the place, came and looked on him, and passed by on the other side. But a certain Samaritan, as he journeyed, came where he was: and when he saw him, he had compassion on him, And went to him, and bound up his wounds, pouring in oil and wine, and set him on his own beast, and brought him to an inn, and took care of him. And on the morrow when he departed, he took out two pence, and gave them to the host, and said unto him, Take care of him; and whatsoever thou spendest more, when I come again, I will repay thee. Which now of these three, thinkest thou, was neighbour unto him that fell among the thieves? And he said, He that shewed mercy on him. Then said Jesus unto him, Go, and do thou likewise.
The Fourteenth Sunday after Trinity.
\subsection{\stylesubsec{}{The Collect.}{}}
ALMIGHTY and everlasting God, give unto us the increase of faith, hope, and charity; and, that we may obtain that which thou dost promise, make us to love that which thou dost command; through Jesus Christ our Lord. Amen.
The Epistle. Gal. 5. 16.
ISAY then, Walk in the Spirit, and ye shall not fulfil the lust of the flesh. For the flesh lusteth against the Spirit, and the Spirit against the flesh: and these are contrary the one to the other: so that ye cannot do the things that ye would. But if ye be led of the Spirit, ye are not under the law. Now the works of the flesh are manifest, which are these; Adultery, fornication, uncleanness, lasciviousness, Idolatry, witchcraft, hatred, variance, emulations, wrath, strife, seditions, heresies, Envyings, murders, drunkenness, revellings, and such like: of the which I tell you before, as I have also told you in time past, that they which do such things shall not inherit the kingdom of God. But the fruit of the Spirit is love, joy, peace, longsuffering, gentleness, goodness, faith, Meekness, temperance: against such there is no law. And they that are Christ's have crucified the flesh with the affections and lusts.
The Gospel. St. Luke 17. 11.
AND it came to pass, as he went to Jerusalem, that he passed through the midst of Samaria and Galilee. And as he entered into a certain village, there met him ten men that were lepers, which stood afar off: And they lifted up their voices, and said, Jesus, Master, have mercy on us. And when he saw them, he said unto them, Go shew yourselves unto the priests. And it came to pass, that, as they went, they were cleansed. And one of them, when he saw that he was healed, turned back, and with a loud voice glorified God, And fell down on his face at his feet, giving him thanks: and he was a Samaritan. And Jesus answering said, Were there not ten cleansed? but where are the nine? There are not found that returned to give glory to God, save this stranger. And he said unto him, Arise, go thy way: thy faith hath made thee whole.
The Fifteenth Sunday after Trinity.
\subsection{\stylesubsec{}{The Collect.}{}}
KEEP, we beseech thee, O Lord, thy Church with thy perpetual mercy: and, because the frailty of man without thee cannot but fall, keep us ever by thy help from all things hurtful, and lead us to all things profitable to our salvation; through Jesus Christ our Lord. Amen.
The Epistle. Gal. 6. 11.
YE see how large a letter I have written unto you with mine own hand. As many as desire to make a fair shew in the flesh, they constrain you to be circumcised; only lest they should suffer persecution for the cross of Christ. For neither they themselves who are circumcised keep the law; but desire to have you circumcised, that they may glory in your flesh. But God forbid that I should glory, save in the cross of our Lord Jesus Christ, by whom the world is crucified unto me, and I unto the world. For in Christ Jesus neither circumcision availeth any thing, nor uncircumcision, but a new creature. And as many as walk according to this rule, peace be on them, and mercy, and upon the Israel of God. From henceforth let no man trouble me: for I bear in my body the marks of the Lord Jesus. Brethren, the grace of our Lord Jesus Christ be with your spirit. Amen.
The Gospel. St. Matth. 6. 24.
NO man can serve two masters: for either he will hate the one, and love the other; or else he will hold to the one, and despise the other. Ye cannot serve God and mammon. Therefore I say unto you, Take no thought for your life, what ye shall eat, or what ye shall drink; nor yet for your body, what ye shall put on. Is not the life more than meat, and the body than raiment? Behold the fowls of the air: for they sow not, neither do they reap, nor gather into barns; yet your heavenly Father feedeth them. Are ye not much better than they? Which of you by taking thought can add one cubit unto his stature? And why take ye thought for raiment? Consider the lilies of the field, how they grow; they toil not, neither do they spin: And yet I say unto you, That even Solomon in all his glory was not arrayed like one of these. Wherefore, if God so clothe the grass of the field, which to day is, and to morrow is cast into the oven, shall he not much more clothe you, O ye of little faith? Therefore take no thought, saying, What shall we eat? or, What shall we drink? or, Wherewithal shall we be clothed? (For after all these things do the Gentiles seek:) for your heavenly Father knoweth that ye have need of all these things. But seek ye first the kingdom of God, and his righteousness; and all these things shall be added unto you. Take therefore no thought for the morrow: for the morrow shall take thought for the things of itself. Sufficient unto the day is the evil thereof.
The Sixteenth Sunday after Trinity.
\subsection{\stylesubsec{}{The Collect.}{}}
OLORD, we beseech thee, let thy continual pity cleanse and defend thy Church; and, because it cannot continue in safety without thy succour, preserve it evermore by thy help and goodness; through Jesus Christ our Lord. Amen.
The Epistle. Ephes. 3. 13.
IDESIRE that ye faint not at my tribulations for you, which is your glory. For this cause I bow my knees unto the Father of our Lord Jesus Christ, Of whom the whole family in heaven and earth is named, That he would grant you, according to the riches of his glory, to be strengthened with might by his Spirit in the inner man; That Christ may dwell in your hearts by faith; that ye, being rooted and grounded in love, May be able to comprehend with all saints what is the breadth, and length, and depth, and height; And to know the love of Christ, which passeth knowledge, that ye might be filled with all the fulness of God. Now unto him that is able to do exceeding abundantly above all that we ask or think, according to the power that worketh in us, Unto him be glory in the church by Christ Jesus throughout all ages, world without end. Amen.
The Gospel. St. Luke 7. 11.
AND it came to pass the day after, that he went into a city called Nain; and many of his disciples went with him, and much people. Now when he came nigh to the gate of the city, behold, there was a dead man carried out, the only son of his mother, and she was a widow: and much people of the city was with her. And when the Lord saw her, he had compassion on her, and said unto her, Weep not. And he came and touched the bier: and they that bare him stood still. And he said, Young man, I say unto thee, Arise. And he that was dead sat up, and began to speak. And he delivered him to his mother. And there came a fear on all: and they glorified God, saying, That a great prophet is risen up among us; and, That God hath visited his people. And this rumour of him went forth throughout all Judaea, and throughout all the region round about.
The Seventeenth Sunday after Trinity.
\subsection{\stylesubsec{}{The Collect.}{}}
LORD, we pray thee that thy grace may always prevent and follow us, and make us continually to be given to all good works; through Jesus Christ our Lord. Amen.
The Epistle. Ephes. 4. 1.
ITHEREFORE, the prisoner of the Lord, beseech you that ye walk worthy of the vocation wherewith ye are called, With all lowliness and meekness, with longsuffering, forbearing one another in love; Endeavouring to keep the unity of the Spirit in the bond of peace. There is one body, and one Spirit, even as ye are called in one hope of your calling; One Lord, one faith, one baptism, One God and Father of all, who is above all, and through all, and in you all.
The Gospel. St. Luke 14. 1.
IT came to pass, as he went into the house of one of the chief Pharisees to eat bread on the sabbath day, that they watched him. And, behold, there was a certain man before him which had the dropsy. And Jesus answering spake unto the lawyers and Pharisees, saying, Is it lawful to heal on the sabbath day? And they held their peace. And he took him, and healed him, and let him go; And answered them, saying, Which of you shall have an ass or an ox fallen into a pit, and will not straightway pull him out on the sabbath day? And they could not answer him again to these things. And he put forth a parable to those which were bidden, when he marked how they chose out the chief rooms; saying unto them. When thou art bidden of any man to a wedding, sit not down in the highest room; lest a more honourable man than thou be bidden of him; And he that bade thee and him come and say to thee, Give this man place; and thou begin with shame to take the lowest room. But when thou art bidden, go and sit down in the lowest room; that when he that bade thee cometh, he may say unto thee, Friend, go up higher: then shalt thou have worship in the presence of them that sit at meat with thee. For whosoever exalteth himself shall be abased; and he that humbleth himself shall be exalted.
The Eighteenth Sunday after Trinity.
\subsection{\stylesubsec{}{The Collect.}{}}
LORD, we beseech thee, grant thy people grace to withstand the temptations of the world, the flesh, and the devil, and with pure hearts and minds to follow thee the only God; through Jesus Christ our Lord. Amen.
The Epistle. 1 Cor. 1. 4.
ITHANK my God always on your behalf, for the grace of God which is given you by Jesus Christ; That in every thing ye are enriched by him, in all utterance, and in all knowledge; Even as the testimony of Christ was confirmed in you: So that ye come behind in no gift; waiting for the coming of our Lord Jesus Christ, Who shall also confirm you unto the end, that ye may be blameless in the day of our Lord Jesus Christ.
The Gospel. St. Matth. 22. 34.
WHEN the Pharisees had heard that he had put the Sadducees to silence, they were gathered together. Then one of them, which was a lawyer, asked him a question, tempting him, and saying, Master, which is the great commandment in the law? Jesus said unto him, Thou shalt love the Lord thy God with all thy heart, and with all thy soul, and with all thy mind. This is the first and great commandment. And the second is like unto it, Thou shalt love thy neighbour as thyself. On these two commandments hang all the law and the prophets. While the Pharisees were gathered together, Jesus asked them, Saying, What think ye of Christ? whose son is he? They say unto him, The son of David. He saith unto them, How then doth David in spirit call him Lord, saying, The Lord said unto my Lord, Sit thou on my right hand, till I make thine enemies thy footstool? If David then call him Lord, how is he his son? And no man was able to answer him a word, neither durst any man from that day forth ask him any more questions.
The Nineteenth Sunday after Trinity.
The Collect.
OGOD, forasmuch as without thee we are not able to please thee; Mercifully grant, that thy Holy Spirit may in all things direct and rule our hearts; through Jesus Christ our Lord. Amen.
The Epistle. Ephes. 4. 17.
THIS I say therefore, and testify in the Lord, that ye henceforth walk not as other Gentiles walk, in the vanity of their mind, Having the understanding darkened, being alienated from the life of God through the ignorance that is in them, because of the blindness of their heart: Who being past feeling have given themselves over unto lasciviousness, to work all uncleanness with greediness. But ye have not so learned Christ; If so be that ye have heard him, and have been taught by him, as the truth is in Jesus: That ye put off concerning the former conversation the old man, which is corrupt according to the deceitful lusts; And be renewed in the spirit of your mind; And that ye put on the new man, which after God is created in righteousness and true holiness. Wherefore putting away lying, speak every man truth with his neighbour: for we are members one of another. Be ye angry, and sin not: let not the sun go down upon your wrath: Neither give place to the devil. Let him that stole steal no more: but rather let him labour, working with his hands the thing which is good, that he may have to give to him that needeth. Let no corrupt communication proceed out of your mouth, but that which is good to the use of edifying, that it may minister grace unto the hearers. And grieve not the holy Spirit of God, whereby ye are sealed unto the day of redemption. Let all bitterness, and wrath, and anger, and clamour, and evil speaking, be put away from you, with all malice: And be ye kind one to another, tenderhearted, forgiving one another, even as God for Christ's sake hath forgiven you.
The Gospel. St. Matth. 9. 1.
JESUS entered into a ship, and passed over, and came into his own city. And, behold, they brought to him a man sick of the palsy, lying on a bed: and Jesus seeing their faith said unto the sick of the palsy; Son, be of good cheer; thy sins be forgiven thee. And, behold, certain of the scribes said within themselves, This man blasphemeth. And Jesus knowing their thoughts said, Wherefore think ye evil in your hearts? For whether is easier, to say, Thy sins be forgiven thee; or to say, Arise, and walk? But that ye may know that the Son of man hath power on earth to forgive sins, (then saith he to the sick of the palsy,) Arise, take up thy bed, and go unto thine house. And he arose, and departed to his house. But when the multitudes saw it, they marvelled, and glorified God, which had given such power unto men.
The Twentieth Sunday after Trinity.
\subsection{\stylesubsec{}{The Collect.}{}}
OALMIGHTY and most merciful God, of thy bountiful goodness keep us, we beseech thee, from all things that may hurt us; that we, being ready both in body and soul, may cheerfully accomplish those things that though wouldest have done; through Jesus Christ our Lord. Amen.
The Epistle. Ephes. 5. 15.
SEE then that ye walk circumspectly, not as fools, but as wise, Redeeming the time, because the days are evil. Wherefore be ye not unwise, but understanding what the will of the Lord is. And be not drunk with wine, wherein is excess; but be filled with the Spirit; Speaking to yourselves in psalms and hymns and spiritual songs, singing and making melody in your heart to the Lord; Giving thanks always for all things unto God and the Father in the name of our Lord Jesus Christ; Submitting yourselves one to another in the fear of God.
The Gospel. St. Matth. 22. 1.
JESUS said, The kingdom of heaven is like unto a certain king, which made a marriage for his son, And sent forth his servants to call them that were bidden to the wedding: and they would not come. Again, he sent forth other servants, saying, Tell them which are bidden, Behold, I have prepared my dinner: my oxen and my fatlings are killed, and all things are ready: come unto the marriage. But they made light of it, and went their ways, one to his farm, another to his merchandise: And the remnant took his servants, and entreated them spitefully, and slew them. But when the king heard thereof, he was wroth: and he sent forth his armies, and destroyed those murderers, and burned up their city. Then saith he to his servants, The wedding is ready, but they which were bidden were not worthy. Go ye therefore into the highways, and as many as ye shall find, bid to the marriage. So those servants went out into the highways, and gathered together all as many as they found, both bad and good: and the wedding was furnished with guests. And when the king came in to see the guests, he saw there a man which had not on a wedding garment: And he saith unto him, Friend, how camest thou in hither not having a wedding garment? And he was speechless. Then said the king to the servants, Bind him hand and foot, and take him away, and cast him into outer darkness, there shall be weeping and gnashing of teeth. For many are called, but few are chosen.
The Twenty-first Sunday after Trinity.
\subsection{\stylesubsec{}{The Collect.}{}}
GRANT, we beseech thee, merciful Lord, to thy faithful people pardon and peace, that they may be cleansed from all their sins, and serve thee with a quiet mind; through Jesus Christ our Lord. Amen.
The Epistle. Ephes. 6. 10.
MY brethren, be strong in the Lord, and in the power of his might. Put on the whole armour of God, that ye may be able to stand against the wiles of the devil. For we wrestle not against flesh and blood, but against principalities, against powers, against the rulers of the darkness of this world, against spiritual wickedness in high places. Wherefore take unto you the whole armour of God, that ye may be able to withstand in the evil day, and having done all, to stand. Stand therefore, having your loins girt about with truth, and having on the breastplate of righteousness; And your feet shod with the preparation of the Gospel of peace; Above all, taking the shield of faith, wherewith ye shall be able to quench all the fiery darts of the wicked. And take the helmet of salvation, and the sword of the Spirit, which is the word of God: Praying always with all prayer and supplication in the Spirit, and watching thereunto with all perseverance and supplication for all saints; And for me, that utterance may be given unto me, that I may open my mouth boldly, to make known the mystery of the Gospel, For which I am an ambassador in bonds: that therein I may speak boldly, as I ought to speak.
The Gospel. St. John 4. 46.
THERE was a certain nobleman, whose son was sick at Capernaum. When he heard that Jesus was come out of Judaea into Galilee, he went unto him, and besought him that he would come down, and heal his son: for he was at the point of death. Then said Jesus unto him, Except ye see signs and wonders, ye will not believe. The nobleman saith unto him, Sir, come down ere my child die. Jesus saith unto him, Go thy way; thy son liveth. And the man believed the word that Jesus had spoken unto him, and he went his way. And as he was now going down, his servants met him, and told him, saying, Thy son liveth. Then enquired he of them the hour when he began to amend. And they said unto him, Yesterday at the seventh hour the fever left him. So the father knew that it was at the same hour, in the which Jesus said unto him, Thy son liveth: and himself believed, and his whole house. This is again the second miracle that Jesus did, when he was come out of Judaea into Galilee.
The Twenty-second Sunday after Trinity.
\subsection{\stylesubsec{}{The Collect.}{}}
LORD, we beseech thee to keep thy household the Church in continual godliness; that through thy protection it may be free from all adversities, and devoutly given to serve thee in good works, to the glory of thy Name; through Jesus Christ our Lord. Amen.
The Epistle. Phil. 1. 3.
ITHANK my God upon every remembrance of you, Always in every prayer of mine for you all making request with joy, For your fellowship in the Gospel from the first day until now; Being confident of this very thing, that he which hath begun a good work in you will perform it until the day of Jesus Christ: Even as it is meet for me to think this of you all, because I have you in my heart; inasmuch as both in my bonds, and in the defence and confirmation of the Gospel, ye all are partakers of my grace. For God is my record, how greatly I long after you all in the bowels of Jesus Christ. And this I pray, that your love may abound yet more and more in knowledge and in all judgment; That ye may approve things that are excellent; that ye may be sincere and without offence till the day of Christ. Being filled with the fruits of righteousness, which are by Jesus Christ, unto the glory and praise of God.
The Gospel. St. Matth. 18. 21.
PETER said unto Jesus, Lord, how oft shall my brother sin against me, and I forgive him? till seven times? Jesus saith unto him, I say not unto thee, Until seven times: but, Until seventy times seven. Therefore is the kingdom of heaven likened unto a certain king, which would take account of his servants. And when he had begun to reckon, one was brought unto him, which owed him ten thousand talents. But forasmuch as he had not to pay, his Lord commanded him to be sold, and his wife, and children, and all that he had, and payment to be made. The servant therefore fell down, and worshipped him, saying, Lord, have patience with me, and I will pay thee all. Then the Lord of that servant was moved with compassion, and loosed him, and forgave him the debt. But the same servant went out, and found one of his fellowservants, which owed him an hundred pence: and he laid hands on him, and took him by the throat, saying, Pay me that thou owest. And his fellowservant fell down at his feet, and besought him, saying, Have patience with me, and I will pay thee all. And he would not: but went and cast him into prison, till he should pay the debt. So when his fellowservants saw what was done, they were very sorry, and came and told unto their Lord all that was done. Then his Lord, after that he had called him, said unto him, O thou wicked servant, I forgave thee all that debt, because thou desiredst me: Shouldest not thou also have had compassion on thy fellowservant, even as I had pity on thee? And his Lord was wroth, and delivered him to the tormentors, till he should pay all that was due unto him. So likewise shall my heavenly Father do also unto you, if ye from your hearts forgive not every one his brother their trespasses.
The Twenty-third Sunday after Trinity.
\subsection{\stylesubsec{}{The Collect.}{}}
OGOD, our refuge and strength, who art the author of all godliness; Be ready, we beseech thee, to hear the devout prayers of thy Church; and grant that those things which we ask faithfully we may obtain effectually; through Jesus Christ our Lord. Amen.
The Epistle. Phil. 3. 17.
BRETHREN, be followers together of me, and mark them which walk so as ye have us for an ensample. (For many walk, of whom I have told you often, and now tell you even weeping, that they are the enemies of the cross of Christ: Whose end is destruction, whose God is their belly, and whose glory is in their shame, who mind earthly things.) For our conversation is in heaven; from whence also we look for the Saviour, the Lord Jesus Christ: Who shall change our vile body, that it may be fashioned like unto his glorious body, according to the working whereby he is able even to subdue all things unto himself.
The Gospel. St. Matth. 22. 15.
THEN went the Pharisees, and took counsel how they might entangle him in his talk. And they sent out unto him their disciples with the Herodians, saying, Master, we know that thou art true, and teachest the way of God in truth, neither carest thou for any man: for thou regardest not the person of men. Tell us therefore, What thinkest thou? Is it lawful to give tribute unto Caesar, or not? But Jesus perceived their wickedness, and said, Why tempt ye me, ye hypocrites? Shew me the tribute money. And they brought unto him a penny. And he saith unto them, Whose is this image and superscription? They say unto him, Caesar's. Then saith he unto them, Render therefore unto Caesar the things which are Caesar's; and unto God the things that are God's. When they had heard these words, they marvelled, and left him, and went their way.
The Twenty-fourth Sunday after Trinity.
\subsection{\stylesubsec{}{The Collect.}{}}
OLORD, we beseech thee, absolve thy people from their offences; that through thy bountiful goodness we may all be delivered from the bands of those sins, which by our frailty we have committed: Grant this, O heavenly Father, for Jesus Christ's sake, our blessed Lord and Saviour. Amen.
The Epistle. Col. 1. 3.
WE give thanks to God and the Father of our Lord Jesus Christ, praying always for you, Since we heard of your faith in Christ Jesus, and of the love which ye have to all the saints, For the hope which is laid up for you in heaven, whereof ye heard before in the word of the truth of the Gospel; Which is come unto you, as it is in all the world; and bringeth forth fruit, as it doth also in you, since the day ye heard of it, and knew the grace of God in truth: As ye also learned of Epaphras our dear fellowservant, who is for you a faithful minister of Christ; Who also declared unto us your love in the Spirit. For this cause we also, since the day we heard it, do not cease to pray for you, and to desire that ye might be filled with the knowledge of his will in all wisdom and spiritual understanding; That ye might walk worthy of the Lord unto all pleasing, being fruitful in every good work, and increasing in the knowledge of God; Strengthened with all might, according to his glorious power, unto all patience and longsuffering with joyfulness; Giving thanks unto the Father, which hath made us meet to be partakers of the inheritance of the saints in light.
The Gospel. St. Matth. 9. 18.
WHILE he spake these things unto them, behold, there came a certain ruler, and worshipped him, saying, My daughter is even now dead: but come and lay thy hand upon her, and she shall live. And Jesus arose, and followed him, and so did his disciples. And, behold, a woman, which was diseased with an issue of blood twelve years, came behind him, and touched the hem of his garment: For she said within herself, If I may but touch his garment, I shall be whole. But Jesus turned him about, and when he saw her, he said, Daughter, be of good comfort; thy faith hath made thee whole. And the woman was made whole from that hour. And when Jesus came into the ruler's house, and saw the minstrels and the people making a noise, He said unto them, Give place: for the maid is not dead, but sleepeth. And they laughed him to scorn. But when the people were put forth, he went in, and took her by the hand, and the maid arose. And the fame hereof went abroad into all that land.
The Twenty-fifth Sunday after Trinity.
\subsection{\stylesubsec{}{The Collect.}{}}
\drop{Stir up, we beseech thee, O Lord, the wills of thy faithful people; that they, plenteously bringing forth the fruit of good works, may of thee be plenteously rewarded; through Jesus Christ our Lord. \R Amen.}

For the Epistle. Jer. 23. 5.
\drop{Behold, the days come, saith the Lord, that I will raise unto David a righteous Branch, and a King shall reign and prosper, and shall execute judgment and justice in the earth. In his days Judah shall be saved, and Israel shall dwell safely: and this is his name whereby he shall be called, THE LORD OUR RIGHTEOUSNESS. Therefore, behold, the days come, saith the Lord, that they shall no more say, The Lord liveth, which brought up the children of Israel out of the land of Egypt; But, The Lord liveth, which brought up and which led the seed of the house of Israel out of the north country, and from all countries whither I had driven them; and they shall dwell in their own land.}

The Gospel. St. John 6. 5.
\drop{When Jesus then lifted up his eyes, and saw a great company come unto him, he saith unto Philip, Whence shall we buy bread, that these may eat? And this he said to prove him: for he himself knew what he would do. Philip answered him, Two hundred pennyworth of bread is not sufficient for them, that every one of them may take a little. One of his disciples, Andrew, Simon Peter's brother, saith unto him, There is a lad here, which hath five barley loaves, and two small fishes: but what are they among so many? And Jesus said, Make the men sit down. Now there was much grass in the place. So the men sat down, in number about five thousand. And Jesus took the loaves; and when he had given thanks, he distributed to the disciples, and the disciples to them that were set down; and likewise of the fishes as much as they would. When they were filled, he said unto his disciples, Gather up the fragments that remain, that nothing be lost. Therefore they gathered them together, and filled twelve baskets with the fragments of the five barley loaves, which remained over and above unto them that had eaten. Then those men, when they had seen the miracle that Jesus did, said, This is of a truth that prophet that should come into the world.}
If there be any more Sundays before Advent-Sunday, the Service of some of those Sundays that were omitted after the Epiphany shall be taken in to supply so many as are here wanting. And if there be fewer, the overplus may be omitted : Provided that this last Collect, Epistle, and Gospel shall always be used upon the Sunday next before Advent.



Saint Andrew's Day.
The Collect.
ALMIGHTY God, who didst give such grace unto thy holy Apostle Saint Andrew, that he readily obeyed the calling of thy Son Jesus Christ, and followed him without delay; Grant unto us all, that we, being called by thy holy Word, may forthwith give up ourselves obediently to fulfil the holy commandments; through the same Jesus Christ our Lord. Amen.
(The Collect from the First Sunday in Advent is to be repeated every day, with the other Collects in Advent, from Advent Sunday until Christmas-Eve.)
The Epistle. Rom. 10. 9.
IF thou shalt confess with thy mouth the Lord Jesus, and shalt believe in thine heart that God hath raised him from the dead, thou shalt be saved. For with the heart man believeth unto righteousness; and with the mouth confession is made unto salvation. For the scripture saith, Whosoever believeth on him shall not be ashamed. For there is no difference between the Jew and the Greek: for the same Lord over all is rich unto all that call upon him. For whosoever shall call upon the name of the Lord shall be saved. How then shall they call on him in whom they have not believed? and how shall they believe in him of whom they have not heard? and how shall they hear without a preacher? And how shall they preach, except they be sent? as it is written, How beautiful are the feet of them that preach the Gospel of peace, and bring glad tidings of good things! But they have not all obeyed the Gospel. For Esaias saith, Lord, who hath believed our report? So then faith cometh by hearing, and hearing by the word of God. But I say, Have they not heard? Yes verily, their sound went into all the earth, and their words unto the ends of the world. But I say, Did not Israel know? First Moses saith, I will provoke you to jealousy by them that are no people, and by a foolish nation I will anger you. But Esaias is very bold, and saith, I was found of them that sought me not; I was made manifest unto them that asked not after me. But to Israel he saith, All day long I have stretched forth my hands unto a disobedient and gainsaying people.
The Gospel. St. Matth. 4. 18.
JESUS, walking by the sea of Galilee, saw two brethren, Simon called Peter, and Andrew his brother, casting a net into the sea: for they were fishers. And he saith unto them, Follow me, and I will make you fishers of men. And they straightway left their nets, and followed him. And going on from thence, he saw other two brethren, James the son of Zebedee, and John his brother, in a ship with Zebedee their father, mending their nets; and he called them. And they immediately left the ship and their father, and followed him.
Saint Thomas the Apostle.
The Collect.
ALMIGHTY and everliving God, who for the more confirmation of the faith didst suffer thy holy Apostle Thomas to be doubtful in thy Son's resurrection; Grant us so perfectly, and without all doubt, to believe in thy Son Jesus Christ, that our faith in thy sight may never be reproved. Hear us, O Lord, through the same Jesus Christ, to whom, with thee and the Holy Ghost, be all honour and glory, now and for evermore. Amen.
The Collect from the First Sunday in Advent is to be repeated every day, with the other Collects in Advent, until Christmas-Eve.
The Epistle. Ephes. 2. 19.
NOW therefore ye are no more strangers and foreigners, but fellow-citizens with the saints, and of the household of God; And are built upon the foundation of the apostles and prophets, Jesus Christ himself being the chief corner stone; In whom all the building fitly framed together groweth unto an holy temple in the Lord: In whom ye also are builded together for an habitation of God through the Spirit.
The Gospel. St. John 20. 24.
THOMAS, one of the twelve, called Didymus, was not with them when Jesus came. The other disciples therefore said unto him, We have seen the Lord. But he said unto them, Except I shall see in his hands the print of the nails, and put my finger into the print of the nails, and thrust my hand into his side, I will not believe. And after eight days again his disciples were within, and Thomas with them: then came Jesus, the doors being shut, and stood in the midst, and said, Peace be unto you. Then saith he to Thomas, Reach hither thy finger, and behold my hands; and reach hither thy hand, and thrust it into my side: and be not faithless, but believing. And Thomas answered and said unto him, My Lord and my God. Jesus saith unto him, Thomas, because thou hast seen me, thou hast believed: blessed are they that have not seen, and yet have believed. And many other signs truly did Jesus in the presence of his disciples, which are not written in this book: But these are written, that ye might believe that Jesus is the Christ, the Son of God; and that believing ye might have life through his name.
The Conversion of Saint Paul.
The Collect.
OGOD, who, through the preaching of the blessed Apostle Saint Paul, hast caused the light of the Gospel to shine throughout the world; Grant, we beseech thee, that we, having his wonderful conversion in remembrance, may shew forth our thankfulness unto thee for the same, by following the holy doctrine which he taught; through Jesus Christ our Lord. Amen.
For the Epistle. Acts 9. 1.
AND Saul, yet breathing out threatenings and slaughter against the disciples of the Lord, went unto the high priest, And desired of him letters to Damascus to the synagogues, that if he found any of this way, whether they were men or women, he might bring them bound unto Jerusalem. And as he journeyed, he came near Damascus: and suddenly there shined round about him a light from heaven: And he fell to the earth, and heard a voice saying unto him, Saul, Saul, why persecutest thou me? And he said, Who art thou, Lord? And the Lord said, I am Jesus whom thou persecutest: it is hard for thee to kick against the pricks. And he trembling and astonished said, Lord, what wilt thou have me to do? And the Lord said unto him, Arise, and go into the city, and it shall be told thee what thou must do. And the men which journeyed with him stood speechless, hearing a voice, but seeing no man. And Saul arose from the earth; and when his eyes were opened, he saw no man: but they led him by the hand, and brought him into Damascus. And he was three days without sight, and neither did eat nor drink. And there was a certain disciple at Damascus, named Ananias; and to him said the Lord in a vision, Ananias. And he said, Behold, I am here, Lord. And the Lord said unto him, Arise, and go into the street which is called Straight, and enquire in the house of Judas for one called Saul, of Tarsus: for, behold, he prayeth, And hath seen in a vision a man named Ananias coming in, and putting his hand on him, that he might receive his sight. Then Ananias answered, Lord, I have heard by many of this man, how much evil he hath done to thy saints at Jerusalem: And here he hath authority from the chief priests to bind all that call on thy name. But the Lord said unto him, Go thy way: for he is a chosen vessel unto me, to bear my name before the Gentiles, and kings, and the children of Israel: For I will shew him how great things he must suffer for my name's sake. And Ananias went his way, and entered into the house; and putting his hands on him said, Brother Saul, the Lord, even Jesus, that appeared unto thee in the way as thou camest, hath sent me, that thou mightest receive thy sight, and be filled with the Holy Ghost. And immediately there fell from his eyes as it had been scales: and he received sight forthwith, and arose, and was baptized. And when he had received meat, he was strengthened. Then was Saul certain days with the disciples which were at Damascus. And straightway he preached Christ in the synagogues, that he is the Son of God. But all that heard him were amazed, and said; Is not this he that destroyed them which called on this name in Jerusalem, and came hither for that intent, that he might bring them bound unto the chief priests? But Saul increased the more in strength, and confounded the Jews which dwelt at Damascus, proving that this is very Christ.
The Gospel. St. Matth. 19. 27.
PETER answered and said unto him, Behold, we have forsaken all, and followed thee; what shall we have therefore? And Jesus said unto them, Verily I say unto you, That ye which have followed me, in the regeneration when the Son of man shall sit in the throne of his glory, ye also shall sit upon twelve thrones, judging the twelve tribes of Israel. And every one that hath forsaken houses, or brethren, or sisters, or father, or mother, or wife, or children, or lands, for my name's sake, shall receive an hundredfold, and shall inherit everlasting life. But many that are first shall be last; and the last shall be first.
The Presentation of Christ in the Temple,
Commonly Called, The Purification of Saint Mary the Virgin.
The Collect.
ALMIGHTY and everliving God, we humbly beseech thy Majesty, that, as thy only-begotten Son was this day presented in the temple in substance of our flesh, so we may be presented unto thee with pure and clean hearts, by the same thy Son Jesus Christ our Lord. Amen.
For the Epistle. Mal. 3. 1.
BEHOLD, I will send my messenger, and he shall prepare the way before me: and the Lord, whom ye seek, shall suddenly come to his temple, even the messenger of the covenant, whom ye delight in: behold, he shall come, saith the Lord of hosts. But who may abide the day of his coming? and who shall stand when he appeareth? for he is like a refiner's fire, and like fullers' soap: And he shall sit as a refiner and purifier of silver: and he shall purify the sons of Levi, and purge them as gold and silver, that they may offer unto the Lord an offering in righteousness. Then shall the offering of Judah and Jerusalem be pleasant unto the Lord, as in the days of old, and as in former years. And I will come near to you to judgment; and I will be a swift witness against the sorcerers, and against the adulterers, and against false swearers, and against those that oppress the hireling in his wages, the widow, and the fatherless, and that turn aside the stranger from his right, and fear not me, saith the Lord of hosts.
The Gospel. St. Luke 2. 22.
AND when the days of her purification according to the law of Moses were accomplished, they brought him to Jerusalem, to present him to the Lord; (As it is written in the law of the Lord, Every male that openeth the womb shall be called holy to the Lord;) And to offer a sacrifice according to that which is said in the law of the Lord, A pair of turtledoves, or two young pigeons. And, behold, there was a man in Jerusalem, whose name was Simeon; and the same man was just and devout, waiting for the consolation of Israel: and the Holy Ghost was upon him. And it was revealed unto him by the Holy Ghost, that he should not see death, before he had seen the Lord's Christ. And he came by the Spirit into the temple: and when the parents brought in the child Jesus, to do for him after the custom of the law, Then took he him up in his arms, and blessed God, and said, Lord, now lettest thou thy servant depart in peace, according to thy word: For mine eyes have seen thy salvation, Which thou hast prepared before the face of all people; A light to lighten the Gentiles, and the glory of thy people Israel. And Joseph and his mother marvelled at those things which were spoken of him. And Simeon blessed them, and said unto Mary his mother, Behold, this child is set for the fall and rising again of many in Israel; and for a sign which shall be spoken against; (Yea, a sword shall pierce through thy own soul also,) that the thoughts of many hearts may be revealed. And there was one Anna, a prophetess, the daughter of Phanuel, of the tribe of Aser: she was of a great age, and had lived with an husband seven years from her virginity; And she was a widow of about fourscore and four years, which departed not from the temple, but served God with fastings and prayers night and day. And she coming in that instant gave thanks likewise unto the Lord, and spake of him to all them that looked for redemption in Jerusalem. And when they had performed all things according to the law of the Lord, they returned into Galilee, to their own city Nazareth. And the child grew, and waxed strong in spirit, filled with wisdom: and the grace of God was upon him.
Saint Matthias's Day.
The Collect.
OALMIGHTY God, who into the place of the traitor Judas didst choose thy faithful servant Matthias to be of the number of the twelve Apostles; Grant that thy Church, being alway preserved from false Apostles, may be ordered and guided by faithful and true pastors; through Jesus Christ our Lord. Amen.
For the Epistle. Acts 1. 15.
IN those days Peter stood up in the midst of the disciples, and said, (the number of names together were about an hundred and twenty,) Men and brethren, this scripture must needs have been fulfilled, which the Holy Ghost by the mouth of David spake before concerning Judas, which was guide to them that took Jesus. For he was numbered with us, and had obtained part of this ministry. Now this man purchased a field with the reward of iniquity; and falling headlong, he burst asunder in the midst, and all his bowels gushed out. And it was known unto all the dwellers at Jerusalem; insomuch as that field is called in their proper tongue, Aceldama, that is to say, The field of blood. For it is written in the book of Psalms, Let his habitation be desolate, and let no man dwell therein: and his bishoprick let another take. Wherefore of these men which have companied with us all the time that the Lord Jesus went in and out among us, Beginning from the baptism of John, unto that same day that he was taken up from us, must one be ordained to be a witness with us of his resurrection. And they appointed two, Joseph called Barsabas, who was surnamed Justus, and Matthias. And they prayed, and said, Thou, Lord, which knowest the hearts of all men, shew whether of these two thou hast chosen, That he may take part of this ministry and apostleship, from which Judas by transgression fell, that he might go to his own place. And they gave forth their lots; and the lot fell upon Matthias; and he was numbered with the eleven Apostles.
The Gospel. St. Matth. 11. 25.
AT that time Jesus answered and said, I thank thee, O Father, Lord of heaven and earth, because thou hast hid these things from the wise and prudent, and hast revealed them unto babes. Even so, Father: for so it seemed good in thy sight. All things are delivered unto me of my Father: and no man knoweth the Son, but the Father; neither knoweth any man the Father, save the Son, and he to whomsoever the Son will reveal him. Come unto me, all ye that labour and are heavy laden, and I will give you rest. Take my yoke upon you, and learn of me; for I am meek and lowly in heart: and ye shall find rest unto your souls. For my yoke is easy, and my burden is light.
The Annunciation of the Blessed Virgin Mary.
The Collect.
WE beseech thee, O Lord, pour thy grace into our hearts; that, as we have known the incarnation of thy Son Jesus Christ by the message of an angel, so by his cross and passion we may be brought unto the glory of his resurrection; through the same Jesus Christ our Lord. Amen.
For the Epistle. Isaiah 7. 10.
MOREOVER, the Lord spake again unto Ahaz, saying, Ask thee a sign of the Lord thy God; ask it either in the depth, or in the height above. But Ahaz said, I will not ask, neither will I tempt the Lord. And he said, Hear ye now, O house of David; Is it a small thing for you to weary men, but will ye weary my God also? Therefore the Lord himself shall give you a sign; Behold, a virgin shall conceive, and bear a son, and shall call his name Immanuel. Butter and honey shall he eat, that he may know to refuse the evil, and choose the good.
The Gospel. St. Luke 1. 26.
AND in the sixth month the angel Gabriel was sent from God unto a city of Galilee, named Nazareth, To a virgin espoused to a man whose name was Joseph, of the house of David; and the virgin's name was Mary. And the angel came in unto her, and said, Hail, thou that art highly favoured, the Lord is with thee: blessed art thou among women. And when she saw him, she was troubled at his saying, and cast in her mind what manner of salutation this should be. And the angel said unto her, Fear not, Mary: for thou hast found favour with God. And, behold, thou shalt conceive in thy womb, and bring forth a son, and shalt call his name JESUS. He shall be great, and shall be called the Son of the Highest: and the Lord God shall give unto him the throne of his father David: And he shall reign over the house of Jacob for ever; and of his kingdom there shall be no end. Then said Mary unto the angel, How shall this be, seeing I know not a man? And the angel answered and said unto her, The Holy Ghost shall come upon thee, and the power of the Highest shall overshadow thee: therefore also that holy thing which shall be born of thee shall be called the Son of God. And, behold, thy cousin Elisabeth, she hath also conceived a son in her old age: and this is the sixth month with her, who was called barren. For with God nothing shall be impossible. And Mary said, Behold the handmaid of the Lord; be it unto me according to thy word. And the angel departed from her.
Saint Mark's Day.
The Collect.
OALMIGHTY God, who hast instructed thy holy Church with the heavenly doctrine of thy Evangelist Saint Mark; Give us grace, that, being not like children carried away with every blast of vain doctrine, we may be established in the truth of thy holy Gospel; through Jesus Christ our Lord. Amen.
The Epistle. Ephes. 4. 7.
UNTO every one of us is given grace according to the measure of the gift of Christ. Wherefore he saith, When he ascended up on high, he led captivity captive, and gave gifts unto men. (Now that he ascended, what is it but that he also descended first into the lower parts of the earth? He that descended is the same also that ascended up far above all heavens, that he might fill all things.) And he gave some, apostles; and some, prophets; and some, evangelists; and some, pastors and teachers; For the perfecting of the saints, for the work of the ministry, for the edifying of the body of Christ: Till we all come in the unity of the faith, and of the knowledge of the Son of God, unto a perfect man, unto the measure of the stature of the fulness of Christ: That we henceforth be no more children, tossed to and fro, and carried about with every wind of doctrine, by the sleight of men, and cunning craftiness, whereby they lie in wait to deceive; But speaking the truth in love, may grow up into him in all things, which is the head, even Christ: From whom the whole body fitly joined together and compacted by that which every joint supplieth, according to the effectual working in the measure of every part, maketh increase of the body unto the edifying of itself in love.
The Gospel. St. John 15. 1.
IAM the true vine, and my Father is the husbandman. Every branch in me that beareth not fruit he taketh away: and every branch that beareth fruit, he purgeth it, that it may bring forth more fruit. Now ye are clean through the word which I have spoken unto you. Abide in me, and I in you. As the branch cannot bear fruit of itself, except it abide in the vine; no more can ye, except ye abide in me. I am the vine, ye are the branches: He that abideth in me, and I in him, the same bringeth forth much fruit: for without me ye can do nothing. If a man abide not in me, he is cast forth as a branch, and is withered; and men gather them, and cast them into the fire, and they are burned. If ye abide in me, and my words abide in you, ye shall ask what ye will, and it shall be done unto you. Herein is my Father glorified, that ye bear much fruit; so shall ye be my disciples. As the Father hath loved me, so have I loved you: continue ye in my love. If ye keep my commandments, ye shall abide in my love; even as I have kept my Father's commandments, and abide in his love. These things have I spoken unto you, that my joy might remain in you, and that your joy might be full.
Saint Philip and Saint James's Day.
The Collect.
OALMIGHTY God, whom truly to know is everlasting life; Grant us perfectly to know thy Son Jesus Christ to be the way, the truth, and the life; that, following the steps of thy holy Apostles, Saint Philip and Saint James, we may stedfastly walk in the way that leadeth to eternal life; through the same thy Son Jesus Christ our Lord. Amen.
The Epistle. St. James 1. 1.
JAMES, a servant of God and of the Lord Jesus Christ, to the twelve tribes which are scattered abroad, greeting. My brethren, count it all joy when ye fall into divers temptations; Knowing this, that the trying of your faith worketh patience. But let patience have her perfect work, that ye may be perfect and entire, wanting nothing. If any of you lack wisdom, let him ask of God, that giveth to all men liberally, and upbraideth not; and it shall be given him. But let him ask in faith, nothing wavering. For he that wavereth is like a wave of the sea driven with the wind and tossed. For let not that man think that he shall receive any thing of the Lord. A double minded man is unstable in all his ways. Let the brother of low degree rejoice in that he is exalted: But the rich, in that he is made low: because as the flower of the grass he shall pass away. For the sun is no sooner risen with a burning heat, but it withereth the grass, and the flower thereof falleth, and the grace of the fashion of it perisheth: so also shall the rich man fade away in his ways. Blessed is the man that endureth temptation: for when he is tried, he shall receive the crown of life, which the Lord hath promised to them that love him.
The Gospel. St. John 14. 1.
AND Jesus said unto his disciples, Let not your heart be troubled: ye believe in God, believe also in me. In my Father's house are many mansions: if it were not so, I would have told you. I go to prepare a place for you. And if I go and prepare a place for you, I will come again, and receive you unto myself; that where I am, there ye may be also. And whither I go ye know, and the way ye know. Thomas saith unto him, Lord, we know not whither thou goest; and how can we know the way? Jesus saith unto him, I am the way, the truth, and the life: no man cometh unto the Father, but by me. If ye had known me, ye should have known my Father also: and from henceforth ye know him, and have seen him. Philip saith unto him, Lord, shew us the Father, and it sufficeth us. Jesus saith unto him, Have I been so long time with you, and yet hast thou not known me, Philip? he that hath seen me hath seen the Father; and how sayest thou then, Shew us the Father? Believest thou not that I am in the Father, and the Father in me? the words that I speak unto you I speak not of myself: but the Father that dwelleth in me, he doeth the works. Believe me that I am in the Father, and the Father in me: or else believe me for the very works' sake. Verily, verily, I say unto you, He that believeth on me, the works that I do shall he do also; and greater works than these shall he do; because I go unto my Father. And whatsoever ye shall ask in my name, that will I do, that the Father may be glorified in the Son. If ye shall ask any thing in my Name, I will do it.
Saint Barnabas the Apostle.
The Collect.
OLORD God Almighty, who didst endue thy holy Apostle Barnabas with singular gifts of the Holy Ghost; Leave us not, we beseech thee, destitute of thy manifold gifts, nor yet of grace to use them alway to thy honour and glory; through Jesus Christ our Lord. Amen.
For the Epistle. Acts 11. 22.
TIDINGS of these things came unto the ears of the church which was in Jerusalem: and they sent forth Barnabas, that he should go as far as Antioch. Who, when he came, and had seen the grace of God, was glad, and exhorted them all, that with purpose of heart they would cleave unto the Lord. For he was a good man, and full of the Holy Ghost and of faith: and much people was added unto the Lord. Then departed Barnabas to Tarsus, for to seek Saul: And when he had found him, he brought him unto Antioch. And it came to pass, that a whole year they assembled themselves with the church, and taught much people. And the disciples were called Christians first in Antioch. And in these days came prophets from Jerusalem unto Antioch. And there stood up one of them named Agabus, and signified by the Spirit that there should be great dearth throughout all the world: which came to pass in the days of Claudius Caesar. Then the disciples, every man according to his ability, determined to send relief unto the brethren which dwelt in Judaea: Which also they did, and sent it to the elders by the hands of Barnabas and Saul.
The Gospel. St. John 15. 12.
THIS is my commandment, That ye love one another, as I have loved you. Greater love hath no man than this, that a man lay down his life for his friends. Ye are my friends, if ye do whatsoever I command you. Henceforth I call you not servants; for the servant knoweth not what his Lord doeth: but I have called you friends; for all things that I have heard of my Father I have made known unto you. Ye have not chosen me, but I have chosen you, and ordained you, that ye should go and bring forth fruit, and that your fruit should remain: that whatsoever ye shall ask of the Father in my Name, he may give it you.
Saint John Baptist's Day.
The Collect.
ALMIGHTY God, by whose providence thy servant John Baptist was wonderfully born, and sent to prepare the way of they Son our Saviour, by preaching of repentance; Make us so to follow his doctrine and holy life, that we may truly repent according to his preaching; and after his example constantly speak the truth, boldly rebuke vice, and patiently suffer for the truth's sake; through Jesus Christ our Lord. Amen.
For the Epistle. Isaiah 40. 1.
COMFORT ye, comfort ye my people, saith your God. Speak ye comfortably to Jerusalem, and cry unto her, that her warfare is accomplished, that her iniquity is pardoned: for she hath received of the Lord's hand double for all her sins. The voice of him that crieth in the wilderness, Prepare ye the way of the Lord, make straight in the desert a highway for our God. Every valley shall be exalted, and every mountain and hill shall be made low: and the crooked shall be made straight, and the rough places plain: And the glory of the Lord shall be revealed, and all flesh shall see it together: for the mouth of the Lord hath spoken it. The voice said, Cry. And he said, What shall I cry? All flesh is grass, and all the goodliness thereof is as the flower of the field: The grass withereth, the flower fadeth: because the spirit of the Lord bloweth upon it: surely the people is grass. The grass withereth, the flower fadeth: but the word of our God shall stand for ever. O Zion, that bringest good tidings, get thee up into the high mountain; O Jerusalem, that bringest good tidings, lift up thy voice with strength; lift it up, be not afraid; say unto the cities of Judah, Behold your God! Behold, the Lord God will come with strong hand, and his arm shall rule for him: behold, his reward is with him, and his work before him. He shall feed his flock like a shepherd: he shall gather the lambs with his arm, and carry them in his bosom, and shall gently lead those that are with young.
The Gospel. St. Luke 1. 57.
ELISABETH'S full time came that she should be delivered; and she brought forth a son. And her neighbours and her cousins heard how the Lord had shewed great mercy upon her; and they rejoiced with her. And it came to pass, that on the eighth day they came to circumcise the child; and they called him Zacharias, after the name of his father. And his mother answered and said, Not so; but he shall be called John. And they said unto her, There is none of thy kindred that is called by this name. And they made signs to his father, how he would have him called. And he asked for a writing table, and wrote, saying, His name is John. And they marvelled all. And his mouth was opened immediately, and his tongue loosed, and he spake, and praised God. And fear came on all that dwelt round about them: and all these sayings were noised abroad throughout all the hill country of Judaea. And all they that heard them laid them up in their hearts, saying, What manner of child shall this be! And the hand of the Lord was with him. And his father Zacharias was filled with the Holy Ghost, and prophesied, saying, Blessed be the Lord God of Israel; for he hath visited and redeemed his people, And hath raised up an horn of salvation for us in the house of his servant David; As he spake by the mouth of his holy prophets, which have been since the world began: That we should be saved from our enemies, and from the hand of all that hate us; To perform the mercy promised to our fathers, and to remember his holy covenant; The oath which he sware to our father Abraham, That he would grant unto us, that we being delivered out of the hand of our enemies might serve him without fear, In holiness and righteousness before him, all the days of our life. And thou, child, shalt be called the prophet of the Highest: for thou shalt go before the face of the Lord to prepare his ways; To give knowledge of salvation unto his people by the remission of their sins, Through the tender mercy of our God; whereby the dayspring from on high hath visited us, To give light to them that sit in darkness and in the shadow of death, to guide our feet into the way of peace. And the child grew, and waxed strong in spirit, and was in the deserts till the day of his shewing unto Israel.
Saint Peter's Day.
The Collect.
OALMIGHTY God, who by thy Son Jesus Christ didst give to thy Apostle Saint Peter many excellent gifts, and commandest him earnestly to feed thy flock; Make, we beseech thee, all Bishops and Pastors diligently to preach thy holy Word, and the people obediently to follow the same, that they may receive the crown of everlasting glory; through Jesus Christ our Lord. Amen.
For the Epistle. Acts 12. 1.
ABOUT that time Herod the king stretched forth his hands to vex certain of the church. And he killed James the brother of John with the sword. And because he saw it pleased the Jews, he proceeded further to take Peter also. (Then were the days of unleavened bread.) And when he had apprehended him, he put him in prison, and delivered him to four quaternions of soldiers to keep him; intending after Easter to bring him forth to the people. Peter therefore was kept in prison: but prayer was made without ceasing of the church unto God for him. And when Herod would have brought him forth, the same night Peter was sleeping between two soldiers, bound with two chains: and the keepers before the door kept the prison. And, behold, the angel of the Lord came upon him, and a light shined in the prison: and he smote Peter on the side, and raised him up, saying, Arise up quickly. And his chains fell off from his hands. And the angel said unto him, Gird thyself, and bind on thy sandals. And so he did. And he saith unto him, Cast thy garment about thee, and follow me. And he went out, and followed him; and wist not that it was true which was done by the angel; but thought he saw a vision. When they were past the first and the second ward, they came unto the iron gate that leadeth unto the city; which opened to them of his own accord: and they went out, and passed on through one street; and forthwith the angel departed from him. And when Peter was come to himself, he said, Now I know of a surety, that the Lord hath sent his angel, and hath delivered me out of the hand of Herod, and from all the expectation of the people of the Jews.
The Gospel. St. Matth. 16. 13.
WHEN Jesus came into the coasts of Caesarea Philippi, he asked his disciples, saying, Whom do men say that I the Son of man am? And they said, Some say that thou art John the Baptist: some, Elias; and others, Jeremias, or one of the prophets. He saith unto them, But whom say ye that I am? And Simon Peter answered and said, Thou art the Christ, the Son of the living God. And Jesus answered and said unto him, Blessed art thou, Simon Barjona: for flesh and blood hath not revealed it unto thee, but my Father which is in heaven. And I say also unto thee, That thou art Peter, and upon this rock I will build my church; and the gates of hell shall not prevail against it. And I will give unto thee the keys of the kingdom of heaven: and whatsoever thou shalt bind on earth shall be bound in heaven: and whatsoever thou shalt loose on earth shall be loosed in heaven.

\section{Saint James the Apostle}
\subsection{\stylesubsec{}{}{[July 25]}}

\subsection{\stylesubsec{}{The Collect.}{}}
\drop{Grant, O merciful God, that as thine holy Apostle Saint James, leaving his father and all that he had, without delay was obedient unto the calling of thy Son Jesus Christ, and followed him; so we, forsaking all worldly and carnal affections, may be evermore ready to follow thy holy commandments; through Jesus Christ our Lord. \R Amen.}

\subsection{\stylesubsec{}{For the Epistle.}{Acts 11.~27, \emph{and part of chapter 12.}}}
\drop{In these days came prophets from Jerusalem unto Antioch. And there stood up one of them named Agabus, and signified by the Spirit that there should be great dearth throughout all the world: which came to pass in the days of Claudius Caesar. Then the disciples, every man according to his ability, determined to send relief unto the brethren which dwelt in Judaea: Which also they did, and sent it to the elders by the hands of Barnabas and Saul. Now about that time Herod the king stretched forth his hands to vex certain of the church. And he killed James the brother of John with the sword. And because he saw it pleased the Jews, he proceeded further to take Peter also.}

\subsection{\stylesubsec{}{The Gospel.}{St.~Matthew 20.~20.}}
\drop{Then came to him the mother of Zebedees children with her sons, worshipping him, and desiring a certain thing of him. And he said unto her, What wilt thou? She saith unto him, Grant that these my two sons may sit, the one on thy right hand, and the other on the left, in thy kingdom. But Jesus answered and said, Ye know not what ye ask. Are ye able to drink of the cup that I shall drink of, and to be baptized with the baptism that I am baptized with? They say unto him, We are able. And he saith unto them, Ye shall drink indeed of my cup, and be baptized with the baptism that I am baptized with: but to sit on my right hand, and on my left, is not mine to give, but it shall be given to them for whom it is prepared of my Father. And when the ten heard it, they were moved with indignation against the two brethren. But Jesus called them unto him, and said, Ye know that the princes of the Gentiles exercise dominion over them, and they that are great exercise authority upon them. But it shall not be so among you: but whosoever will be great among you, let him be your minister; And whosoever will be chief among you, let him be your servant: Even as the Son of man came not to be ministered unto, but to minister, and to give his life a ransom for many.}

\section{Saint Bartholomew the Apostle}
\subsection{\stylesubsec{}{}{[August 24]}}

\subsection{\stylesubsec{}{The Collect.}{}}
\drop{O almighty and everlasting God, who didst give to thine Apostle Bartholomew grace truly to believe and to preach thy Word; Grant, we beseech thee, unto thy Church, to love that Word which he believed, and both to preach and receive the same; through Jesus Christ our Lord. \R Amen.}

\subsection{\stylesubsec{}{For the Epistle.}{Acts 5.~12.}}
\drop{By the hands of the apostles were many signs and wonders wrought among the people; (and they were all with one accord in Solomon's porch. And of the rest durst no man join himself to them: but the people magnified them. And believers were the more added to the Lord, multitudes both of men and women.) Insomuch that they brought forth the sick into the streets, and laid them on beds and couches, that at the least the shadow of Peter passing by might overshadow some of them. There came also a multitude out of the cities round about unto Jerusalem, bringing sick folks, and them which were vexed with unclean spirits: and they were healed every one.}

\subsection{\stylesubsec{}{The Gospel.}{St.~Luke 22.~24.}}
\drop{And there was also a strife among them, which of them should be accounted the greatest. And he said unto them, The kings of the Gentiles exercise Lordship over them; and they that exercise authority upon them are called benefactors. But ye shall not be so: but he that is greatest among you, let him be as the younger; and he that is chief, as he that doth serve. For whether is greater, he that sitteth at meat, or he that serveth? is not he that sitteth at meat? but I am among you as he that serveth. Ye are they which have continued with me in my temptations. And I appoint unto you a kingdom, as my Father hath appointed unto me; That ye may eat and drink at my table in my kingdom, and sit on thrones judging the twelve tribes of Israel.}

\section{Saint Matthew the Apostle}
\subsection{\stylesubsec{}{}{[September 21]}}

\subsection{\stylesubsec{}{The Collect.}{}}
\drop{O almighty God, who by thy blessed Son didst call Matthew from the receipt of custom to be an Apostle and Evangelist; Grant us grace to forsake all covetous desires, and inordinate love of riches, and to follow the same thy Son Jesus Christ, who liveth and reigneth with thee and the Holy Ghost, one God, world without end. \R Amen.}

\subsection{\stylesubsec{}{The Epistle.}{2 Corinthians 4.~1.}}
\drop{Therefore seeing we have this ministry, as we have received mercy, we faint not; But have renounced the hidden things of dishonesty, not walking in craftiness, nor handling the word of God deceitfully; but by manifestation of the truth commending ourselves to every man's conscience in the sight of God. But if our Gospel be hid, it is hid to them that are lost: In whom the God of this world hath blinded the minds of them which believe not, lest the light of the glorious Gospel of Christ, who is the image of God, should shine unto them. For we preach not ourselves, but Christ Jesus the Lord; and ourselves your servants for Jesus' sake. For God, who commanded the light to shine out of darkness, hath shined in our hearts, to give the light of the knowledge of the glory of God in the face of Jesus Christ.}

\subsection{\stylesubsec{}{The Gospel.}{St.~Matthew 9.~9.}}
\drop{And as Jesus passed forth from thence, he saw a man, named Matthew, sitting at the receipt of custom: and he saith unto him, Follow me. And he arose, and followed him. And it came to pass, as Jesus sat at meat in the house, behold, many publicans and sinners came and sat down with him and his disciples. And when the Pharisees saw it, they said unto his disciples, Why eateth your Master with publicans and sinners? But when Jesus heard that, he said unto them, They that be whole need not a physician, but they that are sick. But go ye and learn what that meaneth, I will have mercy, and not sacrifice: for I am not come to call the righteous, but sinners to repentance.}

\section{Saint Michael and all Angels}
\subsection{\stylesubsec{}{}{[September 29]}}

\subsection{\stylesubsec{}{The Collect.}{}}
\drop{O everlasting God, who hast ordained and constituted the services of Angels and men in a wonderful order; Mercifully grant, that as thy holy Angels alway do thee service in heaven, so by thy appointment they may succour and defend us on earth; through Jesus Christ our Lord. \R Amen.}

\subsection{\stylesubsec{}{For the Epistle.}{Revelation 12.~7.}}
\drop{There was war in heaven: Michael and his angels fought against the dragon; and the dragon fought and his angels, And prevailed not; neither was their place found any more in heaven. And the great dragon was cast out, that old serpent, called the Devil, and Satan, which deceiveth the whole world: he was cast out into the earth, and his angels were cast out with him. And I heard a loud voice saying in heaven, Now is come salvation, and strength, and the kingdom of our God, and the power of his Christ: for the accuser of our brethren is cast down, which accused them before our God day and night. And they overcame him by the blood of the Lamb, and by the word of their testimony; and they loved not their lives unto the death. Therefore rejoice, ye heavens, and ye that dwell in them. Woe to the inhabiters of the earth and of the sea! for the devil is come down unto you, having great wrath, because he knoweth that he hath but a short time.}

\subsection{\stylesubsec{}{The Gospel.}{St.~Matthew 18.~1.}}
\drop{At the same time came the disciples unto Jesus, saying, Who is the greatest in the kingdom of heaven? And Jesus called a little child unto him, and set him in the midst of them, And said, Verily I say unto you, Except ye be converted, and become as little children, ye shall not enter into the kingdom of heaven. Whosoever therefore shall humble himself as this little child, the same is greatest in the kingdom of heaven. And whoso shall receive one such little child in my name receiveth me. But whoso shall offend one of these little ones which believe in me, it were better for him that a millstone were hanged about his neck, and that he were drowned in the depth of the sea. Woe unto the world because of offences! for it must needs be that offences come; but woe to that man by whom the offence cometh! Wherefore if thy hand or thy foot offend thee, cut them off, and cast them from thee: it is better for thee to enter into life halt or maimed, rather than having two hands or two feet to be cast into everlasting fire. And if thine eye offend thee, pluck it out, and cast it from thee: it is better for thee to enter into life with one eye, rather than having two eyes to be cast into hell fire. Take heed that ye despise not one of these little ones; for I say unto you, That in heaven their angels do always behold the face of my Father which is in heaven.}


\section{Saint Luke the Evangelist}
\subsection{\stylesubsec{}{}{[October 18]}}

\subsection{\stylesubsec{}{The Collect.}{}}
\drop{Almighty God, who calledst Luke the Physician, whose praise is in the Gospel, to be an Evangelist, and Physician of the soul; May it please thee, that, by the wholesome medicines of the doctrine delivered by him, all the diseases of our souls may be healed; through the merits of thy Son Jesus Christ our Lord. \R Amen.}

\subsection{\stylesubsec{}{The Epistle.}{2 Timothy 4.~5.}}
\drop{Watch thou in all things, endure afflictions, do the work of an evangelist, make full proof of thy ministry. For I am now ready to be offered, and the time of my departure is at hand. I have fought a good fight, I have finished my course, I have kept the faith: Henceforth there is laid up for me a crown of righteousness, which the Lord, the righteous judge, shall give me at that day: and not to me only, but unto all them also that love his appearing. Do thy diligence to come shortly unto me: For Demas hath forsaken me, having loved this present world, and is departed unto Thessalonica; Crescens to Galatia, Titus unto Dalmatia. Only Luke is with me. Take Mark, and bring him with thee: for he is profitable to me for the ministry. And Tychicus have I sent to Ephesus. The cloak that I left at Troas with Carpus, when thou comest, bring with thee, and the books, but especially the parchments. Alexander the coppersmith did me much evil: the Lord reward him according to his works: Of whom be thou ware also; for he hath greatly withstood our words.}

\subsection{\stylesubsec{}{The Gospel.}{St.~Luke 10.~1.}}
\drop{The Lord appointed other seventy also, and sent them two and two before his face into every city and place, whither he himself would come. Therefore said he unto them, The harvest truly is great, but the labourers are few: pray ye therefore the Lord of the harvest, that he would send forth labourers into his harvest. Go your ways: behold, I send you forth as lambs among wolves. Carry neither purse, nor scrip, nor shoes: and salute no man by the way. And into whatsoever house ye enter, first say, Peace be to this house. And if the son of peace be there, your peace shall rest upon it: if not, it shall turn to you again. And in the same house remain, eating and drinking such things as they give: for the labourer is worthy of his hire.}

\section{Saint Simon and Saint Jude, Apostles}
\subsection{\stylesubsec{}{}{[October 28]}}

\subsection{\stylesubsec{}{The Collect.}{}}
\drop{O almighty God, who hast built thy Church upon the foundation of the Apostles and Prophets, Jesus Christ himself being the head corner-stone; Grant us so to be joined together in unity of spirit by their doctrine, that we may be made an holy temple acceptable unto thee; through Jesus Christ our Lord. \R Amen.}

\subsection{\stylesubsec{}{The Epistle.}{St.~Jude 1.}}
\drop{Jude, the servant of Jesus Christ, and brother of James, to them that are sanctified by God the Father, and preserved in Jesus Christ, and called: Mercy unto you, and peace, and love, be multiplied. Beloved, when I gave all diligence to write unto you of the common salvation, it was needful for me to write unto you, and exhort you that ye should earnestly contend for the faith which was once delivered unto the saints. For there are certain men crept in unawares, who were before of old ordained to this condemnation, ungodly men, turning the grace of our God into lasciviousness, and denying the only Lord God, and our Lord Jesus Christ. I will therefore put you in remembrance, though ye once knew this, how that the Lord, having saved the people out of the land of Egypt, afterward destroyed them that believed not. And the angels which kept not their first estate, but left their own habitation, he hath reserved in everlasting chains under darkness unto the judgment of the great day. Even as Sodom and Gomorrha, and the cities about them in like manner, giving themselves over to fornication, and going after strange flesh, are set forth for an example, suffering the vengeance of eternal fire. Likewise also these filthy dreamers defile the flesh, despise dominion, and speak evil of dignities.}

\subsection{\stylesubsec{}{The Gospel.}{St.~John 15.~17.}}
\drop{These things I command you, that ye love one another. If the world hate you, ye know that it hated me before it hated you. If ye were of the world, the world would love his own: but because ye are not of the world, but I have chosen you out of the world, therefore the world hateth you. Remember the word that I said unto you, The servant is not greater than his Lord. If they have persecuted me, they will also persecute you; if they have kept my saying, they will keep yours also. But all these things will they do unto you for my name's sake, because they know not him that sent me. If I had not come and spoken unto them, they had not had sin: but now they have no cloak for their sin. He that hateth me hateth my Father also. If I had not done among them the works which none other man did, they had not had sin: but now have they both seen and hated both me and my Father. But this cometh to pass, that the word might be fulfilled that is written in their law, They hated me without a cause. But when the Comforter is come, whom I will send unto you from the Father, even the Spirit of truth, which proceedeth from the Father, he shall testify of me: And ye also shall bear witness, because ye have been with me from the beginning.
}
\section{All Saints’ Day.}
\subsection{\stylesubsec{}{}{[November 1]}}

\subsection{\stylesubsec{}{The Collect.}{}}
\drop{O almighty God, who hast knit together thine elect in one communion and fellowship, in the mystical body of thy Son Christ our Lord; Grant us grace so to follow thy blessed Saints in all virtuous and godly living, that we may come to those unspeakable joys, which thou hast prepared for them that unfeignedly love thee; through Jesus Christ our Lord. \R Amen.}

\subsection{\stylesubsec{}{For the Epistle.}{Revelation 7.~2.}}
\drop{And I saw another angel ascending from the east, having the seal of the living God: and he cried with a loud voice to the four angels, to whom it was given to hurt the earth and the sea, Saying, Hurt not the earth, neither the sea, nor the trees, till we have sealed the servants of our God in their foreheads. And I heard the number of them which were sealed: and there were sealed an hundred and forty and four thousand of all the tribes of the children of Israel.
    Of the tribe of Juda were sealed twelve thousand.
    Of the tribe of Reuben were sealed twelve thousand.
    Of the tribe of Gad were sealed twelve thousand.
    Of the tribe of Aser were sealed twelve thousand.
    Of the tribe of Nephthalim were sealed twelve thousand.
    Of the tribe of Manasses were sealed twelve thousand.
    Of the tribe of Simeon were sealed twelve thousand.
    Of the tribe of Levi were sealed twelve thousand.
    Of the tribe of Issachar were sealed twelve thousand.
    Of the tribe of Zabulon were sealed twelve thousand.
    Of the tribe of Joseph were sealed twelve thousand.
    Of the tribe of Benjamin were sealed twelve thousand.
    After this I beheld, and, lo, a great multitude, which no man could number, of all nations, and kindreds, and people, and tongues, stood before the throne, and before the Lamb, clothed with white robes, and palms in their hands; And cried with a loud voice, saying, Salvation to our God which sitteth upon the throne, and unto the Lamb. And all the angels stood round about the throne, and about the elders and the four beasts, and fell before the throne on their faces, and worshipped God, Saying, Amen: Blessing, and glory, and wisdom, and thanksgiving, and honour, and power, and might, be unto our God for ever and ever. Amen.}

\subsection{\stylesubsec{}{The Gospel.}{St.~Matthew 5.~1.}}
\drop{Jesus, seeing the multitudes, went up into a mountain: and when he was set, his disciples came unto him: And he opened his mouth, and taught them, saying, Blessed are the poor in spirit: for theirs is the kingdom of heaven. Blessed are they that mourn: for they shall be comforted. Blessed are the meek: for they shall inherit the earth. Blessed are they which do hunger and thirst after righteousness: for they shall be filled. Blessed are the merciful: for they shall obtain mercy. Blessed are the pure in heart: for they shall see God. Blessed are the peacemakers: for they shall be called the children of God. Blessed are they which are persecuted for righteousness' sake: for theirs is the kingdom of heaven. Blessed are ye, when men shall revile you, and persecute you, and shall say all manner of evil against you falsely, for my sake. Rejoice, and be exceeding glad: for great is your reward in heaven: for so persecuted they the prophets which were before you.}

\fleuron

Of a Martyr or Martyrs
The Collect
\drop{Almighty God, by whose grace and power thy holy Martyr {N.} or {M.} triumphed over suffering and despised death: Grant, we beseech thee, that enduring hardness, and waxing valiant in fight, we may with [them] (the noble army of martyrs) receive the crown of everlasting life; through Jesus Christ our Lord.  \R Amen.}
% [1923, 1923]; (1928)
Epistle: Hebrews 11. 32—12. 2. And what shall I . . . the throne of God.
Gospel: St.~Matthew 16. 24—27. Jesus said to his disciples, If any man . . . to his works


Of a Confessor or Doctor
\drop{O God, who hast enlightened thy Church by the [example and] teaching of thy [Confessors and Doctors] (servent {N.}): Enrich it evermore, we beseech thee, with thy heavenly grace, and raise up faithful witnesses, who by their life and doctrine may set forth to all men the truth of thy salvation; through Jesus Christ our Lord.  \R Amen.}
% [1923, 1923]; (1928)
Lesson: Wisdom 7. 7–14.  I prayed, and understanding . . . from learning.
Gospel: St.~Matthew 13. 51–52. Jesus saith . . . new and old.

Bishop: Alcuin [g 65]


Of a Virgin or Matron
Collect
Graciously hear us, O God of our salvation: that like as we do rejoice in the festival of thy blessed [Virgin and Martyr, {or} Virgin, {or} Martyr] Saint N. [and her companions] so we may learn to follow her [{or} them] in all godly and devout affections: through Jesus Christ our Lord. \R Amen.
%1923
See Alcuin page 67





O merciful God, ........ of the Lord; and ..... may indeed be ...... , one God, world....
Merciful Lord, hear the prayers of thy servants who commemorate the Nativity of the Mother of God; and grant that by the incarnation of thy dear Son, we may be indeed made nigh into him, who liveth and reigneth with thee and the Holy Ghost, ever one God, world without end.  Amen.
The Priest's Book of Private Devotion - 1899
The Day-hours of the Church of England - 1891



\chapter{A Devotion}

{\centering\rubric{which may be said by the Priest and people immediately before the celebration of the Holy Communion.}\par}


\pilcrow{The Priest, standing at God’s Board, shall say with the Ministers and the people, all kneeling, as follows.}

\drop{In the name of the Father, \grecross\ and of the Son, and of the Holy Ghost. Amen.}

\rubric{Anthem.} I will go unto the altar of God, \star\ even unto the God of my joy and gladness.

\subsection{\stylesubsec{Psalm 43.}{Judica me, Deus.}{}}

\drop{Give sentence with me, O God, and defend my cause against the ungódly péople; \star\  O deliver me from the decéitful and wícked man.}

2\enspace For thou art the God of my strength, why hast thou pút me fróm thee? \star\  and why go I so heavily, while the enemý opprésseth me?

3\enspace O send out thy light and thy truth, that théy may léad me, \star\  and bring me unto thy holy hill, and tó thy dwélling.

4\enspace And that I may go unto the altar of God, even unto the God of my jóy and gládness; \star\  and upon the harp will I give thanks unto thée, O Gód, my God.

5\enspace Why art thou so héavy, O my sóul? \star\  and why art thou so disquietéd withín me?

6\enspace O pút thy trust in Gód; \star\  for I will yet give him thanks, which is the help of my cóuntenance, ánd my God.

Glory be to the Father, and to the Son, \star\  and to the Holy Ghost;

As it was in the beginning, is now, and ever shall be, \star\  world without end. Amen.

\rubric{Anthem.} I will go unto the altar of God, even unto the God of my joy and gladness.

\medskip

\V Our help standeth in the name of the Lord;  \R Who hath made heaven and earth.

\medskip

\rubric{If desired, the Confession and Absolution may be said here, and ommitted in the Order of Communion.}

\medskip

\V Wilt thou not turn again and quicken us;  \R That thy people may rejoice in thee?

\V O Lord, show thy mercy upon us;  \R And grant us thy salvation.

\V O Lord, hear our prayer;  \V And let our cry come unto thee.

\V The Lord be with you; \V And with thy spirit.

\centerline{Let us pray.}

\pilcrow{Then shall the Priest proceed with the celebration of the Holy Communion}

\chapter[Holy Communion]{\stylechapter{The Order for the Administration of the Lord's Supper, or}{Holy Communion}{Commonly Called the Mass.}}


\pilcrow{It is an ancient and laudable custom of the Church to receive this Holy Sacrament fasting. Yet for the avoidance of all scruple it is hereby declared that such preparation may be used or not used, according to every man’s conscience in the sight of God.}

\medskip
% \pilcrow{So many as intend to be partakers of the holy Communion shall signify their names to the Curate at least some time the day before.}

% And if any of those be an open and notorious evil liver, or have done any wrong to his neighbours by word or deed, so that the Congregation be thereby offended; the Curate having knowledge thereof, shall call him and advertise him, that in any wise he presume not to come to the Lord's Table, until he have openly declared himself to have truly repented and amended his former naughty life, that the Congregation may thereby be satisfied, which before were offended; and that he have recompensed the parties, to whom he hath done wrong; or at least declare himself to be in full purpose so to do, as soon as he conveniently may.

% The same order shall the Curate use with those betwixt whom he perceiveth malice and hatred to reign; not suffering them to be partakers of the Lord's Table. until he know them to be reconciled. And if one of the parties so at variance be content to forgive from the bottom of his heart all that the other hath trespassed against him, and to make amends for that he himself hath offended; and the other party will not be persuaded to a godly unity, but remain still in his frowardness and malice: the Minister in that case ought to admit the penitent person to the holy Communion, and not him that is obstinate.

% And when any person is warned as in the two precedent paragraphs not to come to the Lord's Table, the Minister shall inform him that the case shall be laid before the Bishop of the Diocese without delay, and that pending the judgement of the Bishop he cannot be admitted to the Holy Communion.

% And on every such occasion as is set forth in the three precedent paragraphs, the Minister shall immediately give an account of the case to the Bishop and shall await his directions. And if occasion require, the Ordinary shall proceed against the offending person according to the Canon.

% ¶ The Priest shall say the Service following in a distinct and audible voice.

\pilcrow{The Holy Table, having at the Communion time a fair white linen cloth upon it, with other decent furniture meet for the high Mysteries there to be celebrated, shall stand at the uppermost part of the Chancel or Church. And the Priest, standing at the Holy Table, shall say the Lord’s Prayer, with the collect following for due preparation, the people kneeling.} % 1912 Scott (this only) 


% 1549: Upon the date and at the tyme appoincted for the ministracion of the holy Communion, the Priest that shal execute the holy ministery, shall put upon hym the vesture appoincted for that ministracion, that is to saye: a white Albe plain, with a vestement or Cope. And where there be many Priestes, or Decons, there so many shalbe ready to helpe the Priest, in the ministracion, as shalbee requisite: And shall have upon them lykewise the vestures appointed for their ministery, that is to saye, Albes with tunacles. Then shall the Clerkes syng in Englishe for the office, or Introite, (as they call it,) a Psalme appointed for that daie.

% 1549: The Priest standing humbly afore the middes of the Altar, shall saie the Lordes praier, with this Collect.

\section{The Introduction}

\ourFather

\subsection{\stylesubsec{}{The Collect.}{}}
\drop{Almighty God, unto whom all hearts be open, all desires known, and from whom no secrets are hid; Cleanse the thoughts of our hearts by the inspiration of thy Holy Spirit, that we may perfectly love thee, and worthily magnify thy holy Name; through Christ our Lord. \R Amen.}

\bigskip
%location from green book; text from am28

\pilcrow{Here may be sung a Hymn or an Anthem.}
% he saie a Psalme appointed for the introite: whiche Psalme ended the Priest shall saye, or els the Clerkes shal syng,}

\bigskip

% iii. Lorde have mercie upon us.
% iii. Christ have mercie upon us.
% iii. Lorde have mercie upon us.

% Then the Prieste standyng at Goddes borde shall begin,

% Glory be to God on high.


\pilcrow{Then shall the Priest, turning to the people, rehearse distinctly all the TEN COMMANDMENTS; and the people, still kneeling, shall after every Commandment ask God mercy for their transgression of every duty therein (either according to the letter or according to the spiritual import thereof) for the time past, and grace to keep the same for the time to come, as followeth.} %1923, includes 1912 Scott

\smallskip

\centerline{God spake these words, and said,}
\drop{I am the Lord thy God: Thou shalt have none other gods but me.}

\R Lord, have mercy upon us, and incline our hearts to keep this law.

II. Thou shalt not make to thyself any graven image, nor the likeness of any thing that is in heaven above, or in the earth beneath, or in the water under the earth. Thou shalt not bow down to them, nor worship them.%: for I the Lord thy God am a jealous God, and visit the sins of the fathers upon the children unto the third and fourth generation of them that hate me, and shew mercy unto thousands in them that love me, and keep my commandments.

\R Lord, have mercy upon us, and incline our hearts to keep this law.

III. Thou shalt not take the Name of the Lord thy God in vain%: for the Lord will not hold him guiltless, that taketh his Name in vain.
.

\R Lord, have mercy upon us, and incline our hearts to keep this law.

IV. Remember that thou keep holy the Sabbath-day. Six days shalt thou labour, and do all that thou hast to do; but the seventh day is the Sabbath of the Lord thy God.% [In it thou shalt do no manner of work, thou, and thy son, and thy daughter, thy man-servant, and thy maid-servant, thy cattle, and the stranger that is within thy gates. For in six days the Lord made heaven and earth, the sea, and ail that in them is, and rested the seventh day: wherefore the Lord blessed the seventh day, and hallowed it.]

\R Lord, have mercy upon us, and incline our hearts to keep this law.

V. Honour thy father and thy mother.%; that thy days may be long in the land which the Lord thy God giveth thee.


\R Lord, have mercy upon us, and incline our hearts to keep this law.

VI. Thou shalt do no murder.

\R Lord, have mercy upon us, and incline our hearts to keep this law.

VII. Thou shalt not commit adultery.
    
\R Lord, have mercy upon us, and incline our hearts to keep this law.

VIII. Thou shalt not steal.

\R Lord, have mercy upon us, and incline our hearts to keep this law.

IX. Thou shalt not bear false witness% against thy neighbour.
.

\R Lord, have mercy upon us, and incline our hearts to keep this law.

X. Thou shalt not covet% thy neighbour's house, thou shalt not covet thy neighbour's wife, nor his servant, nor his maid, nor his ox, nor his ass, nor any thing that is his.
.

\R Lord, have mercy upon us, and write all these thy laws in our hearts, we beseech thee.

\medskip

%scottish 1912
\subsubsection{Or the Priest may rehearse, instead of the Ten Commandments, the Summary of the Law as followeth:}

%Irish, Canadian, American
\centerline{Hear what our Lord Jesus Christ saith.}
 
% \drop{Hear, O Israel, the Lord our God is one Lord: and thou shalt love the Lord thy God with all thy heart, and with all thy soul, and with all thy mind, with all thy strength: This is the first commandment; And the second is like, namely this, Thou shalt love thy neighbour as thyself: there is none other commandment greater than these.}\scripture{St.~Mark xij.~29.}
% On these two commandments hang all the Law and the Prophets.

\drop{Thou shalt love the Lord thy God with all thy heart, and with all thy soul, and with all thy mind.  This is the first and great commandment.  And the second is like unto it, Thou shalt love thy neighbour as thyself.  On these two commandments hang all the law and the prophets.}\scripture{St.~Matthew xxij.~37.}

\R Lord, have mercy upon us, and write these thy laws in our hearts, we beseech thee.


%\rubric{Or else, instead of the Ten Commandments or the Summary of the Law, may be sung or said on week-days, not being Great Festivals, as followeth:}

\medskip

\pilcrow{Here, if the Decalogue hath been omitted, shall be said or sung,}

\smallskip

\centerline{Lord, have mercy upon us. \rubric{iij.}}
\centerline{Christ, have mercy upon us. \rubric{iij.}}
\centerline{Lord, have mercy upon us. \rubric{iij.}}
\subsubsection{or}

\centerline{Kyrie eleison. \rubric{iij.}}
\centerline{Christe, eleison. \rubric{iij.}}
\centerline{Kyrie eleison. \rubric{iij.}}

\bigskip

\pilcrow{Then, on Sundays and on Feast days (except in Advent and from Septuagesima to Palm Sunday inclusive), shall be said or sung as follows:}

\drop{Glory be to God on high, and in earth peace, good will towards men. We praise thee, we bless thee, \footnote{Bow} we worship thee, we glorify thee, we give thanks to thee for thy great glory, O Lord God, heavenly King, God the Father Almighty.}
%\emph{and to thee, O God, the only begotten Son Jesu Christ; and to thee, O God, the Holy Ghost. - Scottish}

O Lord, the only-begotten Son Jesu Christ; O Lord God, Lamb of God, Son of the Father, that takest away the sins of the world, have mercy upon us. 
%\emph{Thou that takest away the sins of the world, have mercy upon us. - English} 
Thou that takest away the sins of the world, * receive our prayer. Thou that sittest at the right hand of God the Father, have mercy upon us.

For thou only art holy; thou only art the Lord; thou only, * O {Jesu} Christ, with the Holy Ghost, art most high \grecross\ in the glory of God the Father. Amen.

% Rubric from "The Orange Book" - 1923
{\footnotesize\rubric{This hymn may be omitted here, and sung instead at the end of this Order after the Thanksgiving after Communion.}\par}

\bigskip
%1549, 1912 Scottish:
\pilcrow {Then the priest shall turn him to the people and say,}
\V The Lord be with you. \R And with thy spirit.

% \pilcrow{Then shall follow one of these two Collects for the Queen, the Priest standing as before, and saying,}
\centerline{Let us pray.}
% \drop{Almighty God, whose kingdom is everlasting, and power infinite: Have mercy upon the whole Church; and so rule the heart of thy chosen servant ELIZABETH, our Queen and Governor, that she (knowing whose minister she is) may above all things seek thy honour and glory; and that we, and all her subjects (duly considering whose authority she hath) may faithfully serve, honour, and humbly obey her, in thee, and for thee, according to thy blessed Word and ordinance; through Jesus Christ our Lord, who with thee and the Holy Ghost liveth and reigneth, ever one God, world without end. \R Amen.}

% Or,

% \drop{Almighty and everlasting God, we are taught by thy Holy Word, that the hearts of Kings are in thy rule and governance, and that thou dost dispose and turn them as it seemeth best to thy godly wisdom: We humbly beseech thee so to dispose and govern the heart of ELIZABETH thy Servant, our Queen and Governor, that, in all her thoughts, words, and works, she may ever seek thy honour and glory, and study to preserve thy people committed to her charge, in wealth, peace, and godliness: Grant this, O merciful Father, for thy dear Son's sake, Jesus Christ our Lord. \R Amen.}

% \pilcrow{Then the Priest, turning to the Holy Table, shall say the \emph{Collect} or Collects.}
%En28
\medskip
\pilcrow{And turning to the Holy Table he shall say the Collect of the Day. Other collects contained in this Book or authorized by the Bishop may follow.}

\medskip

\section{The Ministry of the Word}

\pilcrow{Immediately thereafter he that readeth the Epistle shall say,}
The Epistle [\rubric{or,} The portion of Scripture appointed for the Epistle] is written in the --- chapter of --- beginning at the --- verse. 
\rubric{And the Epistle ended, he shall say,} Here endeth the Epistle.

\medskip
% From the Am28
\pilcrow{Here may be sung a Hymn or an Anthem.}
\medskip

\pilcrow{Then the Deacon or Priest that readeth the Gospel (the people all standing up) shall say,}
\V The Lord be with you. \R And with thy Spirit. 

\drop{The \grealtcross\ Holy Gospel is written in the — chapter of — beginning at the — verse. \R Glory \grecross\ be to thee, O Lord.}
% \pilcrow{He that readeth the Epistle or the Gospel shall so stand and turn himself as he may best be heard of the people.}
% \rubric{And after the Gospel the people may in like manner say or sing,}
% Thanks be to thee, O Lord, for this thy glorious Gospel. %1912: 

\centerline{\rubric{The Gospel ended, there may be said,}}
\centerline{Praise be to thee, O Christ.}
%Am28, Green, En28
% ¶ And after the Gospel may be said,
% Praise be to thee, O Christ.

\bigskip

\pilcrow{Then shall be sung or said the Creed following, the people still standing as before: except that at the discretion of the Minister it may be omitted on any day not being a Sunday or a Holy-day.} %Eng28

\drop{I believe in one God the Father Almighty, Maker of heaven and earth, And of all things visible and invisible:}

And in one Lord Jesus Christ, the only-begotten son of God, Begotten of his Father before all worlds, God of God, Light of Light, Very God of very God, Begotten, not made, Being of one substance with the Father, By whom all things were made: Who for us men, and for our salvation came down from heaven, * And was incarnate by the Holy Ghost of the Virgin Mary, * And was made man, * And was crucified also for us under Pontius Pilate. He suffered and was buried, And the third day he rose again according to the Scriptures, And ascended into heaven, And sitteth on the right hand of the Father. And he shall come again with glory to judge both the quick and the dead: Whose kingdom shall have no end.

And I believe in the Holy Ghost, The Lord and giver of life, Who proceedeth from the Father and the Son, Who with the Father and the Son together is worshipped and glorified, Who spake by the Prophets. And I believe One {Holy} Catholick and Apostolick Church. I acknowledge one Baptism for the remission of sins. And I look for the Resurrection of the dead, * And the life of the world to come. Amen.

\bigskip
\pilcrow{Then the Curate shall declare unto the people what Holy-days, or Fasting-days, are in the week following to be observed. And then also (if occasion be) shall notice be given of the Holy Communion, or of other services; Banns of matrimony may be published, and Briefs, Citations, and Excommunications shall be read, and Bidding of Prayers may be made. And nothing shall be proclaimed or published in the Church during the time of Divine Service, but by the Minister: nor by him any thing, but what is prescribed in the rules of this Book, or enjoined by the %Queen, or by the 
Ordinary of the place.}% adapted toward En1928


\smallskip


\pilcrow{Then may follow the Sermon, or one of the Homilies already set forth, or hereafter to be set forth, by authority.}


\smallskip


%scottish 1912
\pilcrow{When the Priest giveth warning of the Holy Communion he may, at his discretion, use the first or the second of the Exhortations appended to this Liturgy.}


\smallskip


% \pilcrow{The third Exhortation appended to this Liturgy may be used at the discretion of the Presbyter before the Offertory, the people standing.}
\medskip

\pilcrow{At the time of the celebration of the Holy Communion, the communicants being conveniently placed for the receiving of the Holy Sacrament, the Priest may say this Exhortation.}

\drop{Dearly beloved in the Lord, ye that mind to come to the holy Communion of the Body and Blood of our Saviour Christ, must consider how Saint Paul exhorteth all persons diligently to try and examine themselves, before they presume to eat of that Bread, and drink of that Cup. For as the benefit is great, if with a true penitent heart and lively faith we receive that holy Sacrament; \emph{(for then we spiritually eat the flesh of Christ, and drink his blood; then we dwell in Christ, and Christ in us; we are one with Christ, and Christ with us;)} so is the danger great, if we receive the same unworthily.}
\rubric{For then we are guilty of the Body and Blood of Christ our Saviour; we eat and drink our own damnation, not considering the Lord's Body; we kindle God's wrath against us; we provoke him to plague us with divers diseases, and sundry kinds of death.}

Judge therefore yourselves, brethren, that ye be not judged of the Lord; repent you truly for your sins past; have a lively and stedfast faith in Christ our Saviour; amend your lives, and be in perfect charity with all men; so shall ye be meet partakers of those holy mysteries. And above all things ye must give most humble and hearty thanks to God, the Father, the Son, and the Holy Ghost, for the redemption of the world by the death and passion of our Saviour Christ, both God and man; who did humble himself, even to the death upon the Cross, for us, miserable sinners, who lay in darkness and the shadow of death; that he might make us the children of God, and exalt us to everlasting life.

And to the end that we should alway remember the exceeding great love of our Master, and only Saviour, Jesus Christ, thus dying for us, and the innumerable benefits which by his precious blood-shedding he hath obtained to us; he hath instituted and ordained holy mysteries, as pledges of his love, and for a continual remembrance of his death, to our great and endless comfort. To him therefore, with the Father and the Holy Ghost, let us give (as we are most bounden) continual thanks; submitting ourselves wholly to his holy will and pleasure, and studying to serve him in true holiness and righteousness all the days of our life. Amen.


\bigskip

\section{The Offertory}
\label{offertory}
% From the "Green Book"
\pilcrow{Then shall the Priest return to the Lord's Table, and begin the \emph{Offertory}. The Priest shall say, or the Clerks shall sing, one of these Sentences following, or some other convenient sentence taken out of Holy Scripture. A Hymn may follow.}
% \pilcrow{1549 : ¶ Then shall folowe for the Offertory, one or mo, of these Sentences of holy scripture, to bee song whiles the People doo offer, or els one of them to bee saied by the minister, immediatly afore the offeryng.}

% \pilcrow{Then shall the Priest return to the Lord's Table, and begin the Offertory, saying one or more of these Sentences following, as he thinketh most convenient in his discretion.}

\drop{Let your light so shine before men, that they may see your good works, and glorify your Father which is in heaven.}\scripture{St.~Matthew v.~16.}

Lay not up for yourselves treasure upon the earth; where the rust and moth doth corrupt, and where thieves break through and steal: but lay up for yourselves treasures in heaven; where neither rust nor moth doth corrupt, and where thieves do not break through and steal.\scripture{St.~Matthew vj.~19.}

Whatsoever ye would that men should do unto you, even so do unto them; for this is the Law and the Prophets.\scripture{St.~Matthew vij.~12.}

Not every one that saith unto me, Lord, Lord, shall enter into the kingdom of heaven; but he that doeth the will of my Father which is in heaven.\scripture{St.~Matthew vij.~21.}

Remember the words of the Lord Jesus, how he said, It is more blessed to give than to receive.\scripture{Acts xx.~35.}

Godliness is great riches, if a man be content with that he hath: for we brought nothing into the world, neither may we carry any thing out.\scripture{1 Timothy vj.~6.}

Be merciful after thy power. If thou hast much, give plenteously: if thou hast little, do thy diligence gladly to give of that little: for so gatherest thou thyself a good reward in the day of necessity.\scripture{Tobit iv.~8.}

%\emph{Zacchæus stood forth, and said unto the Lord, Behold, Lord, the half of my goods I give to the poor; and if I have done any wrong to any man, I restore four-fold.\scripture{St.~Luke xix.}} Omitted in En28

%\emph{Who goeth a warfare at any time of his own cost? Who planteth a vineyard, and eateth not of the fruit thereof? Or who feedeth a flock, and eateth not of the milk of the flock?\scripture{1 Corithians ix}} Omitted in En28

If we have sown unto you spiritual things, is it a great matter if we shall reap your worldly things?\scripture{1 Corithians ix.~11.}

Do ye not know, that they who minister about holy things live of the sacrifice; and they who wait at the altar are partakers with the altar? Even so hath the Lord also ordained, that they who preach the Gospel should live of the Gospel.\scripture{1 Corithians ix.~13.}

He that soweth little shall reap little; and he that soweth plenteously shall reap plenteously. Let every man do according as he is disposed in his heart, not grudging, or of necessity; for God loveth a cheerful giver.\scripture{2 Corithians ix.~6.}

Let him that is taught in the Word minister unto him that teacheth, in all good things. Be not deceived, God is not mocked: for whatsoever a man soweth that shall he reap.\scripture{Galatians vj.~6.}

While we have time, let us do good unto all men; and specially unto them that are of the household of faith.\scripture{Galatians vj.~10.}

God is not unrighteous, that he will forget your works, and labour that proceedeth of love; which love ye have shewed for his Name’s sake, who have ministered unto the saints, and yet do minister.\scripture{Hebrews vj.~10.}

Lift up your eyes and look upon the fields; for they are white already to harvest.\scripture{St.~John iv.~35.}

Charge them who are rich in this world, that they be ready to give, and glad to distribute; laying up in store for themselves a good foundation against the time to come, that they may attain eternal life.\scripture{1 Timothy vj.~17.}

Whoso hath this world’s good, and seeth his brother have need, and shutteth up his compassion from him, how dwelleth the love of God in him?\scripture{1 St.~John iij.~17}

Blessed be the man that provideth for the sick and needy: the Lord shall deliver him in the time of trouble.\scripture{Psalm xli.~1.}

To do good, and to distribute, forget not; for with such sacrifices God is pleased.\scripture{Hebrews xiij.~16.}

Offer unto God thanksgiving, and pay thy vows unto the most Highest.\scripture{Psalm l.~14.}

I will offer in his dwelling an oblation with great gladness: I will sing and speak praises unto the Lord.\scripture{Psalm xxvij.~6.}

Melchizedek king of Salem brought forth bread and wine; and he was the priest of the most high God.\scripture{Genesis xiv.~18.}

%\emph{Give alms of thy goods, and never turn thy face from any poor man; and then the face of the Lord shall not be turned away from thee. \scripture{Tobit iv.}} Omitted in En28

%\emph{He that hath pity upon the poor lendeth unto the Lord: and look, what he layeth out, it shall be paid him again.\scripture{Proverbs xix.}} Omitted in En28

% \pilcrow{1549 : Where there be Clerkes, thei shall syng one, or many of the sentences above written, accordyng to the length and shortenesse of the tyme, that the people be offeryng.}

% \pilcrow{In the meane time, whyles the Clerkes do syng the Offertory, so many as are disposed, shall offer unto the poore mennes boxe every one accordynge to his habilitie and charitable mynde. And at the offeryng daies appoynted, every manne and woman shall paie to the Curate, the due and accustomed offerynges.}


% \pilcrow{Then so manye as shalbe partakers of the holy Communion, shall tary still in the quire, or in some convenient place nigh the quire, the men on the one side, and the women on the other syde. All other (that mynde not to receive the said holy Communion) shall departe out of the quire, except the ministers and Clerkes.}

% \pilcrow{Than shall the minister take so muche Bread and Wine, as shalt suffice for the persons appoynted to receive the holy Communion, laiyng the breade upon the corporas, or els in the paten, or in some other comely thyng, prepared for that purpose. And puttyng ye wyne into the Chalice, or els in some faire or convenient cup, prepared for that use (if the Chalice will not serve), puttyng thereto a litle pure and cleane water: And settyng both the breade and wyne upon the Alter: then the Priest shall saye.}

\pilcrow{Whilst these Sentences are said or sung, the Deacons, Church-wardens, or other fit person appointed for that purpose, shall receive the alms for the poor, or other devotions of the people,
% in a decent basin to be provided by the Parish for that purpose; 
and reverently bring them to the Priest, who shall humbly present and place them upon the Holy Table in a decent bason to be provided for that purpose.}

\bigskip
% \pilcrow{1912: While the Presbyter distinctly pronounceth one or more of these sentences for the Offertory, the Deacon, or (if no such be present) some other fit person, shall receive the devotions of the people there present, in a bason provided for that purpose. And when all have offered, her shall reverently bring the said bason, with the offerings therein, and deliver it to the Presbyter; who shall humbly present it before the Lord, and set it upon the Holy Table.}

% \pilcrow{And when there is a Communion, the Priest shall then place upon the Table so much Bread and Wine, as he shall think sufficient. After which done, the Priest shall say,}
%1912
\pilcrow{{And when there is a Communion,} the Priest shall then offer up, and place the bread and wine prepared for the Sacrament upon the Lord’s Table; and shall say,}

\drop{Blessed be thou, O {\scshape Lord} God, for ever and ever. Thine, O {\scshape Lord}, is the greatness, and the glory, and the victory, and the majesty: for all that is in the heaven and in the earth, is thine: thine is the kingdom, O {\scshape Lord}, and thou art exalted as head above all: both riches and honour come of thee, and of thine own do we give unto thee. \R Amen.}
\scripture{1 Chronicles xxix.~10}

\bigskip
\centerline{Let us pray for the whole state of Christ's Church.} % militant here in earth.

% \begin{wrapfigure}[lineheight]{position}[overhang]{width}

\drop{Almighty and everliving God, who by thy holy Apostle hast taught us to make prayers, and supplications, and to give thanks for all men;}
% \begin{wrapfigure}[5]{r}{0.40\textwidth}{\footnotesize
% \rubric{* If there be no alms or oblations, then the words \emph{[to accept our alms and oblations]} be left out unsaid.}\par
% }\end{wrapfigure}\noindent
We humbly beseech thee most mercifully [\footnote{\rubric{If there be no alms or oblations, then the words \emph{[to accept our alms and oblations]} be left out unsaid.}}\emph{to accept our alms and \grealtcross\ oblations, and}] to receive these our prayers, which we offer unto thy Divine Majesty; beseeching thee to inspire continually the Universal Church with the spirit of truth, unity, and concord: And grant, that all they that do confess thy holy Name may agree in the truth of thy holy Word, and live in unity, and godly love.



% "M & SA" We beseech thee also to lead all nations in the way of righteousness and peace, so directing all Kings, Presidents, and ruling Authorities, that under them the world may bc godly and quietly governed. And grant unto thy servant George, our King, his Ministers and Parliaments, and all who are set in authority over us, that they may truly and impartially minister justice, to the removing of all wickedness and vice, and the maintenance of thy true religion and virtue.


% We beseech thee also to save and defend all Christian Kings, Princes, and Governours; and specially thy Servant ELIZABETH our Queen; that under her we may be godly and quietly governed: And grant unto her whole Council, and to all that are put in authority under her, that they may truly and impartially minister justice, to the punishment of wickedness and vice, and to the maintenance of thy true religion, and virtue.

%orange book
% We beseech thee also to lead all nations in the way of peace and righteousness, so directing all Kings and ruling authorities that under them the world may be godly and quietly governed.
%SA1954
We beseech thee also to lead all nations into the way of righteousness and peace, and so to direct all 
% Kings, Presidents and Rulers
ruling authorities,
that under them the world may be godly and quietly governed. 
And grant unto all that are put in authority,
% under them, 
% American
% [We beseech thee also, so to direct and dispose the hearts of all Christian Rulers,] 
that they may truly and indifferently minister justice, to the punishment of wickedness and vice, and to the maintenance of thy true religion, and virtue.

Give grace, O heavenly Father, to all Bishops, Priests, and Deacons, 
% 1912 Scot
% and Curates (1662)
% and other Ministers, (1928 Amer)
especially to thy servant \emph{N.} our bishop, %En28
that they may both by their life and doctrine set forth thy true and lively Word, and rightly and duly administer thy holy Sacraments.

And to all thy people give thy heavenly grace,
%; and specially to this congregation here present; 
that with meek heart and due reverence, they may hear and receive thy holy Word; truly serving thee in holiness and righteousness all the days of their life.
And especially we commend unto thy merciful goodness this congregation which is here assembled in thy Name, to celebrate the Commemoration of the most glorious death of thy Son.

And we most humbly beseech thee, of thy goodness, O Lord, to comfort and succour all them, who in this transitory life are in trouble, sorrow, need, sickness, or any other adversity.

%And we also bless thy holy Name for all thy servants departed this life in thy faith and fear; beseeching thee to give us grace so to follow their good examples, that with them we may be partakers of thy heavenly kingFdom.

% {\begin{multicols}{2}
% And here we do give unto thee most high praise and hearty thanks for the wonderful grace and virtue declared in all thy saints from the beginning of the world: And chiefly in the glorious and most blessed Virgin Mary, mother of thy Son Jesus Christ our Lord and God, and in the holy Patriarchs, Prophets, Apostles, and Martyrs, whose examples (O Lord) and stedfastness in the faith, and keeping thy holy commandments, grant us to follow.

% We commend unto thy mercy all other thy servants which are departed from us with the sign of faith, and now to rest in the sleep of peace: Grant unto them, we beseech thee, thy mercy and everlasting peace, and that at the day of the general resurrection, we and all they which be of the mystical body of thy Son, may all together be set on his right hand, and hear that his most joyful voice: Come ye blessed of my Father, and possess the kingdom which is prepared for you from the beginning of the world.



And here we do give unto thee most high praise and hearty thanks for the wonderful grace and virtue declared in all thy Saints, from the beginning of the world: And chiefly in the glorious and most blessed Virgin Mary, Mother of thy Son Jesus Christ our Lord and God, [in \emph{N.},] and in the holy Patriarchs, Prophets, Apostles and Martyrs, whose examples, O Lord, and stedfastness in thy faith, and keeping thy holy commandments, grant us to follow.

We commend unto thy mercy, O Lord, all other thy servants which are departed hence from us, with the sign of faith, and now do rest in the sleep of peace.  Grant unto them, we beseech thee, thy mercy, and everlasting peace, and that, at the day of the general resurrection, we and all thy servants which be of the mystical body of thy Son, may altogether be set on his right hand, and hear that his most joyful voice: Come unto me, O ye that be blessed of my Father, and possess the kingdom, which is prepared for you from the beginning of the world. 


% \end{multicols}}

Grant this, O Father, for Jesus Christ's sake, our only Mediator and Advocate. \R Amen.

\section{The Consecration}

\pilcrow{Turning himself to the people the Priest shall say,}
\V The Lord be with you. \R And with thy spirit.

\V Lift up your hearts.  \R We lift them up unto the Lord.

\V Let us give thanks unto our Lord God. \R It is meet and right so to do.

\centerline{\rubric{Then shall the Priest turn to the Lord's Table, and say,}}

\drop{It is very meet, right, and our bounden duty, that we should at all times, and in all places, give thanks unto thee, O Lord, Holy Father, Almighty, Everlasting God.}
% \rubric{* These words} [Holy Father] \rubric{must be omitted on Trinity Sunday.}

{\footnotesize\centering\rubric{Here shall follow the proper Preface, according to the time, if there be any specially appointed: or else immediately shall follow, \emph{Therefore with Angels \etc}}\par}


% \drop{Therefore with Angels and Archangels, and with all the company of heaven, we laud and magnify thy glorious Name; evermore praising thee, and saying,}
% \drop{Holy, holy, holy, Lord God of hosts, heaven and earth are full of thy glory: Glory be to thee, O Lord most High.}
% Blessed is he that cometh in the Name of the Lord;
% Hosanna in the highest.

\section*{Proper Prefaces}
\prefaceCaption{Upon}{Christmas Day,}{and seven days after.}
\drop{Because thou didst give Jesus Christ thine only Son to be born as at this time for us; who, by the operation of the Holy Ghost, was made very man of the substance of the Virgin Mary his mother; and that without spot of sin, to make us clean from all sin. Therefore with Angels, \etc}

\prefaceCaption{Upon the}{Epiphany,}{and seven days after.}
\drop{Through Jesus Christ our Lord: Who in substance of our mortal flesh manifested forth his glory: That he might bring us out of darkness into his own glorious light.  Therefore with Angels, \etc}

% (sE) Maundy Thursday

\prefaceCaption{Upon}{Easter Day,}{and seven days after.}
\drop{But chiefly are we bound to praise thee for the glorious Resurrection of thy Son Jesus Christ our Lord: for he is the very Paschal Lamb, which was offered for us, and hath taken away the sin of the world; who by his death hath destroyed death, and by his rising to life again hath restored to us everlasting life. Therefore with Angels, \etc}


\prefaceCaption{Upon}{Ascension Day,}{and seven days after.}
\drop{Through thy most dearly beloved Son Jesus Christ our Lord; who after his most glorious Resurrection manifestly appeared to all his Apostles, and in their sight ascended up into heaven to prepare a place for us; that where he is, thither we might also ascend, and reign with him in glory. Therefore with Angels, \etc}


\prefaceCaption{Upon}{Whitsunday,}{and six days after.}
\drop{Through Jesus Christ our Lord; according to whose most true promise, the Holy Ghost came down as at this time from heaven with a sudden great sound, as it had been a mighty wind in the likeness of fiery tongues, lighting upon the Apostles, to teach them, and to lead them to all truth; giving them both the gift of divers languages, and also boldness with fervent zeal constantly to preach the Gospel unto all nations; whereby we have been brought out of darkness and error into the clear light and true knowledge of thee, and of thy Son Jesus Christ. Therefore with Angels, \etc}

\prefaceCaption{Upon the Feast of}{Trinity}{only.}
\drop{Who 
with thine only-begotton Son and the Holy Ghost %AmEn1928
art one God, one Lord,
in Trinity of Persons and Unity of Substance. %AmEn1928
% ; not one only Person, but three Persons in one Substance.
For that which we believe of 
% the glory of the Father, 
thy glory, O Father, %AmEn1928
the same we believe of the Son, and of the Holy Ghost, without any difference or inequality. Therefore with Angels, \etc}


\prefaceCaption{Upon the}{Purification, Annunciation, \rubric{and} Transfiguration}{.}
\drop{Because in the Mystery of the Word made flesh, thou hast caused a new light to shine in our hearts, to give the knowledge of thy glory in the face of thy Son, Jesus Christ our Lord. Therefore with Angels, \etc} %am1928

\prefaceCaption{Upon}{All Saints’ Day}{and the Feasts of Apostles, Evangelists, and \emph{St.\ John Baptist’s Nativity}, except when the Proper Preface of any Principal Feast is appointed.} 
\drop{Who, in the multitude of thy saints, hast compassed us about with so great a cloud of witnesses that we, rejoicing in their fellowship, may run with patience the race that is set before us, and together with them may receive the crown of glory that fadeth not away. Therefore with Angels, \etc }


\subsubsection{After each of which Prefaces shall immediately be sung or said,}

\drop{Therefore with Angels and Archangels, and with all the company of heaven, we laud and magnify thy glorious Name; evermore praising thee, and saying,}
\smallskip
\drop{Holy, holy, holy, Lord God of hosts, heaven and earth are full of thy glory: Glory be to thee, O Lord most High.
\grecross\ Blessed is he that cometh in the Name of the Lord;
Hosanna in the highest.}

\bigskip

\pilcrow{When the Priest, standing before the Table, hath so ordered the Bread and Wine, that he may with the more readiness and decency break the Bread before the people, and take the Cup into his hands, he shall say the Prayer of Consecration, as followeth.}

\drop{All glory be to thee, Almighty God, our heavenly Father, for that thou of thy tender mercy didst give thine only Son Jesus Christ to suffer death upon the Cross for our redemption; who made there (by his one oblation of himself once offered) a full, perfect, and sufficient sacrifice, oblation, and satisfaction, for the sins of the whole world; and did institute, and in his holy Gospel command us to continue, a perpetual memory of that his precious death 
and sacrifice, %scottish
until his coming again;}

% 1549: Hear us, O most merciful Father, and with thy Holy Spirit and word vousafe to bless \grecross and sanctify \grecross these thy gifts and creatures of bread and wine that they may be unto us the body and blood of thy most dearly beloved Son Jesus Christ.

%1637 Heare us, O mercifull Father, we most humbly beseech thee, and of thy almighty goodnesse vouchsafe so to blesse and sanctifie with thy word and holy Spirit these thy gifts and creatures of bread and wine, that they may bee unto us the body and bloud of thy most dearly beloved Son; so that wee receiving them according to thy Sonne our Saviour Jesus Christs holy institution, in remembrance of his death and passion, may be partakers of the same his most precious body and bloud
%English: Hear us, O merciful Father, we most humbly beseech thee; and grant that we receiving these thy creatures of bread and wine, according to thy Son our Saviour Jesus Christ's holy institution, in remembrance of his death and passion, may be partakers of his most blessed Body and Blood: 


% [For, in the night (S)  (a) he took]
\drop{Who, in the same night that he was betrayed, \footnote{\rubric{Here the Priest is to take the Bread into his hands:}}took Bread; and, when he had given thanks, \footnote{\rubric{And here to touch or break the Bread:}}he brake it, and gave it to his disciples, saying, Take, eat, \footnote{\rubric{And here to lay his hand upon all the Bread.}}}

{\centering{\scshape this is my Body which is given for you:} 
Do this in remembrance of me.\par}

\medskip

Likewise after supper he \footnote{\rubric{Here he is to take the Cup into his hand:}}took the Cup; and, when he had given thanks, he gave it to them, saying, Drink ye all of this; \footnote{\rubric{And here to lay his hand upon every vessel (be it Chalice or Flagon) in which there is any Wine to be consecrated.}}

{\centering{\scshape for this is my Blood of the New Testament, which is shed for you and for many for the remission of sins:}

Do this, as oft as ye shall drink it, in remembrance of me.\par}


\centerline{\rubric{The Oblation}}
\drop{Wherefore, O Lord, and heavenly Father, according to the institution of thy dearly beloved Son our Saviour Jesus Christ, we thy humble servants do celebrate and make here before thy divine Majesty, with these thy holy gifts, {\scshape which we now offer unto thee,} the memorial thy Son hath commanded us to make; having in remembrance his blessed passion, and precious death, his mighty resurrection and glorious ascension; rendering unto thee most hearty thanks for the innumerable benefits which he hath procured unto us by the same, 
and looking for his coming again with power and great glory.} %late scottish

% ["invocation" - 1928, 1928 scottish (all) ]
% 1549: Hear us, O most merciful Father, and with thy Holy Spirit and word vousafe to bless \grecross and sanctify \grecross these thy gifts and creatures of bread and wine that they may be unto us the body and blood of thy most dearly beloved Son Jesus Christ.

%1637 Heare us, O mercifull Father, we most humbly beseech thee, and of thy almighty goodnesse vouchsafe so to blesse and sanctifie with thy word and holy Spirit these thy gifts and creatures of bread and wine, that they may bee unto us the body and bloud of thy most dearly beloved Son; so that wee receiving them according to thy Sonne our Saviour Jesus Christs holy institution, in remembrance of his death and passion, may be partakers of the same his most precious body and bloud
%English 1928: Hear us, O Merciful Father, we most humbly beseech thee, and with thy Holy and Life-giving Spirit vouchsafe to bless and sanctify both us and these thy gifts of Bread and Wine, that they may be unto us the Body and Blood of thy Son, our Saviour, Jesus Christ, to the end that, receiving the same, we may be strengthened and refreshed both in body and soul.
%English: Hear us, O merciful Father, we most humbly beseech thee; and grant that we receiving these thy creatures of bread and wine, according to thy Son our Saviour Jesus Christ's holy institution, in remembrance of his death and passion, may be partakers of his most blessed Body and Blood: 

% "Scottish office" - newer than 1637; older than 1912.  The 1637 is actually similar but goes on with the english ending.
\centerline{\rubric{The Invocation}}
\drop{Hear us, O Merciful Father, we most humbly beseech thee, % ENglish
% And we most humbly beseech thee, O merciful Father, to hear us, Scottish
and of thy almighty goodness vouchsafe to \grealtcross\  bless and \grealtcross\  sanctify, with thy Word and Holy Spirit, these thy gifts and creatures of bread and wine, that they may become the Body and Blood of thy most dearly beloved Son, }
to the end that all who shall receive the same may be sanctified both in body and soul, and preserved unto everlasting life. % (late Scottish)]


% (scottish)

\drop{And we entirely %english
% [earnestly]  %scottish
desire thy fatherly goodness mercifully to accept this our sacrifice of praise and thanksgiving; most humbly beseeching thee to grant, that by the merits and death of thy Son Jesus Christ, and through faith in his blood, we and all thy whole Church may obtain remission of our sins, and all other benefits of his passion.}

And here we %[humbly] scottish
offer and present unto thee, O Lord, our selves, our souls and bodies, to be a reasonable, holy, and lively sacrifice unto thee; 
% humbly beseeching thee, that whosoever shall be partakers 
% humbly beseeching thee, that all we, who are partakers 
humbly beseeching thee, that all we, who shall be partakers
% that we, and all others who shall be partakers
of this holy Communion, may
worthily receive the most precious Body and Blood of thy Son Jesus Christ, and
be fulfilled with thy grace and heavenly \grecross\  benediction,
and made one body with him, that he may dwell in us, and we in him.

And although we be unworthy, through our manifold sins, to offer unto thee any sacrifice, yet we beseech thee to accept this our bounden duty and service, and command these our prayers and supplications by the ministry of thy holy Angels to be brought up into thy holy Tabernacle before the sight of thy divine Majesty; not weighing our merits, but pardoning our offences,

Through Jesus Christ our Lord; by whom, and with whom, in the unity of the Holy Ghost, all honour and glory be unto thee, O Father Almighty, world without end. \R Amen.

% \centerline{Let us pray.}
% [And now,]
{\centering\rubric{Here shall the people join with the Priest in the Lord’s Prayer, the Priest first saying,}\par}
As our Saviour Christ hath commanded and taught us, we are bold to say,

\smallskip
% \pilcrow{Then shall the Priest say the Lord's Prayer, the people repeating after him every Petition.}
\drop{Our Father, which art in heaven, Hallowed be thy Name. Thy kingdom come. Thy will be done, in earth as it is in heaven. Give us this day our daily bread. And forgive us our trespasses, As we forgive them that trespass against us. And lead us not into temptation; But deliver us from evil.}

For thine is the kingdom, the power, and the glory, For ever and ever. Amen.

\centerline{\rubric{Here the Priest is to break the consecrated bread.}}

\medskip

\pilcrow{Then shall the Priest say or sing,}
\drop{The Peace of the Lord be always with you.  \R And with thy spirit.}

\medskip

\drop{Christ, our Paschal Lamb, is offered up for us, once for all, when he bare our sins on his body upon the Cross; for he is the very Lamb of God that taketh away the sins of the world, wherefore let us keep a joyful and holy feast unto the Lord.}


\section{The Communion}

\pilcrow{Then shall the Minister say to them that come to receive the Holy Communion,}

\drop{Ye that do truly and earnestly repent you of your sins, and are in love and charity with your neighbours, and intend to lead a new life, following the commandments of God, and walking from henceforth in his holy ways; Draw near with faith, and take this holy Sacrament to your comfort; and make your humble confession to Almighty God, meekly kneeling upon your knees.}

\medskip
%1662
% \pilcrow{Then shall this general Confession be made, in the name of all those that are minded to receive the holy Communion, by one of the Ministers; both he and all the people kneeling humbly upon their knees, and saying,}
%Scottish 1912
% Then shall this general confession be made by the people, along with the Presbyter; he first kneeling down.
%american 1928
\pilcrow{Then shall this \emph{General Confession be made}, by the Priest and all those who are minded to receive the Holy Communion, humbly kneeling.}

\drop{Almighty God, Father of our Lord Jesus Christ, Maker of all things, judge of all men; We acknowledge and bewail our manifold sins and wickedness, Which we, from time to time, most grievously have committed, By thought, word, and deed, Against thy Divine Majesty, Provoking most justly thy wrath and indignation against us. We do earnestly repent, And are heartily sorry for these our misdoings; The remembrance of them is grievous unto us; The burden of them is intolerable. Have mercy upon us, Have mercy upon us, most merciful Father; For thy Son our Lord Jesus Christ's sake, Forgive us all that is past; And grant that we may ever hereafter Serve and please thee In newness of life, To the honour and glory of thy Name; Through Jesus Christ our Lord. Amen.}


\medskip


\pilcrow{Then shall the Priest (or the Bishop, being present,) standing up, and turning himself to the people, pronounce this Absolution.}
\drop{Almighty God, our heavenly Father, who of his great mercy hath promised forgiveness of sins to all them that with hearty repentance and true faith turn unto him; Have mercy upon you; pardon  \grealtcross\  and deliver you from all your sins; confirm and strengthen you in all goodness; and bring you to everlasting life; through Jesus Christ our Lord. \R Amen.}

\medskip

\centerline{\pilcrow{Then shall the Priest say,}}
Hear what comfortable words our Saviour Christ saith unto all that truly turn to him.
\drop{Come unto me all that travail and are heavy laden, and I will refresh you.}\scripture{St.~Matthew xj.~28}

So God loved the world, that he gave his only-begotten Son, to the end that all that believe in him should not perish, but have everlasting life.\scripture{St.~John iij.~16}

\centerline{Hear also what Saint Paul saith.}

This is a true saying, and worthy of all men to be received, That Christ Jesus came into the world to save sinners.\scripture{1 Timothy i.~15.}

\centerline{Hear also what Saint John saith.}

If any man sin, we have an Advocate with the Father, Jesus Christ the righteous; and he is the propitiation for our sins.\scripture{1 St.~John ij.~1.}

\medskip


{\centering\footnotesize\rubric{Then shall the Priest, turning him to the Altar, kneel down, and say, in the name of all them that shall communicate, this Collect of humble access to the Holy Communion, as followeth:}\par} % Scottish rubric

\drop{We do not presume to come to this thy Table, O merciful Lord, trusting in our own righteousness, but in thy manifold and great mercies. We are not worthy so much as to gather up the crumbs under thy Table. But thou art the same Lord, whose property is always to have mercy: Grant us therefore, gracious Lord, so to eat the flesh of thy dear Son Jesus Christ, and to drink his blood, that our sinful bodies may be made clean by his body, and our souls washed through his most precious blood, and that we may evermore dwell in him, and he in us. \R Amen.}


\medskip
% Scottish Position:
\pilcrow{Here may be sung or said:} %scottish rubric
% \pilcrow{In the Communion time the choir and people may say or sing the following, beginning so soon as the Priest doth receive the holy Communion.}

\drop{O Lamb of God, that takest away the sins of the world: have mercy upon us.

O Lamb of God that, takest away the sins of the world: have mercy upon us.

O Lamb of God, that takest away the sins of the world: grant us thy peace.}
\bigskip

\pilcrow{Then shall the Minister first receive the Communion in both kinds himself, and then proceed to deliver the same to the Bishops, Priests, and Deacons, in like manner, (if any be present,) and after that to the people also in order, 
into their hands, % Percy: This has always been the custom with us.
all meekly kneeling. And, when he delivereth the Bread to any one, he shall say,}
\drop{The Body of our Lord Jesus Christ, which was given for thee, preserve thy body and soul unto everlasting life. \R Amen.}

\smallskip

{\centering\footnotesize\rubric{Take and eat this in remembrance that Christ died for thee, and feed on him in thy heart by faith with thanksgiving.}\par}

\bigskip

{\centering\footnotesize\rubric{And the Minister that delivereth the Cup to any one shall say,}\par}

\drop{The Blood of our Lord Jesus Christ, which was shed for thee, preserve thy body and soul unto everlasting life. \R Amen.}

\smallskip

{\centering\footnotesize\rubric{Drink this in remembrance that Christ's Blood was shed for thee, and be thankful.}\par}

\bigskip


% \pilcrow{If the consecrated Bread or Wine be all spent before all have communicated, the Priest is to consecrate more cording to the Form before prescribed: Beginning at [Our Saviour Christ in the same night, etc.] for the blessing of the Bread ; and at [Likewise after Supper, etc.] for the blessing of the Cup.}

\pilcrow{When all have communicated, the Minister shall return to the Lord's Table, and reverently place upon it what remaineth of the consecrated Elements, covering the same with a fair linen cloth.}

\medskip
% Green book.
\pilcrow{Here may follow an Anthem or Hymn.}

\section{The Thanksgiving}

\pilcrow{Then shall the Priest give thanks to God in the name of all them that have communicated, turning him first to the people, and saying,}

%1923 (1929 option, along with 1549)
\V O give thanks unto the Lord, for he is gracious:  \R And his mercy endureth for ever.

%1549
\V The Lord be with you. \R And with thy spirit.

\centerline{Let us pray.}

%proposed 1928, shortened from Scottish.
% Having now 
% by faith <- added! 
% received the precious Body and Blood of Christ, let us give thanks unto our Lord God,

\drop{Almighty and everliving God, we most heartily thank thee, for that thou dost vouchsafe to feed us, who have duly received these holy mysteries, with the spiritual food of the most precious Body and Blood of thy Son our Saviour Jesus Christ; and dost assure us thereby of thy favour and goodness towards us; and that we are very members incorporate in the mystical body of thy Son, which is the blessed company of all faithful people; and are also heirs through hope of thy everlasting kingdom, by the merits of the most precious death and passion of thy dear Son. And we most humbly beseech thee, O heavenly Father, so to assist us with thy grace, that we may continue in that holy fellowship, and do all such good works as thou hast prepared for us to walk in; through Jesus Christ our Lord, to whom, with thee and the Holy Ghost, be all honour and glory, world without end. Amen.}

\bigskip
% Rubric from "The Orange Book" - 1923
\pilcrow{The \emph{Gloria in excelsis} may be omitted after the \emph{Kyrie eleison}, and sung here instead: provided that it be always said or sung in one or other position on Holy-days and on all Sundays except those in Advent and from Septuagesima to Palm Sunday inclusive.}
\bigskip

\centerline{\rubric{For the Post-Communions see pages X-X}} %scottish
\bigskip

\V The Lord be with you. \R And with thy spirit.

\V Depart in peace, \rubric{or} Let us bless the Lord.

\R Thanks be to God.

\medskip
\pilcrow{Then the Priest (or Bishop if he be present) shall let them depart with this Blessing.}
\drop{The peace of God, which passeth all understanding, keep your hearts and minds in the knowledge and love of God, and of his son Jesus Christ our Lord: and the blessing of God Almighty, the Father, \grealtcross\ the Son, and the Holy Ghost, be amongst you and remain with you always. \R Amen.}

\medskip

\fleuron

\bigskip

\section{Collects}
\pilcrow{To be said after the Offertory, when there is no Communion, every such day one or more; and the same may be said also, as often as occasion shall serve, after the Collects either of Morning or Evening Prayer, or the Litany, or immediately before the Blessing at Holy Communion, by the discretion of the Minister.}

% Scottish:
\drop{O Almighty Father, wellspring of life to all things that have being, from amid the unwearied praises of Cherubim and Seraphim who stand about thy throne of light which no man can approach unto, give ear, we humbly beseech thee, to the supplications of thy people who put their sure trust in thy mercy, through Jesus Christ our Lord.  \R Amen.}
% [From the Book of Deer.]}

\smallskip


\drop{O Lord Jesus Christ, before whose judgement-seat we must all appear and give account of the things done in the body: Grant, we beseech thee, that when the books are opened in that day, the faces of thy servants may not be ashamed, through thy merits, O blessed Saviour, who livest and reignest with the Father and the Holy Spirit, one God, world without end.  \R Amen.}
% [From the Altus of St Columba]}

\smallskip

\drop{Assist us mercifully, O Lord, in these our supplications and prayers, and dispose the way of thy servants towards the attainment of everlasting salvation; that, among all the changes and chances of this mortal life, they may ever be defended by thy most gracious and ready help; through Jesus Christ our Lord. \R Amen.}

\smallskip

\drop{O Almighty Lord, and everlasting God, vouchsafe, we beseech thee, to direct, sanctify, and govern, both our hearts and bodies, in the ways of thy laws, and in the works of thy commandments; that through thy most mighty protection, both here and ever, we may be pre- served in body and soul; through our Lord and Saviour Jesus Christ. \R Amen.}

\smallskip


\drop{Grant, we beseech thee, Almighty God, that the words, which we have heard this day with our outward ears, may through thy grace be so grafted inwardly in our hearts, that they may bring forth in us the fruit of good living, to the honour and praise of thy Name; through Jesus Christ our Lord. \R Amen.}

\smallskip

\drop{Prevent us O Lord, in all our doings with thy most gracious favour, and further us with thy continual help; that in all our works begun, continued, and ended in thee, we may glorify thy holy Name, and finally by thy mercy obtain everlasting life; through Jesus Christ our Lord. \R Amen.}

\smallskip

\drop{Almighty God, the fountain of all wisdom, who knowest our necessities before we ask, and our ignorance in asking; We beseech thee to have compassion upon our infirmities; and those things, which for our unworthiness we dare not, and for our blindness we cannot ask, vouchsafe to give us, for the worthiness of thy Son Jesus Christ our Lord. \R Amen.}

\smallskip

% The two following collects may be said before the Blessing.

\drop{O Lord, our God, thou Saviour of the world, through whom we have celebrated these sacred mysteries: Receive our humble thanksgiving, and of thy great mercy vouchsafe to sanctify us evermore in body and soul, who livest and reignest with the Father and the Holy Spirit, one God, world without end.  \R Amen.}

\smallskip

\drop{Almighty God, who hast promised to hear the petitions of them that ask in thy Son's Name; We beseech thee mercifully to incline thine ears to us that have made now our prayers and supplications unto thee; and grant, that those things, which we have faithfully asked according to thy will, may effectually be obtained, to the relief of our necessity, and to the setting forth of thy glory; through Jesus Christ our Lord. \R Amen.}

    %Scottish 1912
\subsection{Prayers for Certain Festivals and Seasons}

\prefaceCaption{}{Advent.}{} % Proofed vs 1929 on 11-29-21
\drop{Grant, O Almighty God, that as thy blessed Son Jesus Christ at his first advent came to seek and to save that which was lost, so at his second and glorious appearing he may find in us the fruits of the redemption which he wrought; who liveth and reigneth, with thee and the Holy Spirit, one God, world without end. Amen.}

\prefaceCaption{}{Christmas Day,}{and seven days after.}
\drop{O God, who hast given us grace at this time to celebrate the birth of our Saviour, Jesus Christ: We laud and magnify thy glorious Name for the countless blessings which he hath brought unto us; and we beseech thee to grant that We may ever set forth thy praise in joyful obedience to thy will; through the same Jesus Christ our Lord. \R Amen.}

%1928 prop
\prefaceCaption{}{New Year’s Day}{}
\drop{O eternal Lord God, who hast brought thy servants to the beginning of another year: Pardon, we humbly beseech thee, our transgressions in the past, and graciously abide with us all the days of our life; through Jesus Christ our Lord.  \R Amen}

\prefaceCaption{}{Epiphany,}{and seven days after.}
\drop{Almighty God, who at the baptism of thy blessed Son Jesus Christ in the river Jordan didst manifest his glorious Godhead: Grant, we beseech thee, that the brightness of his presence may shine in our hearts, and his glory be set forth in our lives; through the same Jesus Christ our Lord. \R Amen.}

\prefaceCaption{}{Easter Day,}{and seven days after.}
\drop{O Lord God Almighty, whose blessed Son, our Saviour, Jesus Christ, did on the third day rise triumphant over death: Raise us, we beseech thee, from the death of sin unto the life of righteousness, that we may seek those things which are above, where he sitteth on thy right hand in glory; and this we beg for the sake of the same, thy Son, Jesus Christ our Lord. \R Amen.}

\prefaceCaption{}{Ascension Day,}{and seven days after.}
\drop{Almighty God, whose blessed Son, our Saviour, Jesus Christ, ascended far above all heavens that he might fill all things: Mercifully give us faith to perceive that according to his promise he abideth with his Church on earth, even unto the end of the world; through the same Jesus Christ our Lord. \R Amen.}

\prefaceCaption{}{Whitsunday,}{and six days after.}
\drop{O Almighty God, who on the day of Pentecost didst send the Holy Ghost the Comforter to abide in thy Church unto the end: Bestow upon us and upon all thy faithful people his manifold gifts of grace, that with minds enlightened by his truth and hearts purified by his presence, we may day by day be strengthened with power in the inward man; through Jesus Christ our Lord, who with thee and the same Spirit liveth and reigneth, one God, world without end. \R Amen.}


\prefaceCaption{}{Trinity Sunday.}{}
\drop{O Lord God Almighty, Eternal, Immortal, Invisible, the mysteries of whose being are unsearchable: Accept, we beseech thee, our praises for the revelation which thou hast made of thyself, Father, Son, and Holy Ghost, three Persons, and one God; and mercifully grant, that ever holding fast this faith, we may magnify thy glorious Name; who livest and reignest, one God, world without end. \R Amen.}




\section{Exhortations}

\pilcrow{When the Minister giveth warning for the celebration of the holy Communion, (which he shall always do upon the Sunday, or some Holy-day, immediately preceding,) after the Sermon or Homily ended, he shall read this Exhortation following.}
\drop{Dearly beloved, on — I purpose, through God’s assistance, to administer to all such as shall be religiously and devoutly disposed the most comfortable Sacrament of the Body and Blood of Christ; to be by them received in remembrance of his meritorious Cross and Passion; whereby alone we obtain remission of our sins, and are make partakers of the Kingdom of heaven.} 

Wherefore it is our duty to render most humble and hearty thanks to Almighty God our heavenly Father, for that he hath given his Son our Saviour Jesus Christ, not only to die for us, but also to be our spiritual food and sustenance in that holy Sacrament.

Which being so divine and comfortable a thing to them who receive it worthily, and so dangerous to them that will presume to receive it unworthily; my duty is to exhort you in the mean season to consider the dignity of that holy mystery, and the great peril of the unworthy receiving thereof; and so to search and examine your own consciences, (and that nor lightly, and after the manner of dissemblers with God; but so) that ye may come holy and clean to such a heavenly Feast, in the marriage-garment required by God in holy Scripture, and be received as worthy partakers of that holy Table.

The way and means thereto is; First, to examine your lives and conversations by the rule of God’s commandments; and wherein soever ye shall perceive yourselves to have offended, either by will, word, or deed, there to bewail your own sinfulness, and to confess yourselves to Almighty God, with full purpose of amendment of life. 

And if ye shall perceive your offences to be such as are not only against God, but also against your neighbours; then ye shall reconcile yourselves unto them; being ready to make restitution and satisfaction, according to the uttermost of your powers, for all injuries and wrongs done by you to any other; and being likewise ready to forgive others that have offended you, as ye would have forgiveness of your offences at God’s hand: \emph{[for otherwise the receiving of the holy Communion doth nothing else but increase your \emph{damnation} [guilt, condemnation].]}

Therefore if any of you be a blasphemer of God, an hinderer or slanderer of his Word, an adulterer, or be in malice, or envy, or in any other grievous crime, repent you of your sins, or else come not to that holy Table\rubric{; lest, after the taking of that holy Sacrament, the devil enter into you, as he entered into Judas, and fill you full of all iniquities, and bring you to destruction both of body and soul}.

And because it is requisite, that no man should come to the holy Communion, but with a full trust in God's mercy, and with a quiet conscience; therefore if there be any of you, who by this means cannot quiet his own conscience herein, but requireth further comfort or counsel, let him come to me, or to some other discreet and learned Minister of God's Word, and open his grief; that by the ministry of God's holy Word he may receive the benefit of absolution, together with ghostly counsel and advice, to the quieting of his conscience, and avoiding of all scruple and doubtfulness.

% Rubric moved to Confession.

\medskip
%a new form, composed apparently by Peter Martyr at the instance of Bucer. (i)

{\centering\rubric{Or, in case he shall see the people negligent to come to the Holy Communion, instead of the former, he may use this Exhortation.}\par}


\drop{Dearly beloved brethren, on — I intend, by God’s grace, to celebrate the Lord’s Supper: unto which, in God’s behalf, I bid you all that are here present; and beseech you, for the Lord Jesus Christ’s sake, that ye will not refuse to come thereto, being so lovingly called and bidden by God himself.}

Ye know how grievous and unkind a thing it is, when a man hath prepared a rich feast, decked his table with all kind of provision, so that there lacketh nothing but the guests to sit down; and yet they who are called (without any cause) most unthankfully refuse to come. Which of you in such a case would not be moved? Who would not think a great injury and wrong done unto him? Wherefore, most dearly beloved in Christ, take ye good heed, lest ye, withdrawing yourselves from this holy Supper, provoke God’s indignation against you. It is an easy matter for a man to say, I will not communicate, because I am otherwise hindered with worldly business. But such excuses are not so easily accepted and allowed before God. If any man say, I am a grievous sinner, and therefore am afraid to come: wherefore then do ye not repent and amend? When God calleth you, are ye not ashamed to say ye will not come? When ye should return to God, will ye excuse yourselves, and say ye are not ready? Consider earnestly with yourselves how little such feigned excuses will avail before God. They that refused the feast in the Gospel, because they had bought a farm, or would try their yokes of oxen, or because they were married, were not so excused, but counted unworthy of the heavenly feast.

I, for my part, shall be ready; and, according to mine office, I bid you in the Name of God, I call you in Christ’s behalf, I exhort you, as ye love your own salvation, that ye will be partakers of this Holy Communion. And as the Son of God did vouchsafe to yield up his soul by death upon the Cross for your salvation; so it is your duty to receive the Communion in remembrance of the sacrifice of his death, as he himself hath commanded: which if ye shall neglect to do, consider with yourselves how great injury ye do unto God, and how sore punishment hangeth over your heads for the same; when ye wilfully abstain from the Lord’s Table, and separate from your brethren, who come to feed on the banquet of that most heavenly food.

These things if ye earnestly consider, ye will by God’s grace return to a better mind: for the obtaining whereof we shall not cease to make our humble petitions unto Almighty God our heavenly Father.


\section{General Rubricks}

\pilcrow{Upon the Sundays and other Holy-days (if there be no Communion) shall be said all that is appointed at the Communion, until the end of the general Prayer \emph{For the whole state of Christ’s Church} together with one or more of these Collects last before rehearsed, concluding with the Blessing.}


\pilcrow{And whenever this Service is used, Collects, contained in this Book, or sanctioned by the Bishop, may be said after \emph{The Intercession}, or before the Blessing.}

\medskip

% 1928 Rubrics:
\pilcrow{If any of the consecrated Bread and Wine remain, apart from that which may be reserved,
% for the Communion of the sick,
it shall not be carried out of the church; but the Priest, and such other of the communicants as he shall call unto him, shall, immediately after the Blessing, reverently eat and drink the same.}

% \pilcrow{And there shall be no celebration of the Lord's Supper, except there be a convenient number to communicate with the Priest, according to his discretion.}

% \pilcrow{And if there be not above twenty persons in the Parish of discretion to receive the Communion: yet there shall be no Communion, except four (or three at the least) communicate with the Priest.}

% \pilcrow{And in Cathedral and Collegiate Churches, and Colleges, where there are many Priests and Deacons, they shall all receive the Communion with the Priest every Sunday at the least, except they have a reasonable cause to the contrary.}

% \pilcrow{And to take away all occasion of dissension, and superstition, which any person hath or might have concerning the Bread and Wine, it shall suffice that the Bread be such as is usual to be eaten; but the best and purest Wheat Bread that conveniently may be gotten.}

% \pilcrow{And if any of the Bread and Wine remain unconsecrated, the Curate shall have it to his own use: but if any remain of that which was consecrated, it shall not be carried out of the Church, but the Priest, and such other of the Communicants as he shall then call unto him, shall, immediately after the Blessing, reverently eat and drink the same.}

\pilcrow{The Bread and Wine for the Communion shall be provided by the Curate and the Church-wardens at the charges of the Parish.}


\pilcrow{After the Divine Service ended, the money given at the Offertory shall be disposed of to such pious and charitable uses, as the Minister and Church-wardens shall think fit. Wherein if they disagree, it shall be disposed of as the Ordinary shall appoint.}

\medskip

% all following from 1928Eng
\pilcrow{And note, that every confirmed member of the Church shall communicate at the least three times in the year, of which Easter to be one.} 


\pilcrow{If any be an open and notorious evil liver, or have done any wrong to his neighbours by word or deed, so that the Congregation be thereby offended; the Curate, having knowledge thereof, shall call him and advertise him, that in any wise he presume not to come to the Lord’s Table, until he hath openly declared himself to have truly repented and amended his former naughty life, that the Congregation may thereby be satisfied, which before were offended; and that he hath recompensed the parties, to whom he hath done wrong; or at least declare himself to be in full purpose so to do, as soon as he conveniently may.}


\pilcrow{The same order shall the Curate use with those betwixt whom he perceiveth malice and hatred to reign; not suffering them to be partakers of the Lord’s Table, until he known them to be reconciled. And if one of the parties so at variance be content to forgive from the bottom of his heart all that the other hath trespassed against him, and to make amends for that he himself hath offended, and the other party will not be persuaded to a godly unity, but remain still in his frowardness and malice: the Minister in that case ought to admit the penitent person to the holy Communion, and not him that is obstinate.}


\pilcrow{Provided that every Minister so advertising repelling any, as is specified in the two precedent paragraphs, shall be obliged forthwith to give an account of the same to the Bishop, and therein to obey his order and direction.}

\fleuron

%All the following from the Indian Supplement
\chapter[Order of Communion]{\stylechapter{The}{Order of Communion}{with the Reserved Sacrament.}}
\label{reservedSacrament}
\pilcrow{In the absence of a Priest, a Deacon may administer from the Reserved Sacrament, but he shall use the Collect for the Twenty-first Sunday after Trinity instead of the Absolution, and \emph{The Grace of our Lord Jesus Christ  \etc} instead of the Blessing.}

\pilcrow{When this Order is used for those unable to be present at a celebration of the Lord’s Supper but who are not sick, it shall be used with the Collect, Epistle and Gospel of the Day. At the discretion of the Minister other parts of the Order for the Lord’s Supper (except the setting apart of the Bread and Wine and the Prayer of Consecration) may be added.}

\pilcrow{When this Order is used for the sick, it is fitting that a table be prepared at a convenient place in the sick person’s house. The table shall be covered with a clean white cloth, whereon is to be placed a vessel containing a little water. The Priest shall come at the appointed time, bearing the Reserved Sacrament, and shall place it upon the table.}

% If the sick person is not too weak, the Priest may begin with the Collect, Epistle and Gospel of the Day, or else with the following:

% The Collett

% ALMIGHTY Father, giver of life and health: Look mercifully, we beseech thee, on this thy servant, that by thy blessing upon him and upon those who minister to him, he may speedily be restored to health, if it be thy gracious will, and live to the glory of thy holy name; through Jesus Christ our Lord. Amen.

% Or this,

% ASSIST us mercifully, O Lord, in these our supplications and prayers, and dispose the way of thy servant towards the attainment of everlasting salvation; that among all the changes and chances of this mortal life, he may ever be defended by thy most gracious and ready help; through Jesus Christ our Lord. Amen.

% The Epistle. 2 Corinthians 1. 3

% BLESSED be God, even the Father of our Lord Jesus Christ, the Father of mercies, and the God of all comfort; who comforteth us in all our tribulation, that we may be able to comfort them which are in any trouble, by the comfort wherewith we ourselves are comforted of God. For as the sufferings of Christ abound in us, so our consolation also aboundeth by Christ.

% The Gospel. St John 10. 15, 27-30

% IAM the good shepherd; and I know mine own, and mine own know me, even as the Father knoweth me, and I know the Father; and I lay down my life for the sheep. My sheep hear my voice, and I know them, and they follow me: and I give unto them eternal life; and they shall never perish, and no one shall pluck them out of my hand. My Father, which hath given them unto me, is greater than all; and no one is able to pluck them out of the Father's hand. I and the Father are one.
 

 
\pilcrow{Then the Priest and those present shall say the \emph{General Confession} from the Order for the Lord’s Supper, the Priest adding the \emph{Absolution} and the \emph{Comfortable Words}, or else the following shortened Confession and Absolution may be used:}
\drop{Draw near with faith, and take this Holy Sacrament to your comfort; and make your humble confession to Almighty God.}

\medskip
\drop{We confess to God Almighty, the Father, the Son, and the Holy Ghost, that we have sinned in thought, word, and deed, through our own grievous fault. Wherefore we pray God to have mercy upon us.}

\medskip
{\centering\footnotesize\rubric{After which the Priest shall pronounce this Absolution,}\par} 
\drop{Almighty God have mercy upon you, forgive \grealtcross\ you all your sins, and deliver you from all evil, confirm and strengthen you in all goodness, and bring you to life everlasting; through Jesus Christ our Lord. \R Amen.}

\medskip
{\centering\footnotesize\rubric{Then shall the Priest and those who are to receive the Sacrament say together:}\par} 
\drop{We do not presume to come to this thy Table, O merciful Lord, trusting in our own righteousness, but in thy manifold and great mercies. We are not worthy so much as to gather up the crumbs under thy Table. But thou art the same Lord, whose property is always to have mercy: Grant us therefore, gracious Lord, so to eat the flesh of thy dear Son Jesus Christ, and to drink his blood, that our sinful bodies may be made clean by his body, and our souls washed through his most precious blood, and that we may evermore dwell in him, and he in us. \R Amen.}


\medskip
% Scottish Position:
\pilcrow{Then may be sung or said:} %scottish rubric

\drop{O Lamb of God, that takest away the sins of the world: have mercy upon us.

O Lamb of God that, takest away the sins of the world: have mercy upon us.

O Lamb of God, that takest away the sins of the world: grant us thy peace.}
\bigskip

\medskip
\rubric{After a short space of silence the Priest shall deliver the Sacrament with the customary words of administration.}


\medskip

{\centering\footnotesize\rubric{After a short silence the Priest and all present shall repeat together the Lord's Prayer,}\par} 
\ourFather


\medskip
{\centering\footnotesize\rubric{Then the Priest shall say,}\par} 
\drop{Almighty and everliving God, we most heartily thank thee, for that thou dost vouchsafe to feed us, who have duly received these holy mysteries, with the spiritual food of the most precious Body and Blood of thy Son our Saviour Jesus Christ; and dost assure us thereby of thy favour and goodness towards us; and that we are very members incorporate in the mystical body of thy Son, which is the blessed company of all faithful people; and are also heirs through hope of thy everlasting kingdom, by the merits of the most precious death and passion of thy dear Son. And we most humbly beseech thee, O heavenly Father, so to assist us with thy grace, that we may continue in that holy fellowship, and do all such good works as thou hast prepared for us to walk in; through Jesus Christ our Lord, to whom, with thee and the Holy Ghost, be all honour and glory, world without end. Amen.}

\medskip
{\centering\footnotesize\rubric{And the Blessing,}\par}
\drop{The peace of God, which passeth all understanding, keep your hearts and minds in the knowledge and love of God, and of his son Jesus Christ our Lord: and the blessing of God Almighty, the Father, \grealtcross\ the Son, and the Holy Ghost, be amongst you and remain with you always. \R Amen.}


\fleuron


% \chapter{ An Alternative Order of the Communion}
% \ourFather

% \subsection{\stylesubsec{}{The Collect.}{}}
% \drop{Almighty God, unto whom all hearts be open, all desires known, and from whom no secrets are hid; Cleanse the thoughts of our hearts by the inspiration of thy Holy Spirit, that we may perfectly love thee, and worthily magnify thy holy Name; through Christ our Lord. \R Amen.}

% \medskip
% \centerline{\pilcrow{Here may be sung a Hymn or an Anthem.}}

% \medskip

% \centerline{Lord, have mercy upon us. \rubric{iij.}}
% \centerline{Christ, have mercy upon us. \rubric{iij.}}
% \centerline{Lord, have mercy upon us. \rubric{iij.}}


% \bigskip
% \pilcrow {Then the priest shall turn him to the people and say,}
% \V The Lord be with you. \R And with thy spirit.

% \centerline{Let us pray.}

% \medskip
% \pilcrow{And turning to the Holy Table he shall say the Collect of the Day. Other collects contained in this Book or authorized by the Bishop may follow.}

% \medskip

% \section{The Ministry of the Word}

% \pilcrow{Immediately thereafter he that readeth the Epistle shall say,}
% The Epistle [\rubric{or,} The portion of Scripture appointed for the Epistle] is written in the --- chapter of --- beginning at the --- verse. 
% \rubric{And the Epistle ended, he shall say,} Here endeth the Epistle.

% \medskip
% \centerline{\pilcrow{Here may be sung a Hymn or an Anthem.}}
% \medskip

% \pilcrow{Then the Deacon or Priest that readeth the Gospel (the people all standing up) shall say,}
% \V The Lord be with you. \R And with thy Spirit. 

% \drop{The \grealtcross\ Holy Gospel is written in the — chapter of — beginning at the — verse. \R Glory \grecross\ be to thee, O Lord.}

% \centerline{\rubric{The Gospel ended, there may be said,}}
% \centerline{Praise be to thee, O Christ.}


% \bigskip

% \pilcrow{Then shall be sung or said the Creed following, the people still standing as before: except that at the discretion of the Minister it may be omitted on any day not being a Sunday or a Holy-day.} 

% \drop{I believe in one God the Father Almighty, Maker of heaven and earth, And of all things visible and invisible:}

% And in one Lord Jesus Christ, the only-begotten son of God, Begotten of his Father before all worlds, God of God, Light of Light, Very God of very God, Begotten, not made, Being of one substance with the Father, By whom all things were made: Who for us men, and for our salvation came down from heaven, * And was incarnate by the Holy Ghost of the Virgin Mary, * And was made man, * And was crucified also for us under Pontius Pilate. He suffered and was buried, And the third day he rose again according to the Scriptures, And ascended into heaven, And sitteth on the right hand of the Father. And he shall come again with glory to judge both the quick and the dead: Whose kingdom shall have no end.

% And I believe in the Holy Ghost, The Lord and giver of life, Who proceedeth from the Father and the Son, Who with the Father and the Son together is worshipped and glorified, Who spake by the Prophets. And I believe One {Holy} Catholick and Apostolick Church. I acknowledge one Baptism for the remission of sins. And I look for the Resurrection of the dead, * And the life of the world to come. Amen.

% \bigskip
% \pilcrow{Then the Curate shall declare unto the people what Holy-days, or Fasting-days, are in the week following to be observed. And then also (if occasion be) shall notice be given of the Holy Communion, or of other services; Banns of matrimony may be published, and Briefs, Citations, and Excommunications shall be read, and Bidding of Prayers may be made. And nothing shall be proclaimed or published in the Church during the time of Divine Service, but by the Minister: nor by him any thing, but what is prescribed in the rules of this Book, or enjoined by the %Queen, or by the 
% Ordinary of the place.}% adapted toward En1928

% \smallskip

% \pilcrow{Then may follow the Sermon, or one of the Homilies already set forth, or hereafter to be set forth, by authority.}


% \bigskip

% \section{The Offertory}

% \pilcrow{Then shall the Priest return to the Lord's Table, and begin the \emph{Offertory}. The Priest shall say, or the Clerks shall sing, one of these Sentences following, or some other convenient sentence taken out of Holy Scripture. A Hymn may follow.}

% \drop{To do good, and to distribute, forget not; for with such sacrifices God is pleased.\scripture{Hebrews xiij.~16.}}

% Lay not up for yourselves treasure upon the earth; where the rust and moth doth corrupt, and where thieves break through and steal: but lay up for yourselves treasures in heaven; where neither rust nor moth doth corrupt, and where thieves do not break through and steal.\scripture{St.~Matthew vj.~19.}

% Not every one that saith unto me, Lord, Lord, shall enter into the kingdom of heaven; but he that doeth the will of my Father which is in heaven.\scripture{St.~Matthew vij.~21.}

% If we have sown unto you spiritual things, is it a great matter if we shall reap your worldly things?\scripture{1 Corithians ix.~11.}

% Do ye not know, that they who minister about holy things live of the sacrifice; and they who wait at the altar are partakers with the altar? Even so hath the Lord also ordained, that they who preach the Gospel should live of the Gospel.\scripture{1 Corithians ix.~13.}

% He that soweth little shall reap little; and he that soweth plenteously shall reap plenteously. Let every man do according as he is disposed in his heart, not grudging, or of necessity; for God loveth a cheerful giver.\scripture{2 Corithians ix.~6.}

% Let him that is taught in the Word minister unto him that teacheth, in all good things. Be not deceived, God is not mocked: for whatsoever a man soweth that shall he reap.\scripture{Galatians vj.~6.}

% While we have time, let us do good unto all men; and specially unto them that are of the household of faith.\scripture{Galatians vj.~10.}

% God is not unrighteous, that he will forget your works, and labour that proceedeth of love; which love ye have shewed for his Name’s sake, who have ministered unto the saints, and yet do minister.\scripture{Hebrews vj.~10.}

% Charge them who are rich in this world, that they be ready to give, and glad to distribute; laying up in store for themselves a good foundation against the time to come, that they may attain eternal life.\scripture{1 Timothy vj.~17.}

% Whoso hath this world’s good, and seeth his brother have need, and shutteth up his compassion from him, how dwelleth the love of God in him?\scripture{1 St.~John iij.~17}



% \pilcrow{Whilst these Sentences are said or sung, the Deacons, Church-wardens, or other fit person appointed for that purpose, shall receive the alms for the poor, or other devotions of the people, and reverently bring them to the Priest, who shall humbly present and place them upon the Holy Table in a decent bason to be provided for that purpose.}

% \bigskip

% \pilcrow{{And when there is a Communion,} the Priest shall then offer up, and place the bread and wine prepared for the Sacrament upon the Lord’s Table; and shall say,}

% \drop{Blessed be thou, O {\scshape Lord} God, for ever and ever. Thine, O {\scshape Lord}, is the greatness, and the glory, and the victory, and the majesty: for all that is in the heaven and in the earth, is thine: thine is the kingdom, O {\scshape Lord}, and thou art exalted as head above all: both riches and honour come of thee, and of thine own do we give unto thee. \R Amen.}
% \scripture{1 Chronicles xxix.~10}

% \bigskip
% \centerline{Let us pray for the whole state of Christ's Church.}

% \drop{Almighty and everliving God, who by thy holy Apostle hast taught us to make prayers, and supplications, and to give thanks for all men;}
% We humbly beseech thee most mercifully [\footnote{\rubric{If there be no alms or oblations, then the words \emph{[to accept our alms and oblations]} be left out unsaid.}}\emph{to accept our alms and \grealtcross\ oblations, and}] to receive these our prayers, which we offer unto thy Divine Majesty; beseeching thee to inspire continually the Universal Church with the spirit of truth, unity, and concord: And grant, that all they that do confess thy holy Name may agree in the truth of thy holy Word, and live in unity, and godly love.

% We beseech thee also to lead all nations into the way of righteousness and peace, and so to direct all 
% ruling authorities, that under them the world may be godly and quietly governed. 
% And grant unto all that are put in authority, 
% that they may truly and indifferently minister justice, to the punishment of wickedness and vice, and to the maintenance of thy true religion, and virtue.

% Give grace, O heavenly Father, to all Bishops, Priests, and Deacons, especially to thy servant \emph{N.} our bishop, that they may both by their life and doctrine set forth thy true and lively Word, and rightly and duly administer thy holy Sacraments.

% And to all thy people give thy heavenly grace; and specially to this congregation here present; that with meek heart and due reverence, they may hear and receive thy holy Word; truly serving thee in holiness and righteousness all the days of their life.

% And we most humbly beseech thee, of thy goodness, O Lord, to comfort and succour all them, who in this transitory life are in trouble, sorrow, need, sickness, or any other adversity.

% And we commend to thy gracious keeeping, O Lord, all thy servants departed this life in thy faith and fear, beseeching them to grant them everlasting light and peace.

% And here we give thee most high praise and hearty thanks for all thy Saints, who have been the chosen vessels of thy grace, and lights of the world in their several generations; and we pray, that rejoicing in their fellowship, and following their good examples, we may be partakers with them of thy heavenly kingdom.

% Grant this, O Father, for Jesus Christ's sake, our only Mediator and Advocate. \R Amen.

% \section{The Consecration}

% \pilcrow{Turning himself to the people the Priest shall say,}

% \V The Lord be with you. \R And with thy spirit.

% \V Lift up your hearts.  \R We lift them up unto the Lord.

% \V Let us give thanks unto our Lord God. \R It is meet and right so to do.


% \centerline{\rubric{Then shall the Priest turn to the Lord’s Table, and say,}}

% \drop{It is very meet, right, and our bounden duty, that we should at all times, and in all places, give thanks unto thee, O Lord, Holy Father, Almighty, Everlasting God.


% {\centering\rubric{Here shall follow the Proper Preface, according to the time, if there be any specially appointed, or else immediately shall follow,}\par}


% Therefore with Angels and Archangels, and with all the company of heaven, we laud and magnify thy glorious Name; evermore praising thee, and saying,}
% \smallskip

% \drop{Holy, holy, holy, Lord God of hosts, heaven and earth are full of thy glory: Glory be to thee, O Lord most High.}
% \begin{leftbar}
%     \grecross\ Blessed is he that cometh in the Name of the Lord; Hosanna in the highest.
% \end{leftbar}

% \centerline{\rubric{Then shall the priest continue thus.}}
% \drop{All glory be to thee, Almighty God, our heavenly Father, for that thou of thy tender mercy didst give thine only Son Jesus Christ to suffer death upon the Cross for our redemption; who made there (by his one oblation of himself once offered) a full, perfect, and sufficient sacrifice, oblation, and satisfaction, for the sins of the whole world; and did institute, and in his holy Gospel command us to continue, a perpetual memory of that his precious death, until his coming again;

% Hear us, O merciful Father, we most humbly beseech thee; and grant that we receiving these thy creatures of bread and wine, according to thy Son our Saviour Jesus Christ’s holy institution, in remembrance of his death and passion, may be partakers of his most blessed \grealtcross\ Body and \grealtcross\ Blood: 

% Who, in the same night that he was betrayed, \footnote{\rubric{Here the Priest is to take the Paten unto his hands:}}took Bread; and, when he had given thanks, \footnote{\rubric{And here to break the Bread:}}he brake it, and gave it to his disciples, saying, Take, eat, \footnote{\rubric{And here to lay his hand upon all the Bread.}}{\scshape this is my Body which is given for you}: Do this in remembrance of me. Likewise after supper he \footnote{\rubric{Here he is to take the Cup into his hand:}}took the Cup; and, when he had given thanks, he gave it to them, saying, Drink ye all of this; \footnote{\rubric{And here to lay his hand upon every vessel (be it Chalice or Flagon) in which there is any Wine to be consecrated.}}{\scshape for this is my Blood of the New Testament, which is shed for you and for many for the remission of sins}: Do this, as oft as ye shall drink it, in remembrance of me.}

% Wherefore, O Lord and heavenly Father, we thy humble servants, having in remembrance the precious death of thy dear Son, his mighty resurrection and glorious ascension, looking also for his coming again, do render unto thee most hearty thanks for the innumerable benefits which he hath procured unto us; and we entirely desire thy fatherly goodness mercifully to accept this our sacrifice of praise and thanksgiving; most humbly beseeching thee to grant, that by the merits and death of thy Son Jesus Christ, and through faith in his blood, we and all thy whole Church may obtain remission of our sins, and all other benefits of his passion.

% And here we offer and present unto thee, O Lord, ourselves, our souls and bodies, to be a reasonable, holy, and lively sacrifice unto thee; 
% and we pray thee of thine almighty goodness to send upon us, and upon these thy gifts, thy holy and blessed Spirit, who is the Sanctifier and the Giver of life; humbly beseeching thee, that all we, who are partakers of this holy Communion, may be fulfilled with thy grace and heavenly \grecross\ benediction. 

% And although we be unworthy, through our manifold sins, to offer unto thee any sacrifice, yet we beseech thee to accept this our bounden duty and service; not weighing our merits, but pardoning our offences;

% Through Jesus Christ our Lord; by whom, and with whom, in the unity of the Holy Ghost, all honour and glory be unto thee, O Father Almighty, world without end. \R Amen.

% \smallskip
% {\centering\footnotesize\rubric{Here shall the people join with the Priest in the Lord’s Prayer, the Priest first saying,}\par}
% As our Saviour Christ hath commanded and taught us we are bold to say,
% \drop{Our Father, which art in heaven, Hallowed be thy Name. Thy kingdom come. Thy will be done, in earth as it is in heaven. Give us this day our daily bread. And forgive us our trespasses, As we forgive them that trespass against us. And lead us not into temptation; But deliver us from evil.}
% For thine is the kingdom, the power, and the glory, For ever and ever. Amen.

% \medskip
% \begin{leftbar}
%     \pilcrow{Then may the Priest say,}
%     \drop{The Peace of the Lord be always with you.  \R And with thy spirit.}
% \end{leftbar}

% \bigskip
% \pilcrow{Then shall the Priest say to them that come to receive the holy Communion,}
% \drop{Draw near with faith, and take this Holy Sacrament to your comfort; and make your humble confession to Almighty God, meekly kneeling upon your knees.}

% \smallskip
% \rubric{Then shall be said by the Minister and people together,}
% \drop{We confess to God Almighty, the Father, the Son, and the Holy Ghost, that we have sinned in thought, word, and deed, through our own grievous fault.  Wherefore we pray God to have mercy upon us.}

% \medskip
% {\centering\footnotesize\rubric{Then shall the Priest standing up, and turning himself to the people, pronounce this Absolution.}\par}
% \drop{Almighty God have mercy upon you, \grealtcross\ forgive you all your sins, and deliver you from all evil, confirm and strengthen you in all goodness, and bring you to everlasting life; through Jesus Christ our Lord. \R Amen.}

% \centerline{\pilcrow{Then shall the Priest say,}}
% Hear what comfortable words our Saviour Christ saith unto all that truly turn to him.
% \drop{Come unto me all that travail and are heavy laden, and I will refresh you.}\scripture{St.~Matthew xj.~28}

% So God loved the world, that he gave his only-begotten Son, to the end that all that believe in him should not perish, but have everlasting life.\scripture{St.~John iij.~16}

% \centerline{Hear also what Saint Paul saith.}

% This is a true saying, and worthy of all men to be received, That Christ Jesus came into the world to save sinners.\scripture{1 Timothy i.~15.}

% \centerline{Hear also what Saint John saith.}

% If any man sin, we have an Advocate with the Father, Jesus Christ the righteous; and he is the propitiation for our sins.\scripture{1 St.~John ij.~1.}

% \medskip

% {\centering\footnotesize\rubric{Then shall the Priest, kneeling down at the Lord’s Table, say in the name of all them that shall receive the Communion this Prayer following.}\par}
% \drop{We do not presume to come to this thy Table, O merciful Lord, trusting in our own righteousness, but in thy manifold and great mercies. We are not worthy so much as to gather up the crumbs under thy Table. But thou art the same Lord, whose property is always to have mercy: Grant us therefore, gracious Lord, so to eat the flesh of thy dear Son Jesus Christ, and to drink his blood, that our sinful bodies may be made clean by his body, and our souls washed through his most precious blood, and that we may evermore dwell in him, and he in us. Amen.}

% \medskip

% \pilcrow{Then shall the Minister first receive the Communion in both kinds himself, and then proceed to deliver the same to the Bishops, Priests, and Deacons, in like manner, (if any be present,) and after that to the people also in order, into their hands, all meekly kneeling. And, when he delivereth the Bread to any one, he shall say,}
% \drop{The Body of our Lord Jesus Christ, which was given for thee, preserve thy body and soul unto everlasting life. \R Amen.}

% \smallskip
% {\centering\footnotesize\rubric{And the Minister that delivereth the Cup to any one shall say,}\par}
% \drop{The Blood of our Lord Jesus Christ, which was shed for thee, preserve thy body and soul unto everlasting life.}

% \medskip
% \centerline{\pilcrow{Here may be sung a Hymn or an Anthem.}}
% \medskip

% {\footnotesize\rubric{When all have communicated, the Minister shall return to the Lord’s Table, and reverently place upon it what remaineth of the consecrated Elements, covering the same with a fair linen cloth.}\par}

% \medskip
% {\centering\footnotesize\rubric{Then the Priest shall say,}\par} 
% \drop{Almighty and everliving God, we most heartily thank thee, for that thou dost vouchsafe to feed us, who have duly received these holy mysteries, with the spiritual food of the most precious Body and Blood of thy Son our Saviour Jesus Christ; and dost assure us thereby of thy favour and goodness towards us; and that we are very members incorporate in the mystical body of thy Son, which is the blessed company of all faithful people; and are also heirs through hope of thy everlasting kingdom, by the merits of the most precious death and passion of thy dear Son. And we most humbly beseech thee, O heavenly Father, so to assist us with thy grace, that we may continue in that holy fellowship, and do all such good works as thou hast prepared for us to walk in; through Jesus Christ our Lord, to whom, with thee and the Holy Ghost, be all honour and glory, world without end. Amen.}

% \medskip
% \pilcrow{Then, on Sundays and on Feast days (except in Advent and from Septuagesima to Palm Sunday inclusive), shall be said or sung as follows:}

% \drop{Glory be to God on high, and in earth peace, good will towards men. We praise thee, we bless thee, \footnote{Bow} we worship thee, we glorify thee, we give thanks to thee for thy great glory, O Lord God, heavenly King, God the Father Almighty.}

% O Lord, the only-begotten Son Jesu Christ; O Lord God, Lamb of God, Son of the Father, that takest away the sins of the world, have mercy upon us. 
% Thou that takest away the sins of the world, * receive our prayer. Thou that sittest at the right hand of God the Father, have mercy upon us.

% For thou only art holy; thou only art the Lord; thou only, * O {Jesu} Christ, with the Holy Ghost, art most high \grecross\ in the glory of God the Father. Amen.


% {\centering\rubric{This hymn may be omitted here, and sung instead at the beginning of this Order after the \emph{Kyrie eleison}.}\par}


% \bigskip
% {\centering\rubric{Then the Priest shall let them depart with this Blessing.}\par}
% \drop{The peace of God, which passeth all understanding, keep your hearts and minds in the knowledge and love of God, and of his son Jesus Christ our Lord: and the blessing of God Almighty, the Father, \grealtcross\ the Son, and the Holy Ghost, be amongst you and remain with you always. \R Amen.}

% {\centering\rubric{Or, if there be no congregation,}\par}
% \drop{In the Name of the Father, \grealtcross\ and of the Son, and of the Holy Ghost. \R Amen.}



\chapter[The Order of Baptism]{\stylechapter{The Ministration of}{Holy Baptism}{To be Used in the Church }}

% 1549
\pilcrow{It appeareth by ancient writers that the Sacrament of Bapitism in the old time was not commonly ministered but at two times in the year, at Easter and Whitsuntide, at which times it was only ministered in presence of all the congregation: which custom now being grown out of use, although it cannot for many considerations be well restored again, yet it is thought good to follow the same, as near as conveniently may be: wherefore the people are to be admonished, that it is most convenient that Baptism should not be ministered but upon Sundays and other Holy Days, when the most number of people may come together, as well for that the Congregation there present may testify the receiving of them that be newly baptized into the number of Christ’s Church, as also because in the Baptism of Infants, every man present may be put in remembrance of his own profession made to God in his Baptism. Nevertheless (if necessity so require) children may at all times be baptized at home.}

\smallskip

\pilcrow{Due notice, normally of at least a week, shall be given before a child is brought to the church to be baptized.}

\medskip


\pilcrow{At the time appointed the godfathers and godmothers and [the parents or guardians with] the \emph{child} must be ready at the church door, either immediately before the last Canticle at Mattins or else immediately before the last Canticle at Evensong, as the Curate by his discretion shall appoint.  And the Priest standing there shall proceed as follows.}

\drop{Hath this Child [\rubric{or} Person] been already baptized or no?}

\smallskip

\centerline{\pilcrow{If they answer, \emph{No}: Then shall the Priest proceed as followeth.}}



\section{Admission to the Catechuminate}
\drop{Dearly beloved, forasmuch as all men are conceived and born in sin: and that our Saviour Christ saith, None can enter into the kingdom of God, except he be regenerate and born anew of Water and of the Holy Ghost: I beseech you to call upon God the Father, through our Lord Jesus Christ, that of his bounteous mercy he will grant to \emph{these Children} [\rubric{or} \emph{these persons}] that thing which by nature \emph{they} cannot have; that \emph{they} may be baptized with Water and the Holy Ghost, and received into Christ’s holy Church, and be made \emph{lively members} of the same.}

\centerline{\pilcrow{Then shall the Priest say,}}
\centerline{Let us pray.}
\drop{Almighty and everlasting God, who of thy great mercy didst save Noah and his family in the ark from perishing by water; and also didst safely lead the children of Israel thy people through the Red Sea, figuring thereby thy holy Baptism; and by the Baptism of thy well-beloved Son Jesus Christ, in the river Jordan, didst sanctify Water to the mystical washing away of sin: We beseech thee, for thine infinite mercies, that thou wilt mercifully look upon \emph{these Children} [\rubric{or} \emph{these} thy \emph{Servants and Handmaidens}]; wash \emph{them} and sanctify \emph{them} with the Holy Ghost; that \emph{they}, being delivered from thy wrath, may be received into the ark of Christ’s Church; and being stedfast in faith, joyful through hope, and rooted in charity, may so pass the waves of this troublesome world, that finally \emph{they} may come to the land of everlasting life, there to reign with thee world without end; through Jesus Christ our Lord. \R Amen.}

\medskip

%% <1549>
\pilcrow{Here shall the priest ask what shall be the name of the \emph{child}, [and when the Godfathers and Godmothers have told the name,] then shall he make a cross upon the \emph{child’s} forehead and breast, saying,}

\lettrine{\emph{N.}}{ receive} the sign of the holy Cross, both in thy \grealtcross\ forehead, and in thy \grealtcross\ breast, in token that thou shalt not be ashamed to confess thy faith in Christ crucified, and manfully to fight under his banner, against sin, the world, and the devil; and to continue Christ’s faithful soldier and servant unto thy life’s end. \R Amen.

{\footnotesize\rubric{And this he shall do and say to as many as be presented to be baptized, one after another.}\par}

\medskip


\centerline{Let us pray.}
%% </1549>
\drop{Almighty and immortal God, the aid of all that need, the helper of all that flee to thee for succour, the life of them that believe, and the resurrection of the dead: We call upon thee for \emph{these Infants} [\rubric{or} \emph{these Persons}], that \emph{they}, coming to thy holy Baptism, may receive remission of \emph{their} sins by spiritual regeneration. Receive \emph{them}, O Lord, as thou hast promised by thy well-beloved Son, saying, Ask, and ye shall have; seek, and ye shall find; knock, and it shall be opened unto you: So give now unto us that ask; let us that seek find; open the gate unto us that knock; that \emph{these Infants} [\rubric{or} \emph{these Persons}] may enjoy the everlasting benediction of thy heavenly washing, and may come to the eternal kingdom which thou hast promised by Christ our Lord. \R Amen.}


\medskip


%% <1549>
\pilcrow{Then let the priest, looking upon the \emph{Children}, say,} 

\drop{I command thee, unclean spirit, in the name of the Father, of the Son, and of the Holy Ghost, that thou come out, and depart from \emph{these Infants} [\rubric{or} \emph{these Persons}], whom our Lord Jesus Christ hath vouchsafed to call to his holy Baptism, to be made \emph{members} of his body, and of his holy congregation. Therefore thou cursed spirit, remember thy sentence, remember thy judgment, remember the day to be at hand wherein thou shalt burn in fire everlasting, prepared for thee and thy angels. And presume not hereafter to exercise any tyranny towards \emph{these Infants} [\rubric{or} \emph{these Persons}], whom Christ hath bought with his precious blood, and by this his holy Baptism, called to be of his flock, In the name of the same our Lord Jesus Christ, who shall come to judge the quick and the dead and the world by fire. \R Amen.}
%% </1549>


\medskip


\pilcrow{Then, if all candidates be children, the Priest shall say the Gospel and Exhortation as follow,}
\V The Lord be with you.  \R And with thy spirit.

\drop{Hear the words of the \grecross\ Gospel, written by Saint Mark, in the tenth chapter, at the thirteenth verse. \R Glory be to thee, O Lord.}

\smallskip

\drop{They brought young children to Christ, that he should touch them; and his disciples rebuked those that brought them. But when Jesus saw it, he was much displeased, and said unto them, Suffer the little children to come unto me, and forbid them not; for of such is the kingdom of God. Verily I say unto you, Whosoever shall not receive the kingdom of God as a little child, he shall not enter therein. And he took them up in his arms, put his hands upon them, and blessed them.}

\centerline{\R Praise be to thee, O Christ.}

\medskip
\pilcrow{After the Gospel is read, the Minister shall make this brief Exhortation upon the words of the Gospel.}

\drop{Beloved, ye hear in this Gospel the words of our Saviour Christ, that he commanded the children to be brought unto him; how he blamed those that would have kept them from him; how he exhorteth all men to follow their innocency. Ye perceive how by his outward gesture and deed he declared his good will toward them; for he embraced them in his arms, he laid his hands upon them, and blessed them. Doubt ye not therefore, but earnestly believe, that he will likewise favourably receive \emph{these} present \emph{Infants}; that he will embrace \emph{them} with the arms of his mercy; that he will give unto \emph{them} the blessing of eternal life, and make \emph{them} partaker\emph{s} of his everlasting kingdom.

Wherefore we being thus persuaded of the good will of our heavenly Father towards \emph{these Infants}, declared by his Son Jesus Christ; and nothing doubting but that he favourably alloweth this charitable work of ours in bringing \emph{these Infants} to his holy Baptism; let us faithfully and devoutly give thanks unto him, and say the prayer which the Lord himself taught. And in declaration of our faith, let us also recite the articles contained in our Creed.} %and say,}


\begin{leftbar}
\centerline{\pilcrow{Or, for those of Riper Years,}}
\V The Lord be with you.  \R And with thy spirit.

\drop{Hear the words of the \grecross\ Gospel, written by Saint John, in the third chapter, beginning at the first verse.  \R Glory be to thee, O Lord.}

\smallskip

\drop{There was a man of the Pharisees, named Nicodemus, a ruler of the Jews. The same came to Jesus by night, and said unto him, Rabbi, we know that thou art a teacher come from God; for no man can do these miracles that thou doest, except God be with him. Jesus answered and said unto him, Verily, verily I say unto thee, except a man be born again, he cannot see the kingdom of God. Nicodemus saith unto him, How can a man be born when he is old? Can he enter the second time into his mother’s womb, and be born? Jesus answered, Verily, verily I say unto thee, except a man be born of water and of the Spirit, he cannot enter into the kingdom of God. That which is born of the flesh is flesh; and that which is born of the Spirit is spirit. Marvel not that I said unto thee, Ye must be born again. The wind bloweth where it listeth, and thou hearest the sound thereof; but canst not tell whence it cometh, and whither it goeth: so is every one that is born of the Spirit.}

\centerline{\R Praise be to thee, O Christ.}

\medskip

\pilcrow{After which he shall say this Exhortation following.}
\drop{Beloved, ye hear in this Gospel the express words of our Saviour Christ, that except a man be born of water and of the Spirit, he cannot enter into the kingdom of God. Whereby ye may perceive the great necessity of this Sacrament, where it may be had. Likewise, immediately before his ascension into heaven, (as we read in the last chapter of Saint Mark’s Gospel,) he gave command to his disciples, saying, Go ye into all the world, and preach the Gospel to every creature. He that believeth and is baptized shall be saved; but he that believeth not shall be condemned. Which also sheweth unto us the great benefit we reap thereby. For which cause Saint Peter the Apostle, when upon his first preaching of the Gospel many were pricked at the heart, and said to him and the rest of the Apostles, Men and brethren, what shall we do? replied and said unto them, Repent, and be baptized every one of you for the remission of sins, and ye shall receive the gift of the Holy Ghost. For the promise is to you and your children, and to all that are afar off, even as many as the Lord our God shall call. \footnote{\rubric{These words may be omitted.}}[And with many other words exhorted he them, saying, Save yourselves from this untoward generation. For (as the same Apostle testifieth in another place) even Baptism doth also now save us, (not the putting away of the filth of the flesh, but the answer of a good conscience towards God,) by the resurrection of Jesus Christ.] Doubt ye not therefore, but earnestly believe, that he will favourably receive these present persons, truly repenting, and coming unto him by faith; that he will grant them remission of their sins, and bestow upon them the Holy Ghost; that he will give them the blessing of eternal life, and make them partakers of his everlasting kingdom.}

Wherefore we being thus persuaded of the good will of our heavenly Father towards \emph{these persons}, declared by his Son Jesus Christ; let us faithfully and devoutly give thanks to him, and say the prayer which the Lord himself taught. And in declaration of our faith, let us also recite the articles contained in our Creed.
\end{leftbar}


\pilcrow{Here the minister with the Godfathers, Godmothers, and people present, shall say,}

\drop{Our Father, which art in heaven, Hallowed be thy Name. Thy kingdom come. Thy will be done, in earth as it is in heaven. Give us this day our daily bread. And forgive us our trespasses, As we forgive them that trespass against us. And lead us not into temptation; But deliver us from evil. Amen.}
 

% And then shall say openly.
\medskip

\drop{I believe in God the Father Almighty, Maker of heaven and earth:}

And in Jesus Christ his only Son our Lord: Who was conceived by the Holy Ghost, Born of the Virgin Mary: Suffered under Pontius Pilate, Was crucified, dead, and buried: He descended into hell; The third day he rose again from the dead: He ascended into heaven, And sitteth on the right hand of God the Father Almighty: From thence he shall come to judge the quick and the dead.

I believe in the Holy Ghost: The holy Catholick Church; The Communion of Saints: The Forgiveness of sins: The Resurrection of the body, And the Life everlasting. Amen.

\smallskip

\centerline{\rubric{The priest shall add also this prayer.}}
\drop{Almighty and everlasting God, heavenly Father, we give thee humble thanks, for that thou hast vouchsafed to call us to the knowledge of thy grace, and faith in thee: Increase this knowledge, and confirm this faith in us evermore. Give thy Holy Spirit to \emph{these Infants} [\rubric{or} \emph{these Persons}], that \emph{they} may be born again, and be made \emph{an heir} of everlasting salvation; through our Lord Jesus Christ, who liveth and reigneth with thee and the Holy Spirit, now and for ever. \R Amen.}

\medskip

\pilcrow{Then let the priest take [one of] the \emph{children} by the right hand, [the others being brought after him]. And coming into the Church say,}

\drop{The Lord vouchsafe to receive \emph{you} into his holy household, and to keep and govern \emph{you} alway in the same, that \emph{ye may} have everlasting life. Amen.}




% It consisted of three acts. First the closing exorcism — the Bishop stretching his hands over them as they knelt facing eastwards, prayed for the last time for the ejection of the evil spirit from them; secondly, the exsufflation — he breathed in their faces; thirdly, the Effeta — he touched each candidate on the mouth, ears, \etc, with spittle or oil, after the example of our Lord’s action in; healing the deaf and dumb man.9

% sarum:
% Exorcism, Gospel, Effeta, LordsPrayer/creed. Signing on right hand + blessing, introduction to church.




\section{The Promises}

\pilcrow{Here may be sung a Hymn or an Anthem, \emph{or} the Invocations and Conclusion from the Litany.}

\medskip

\pilcrow{Then, standing at the Font, shall the Priest speak unto the Godfathers and Godmothers on this wise.}

\drop{Dearly beloved, ye have brought \emph{these Children} here to be baptized, ye have prayed that our Lord Jesus Christ would vouchsafe to receive \emph{them}, to release \emph{them} of \emph{their} sins, to sanctify \emph{them} with the Holy Ghost, to give \emph{them} the kingdom of heaven, and everlasting life. Ye have heard also that our Lord Jesus Christ hath promised in his Gospel to grant all these things that ye have prayed for: which promise he, for his part, will most surely keep and perform. 

Wherefore, after this promise made by Christ, \emph{these Infants} must also faithfully, for \emph{their} part, promise by you that are \emph{their} sureties, (until \emph{they} come of age to take it upon \emph{themselves},) that \emph{they} will renounce the devil and all his works, and constantly believe God’s holy Word, and obediently keep his commandments.}


\begin{leftbar}
\centerline{\rubric{Or, for those of Riper Years,}}

\pilcrow{Then the Priest shall speak to the persons to be baptized on this wise}

\drop{Well-beloved, who are come hither desiring to receive Holy Baptism, \emph{ye have} heard how the congregation hath prayed, that our Lord Jesus Christ would vouchsafe to receive \emph{you} and bless \emph{you}, to release \emph{you} of \emph{your} sins, to give \emph{you} the kingdom of heaven, and everlasting life. \emph{Ye have} heard also, that our Lord Jesus Christ hath promised in his holy Word to grant all those things that we have prayed for; which promise he, for his part, will most surely keep and perform.}

Wherefore, after this promise made by Christ, \emph{ye} must also faithfully, for \emph{your} part, promise in the presence of these \emph{your} Witnesses, and this whole congregation, that \emph{ye will} renounce the devil and all his works, and constantly believe God’s holy Word, and obediently keep his commandments.
\end{leftbar}

\medskip
\pilcrow{Then shall the Priest demand of each of the \emph{Children} to be baptized, severally, these questions following:}

\smallskip
\centerline{\rubric{Question.}}
\centerline{I demand therefore,}

\drop{Dost thou %, in the name of this Child, 
renounce the devil and all his works?\\
\qa{Answer.}  I renounce him.

\smallskip
\centerline{\rubric{Question.}}
Dost thou renounce the vain pomp and glory of the world, with all covetous desires of the same?

\qa{Answer.}  I renounce them.

\smallskip
\centerline{\rubric{Question.}}

Does thou renounce the sinful desires of the flesh, so that thou wilt not follow, nor be led by them?}

\qa{Answer.}  I renounce them all.

\medskip

% [sarum: Unction on breast and back)]


\centerline{\rubric{Question.}}
\drop{Dost thou believe in God the Father Almighty, Maker of heaven and earth?}

\qa{Answer.}  This I believe.

\smallskip
\centerline{\rubric{Question.}}
%And
Dost thou believe 
in Jesus Christ his only-begotten Son our Lord? And that he was conceived by the Holy Ghost; born of the Virgin Mary; that he suffered under Pontius Pilate, was crucified, dead, and buried; that he went down into hell, and also did rise again the third day; that he ascended into heaven, and sitteth at the right hand of God the Father Almighty; and from thence shall come again at the end of the world, to judge the quick and the dead?

\qa{Answer.}  This I believe.

\smallskip
\centerline{\rubric{Question.}}

And dost thou believe in the Holy Ghost; the holy Catholick Church; the Communion of Saints; the Remission of sins; the Resurrection of the flesh; and everlasting life after death?

\qa{Answer.} All this I stedfastly believe.

\medskip
\centerline{\rubric{Question.}}

\drop{Wilt thou be baptized in this faith?\\
\qa{Answer.} That is my desire.}

\smallskip
\centerline{\rubric{Question.}}

\drop{Wilt thou then obediently keep God’s holy will and commandments, and walk in the same all the days of thy life?}

\qa{Answer.} I will endeavour so to do, God being my helper. % 1928En

% \qa{Answer.} I will. 1662



% \begin{leftbar}

% \centerline{\rubric{Question.}}

% \drop{Dost thou renounce the devil and all his works, the vain pomp and glory of the world, with all covetous desires of the same, and the sinful desires of the flesh, so that thou wilt not follow, nor be led by them?}

% \qa{Answer.} I renounce them all.

% \centerline{\rubric{Question.}}

% Dost thou profess the Christian faith?

% \qa{Answer.} I do.

% \rubric{Then shall be said by the Candidates with the Priest and the Witnesses as followeth}

% \drop{I believe in God the Father Almighty, Maker of heaven and earth:}

% And in Jesus Christ his only Son our Lord, Who was conceived by the Holy Ghost, Born of the Virgin Mary, Suffered under Pontius Pilate, Was crucified, dead, and buried, He descended into hell; The third day he rose again from the dead, He ascended into heaven, And sitteth on the right hand of God the Father Almighty; From thence he shall come to judge the quick and the dead.

% I believe in the Holy Ghost; The holy Catholick Church; The Communion of Saints ; The Forgiveness of sins ; The Resurrection of the body; And the Life everlasting. Amen.

% \centerline{\rubric{Question.}}

% \drop{Wilt thou be baptized in this faith?\\
% \qa{Answer.} That is my desire.}

% \centerline{\rubric{Question.}}

% \drop{Wilt thou then obediently keep God’s holy will and commandments, and walk in the same all the days of thy life?}

% \qa{Answer.} I will endeavour so to do, God being my helper.
% \end{leftbar}








\section{The Blessing of the Water}


\centerline{\pilcrow{After which the Priest shall proceed, saying,}}
% The water in the fonte shalbe chaunged every moneth once at the lest, and afore any child be Baptized in the water so chaunged, the priest shall say at the font these prayers folowing.

\V The Lord be with you.  \R And with thy spirit.


\centerline{Let us pray.}
\drop{O most merciful God our Saviour Jesu Christ, who hast ordained the element of water for the regeneration of thy faithful people, Upon whom, being baptized in the river of Jordan, the Holy Ghost came down in likeness of a dove: Send down, we beseech thee, the same thy Holy Spirit to assist us, and to be present at this our invocation of thy Holy Name: Sanctify \grealtcross\ this fountain of baptism, thou that art the Sanctifier of all things, that by the power of thy Word all those that shall be baptized therein may be spiritually regenerated, and made the children of everlasting adoption. \R Amen.}

\smallskip
\centerline{\pilcrow{Then shall the Priest say,}}
\drop{O merciful God, grant that the old Adam in \emph{these Children} [\rubric{or} \emph{these Persons}] may be so buried, that the new man may be raised up in \emph{them}.  \R Amen.}

Grant that all %carnal
sinful %20th c. books
affections may die in \emph{them}, and that all things belonging to the Spirit may live and grow in \emph{them}. \R Amen.

Grant that \emph{they} may have power and strength to have victory, and to triumph, against the devil, the world, and the flesh. \R Amen.

%<1549>
Whosoever shall confess thee, O Lord, recognise him also in thy kingdom. \R Amen.

Grant that all sin and vice here may be so extinct, that they never have power to reign in thy servants. \R Amen.

Grant that whosoever here shall begin to be of thy flock, may evermore continue in the same. \R Amen.

Grant that all they which for thy sake in this life do deny and forsake themselves, may win and purchase thee, O Lord, which art everlasting treasure. \R Amen.
%</1549>

Grant that whosoever is here dedicated to thee by our office and ministry may also be endued with heavenly virtues, and everlastingly rewarded, through thy mercy, O blessed Lord God, who dost live, and govern all things, world without end. \R Amen.

\medskip
%1549
\V The Lord be with you.  \R And with thy spirit.

% 1928En, and other modern:
\V Lift up your hearts.  \R We lift them up unto the Lord.

\V Let us give thanks unto our Lord God. \R It is meet and right so to do.
\drop{It is very meet, right, and our bounden duty, that we should % [at all times, and in all places,] (all sources omit)
give thanks unto thee, O Lord, Holy Father, Almighty, Everlasting God, for that thy
% \drop{Almighty, everliving God, whose most * 
dearly beloved Son Jesus Christ, for the forgiveness of our sins, did shed out of his most precious side both water and blood; and gave commandment to his disciples, that they should go teach all nations, and baptize them In the Name of the Father, and of the Son, and of the Holy Ghost: Regard, we beseech thee, the supplications of thy congregation;}
%1662
Sanctify \grealtcross\ this Water to the mystical washing away of sin; and grant that \emph{these Children} [\rubric{or} \emph{these} thy \emph{Servants}], now to be baptized therein,
% Or. 1549
% and grant that all thy servants which shall be baptized in this water, prepared for the ministration of thy holy sacrament, 
may receive the fulness of thy grace, and ever remain in the number of thy faithful and elect children;
%  through Jesus Christ our Lord. \R Amen.}
% 1928Am:
through the same Jesus Christ our Lord, to whom, with thee, in the unity of the Holy Spirit, be all honour and glory, now and evermore. \R Amen.


\section{The Baptism}

\pilcrow{Then the Minister shall take the Child into his hands, and shall say to the Godfathers and Godmothers,}
\centerline{Name this Child.}

{\footnotesize\rubric{And then naming it after them (if they shall certify him that the Child may well endure it) % he shall dip it in the Water discreetly and warily, saying,
he shall dip it in the Water thrice. First dipping the right side, second the left side, the third time dipping the face towards the font, so it be discreetly and warily done, saying,}\par}

% And then naming it after them, he shall dip it in the water, or pour water upon it, saying,
% he shall dip it in the Water discreetly and warily, saying,

\lettrine{\emph{N.}}{ I} baptize thee in the Name of the Father, and of the Son, and of the Holy Ghost. Amen.

{\footnotesize\rubric{But if they certify that the Child is weak, it shall suffice to pour Water upon it, saying the foresaid words, N. \emph{I baptize thee \etc}}\par}

\smallskip
\begin{leftbar}
\pilcrow{But NOTE, That if the Person to be baptized be of Riper Years, the Minister shall take \emph{him} by the right hand, and placing \emph{him} conveniently by the Font, according to his discretion, shall ask the Godfathers and Godmothers the Name; and then shall dip \emph{him} in the Water, or pour Water upon \emph{him}, using the same form of words.}
\end{leftbar}

\medskip

\pilcrow{Then the Priest shall annoint the \emph{Infant} upon the head with chrism, saying,}
    
\drop{Almighty God, the Father of our Lord Jesus Christ, who hath regenerated thee by water and the Holy Ghost, and hath given unto thee remission of all thy sins, may he vouchsafe to anoint thee with the unction \grealtcross\ of his Holy Spirit, and bring thee to the inheritance of everlasting life. \R Amen.}


% \centerline{\rubric{Then the Priest shall say,}}
% \drop{We receive this \emph{Child} into the congregation of Christ’s flock, \grealtcross\ and do sign \emph{him} with the sign of the Cross, in token that hereafter \emph{he} shall not be ashamed to confess the faith of Christ crucified, and manfully to fight under his banner, against sin, the world, and the devil; and to continue Christ’s faithful soldier and servant unto \emph{his} life’s end. \R Amen.}
% * here the Priest shall make a Cross upon the Child's forehead.

\medskip

%1549
\pilcrow{Then the Godfathers and Godmothers shall take and lay their hands upon the \emph{Child}, and the Priest shall put upon \emph{him his} white vesture, commonly called the chrysom; and say}

\drop{Take this white vesture as a token of the innocency which, by God’s grace, in this holy sacrament of Baptism, is given unto thee; and for a sign whereby thou art admonished, so long as thou livest, to give thyself to innocency of living: that after this transitory life thou mayest be partaker of the life everlasting.}
     
\medskip

%india supplement.
\pilcrow{Then the Minister may deliver to the \emph{Child} a burning light, saying,}
% \drop{Receive this burning light, and walk in the light ft by faith in Jesus Christ. Amen.}
% or, South Africa % and Alcuin
\drop{Receive the light of Christ, that when the Bridegroom cometh thou mayest go forth with all the saints to meet him; and see that thou keep the grace of thy baptism.}

\medskip

\pilcrow{When there are many to be baptized, this order of baptizing, annointing, putting on the chrysom, and delivering the light, shall be used severally with every \emph{Child}. Those that be first baptized departing from the font, and remainyng in some convenient place within the Church until all be baptized.}


\section{The Thanksgiving}
\centerline{\pilcrow{Then shall the Priest say,}}
\drop{Seeing now, dearly beloved brethren, that \emph{these Children} [\rubric{or} \emph{these Persons}] \emph{are} regenerate, and grafted into the body of Christ’s Church, let us give thanks unto Almighty God for these benefits; and with one accord make our prayers unto him, that \emph{these Children} [\rubric{or} \emph{these Persons}] may lead the rest of \emph{their} life according to this beginning.}


% Then shall be said, all kneeling;
% \drop{Our Father}
\smallskip

\centerline{\rubric{Then shall the Priest say,}}
\drop{We yield thee hearty thanks, most merciful Father, that it hath pleased thee to regenerate \emph{these Infants} [\rubric{or} \emph{these Persons}] with thy Holy Spirit, to receive \emph{them} for thine own \emph{Children} by adoption, and to incorporate \emph{them} into thy holy Church. And humbly we beseech thee to grant, that \emph{they}, being dead unto sin, and living unto righteousness, and being buried with Christ in his death, may crucify the old man, and utterly abolish the whole body of sin; and that, as \emph{they are} made partaker of the death of thy Son, \emph{they} may also be partaker of his resurrection; so that finally, with the residue of thy holy Church, \emph{they} may be \emph{inheritors} of thine everlasting kingdom; through Christ our Lord. \R Amen.}


\bigskip

\section{The Duties of the Godfathers and Godmothers}


\pilcrow{Then the Priest shall say to the Godfathers and Godmothers and Parents this Exhortation following.}
\drop{Forasmuch as \emph{these Children have} promised by you \emph{their} sureties to renounce the devil and all his works, to believe in God, and to serve him: ye must remember, that it is your parts and duties to see that \emph{these Infants} be taught, so soon as \emph{they} shall be able to learn, what a solemn vow, promise, and profession, \emph{they have} here made by you. And that \emph{they} may know these things the better, ye shall call upon \emph{them} to hear Sermons; and chiefly ye shall provide, that \emph{them} may learn the Creed, the Lord’s Prayer, and the Ten Commandments, in the vulgar tongue, and all other things which a Christian ought to know and believe to his soul’s health; and that \emph{these Children} may be virtuously brought up to lead a godly and a Christian life.}


Will you help \emph{them} to learn and to do all these things?

\qa{Answer.}  I will, the Lord being my helper.

Remember always that Baptism doth represent unto us our profession; which is, to follow the example of our Saviour Christ, and to be made like unto him; that, as he died, and rose again for us, so should we, who are baptized, die from sin, and rise again unto righteousness; continually mortifying all our evil and corrupt affections and daily proceeding in all virtue and godliness of living.

\medskip
% the following is a repetition of the previous exhortation, and the following rubric; and doesn't survive into the modern books.
% \centerline{\rubric{Then shall he add and say,}}
% \drop{Ye are to take care that this Child be brought to the Bishop to be confirmed by him, so soon as he can say the Creed, the Lord's Prayer, and the Ten Commandments, in the vulgar tongue, and be further instructed in the Church-Catechism set forth for that purpose.}
% It is certain by God's Word, that children which are baptized, dying before they commit actual sin, are undoubtedly saved.


\pilcrow{The Minister shall command that the chrysoms be brought to the church, and delivered to the Priests after the accustomed manner, at the purification of the mother of every child: and that the children be brought to the Bishop to be confirmed of him, so soon as they can say, in their vulgar tongue, the Articles of the Faith, the Lord’s Prayer, and the Ten Commandments, and further be instructed in the Catechism set forth for that purpose accordingly as it is there expressed.}

% And so let the congregation depart in the name of the Lord.
\begin{leftbar}
\section{The Duties of the Witnesses and of the New Baptized}
\pilcrow{Then, all standing up, the Priest shall use this exhortation following; speaking to the Godfathers and Godmothers first.}
\drop{Forasmuch as \emph{these persons have} promised in your presence to renounce the devil and all his works, to believe in God, and to serve him: ye must remember, that it is your part and duty to put \emph{them} in mind, what a solemn vow, promise, and profession \emph{they have} now made before this congregation, and especially before you \emph{their} chosen witnesses. And ye are also to call upon \emph{them} to use all diligence to be rightly instructed in God’s holy Word; that so \emph{they} may grow in grace, and in the knowledge of our Lord Jesus Christ, and live godly, righteously, and soberly in this present world.}


{\footnotesize\rubric{(And then, speaking to the new baptized \emph{persons}, he shall proceed, and say,)}\par}


\drop{And as for \emph{you}, who \emph{have} now by Baptism put on Christ, it is \emph{your} part and duty also, being made \emph{the children} of God and of the light by faith in Jesus Christ, to walk answerably to \emph{your} Christian calling, and as becometh the children of light; remembering always that Baptism representeth unto us our profession; which is, to follow the example of our Saviour Christ, and to be made like unto him; that as he died, and rose again for us; so should we, who are baptized, die from sin, and rise again unto righteousness; continually mortifying all our evil and corrupt affections, and daily proceeding in all virtue and godliness of living.}

\medskip
\pilcrow{It is expedient that every person, thus baptized, should be confirmed by the Bishop so soon after his Baptism as conveniently may be.}
\end{leftbar}

\fleuron


\section{Private Baptism}
\pilcrow{When, in consideration of extreme sickness, necessity may require, then the following form shall suffice:}

\pilcrow{The Child (or Person) being named by some one who is present, the Minister shall pour Water upon him, saying these words:}

\lettrine{\emph{N.}}{ I} baptize thee In the Name of the Father, and of the Son, and of the Holy Ghost. Amen.

\pilcrow{After which shall be said the Lord’s Prayer, and the Thanksgiving from the Office, beginning, \emph{We yield thee hearty thanks, \etc}}

\begin{leftbar}
    \pilcrow{But NOTE, That in the case of an Adult, the Minister shall first ask the questions provided in this Office for the Baptism of Adults.}
\end{leftbar}
\pilcrow{In cases of extreme sickness, or any imminent peril, if a Minister cannot be procured, then any person present may administer holy Baptism, using the foregoing form. Such Baptism shall be promptly reported to the authorities.}

\pilcrow{And let them not doubt, but that the \emph{Child} so baptized is lawfully and sufficiently baptized, and ought not to be baptized again.}

\section{The Receiving of one Privately Baptized}
\pilcrow{It is expedient that a Child or Person so baptized be afterward brought to the Church, at which time these parts of the foregoing service shall be used:}

{\footnotesize\rubric{The Gospel, the Questions (omitting the question \emph{Wilt thou be baptized in this Faith?} and the answer thereto), the Annointing, and the remainder of the Office.}\par}



%Am1928
\section{Conditional Baptism}
\pilcrow{If there be reasonable doubt whether any Person were baptized with Water, \emph{In the Name of the Father, and of the Son, and of the Holy Ghost} (which are essential parts of Baptism), such Person may be baptized in the manner herein appointed; saving that, at the immersion or the pouring of water, the Minister shall use this Form of words.}

\drop{If thou art not already baptized, \emph{N.}, I baptize thee In the Name of the Father, and of the Son, and of the Holy Ghost. Amen.}



\section{General Rubrics}

\pilcrow{For every child to be baptized there shall be not fewer than three godparents, of whom at least two shall be of the same sex as the child and of whom at least one shall be of the opposite sex; save that, when three cannot be conveniently had, one godparent shall suffice. Parents may be godparents for their own children provided that the child shall have at least one other godparent. The godparents shall be persons who have been baptized and confirmed and will faithfully fulfil their responsibilities both by their care for the child committed to their charge and by the example of their own godly living. Nevertheless the Minister shall have power to dispense with the requirement of confirmation in any case in which in his judgement need so requires.}

\pilcrow{The Minister shall instruct the parents or guardians of an infant to be admitted to Holy Baptism that the same responsibilities rest on them as are in the service of Holy Baptism required of the godparents.}

\pilcrow{No Minister shall refuse or, save for the purpose of preparing or instructing the parents or guardians or godparents, delay to baptize any infant within his cure that is brought to the church to be baptized, provided that due notice has been given and the provisions relating to godparents are observed. If the Minister shall refuse or unduly delay to baptize any such infant, the parents or guardians may apply to the Bishop of the diocese who shall, after consultation with the Minister, give such directions as he thinks fit.}

\pilcrow{The Minister, before proceeding to the Baptism, shall have satisfied himself that the child presented to him has not already been baptized.}
\begin{leftbar}    
\pilcrow{When any such persons, as are of riper years, are to be baptized, timely notice shall be given to the Bishop, or whom he shall appoint for that purpose, a week before at the least, by the Minister of the Parish, the parents, or some other discreet persons; that so due care may be taken for their examination, whether they be sufficiently instructed in the principles of the Christian Religion; and that they may be exhorted to prepare themselves with prayers and fasting for the receiving of this holy Sacrament.}


\pilcrow{And if they shall be found fit, they shall each choose three, or at the least one, to be their Witnesses, who shall be ready to present them at the Font, immediately after the Second Lesson, either at Morning or Evening Prayer, or (if need so require) at such other time as the Minister in his discretion shall think fit.}

\smallskip

\pilcrow{It is convenient that the Admission to the Catechuminate should precede the Baptism by a week or more.}
\end{leftbar}



\fleuron

\include{catechumens.tex}
% Form of admitting Catechumens (SA)
% The Ministration of Baptism for those of Riper Years

\newcommand{\qa}[1]{{\itshape\small\red#1}}

\chapter{A Catechism}

That is to say
an Instruction to be Learned of Every Person before he be Brought to be Confirmed by the Bishop

\centerline{\qa{Question.}}
\drop{What is your Name?\\
\qa{Answer.} \emph{N.} or \emph{M.}}

\qa{Question.} Who gave you this Name?

\qa{Answer.} My Godfathers and Godmothers in my Baptism; wherein I was made a member of Christ, the child of God, and an inheritor of the kingdom of heaven.

\qa{Question.} What did your Godfathers and Godmothers then for you?

\qa{Answer.} They did promise and vow three things in my name. First, that I should renounce the devil and all his works, the pomps and vanity of this wicked world, and all the sinful lusts of the flesh. Secondly, that I should believe all the Articles of the Christian Faith. And thirdly, that I should keep God's holy will and commandments, and walk in the same all the days of my life.

\qa{Question.} Dost thou not think that thou art bound to believe, and to do, as they have promised for thee?

\qa{Answer.} Yes verily: and by God's help so I will. And I heartily thank our heavenly Father, that he hath called me to this state of salvation, through Jesus Christ our Saviour. And I pray unto God to give me his grace, that I may continue in the same unto my life's end.


\medskip
\centerline{\qa{Catechist.}}
\centerline{Rehearse the Articles of thy Belief.}


\centerline{\qa{Answer.}}
\drop{I believe in God the Father Almighty, Maker of heaven and earth:}

And in Jesus Christ his only Son our Lord, Who was conceived by the Holy Ghost, Born of the Virgin Mary, Suffered under Pontius Pilate, Was crucified, dead, and buried, He descended into hell; The third day he rose again from the dead, He ascended into heaven, And sitteth at the right hand of God the Father Almighty; From thence he shall come to judge the quick and the dead.

I believe in the Holy Ghost; The holy Catholick Church; The Communion of Saints; The Forgiveness of sins; The Resurrection of the body; And the Life everlasting. Amen.

\centerline{\qa{Question.}}
What dost thou chiefly learn in these Articles of thy Belief?

\qa{Answer.} First, I learn to believe in God the Father, who hath made me, and all the world.

Secondly, in God the Son, who hath redeemed me, and all mankind.

Thirdly, in God the Holy Ghost, who sanctifieth me, and all the elect people of God.

\medskip
\centerline{\qa{Question.}}
You said, that your Godfathers and Godmothers did promise for you, that you should keep God's commandments. Tell me how many there be?

\qa{Answer.} Ten.

\qa{Question.} Which be they?

\centerline{\qa{Answer.}}
\drop{The same which God spake in the twentieth Chapter of Exodus, saying, I am the Lord thy God, who brought thee out of the land of Egypt, out of the house of bondage.}

I. Thou shalt have none other gods but me.

II. Thou shalt not make to thyself any graven image, nor the likeness of any thing that is in heaven above, or in the earth beneath, or in the water under the earth. Thou shalt not bow down to them, nor worship them: for I the Lord thy God am a jealous God, and visit the sins of the fathers upon the children unto the third and fourth generation of them that hate me, and shew mercy unto thousands in them that love me, and keep my commandments.

III. Thou shalt not take the Name of the Lord thy God in vain: for the Lord will not hold him guiltless that taketh his Name in vain.

IV. Remember that thou keep holy the Sabbath-day. Six days shalt thou labour, and do all that thou hast to do; but the seventh day is the Sabbath of the Lord thy God. In it thou shalt do no manner of work, thou, and thy son, and thy daughter, thy man-servant, and thy maid-servant, thy cattle, and the stranger that is within thy gates. For in six days the Lord made heaven and earth, the sea, and all that in them is, and rested the seventh day; wherefore the Lord blessed the seventh day, and hallowed it.

V. Honour thy father and thy mother, that thy days may be long in the land which the Lord thy God giveth thee.

VI. Thou shalt do no murder.

VII. Thou shalt not commit adultery.

VIII. Thou shalt not steal.

IX. Thou shalt not bear false witness against thy neighbour.

X. Thou shalt not covet thy neighbour's house, thou shalt not covet thy neighbour's wife, nor his servant, nor his maid, nor his ox, nor his ass, nor any thing that is his.

\medskip
\centerline{\qa{Question.}}
What dost thou chiefly learn by these Commandments?

\qa{Answer.} I learn two things: my duty towards God, and my duty towards my neighbour.

\qa{Question.} What is thy duty towards God?

\qa{Answer.} My duty towards God, is to believe in him, to fear him, and to love him with all my heart, with all my mind, with all my soul, and with all my strength; to worship him, to give him thanks, to put my whole trust in him, to call upon him, to honour his holy name and his Word, and to serve him truly all the days of my life.

\qa{Question.} What is thy duty towards thy Neighbour?

\qa{Answer.} My duty towards my neighbour, is to love him as myself, and to do to all men, as I would they should do unto me: To love, honour, and succour my father and mother: To honour and obey the \emph{civil authority}%,  and all that are put in authority under her
: To submit myself to all my governors, teachers, spiritual pastors and masters: To order myself lowly and reverently to all my betters: To hurt no body by word nor deed: To be true and just in all my dealing: To bear no malice nor hatred in my heart: To keep my hands from picking and stealing, and my tongue from evil-speaking, lying, and slandering: To keep my body in temperance, soberness, and chastity: Not to covet nor desire other men's goods; but to learn and labour truly to get mine own living, and to do my duty in that state of life, unto which it shall please God to call me.

\medskip
\centerline{\qa{Catechist.}}
My good child, know this, that thou art not able to do these things of thyself, nor to walk in the Commandments of God, and to serve him, without his special grace; which thou must learn at all times to call for by diligent prayer. Let me hear therefore, if thou canst say the Lord's Prayer.

\centerline{\qa{Answer.}}
\drop{Our Father, which art in heaven, Hallowed be thy Name. Thy kingdom come. Thy will be done, in earth as it is in heaven. Give us this day our daily bread. And forgive us our trespasses, As we forgive them that trespass against us. And lead us not into temptation; But deliver us from evil. Amen.}

\qa{Question.} What desirest thou of God in this Prayer?

\qa{Answer.} I desire my Lord God our heavenly Father, who is the giver of all goodness, to send his grace unto me, and to all people; that we may worship him, serve him, and obey him, as we ought to do. And I pray unto God, that he will send us all things that be needful both for our souls and bodies; and that he will be merciful unto us, and forgive us our sins; and that it will please him to save and defend us in all dangers ghostly and bodily; and that he will keep us from all sin and wickedness, and from our ghostly enemy, and from everlasting death. And this I trust he will do of his mercy and goodness, through our Lord Jesus Christ. And therefore I say, Amen, So be it.

\medskip
\centerline{\qa{Question.}}
\drop{How many Sacraments hath Christ ordained in his Church?}

\qa{Answer.} Two only, as generally necessary to salvation, that is to say, Baptism, and the Supper of the Lord.

\qa{Question.} What meanest thou by this word Sacrament?

\qa{Answer.} I mean an outward and visible sign of an inward and spiritual grace given unto us, ordained by Christ himself, as a means whereby we receive the same, and a pledge to assure us thereof.

\qa{Question.} How many parts are there in a Sacrament?

\qa{Answer.} Two; the outward visible sign, and the inward spiritual grace.

\qa{Question.} What is the outward visible sign or form in Baptism?

\qa{Answer.} Water; wherein the person is baptized \emph{In the Name of the Father, and of the Son, and of the Holy Ghost.}

\qa{Question.} What is the inward and spiritual grace?

\qa{Answer.} A death unto sin, and a new birth unto righteousness: for being by nature born in sin, and the children of wrath, we are hereby made the children of grace.

\qa{Question.} What is required of persons to be baptized?

\qa{Answer.} Repentance, whereby they forsake sin; and Faith, whereby they stedfastly believe the promises of God made to them in that Sacrament.

\qa{Question.} Why then are Infants baptized, when by reason of their tender age they cannot perform them?

\qa{Answer.} Because they promise them both by their sureties; which promise, when they come to age, themselves are bound to perform.

\qa{Question.} Why was the Sacrament of the Lord's Supper ordained?

\qa{Answer.} For the continual remembrance of the sacrifice of the death of Christ and of the benefits which we receive thereby.


\qa{Question.} What is the outward part or sign of the Lord's Supper?


\qa{Answer.} Bread and Wine, which the Lord hath commanded to be received.


\qa{Question.} What is the inward part, or thing signified?

\qa{Answer.} The Body and Blood of Christ, which are verily and indeed taken and received by the faithful in the Lord's Supper.

\qa{Question.} What are the benefits whereof we are partakers thereby?

\qa{Answer.} The strengthening and refreshing of our souls by the Body and Blood of Christ, as our bodies are by the Bread and Wine.

\qa{Question.} What is required of them who come to the Lord's Supper?

\qa{Answer.} To examine themselves, whether they repent them truly of their former sins, stedfastly purposing to lead a new life; have a lively faith in God's mercy through Christ, with a thankful remembrance of his death; and be in charity with all men.

\pilcrow{The Curate of every Parish shall diligently upon Sundays and Holy-days, after the second Lesson at Evening Prayer, openly in the Church instruct and examine so many Children of his Parish sent unto him, as he shall think convenient, in some Part of this Catechism.}

\pilcrow{And all Fathers, Mothers, Masters, and Dames, shall cause their Children, Servants, and Prentices (which have not learned their Catechism,) to come to the Church at the time appointed, and obediently to hear, and be ordered by the Curate, until such time as they have learned all that is here appointed for them to learn.}

\pilcrow{So soon as children are come to a competent age, and can say, in their mother tongue, the Creed, the Lord's Prayer, and the Ten Commandments; and also can answer to the other questions of this short Catechism; they shall be brought to the Bishop. And every one shall have a Godfather, or a Godmother, as a witness of their Confirmation.}

\pilcrow{And whensoever the Bishop shall give knowledge for Children to be brought unto him for their Confirmation, the Curate of every Parish shall either bring, or send in writing, with his hand subscribed thereunto, the names of all such persons within his Parish, as he shall think fit to be presented to the Bishop to be confirmed. And, if the Bishop approve of them, he shall confirm them in manner following.}

\fleuron

\chapter[Confirmation]{\stylechapter{}{The Order of Confirmation}{Or Laying on of Hands upon Those That Are Baptized and Come to Years of Discretion\\ }}



\pilcrow{Upon the day appointed, all that are to be then confirmed, being placed, and standing in order, before the Bishop; he (or some other Minister appointed by him) shall read this Preface following.}

% 1549 rubric, then become a preface.
% \drop{To the end that Confirmation may be ministered to the more edifying of such as shall receive it, the Church hath thought good to order, That none hereafter shall be Confirmed, but such as can say the Creed, the Lord's Prayer, and the Ten Commandments; and can also answer to such other Questions, as in the short Catechism are contained; which order is very convenient to be observed; to the end, that children, being now come to the years of discretion, and having learned what their Godfathers and Godmothers promised for them in Baptism, they may themselves, with their own mouth and consent, openly before the Church, ratify and confirm the same; and also promise, that by the grace of God they will evermore endeavour themselves faithfully to observe such things, as they, by their own confession, have assented unto.}
%En28, Sc29

\drop{Dearly beloved in the Lord, in ministering Confirmation the Church doth follow the example of the Apostles of Christ. For in the eighth chapter of the Acts of the Apostles we thus read:—}

They therefore that were scattered abroad went about preaching the word. And Philip went down to the city of Samaria, and proclaimed unto them the Christ. When they believed Philip preaching good tidings concerning the kingdom of God and the name of Jesus Christ, they were baptized, both men and women. Now when the Apostles which were at Jerusalem heard that Samaria had received the word of God, they sent unto them Peter and John; who, when they were come down, prayed for them, that they might receive the Holy Ghost: for as yet he was fallen upon none of them; only they had been baptized into the Name of the Lord Jesus. Then laid they their hands on them, and they received the Holy Ghost.

The Scripture here teacheth us that a special gift of the Holy Spirit is bestowed through laying on of hands with prayer. And forasmuch as this gift cometh from God alone, let us that are here present pray to Almighty God, that he will strengthen with his Holy Spirit in Confirmation those who in Baptism were made his children.

You, then, who are to be confirmed must now declare before this congregation that you are stedfastly purposed, with the help of this gift, to lead your life in the faith of Christ and in obedience to God’s will and commandments; and must openly acknowledge yourselves bound to fulfil the Christian duties to which your Baptism hath pledged you.

\section{The Renewal of Baptismal Vows}

\centerline{\pilcrow{Then shall the Bishop say,}}

\lettrine{D}{\emph{o}} \emph{ye} here, in the presence of God, and of this congregation, renew the solemn promise and vow that was made in \emph{your} name at \emph{your} Baptism; ratifying and confirming the same in \emph{your} own persons, and acknowledging \emph{yourselves} bound to believe, and to do, all those things, which \emph{your} Godfathers and Godmothers then undertook for \emph{you}?

\centerline{\rubric{And every one shall audibly answer,}}
\centerline{I do.}

\medskip

\centerline{\rubric{Or else the Bishop shall say,}}
%En1928
% \drop{Do ye here, in the presence of God, and of this congregation, renounce the devil and all his works, the pomps and vanity of this wicked world, and all the sinful lusts of the flesh, so that ye will not follow nor be led by them?}
% Scottish / 28Proposed
\lettrine{D}{\emph{o}} \emph{ye} here, in the presence of God, and of this congregation, renounce the devil and all his works, the vain pomp and glory of the world, with all covetous desires of the same, and the sinful desires of the flesh, so that \emph{ye will} not follow, nor be led by them?

\R I do.

\lettrine{D}{\emph{o}} \emph{ye} believe all the Articles of the Christian Faith as contained in the Apostles’ Creed?

\R I do.

\lettrine{D}{\emph{o}} \emph{ye} promise that \emph{ye will} endeavour to keep God’s holy will and commandments, and to walk in the same all the days of \emph{your} life?

% \drop{Will ye endeavour to keep God’s holy will and commandments, and to walk in the same all the days of your life?}

\R I do.

\medskip

\section{The Confirmation}

\centerline{\pilcrow{The Bishop.}}

\drop{Our \grecross\ help is in the Name of the Lord;  \R Who hath made heaven and earth.}

\V Blessed be the Name of the Lord;  \R Henceforth, world without end.

\V Lord, hear our prayers.  \R And let our cry come unto thee. %1662

\V The Lord be with you.  \R And with thy spirit. %1549

\centerline{Let us pray.}
\drop{Almighty and everliving God, who hast vouchsafed to regenerate these thy servants by Water and the Holy Ghost, and hast given unto them forgiveness of all their sins: Strengthen them, we beseech thee, O Lord, with the Holy Ghost the Comforter, and daily increase in them thy manifold gifts of grace;

The Spirit of wisdom and understanding;

The Spirit of counsel and ghostly strength;

The Spirit of knowledge and true godliness;

and fill them, O Lord, with the Spirit of thy holy fear, now and for ever. \R Amen.}

\smallskip

% 1549 & 1929
\drop{Sign them, O Lord, and mark them to be thine for ever by the virtue of the Holy Cross; mercifully confirm them with the inward unction of the Holy Ghost, that they may attain unto everlasting life.  \R Amen.}

\medskip

\pilcrow{Then all of them in order kneeling before the Bishop, he shall lay his hand upon the head of every one severally, saying}

%1912, 1549
\lettrine{\emph{N.}}{ I} sign thee with the sign of the \grealtcross\ Cross\footnote{\rubric{Here the Bishop shall sign the person with the sign of the Cross on the forehead with the holy Chrism.}} and I lay my hands [\rubric{or} hand] upon thee, in the Name of the Father, and of the Son, and of the Holy Ghost.

% Sarum
% and I confirm thee with the chrism of salvation.  In the name...
\smallskip

% The following from 1552, etc.
\drop{Defend, O Lord, this thy Child [\rubric{or} this thy Servant \rubric{or} Handmaiden] with thy heavenly grace, that \emph{he} may continue thine for ever; and daily increase in thy Holy Spirit more and more, until \emph{he} come unto thy everlasting kingdom. \R Amen.}

\medskip
\centerline{\pilcrow{Then shall the Bishop say,}}

\V Peace be with you.  \R And with thy spirit. %1549; 1552 on it was "The Lord be with you.

\centerline{\rubric{And (all kneeling down) the Bishop shall add,}}

\centerline{Let us pray.}

\begin{leftbar} % added in 1662
    \ourFather

\centerline{\rubric{And this Collect.}}
\end{leftbar}


\drop{Almighty and everliving God, who makest us both to will and to do those things that be good and acceptable unto thy divine Majesty; We make our humble supplications unto thee for these thy servants, upon whom (after the example of thy holy Apostles) we have now laid our hands, to certify them (by this sign) of thy favour and gracious goodness towards them. Let thy fatherly hand, we beseech thee, ever be over them; let thy Holy Spirit ever be with them; and so lead them in the knowledge and obedience of thy Word, that in the end they may obtain everlasting life; through our Lord Jesus Christ, who with thee and the Holy Ghost liveth and reigneth, ever one God, world without end. \R Amen.}

\section{The Conclusion}
\drop{O almighty Lord, and everlasting God, vouchsafe, we beseech thee, to direct, sanctify, and govern, both our hearts and bodies, in the ways of thy laws, and in the works of thy commandments; that, through thy most mighty protection both here and ever, we may be preserved in body and soul; through our Lord and Saviour Jesus Christ. \R Amen.}

\medskip
\centerline{\pilcrow{Then the Bishop shall bless them, saying thus,}}
% En1928 only
\begin{leftbar}
    
\drop{Go forth into the world in peace; be of good courage; hold fast that which is good; render to no man evil for evil; strengthen the fainthearted; support the weak; help the afflicted; honour all men; love and serve the Lord, rejoicing in the power of the Holy Spirit.}
\end{leftbar}

And the Blessing of God Almighty, the \grealtcross\ Father, the \grealtcross\ Son, and the Holy \grealtcross\ Ghost, be upon you, and remain with you for ever. \R Amen.

\medskip

\pilcrow{And there shall none be admitted to the holy Communion, until such time as he be confirmed, or be ready and desirous to be confirmed.}

\fleuron



\chapter[Of Confession]{The Order for the Reconciliation of a Penitent}
\subsubsection{commonly called Confession}

\pilcrow{The Penitent, kneeling, begins,}
Bless me, for I have sinned.

\subsubsection{The Priest gives the blessing,}
\drop{The Lord be in thy heart and upon thy lips, that so thou mayest worthily and rightly confess all thy sins, in the Name of the Father, and of the Son, and of the Holy Ghost. Amen.}

\medskip
\pilcrow{The Penitent then makes \emph{his} confession, saying,}

\drop{I confess to God Almighty, the Father, the Son, and the Holy Ghost, that I have sinned in thought, word, and deed, through my own grievous fault. Wherefore I pray God to have mercy upon me.}

And especially I have sinned in these ways . . .

% \rubric{The penitent then states the specific sins he can remember, and should end with the following:}

For these and all other sins which I cannot now remember, I am truly sorry. I pray God to have mercy on me. I firmly intend amendment of life, and I humbly beg forgiveness of God and his Church, and ask thee for penance, counsel, and absolution.
\medskip
\pilcrow{After the confession, the Priest may find it helpful to question the penitent, so that advice about possible reparation, or restitution, or how to face the future more successfully may be given.}

\pilcrow{Then some form of penance is given. This is not a penalty but some useful act which aids the penitent to make outward embodiment of his contrite purpose.}

%    Here the Priest may offer counsel, direction, and comfort.
\medskip
\pilcrow{The Priest then pronounces this absolution:}

\drop{Our Lord Jesus Christ, who hath left power to his Church to absolve all sinners who truly repent and believe in him, of his great mercy forgive thee thine offences: And by his authority committed to me, I absolve thee from all thy sins, \grealtcross\  In the Name of the Father, and of the Son, and of the Holy Ghost. \R Amen.}

\smallskip
The Lord hath put away all thy sins. \R Thanks be to God.

\subsubsection{The Priest concludes,}

Go in peace, and pray for me, a sinner.

\fleuron

% Book of Common Prayer (1662) Absolution
% Book of Common Prayer (proposed, 1923, Visitation of the Sick) Confession
% The Armed Forces Prayer Book (1951) - Rubrics & confession (parts)
% Book of Common Prayer (1979) - traditionalized.
% Anglican Service Book (some tweaks)



% ¶ Then let him tell his sins, which being ended the Priest shall say—God Almighty have mercy, and The Almighty and merciful Lord, as in the Ordinary of the Mass.

% {Almighty God have mercy upon thee, forgive thee thy sins, and bring thee to everlasting life.  Amen.}
% {May the Almighty and Merciful Lord grant thee pardon, absolution, and remission of thy sins.  Amen.}

% The Passion of our Lord Jesus Christ, the merits of the Blessed Virgin Mary, and of all the Saints, whatsoever good thou hast done, or evil thou hast endured, be to the for the remission of sins, the increase of grace, and the reward of eternal life.  Amen.

% ¶ Here let him enjoin the Penance, saying—
% And for a special Penance thou shalt say or do this or that.
% ¶ Then let him absolve him, and say—








\chapter{The Form of Solemnization of Matrimony}

\pilcrow{The laws respecting Matrimony
%, whether by publishing the Banns in Churches, or by Licence, 
being different in the several States, every Minister is left to the direction of those laws, in every thing that regards the civil contract between the parties.} %1928 American

\medskip
% And when the Banns are published, it shall be in the following form: I publish the Banns of Marriage between N. of —, and N. of —. If any of you know cause, or just impediment, why these two persons should not be joined together in holy Matrimony, ye are to declare it. This is the first [second or third] time of asking. %1928 American

\pilcrow{First, the Banns of all that are to be married together, 
must %1662, etc
should % ICCC
be published in the Church three several Sundays, or Holy-days in the time of Divine Service, immediately before the the sentences for the Offertory: the Curate saying after the accustomed manner.} % Parson's Handbook

I publish the Banns of Marriage between \emph{N.} of \emph{this Parish} and \emph{N.} of —. If any of you know cause, or just impediment, why these two persons should not be joined together in holy Matrimony, ye are to declare it.  This is the first [second \rubric{or} third] time of asking.% Parson's Handbook

% I publish the Banns of Marriage between M. of ----- and N. of -----. If any of you know cause, or just impediment, why these two persons should not be joined together in holy Matrimony, ye are to declare it. This is the first [second, or third] time of asking.


% \pilcrow{And if the persons that are to be married dwell in divers Parishes, the Banns must be asked in both Parishes; and the Curate of the one Parish shall not solemnize Matrimony betwixt them, without a Certificate of the Banns being thrice asked, from the Curate of the other Parish.}

\section{The Introduction}

\pilcrow{At the day and time appointed for solemnization of Matrimony, the persons to be married shall come 
% into the body of the Church %1662
to the Door of the Church, or some other convenient location, %ICCC
with their friends and neighbours: and there standing together, the Man on the right hand, and the Woman on the left, the Priest shall say,}

\drop{Dearly beloved, we are gathered together here in the sight of God, and in the face of this congregation, to join together \emph{this Man and this Woman} in holy Matrimony; which is an honourable estate, instituted of God}
% in the time of man’s innocency, %1662, scottish
himself, %1928 proposed
signifying unto us the mystical union that is betwixt Christ and his Church; which holy estate Christ adorned and beautified with his presence, and first miracle that he wrought, in Cana of Galilee; and is commended of Saint Paul to be honourable among all men: and therefore is not by any to be enterprised, nor taken in hand, unadvisedly, lightly, or wantonly,
%to satisfy men's carnal lusts and appetites, like brute beasts that have no understanding;
but reverently, discreetly, advisedly, soberly, and in the fear of God;
\begin{leftbar}
duly considering the 
chief %scottish
causes for which Matrimony was ordained.

% First, 
It was ordained for the 
%procreation of children, to %1662
increase of mankind according to the will of God, and that children might %1912
be brought up in the fear and nurture of the Lord, and to the praise of his holy Name.

% Secondly, It was ordained for a remedy against sin, and to avoid fornication; that such persons as have not the gift of continency might marry, and keep themselves undefiled members of Christ’s body.

% Thirdly, 
It was 
also %1912
ordained for the mutual society, help, and comfort, that the one ought to have of the other, both in prosperity and adversity.
\end{leftbar}

Into which holy estate these two persons present come now to be joined.

Therefore if any man can shew any just cause, why they may not lawfully be joined together, let him now speak, or else hereafter for ever hold his peace.

\smallskip
\pilcrow{And also, speaking unto the persons that shall be married, he shall say,}
\drop{I require and charge you both, as ye will answer at the dreadful day of judgement when the secrets of all hearts shall be disclosed, that if either of you know any impediment, why ye may not be lawfully joined together in Matrimony, ye do now confess it.} \emph{For be ye well assured, that so many as are coupled together otherwise than God's Word doth allow are not joined together by God; neither is their Matrimony lawful.}


% \pilcrow{At which day of Marriage, if any man do allege and declare any impediment, why they may not be coupled together in Matrimony, by God’s law, or the laws of this Realm; and will be bound, and sufficient sureties with him, to the parties; or else put in a caution (to the full value of such charges as the persons to be married do thereby sustain) to prove his allegation: then the solemnization must be deferred, until such time as the truth be tried.} %1662

{\footnotesize\rubric{The Minister, if he shall have reason to doubt of the lawfulness of the proposed Marriage, may demand sufficient surety for his indemnification.}\par} % 1928

\section{The Marriage}

\pilcrow{If no impediment be alleged, then shall the Curate say unto the \emph{Man},}
\lettrine{\emph{N.}}{ wilt} thou have this \emph{woman} to thy wedded \emph{wife}, to live together after God’s ordinance in the holy estate of Matrimony? Wilt thou love \emph{her}, comfort \emph{her}, honour, and keep \emph{her} in sickness and in health; and, forsaking all other, keep thee only unto \emph{her}, so long as ye both shall live?

\centerline{\rubric{The \emph{Man} shall answer,} I will.}

\medskip

\centerline{\rubric{Then shall the Priest say unto the \emph{Woman},}}
\lettrine{\emph{N.}}{ wilt} wilt thou have this \emph{man} to thy wedded \emph{husband}, to live together after God’s ordinance in the holy estate of Matrimony? Wilt thou obey \emph{him}, and serve \emph{him}, love, honour, and keep \emph{him} in sickness and in health; and, forsaking all other, keep thee only unto \emph{him}, so long as ye both shall live?
\centerline{\rubric{The \emph{Woman} shall answer,} I will.}


\begin{leftbar}
\centerline{\rubric{Then shall the Priest say,}} %1928 proposed
\centerline{Who giveth this woman to be married to this man?}
\end{leftbar}


\pilcrow{Then shall they give their troth to each other in this manner.}

\pilcrow{The Minister, [receiving the Woman at her father’s or friend’s hands,] shall cause the Man with his right hand to take the Woman by her right hand, and to say after him as followeth.}
\drop{I, \emph{N.} take thee \emph{N.} to my wedded \emph{wife}, to have and to hold from this day forward, for better for worse, for richer for poorer, in sickness and in health, to love and to cherish, till death us do part, according to God’s holy ordinance; and thereto I plight thee my troth.}

\pilcrow{Then shall they loose their hands; and the Woman, with her right hand taking the Man by his right hand, shall likewise say after the Minister,}
\drop{I, \emph{N.} take thee \emph{N.} to my wedded \emph{husband}, to have and to hold from this day forward, for better for worse, for richer for poorer, in sickness and in health, to love, cherish, and to obey, till death us do part, according to God’s holy ordinance; and thereto I give thee my troth.}

\bigskip
\pilcrow{Then shall they again loose their hands; and the Man shall give unto the Woman a Ring, [and other tokens of spousage, as gold or silver,] laying the same upon the book.  And the Priest, taking the ring, shall
%  with the accustomed duty to the Priest and Clerk %1662
say,} %Scottish, 1929
\drop{Bless this ring, \grealtcross\ O merciful Lord, that he who giveth and she who weareth it may ever be faithful one to another; through Jesus Christ our Lord. \R Amen} %1929

\pilcrow{Then shall he %scottish 1929
deliver the ring unto the Man, to put it upon the fourth finger of the Woman’s left hand. And the Man holding the Ring there, and taught by the Priest, shall say,}
\drop{With this ring I thee wed, [this gold and silver I thee give,] with my body I thee worship, and with all my worldly goods I thee endow: In the Name of the Father, and of the Son, and of the Holy Ghost. Amen.}



\bigskip
\pilcrow{Then the Man leaving the Ring upon the fourth finger of the Woman’s left hand, they shall both kneel down; and the Priest shall say,} %1928 proposed
\centerline{Let us pray.}
\drop{O eternal God, Creator and Preserver of all mankind, Giver of all spiritual grace, the Author of everlasting life: Send thy blessing upon these thy servants, \emph{this man and this woman}, whom we bless in thy Name; that, 
% as Isaac and Rebecca lived faithfully together, so these persons %1662
living faithfully together, they %1929
may surely perform and keep the vow and covenant betwixt them made, (whereof this Ring given and received is a token and pledge,) and may ever remain in perfect love and peace together, and live according to thy laws; through Jesus Christ our Lord. \R Amen.}

\pilcrow{Then shall the Priest join their right hands together, and say,}
{\centering Those whom God hath joined together let no man put asunder.\par}

\centerline{\rubric{Then shall the Minister speak unto the people.}}
\drop{Forasmuch as \emph{N.} and \emph{N.} have consented together in holy wedlock, and have witnessed the same before God and this company, and thereto have given and pledged their troth either to other, and have declared the same by giving and receiving of a Ring, and by joining of hands; I pronounce that they be \emph{Man and Wife} together, In the Name of the Father, and of the Son, and of the Holy Ghost. Amen.}

\centerline{\pilcrow{And the priest shall add this Blessing.}}
\drop{God the Father, \grealtcross\ God the Son, God the Holy Ghost, bless, preserve, and keep you; the Lord mercifully with his favour look upon you; and so fill you with all spiritual benediction and grace, that ye may so live together in this life, that in the world to come ye may have life everlasting. \R Amen.}

\section{The Benediction}
\pilcrow{Then the Minister or Clerks, going to the Lord’s Table, shall say or sing this Psalm following.}
\subsection[{Psalm 128}]{\stylesubsec{Psalm 128.}{Beati omnes.}{}}
\drop{Blessed are all they that fear the {\scshape Lord}, * and walk in his ways.}

2\enspace For thou shalt eat the labours of thine hands: * O well is thee, and happy shalt thou be.

3\enspace Thy wife shall be as the fruitful vine * upon the walls of thine house.

4\enspace Thy children like the olive-branches * round about thy table.

5\enspace Lo, thus shall the man be blessed * that feareth the {\scshape Lord}.

6\enspace The {\scshape Lord} from out of Sion shall so bless thee, * that thou shalt see Jerusalem in prosperity all thy life long;

7\enspace Yea, that thou shalt see thy children’s children, * and peace upon Israel.

Glory be to the Father, and to the Son, \star\  and to the Holy Ghost;

As it was in the beginning, is now, and ever shall be, \star\  world without end. Amen.

\medskip
\centerline{\rubric{Or this Psalm,}}
\subsection[{Psalm 67}]{\stylesubsec{Psalm 67.}{Deus misereatur.}{}}
\drop{God be merciful unto us, and bless us, \star\ and shew us the light of his countenance, and be merciful unto us:}

2\enspace That thy way may be known upon earth, \star\ thy saving health among all nations.

3\enspace Let the people praise thee, O God; \star\ yea, let all the people praise thee.

4\enspace O let the nations rejoice and be glad; \star\ for thou shalt judge the folk righteously, and govern the nations upon earth.

5\enspace Let the people praise thee, O God; \star\ let all the people práise thee.

6\enspace Then shall the earth bring forth her íncrease; \star\ and God, even our own God, shall gíve us his blessing.

7\enspace God shall bless us; \star\ and all the ends of the world shall fear him.

Glory be to the Father, and to the Son, \star\  and to the Holy Ghost;

As it was in the beginning, is now, and ever shall be, \star\  world without end. Amen.


\medskip
\centerline{\rubric{Or this Psalm,}}
\subsection[{Psalm 133}]{\stylesubsec{Psalm 133.}{Ecce, quam bonum!}{}}
\drop{Behold, how good and joyful a thing it is, \star\ brethren, to dwell together in unity!}

2\enspace It is like the precious ointment upon the head, that ran down unto the beard, \star\ even unto Aaron’s beard, and went down to the skirts of his clothing.

3\enspace Like as the dew of Hermon, \star\ which fell upon the hill of Sion.

4\enspace For there the {\scshape Lord} promised his blessing, \star\ and life for evermore.

Glory be to the Father, and to the Son, \star\  and to the Holy Ghost;

As it was in the beginning, is now, and ever shall be, \star\  world without end. Amen.

\medskip
\centerline{\rubric{Or this Canticle,}}

\subsection[{The Song of Ruth}]{\stylesubsec{The Song of Ruth.}{Ne adverseris mihi.}{Ruth i.}}
\drop{Intreat me not to leave thee, \star\  or to return from following after thee:}

For whither thou goest, I will go; \star\ and where thou lodgest, I will lodge:

Thy people shall be my people, \star\ and thy God my God:

Where thou diest, will I die, \star\ and there will I be buried:

The {\scshape Lord} do so to me, and more also,  \star\  if ought but death part thee and me.

Glory be to the Father, and to the Son, \star\  and to the Holy Ghost;

As it was in the beginning, is now, and ever shall be, \star\  world without end. Amen.

\bigskip

\pilcrow{The Psalm ended, and the \emph{Man and the Woman} kneeling before the Lord’s Table, the Priest standing at the Table, and turning his face towards them, shall say,}
\centerline{Lord, have mercy upon us.}
\centerline{\emph{Christ, have mercy upon us.}}
\centerline{Lord, have mercy upon us.}

\medskip

\drop{Our Father, which art in heaven, Hallowed be thy Name. Thy kingdom come. Thy will be done in earth, As it is in heaven. Give us this day our daily bread. And forgive us our trespasses, As we forgive them that trespass against us. And lead us not into temptation, But deliver us from evil. Amen.}

\smallskip

\V O Lord, save thy \emph{servant, and thy handmaid}; \R Who put their trust in thee.

\V O Lord, send them help from thy holy place; \R And evermore defend them.

\V Be unto them a tower of strength, \R From the face of their enemy.

\V O Lord, hear our prayer. \R And let our cry come unto thee.

\centerline{Let us pray.}
\drop{O God of Abraham, God of Isaac, God of Jacob, bless these thy servants, and sow the seed of eternal life in their hearts; that whatsoever in thy holy Word they shall profitably learn, they may in deed fulfil the same. 
Look, O Lord, mercifully upon them from heaven, and bless them. And as thou didst send thy blessing upon Abraham and Sarah, to their great comfort, so vouchsafe to send thy blessing upon these thy servants; %1662, omitted in 1928 proposed
that they obeying thy Will, and alway being in safety under thy protection, may abide in thy love unto their lives’ end; through Jesus Christ our Lord. \R Amen.}

Or, offer alternatives to Abraham and Sarah?
Ruth and Naomi
David and Jonathan
Mary and Martha
Peter and Paul
Sergius and Bacchus
Perpetua and Felicity

Polyeuct and Nearchus
Symeon of Emessa and John
Vivaldo and Bartolo
Avertanus and Romeo
Brigid and Darlughdach
Protus and Hyacinth, martyrs


\begin{leftbar}
\pilcrow{This Prayer next following shall be omitted, where the woman is past child-bearing.}
% \drop{O merciful Lord, and heavenly Father, by whose gracious gift mankind is increased: We beseech thee, assist with thy blessing these two persons, that they may both be fruitful in procreation of children, and also live together so long in godly love and honesty, that they may see their children Christianly and virtuously brought up, to thy praise and honour; through Jesus Christ our Lord. \R Amen.} %1928 proposed alters toward the below
% 1928 american, 1912, 1929 scottish
\drop{O almighty God, Creator of mankind, who only art the well-spring of life; Bestow upon these thy servants, if it be thy will, the gift and heritage of children; and grant that they may see their children brought up in thy faith and fear, to the honour and glory of thy Name; through Jesus Christ our Lord. \R Amen.}
\end{leftbar}

\centerline{\pilcrow{This Prayer shall follow.}}
% 1928 american
\drop{O God, who hast so consecrated the state of Matrimony that in it is represented the spiritual marriage and unity betwixt Christ and his Church; Look mercifully upon these thy \emph{servants}, that they may love, honour, and cherish each other, and so live together in faithfulness and patience, in wisdom and true godliness, that their home may be a haven of blessing and of peace; through the same Jesus Christ our Lord, who liveth and reigneth with thee and the Holy Spirit ever, one God, world without end. \R Amen.}

% \drop{O God, who by thy mighty power hast made all things of nothing; who also (after other things set in order) didst appoint, that out of man (created after thine own image and similitude) woman should take her beginning; and, knitting them together, didst teach that it should never be lawful to put asunder those whom thou by Matrimony hadst made one: O God, who hast consecrated the state of Matrimony to such an excellent mystery, that in it is signified and represented the spiritual marriage and unity betwixt Christ and his Church: Look mercifully upon these thy servants, that both this man may love his wife, according to thy Word, (as Christ did love his spouse the Church, who gave himself for it, loving and cherishing it even as his own flesh,) and also that this woman may be loving and amiable, faithful and obedient to her husband; and in all quietness, sobriety, and peace, be a follower of holy and godly matrons. O Lord, bless them both, and grant them to inherit thy everlasting kingdom; through Jesus Christ our Lord. \R Amen.}  %1662. Cut down and slightly altered in 1928 proposed and scottish; completely replaced in the american 1928 with the above.


\centerline{\pilcrow{Then shall the Priest say
this Blessing,}} %1928 proposed
\drop{Almighty God, 
% who at the beginning did create our first parents, Adam and Eve, and did sanctify and join them together in marriage; %1662
the Father of our Lord Jesus Christ, %1928 proposed
Pour upon you the riches of his grace, sanctify and \grealtcross\ bless you, that ye may please him both in body and soul, and live together in holy love unto your lives’ end. \R Amen.}

\pilcrow{If there be a Communion, the foregoing Prayer and Blessing shall be said over the \emph{man and woman} immediately before the final Blessing of the congregation at the Communion.}

\bigskip

\pilcrow{If there be no Communion, nor sermon declaring the duties of man and wife, there shall be read some portion of Scripture, or the Priest shall say the Exhortation appended to this Liturgy.  And then the Priest shall dismiss those that are gathered together, saying,}

\centerline{Let us pray.}
\drop{O almighty Lord, and everlasting God, vouchsafe, we beseech thee, to direct, sanctify, and govern, both our hearts and bodies, in the ways of thy laws, and in the works of thy commandments; that through thy most mighty protection, both here and ever, we may be preserved in body and soul; through our Lord and Saviour Jesus Christ. \R Amen.}
\medskip
\drop{The Blessing of God Almighty, \grealtcross\ the Father, the Son, and the Holy Ghost, be amongst you and remain with you always. \R Amen.}


\section{The Communion}

\pilcrow{If there be a Communion, the following Collect, Epistle, and Gospel may be used, immediately after the prayer for child-bearing; or after the Collect preceding it, if that be not said.}

\subsection{\stylesubsec{}{The Collect.}{}}
\drop{O God our Father, who by thy holy apostle hast taught us that love is the fulfilling of the law: Grant to these thy \emph{servants} that, loving one another, they may continue in thy love unto their lives’ end; through Jesus Christ our Lord, who liveth and reigneth with thee in the unity of the Holy Ghost, one God world without end. \R Amen.}

\subsection{\stylesubsec{}{The Epistle.}{Ephesians 3.~14.}}
\drop{For this cause I bow my knees unto the Father of our Lord Jesus Christ, of whom the whole family in heaven and earth is named, that he would grant you, according to the riches of his glory, to be strengthened with might by his Spirit in the inner man; that Christ may dwell in your hearts by faith; that ye, being rooted and grounded in love, May be able to comprehend with all saints what is the breadth, and length, and depth, and height; and to know the love of Christ, which passeth knowledge, that ye might be filled with all the fulness of God.}

\subsection{\stylesubsec{}{The Gospel.}{St.~John 15.~9.}}
\drop{As the Father hath loved me, so have I loved you: continue ye in my love.  If ye keep my commandments, ye shall abide in my love; even as I have kept my Father’s commandments, and abide in his love. These things have I spoken unto you, that my joy might remain in you, and that your joy might be full.  This is my commandment, that ye love one another, as I have loved you.}


{\footnotesize\rubric{The Priest shall then continue the Order of Holy Communion, the new-married persons remaining before the Holy Table until the end of the service. The last prayer and blessing, from \emph{The Benediction}, shall be said over them immediately before \emph{The peace of God,} \etc}\par}



\section{An Exhortation}

\drop{All ye that are married, or that intend to take the holy estate of Matrimony upon you, hear what the holy Scripture doth say as touching the duty of husbands towards their wives, and wives towards their husbands.}

Saint Paul, in his Epistle to the Ephesians, the fifth Chapter, doth give this commandment to all married men; Husbands, love your wives, even as Christ also loved the Church, and gave himself for it, that he might sanctify and cleanse it with the washing of water, by the Word; that he might present it to himself a glorious Church, not having spot, or wrinkle, or any such thing; but that it should be holy, and without blemish. So ought men to love their wives as their own bodies. He that loveth his wife loveth himself: for no man ever yet hated his own flesh, but nourisheth and cherisheth it, even as the Lord the Church: for we are members of his body, of his flesh, and of his bones. For this cause shall a man leave his father and mother, and shall be joined unto his wife; and they two shall be one flesh. This is a great mystery; but I speak concerning Christ and the Church. Nevertheless, let every one of you in particular so love his wife, even as himself.


Likewise the same Saint Paul, writing to the Colossians, speaketh thus to all men that are married; Husbands, love your wives, and be not bitter against them.

Hear also what Saint Peter, the Apostle of Christ, who was himself a married man, saith unto them that are married; Ye husbands, dwell with your wives according to knowledge; giving honour unto the wife, as unto the weaker vessel, and as being heirs together of the grace of life, that your prayers be not hindered.

Hitherto ye have heard the duty of the husband toward the wife. Now likewise, ye wives, hear and learn your duties toward your husbands, even as it is plainly set forth in holy Scripture.

Saint Paul, in the aforenamed Epistle to the Ephesians, teacheth you thus; Wives, submit yourselves unto your own husbands, as unto the Lord. For the husband is the head of the wife, even as Christ is the head of the Church: and he is the Saviour of the body. Therefore as the Church is subject unto Christ, so let the wives be to their own husbands in every thing. And again he saith, Let the wife see that she reverence her husband.

And in his Epistle to the Colossians, Saint Paul giveth you this short lesson; Wives, submit yourselves unto your own husbands, as it is fit in the Lord.

Saint Peter also doth instruct you very well, thus saying; Ye wives, be in subjection to your own husbands; that, if any obey not the Word, they also may without the Word be won by the conversation of the wives; while they behold your chaste conversation coupled with fear. Whose adorning, let it not be that outward adorning of plaiting the hair, and of wearing of gold, or of putting on of apparel; but let it be the hidden man of the heart, in that which is not corruptible; even the ornament of a meek and quiet spirit, which is in the sight of God of great price. For after this manner in the old time the holy women also, who trusted in God, adorned themselves, being in subjection unto their own husbands; even as Sarah obeyed Abraham, calling him lord; whose daughters ye are as long as ye do well, and are not afraid with any amazement.

\fleuron
% \chapter{The Blessing of Civil Marriage}
% SA/Indian suppliment
% Churching moved here?

\chapter[The Visitation of the Sick]{\stylechapter{The Order for}{The Visitation of the Sick}{and the Communion of the Sick}}

\section{Visitation}
%These rubrics from 1923,1928,1929
\pilcrow{When any person is sick, notice shall be given thereof to the Minister,
%  of the Parish; 
who shall minister to the sick person after the form following, or in like manner.}
\smallskip
\pilcrow{When he cometh into the sick person’s house, he shall say,}
\drop{Peace be to this house, and to all that dwell in it.}

\bigskip
{\centering\pilcrow{When he cometh into the sick person’s presence he shall say, kneeling down,}}


% \centerline{\rubric{Then the Minister shall say,}}
% Let us pray.

\centerline{Lord, have mercy upon us.}
\centerline{\emph{Christ, have mercy upon us.}}
\centerline{Lord, have mercy upon us.}

\medskip
\ourFather


\V O Lord, save thy servant; \R Which putteth \emph{his} trust in thee.

\V Send \emph{him} help from thy holy place; \R  And evermore mightily defend \emph{him}.

\V Let the enemy have no advantage of \emph{him};  \R Nor the wicked approach to hurt \emph{him}.

\V Be unto \emph{him}, O Lord, a strong tower. \R  From the face of \emph{his} enemy.

\V O Lord, hear our prayers.  \R And let our cry come unto thee.

\centerline{Let us pray.}
\drop{O Lord, look down from heaven, behold, visit, and relieve this thy servant. Look upon \emph{him} with the eyes of thy mercy, give \emph{him} comfort and sure confidence in thee, defend \emph{him} from the danger of the enemy, and keep \emph{him} in perpetual peace and safety; through Jesus Christ our Lord. \R Amen.}

\smallskip

Hear us, Almighty and most merciful God and Saviour; extend thy accustomed goodness to this thy servant who is grieved with sickness.  \R Amen. % The following divisions are from 1928 proposed.
\smallskip

% \begin{leftbar}
% restored from 1549
Visit \emph{him}, O Lord, as thou didst visit Peter’s wife’s mother, and the captain’s servant. And as thou preservedst Toby and Sarah by thy Angel from danger, so restore unto this sick person \emph{his} former health, if it be thy will.  \R Amen.
% \end{leftbar}
\smallskip

Sanctify this trial unto \emph{him}; that the sense of \emph{his} weakness may add strength to \emph{his} faith, and seriousness to \emph{his} repentance. \R Amen.

\smallskip

May it be thy good pleasure to restore \emph{him} to \emph{his} former health, that so \emph{he} may lead the residue of \emph{his} life in thy fear, and to thy glory. \R Amen.

\smallskip

And whatsoever the issue that thou shalt ordain for \emph{him}, give \emph{him} grace so to be conformed to thy will, that \emph{he} may be made meet to dwell with thee in life everlasting; through Jesus Christ our Lord. \R Amen.

\section{Exhortation to Faith and Prayer}

{\centering\pilcrow{Then shall the Minister exhort the sick person upon such subjects as the following:}}

\drop{Our Heavenly Father, in his love for all men, uses sickness as a gracious means whereby to correct his children.}

Our Lord Jesus Christ, ever present with us, is ready to impart to us spiritual strength to use sickness well to the glory of God.

Our Lord, manifested in the Gospel as the healer of disease, is still ready to minister grace for the healing of the body.

Our Lord himself, though sinless, was made perfect through sufferings; and sinful man needs discipline in order to correct and amend in him whatever is amiss in the eyes of our heavenly Father.

The aim of the Christian, whether in health or in sickness, is that God may be glorified in him through Jesus Christ.

There is great honour in suffering if our pain be conformed to the spirit of Jesus Christ; for in the bearing of pain God manifested his will to redeem the world.

In sickness as in health we are to seek constantly the inspiration of God the Holy Ghost, the Spirit of Christ.


% \medskip
% \rubric{Or the Minister may exhort the sick person after this form,}
% \drop{Dearly beloved, know this, that Almighty God is the Lord of life and death, and of all things to them pertaining, as youth, strength, health, age, weakness, and sickness. Wherefore, whatsoever your sickness is, know you certainly, that it is God’s visitation. And for what cause soever this sickness is sent unto you; whether it be to try your patience for the example of others, and that your faith may be found in the day of the Lord laudable, glorious, and honourable, to the increase of glory and endless felicity; or else it be sent unto you to correct and amend in you whatsoever doth offend the eyes of your heavenly Father; know you certainly, that if you truly repent you of your sins, and bear your sickness patiently, trusting in God’s mercy, for his dear Son Jesus Christ’s sake, and render unto him humble thanks for his fatherly visitation, submitting yourself wholly unto his will, it shall turn to your profit, and help you forward in the right way that leadeth unto everlasting life.}

% Take therefore in good part the chastisement of the Lord: For (as Saint Paul saith in the twelfth Chapter to the Hebrews) whom the Lord loveth he chasteneth, and scourgeth every son whom he receiveth. If ye endure chastening, God dealeth with you as with sons; for what son is he whom the father chasteneth not? But if ye be without chastisement, whereof all are partakers, then are ye bastards, and not sons. Furthermore, we have had fathers of our flesh, which corrected us, and we gave them reverence: shall we not much rather be in subjection unto the Father of spirits, and live? For they verily for a few days chastened us after their own pleasure; but he for our profit, that we might be partakers of his holiness. These words, good \emph{brother}, are written in holy Scripture for our comfort and instruction; that we should patiently, and with thanksgiving, bear our heavenly Father’s correction, whensoever by any manner of adversity it shall please his gracious goodness to visit us. And there should be no greater comfort to Christian persons, than to be made like unto Christ, by suffering patiently adversities, troubles, and sicknesses. For he himself went not up to joy, but first he suffered pain; he entered not into his glory before he was crucified. So truly our way to eternal joy is to suffer here with Christ; and our door to enter into eternal life is gladly to die with Christ; that we may rise again from death, and dwell with him in everlasting life. Now therefore, taking your sickness, which is thus profitable for you, patiently, I exhort you, in the Name of God, to remember the profession which you made unto God in your Baptism. And forasmuch as after this life there is an account to be given unto the righteous judge, by whom all must be judged, without respect of persons, I require you to examine yourself and your estate, both toward God and man; so that, accusing and condemning yourself for your own faults, you may find mercy at our heavenly Father’s hand for Christ’s sake, and not be accused and condemned in that fearful judgement. Therefore I shall rehearse to you the Articles of our Faith, that you may know whether you do believe as a Christian man should, or no.

\medskip
{\centering\rubric{Or if need require he shall explain to him some part of the Christian faith. Which explanation ended, he shall say,}\par}

\drop{I exhort you in the name of God to remember the profession of faith which you made unto God in your baptism, and therefore I shall rehearse to you the Articles of our Faith, that you may shew whether you do believe as a Christian man should.}
\medskip

\bigskip

{\centering\pilcrow{Here the Minister shall rehearse the Articles of the Faith, saying thus,}\par}

\drop{Dost thou believe in God the Father Almighty, Maker of heaven and earth?}


And in Jesus Christ his only-begotten Son our Lord? And that he was conceived by the Holy Ghost, born of the Virgin Mary; that he suffered under Pontius Pilate, was crucified, dead, and buried; that he went down into hell, and also did rise again the third day; that he ascended into heaven, and sitteth at the right hand of God the Father Almighty; and from thence shall come again at the end of the world, to judge the quick and the dead?

And dost thou believe in the Holy Ghost; the holy Catholick Church; the Communion of Saints; the Remission of sins; the Resurrection of the flesh; and everlasting life after death?

\centerline{\rubric{The sick person shall answer,}}
\centerline{All this I stedfastly believe.}

\medskip

%1928 proposed:
\pilcrow{Thereafter, as occasion serves, the Minister shall instruct the sick person so to order his rule of prayer, for himself and others, that his days of sickness may be a time of faithful and loving intercourse with God.}


\section{Exortation to Repentance}
\pilcrow{The Minister shall examine the sick person, whether he repent him truly of his sins, and be in charity with all the world; exhorting him to forgive, from the bottom of his heart, all persons that have offended him; and if he hath offended any other, to ask them forgiveness; and where he hath done injury or wrong to any man, that he make amends to the uttermost of his power.}

\smallskip

\pilcrow{And if he have not before disposed of his goods, let him then be admonished to make his Will, and to declare his Debts, what he oweth, and what is owing unto him; for the better discharging of his conscience, and the quietness of his Executors. But men should often be put in remembrance to take order for the settling of their temporal estates, whilst they are in health.}

% These next two omitted in 1928:
% \pilcrow{These words before rehearsed may be said before the Minister begin his Prayer, as he shall see cause.}

\pilcrow{The Minister should not omit earnestly to move such sick persons as are of ability to be liberal to the poor.}

\medskip
\centerline{\rubric{Then shall the Priest say,}}
%1928, 1929
\drop{Forasmuch as after this life there is an account to be given unto the righteous Judge, by whom all must be judged, without respect of persons, I require you to examine yourself and your state, both toward God and man; so that accusing and condemning yourself for your own faults, you may find mercy at our heavenly Father’s hand for Christ’s sake.}

\medskip

\centerline{\rubric{After such examination he shall say,}}
\drop{Remember not, Lord, our offences, neither take thou vengeance of our sins; spare us, good Lord, spare thy people, whom thou hast redeemed with thy most precious blood, and be not angry with us for ever.}

\R Spare us, good Lord.


\medskip

\pilcrow{Here shall the sick person be moved to make a special confession of his sins, if he feel his conscience troubled with any weighty matter,
in this or other like form.%1928 proposed
}
\drop{I confess to God Almighty, the Father, the Son, and the Holy Ghost, that I have sinned in thought, word, and deed, through my own grievous fault; wherefore I pray God to have mercy on me. And especially I have sinned in these ways....}

\medskip

{\centering\rubric{After which confession, the Priest shall absolve him (if he humbly and heartily desire it) after this sort.}\par}

\drop{Our Lord Jesus Christ, who hath left power to his Church to absolve all sinners who truly repent and believe in him, of his great mercy forgive thee thine offences: And by his authority committed to me, I absolve thee from all thy sins, \grealtcross\  In the Name of the Father, and of the Son, and of the Holy Ghost. \R Amen.}

\medskip
\centerline{\rubric{And then he shall say the Collect following.}}
\centerline{Let us pray.}
\drop{O most merciful God, who, according to the multitude of thy mercies, dost so put away the sins of those who truly repent, that thou rememberest them no more: Open thine eye of mercy upon this thy servant, who most earnestly desireth pardon and forgiveness.
\vspace{-1em}
\begin{leftbar}
Renew in \emph{him}, most loving Father, whatsoever hath been decayed by the fraud and malice of the devil, or by \emph{his} own carnal will and frailness; preserve and continue this sick member in the unity of the Church; consider \emph{his} contrition, accept \emph{his} tears, asswage \emph{his} pain, as shall seem to thee most expedient for \emph{him}.
\end{leftbar}
\vspace{-1em}

And forasmuch as he putteth \emph{his} full trust only in thy mercy, impute not unto \emph{him} \emph{his} former sins, but strengthen \emph{him} with thy blessed Spirit; and, when thou art pleased to take \emph{him} hence, take \emph{him} unto
% thy favour,
thine everlasting favour;
through the merits of thy most dearly beloved Son Jesus Christ our Lord. 
\R Amen.}

%green book
\section{The Blessing and Anointing of the Sick}
% 1928 An Act of Prayer and Blessing
% 1929 ANOINTING, AND LAYING ON OF HANDS
% SA THE ANOINTING OF THE SICK 


\rubric{Anthem.} O Saviour of the world, who by thy Cross and precious Blood hast redeemed us, save us, and help us, we humbly beseech thee, O Lord.

% \subsection[{Psalm 71}]{\stylesubsec{Psalm 71.}{In te, Domine, speravi.}{}}
% \drop{In thee, O {\scshape Lord}, have I put my trust; let me never be put to confusion, \star\ but rid me and deliver me in thy righteousness; incline thine ear unto me, and save me.}

% 2\enspace Be thou my strong hold, whereunto I may alway resort: \star\ thou hast promised to help me, for thou art my house of defence and my castle.

% 3\enspace Deliver me, O my God, out of the hand of the ungodly, \star\ out of the hand of the unrighteous and cruel man.

% 4\enspace For thou, O Lord {\scshape God}, art the thing that I long for: \star\ thou art my hope, even from my youth.

% 5\enspace Through thee have I been holden up ever since I was born: \star\ thou art he that took me out of my mother’s womb: my praise shall be always of thee.

% 6\enspace I am become as it were a monster unto many, \star\ but my sure trust is in thee.

% 7\enspace O let my mouth be filled with thy praise, \star\ that I may sing of thy glory and honour all the day long.

% 8\enspace Cast me not away in the time of age; \star\ forsake me not when my strength faileth me.

% 9\enspace For mine enemies speak against me; and they that lay wait for my soul take their counsel together, saying, \star\ God hath forsaken him; persecute him, and take him, for there is none to deliver him.

% 10\enspace Go not far from me, O God; \star\ my God, haste thee to help me.

% 11\enspace Let them be confounded and perish that are against my soul; \star\ let them be covered with shame and dishonour that seek to do me evil.

% 12\enspace As for me, I will patiently abide alway, \star\ and will praise thee more and more.

% 13\enspace My mouth shall daily speak of thy righteousness and salvation; \star\ for I know no end thereof.

% 14\enspace I will go forth in the strength of the Lord {\scshape God}, \star\ and will make mention of thy righteousness only.

% 15\enspace Thou, O God, hast taught me from my youth up until now; \star\ therefore will I tell of thy wondrous works.

% 16\enspace Forsake me not, O God, in mine old age, when I am gray-headed, \star\ until I have shewed thy strength unto this generation, and thy power to all them that are yet for to come.

% 17\enspace Thy righteousness, O God, is very high, \star\ and great things are they that thou hast done: O God, who is like unto thee!

\subsection[{Psalm 121}]{\stylesubsec{Psalm 121.}{Levavi oculus.}{}}
% \drop{In thee, O {\scshape Lord}, have I put my trust; let me never be put to confusion, \star\ but rid me and deliver me in thy righteousness; incline thine ear unto me, and save me.}


\drop{I will lift up mine eyes unto the hills; \star\ from whence cometh my help?}

2\enspace My help cometh even from the {\scshape Lord}, \star\ who hath made heaven and earth.

3\enspace He will not suffer thy foot to be moved; \star\ and he that keepeth thee will not sleep.

4\enspace Behold, he that keepeth israel \star\ shall neither slumber nor sleep.

5\enspace The {\scshape Lord} himself is thy keeper; \star\ the {\scshape Lord} is thy defence upon thy right hand;

6\enspace So that the sun shall not burn thee by day, \star\ neither the moon by night.

7\enspace The {\scshape Lord} shall preserve thee from all evil; \star\ yea, it is even he that shall keep thy soul.

8\enspace The {\scshape Lord} shall preserve thy going out, and thy coming in, \star\ from this time forth for evermore.

Glory be to the Father, and to the Son, \star\  and to the Holy Ghost;

As it was in the beginning, is now, and ever shall be, \star\  world without end. Amen.

{\centering\rubric{Or any other Psalm, such as the following: \emph{23, 27, 43, 71 (\emph{verses} 1–17), 77, 86, 91, 103, 130, 142, 146.}}\par}

\rubric{Anthem.} O Saviour of the world, who by thy Cross and precious Blood hast redeemed us, save us, and help us, we humbly beseech thee, O Lord.

\medskip

\pilcrow{Then shall the Minister say (laying his hands upon the sick person if desired),} %1928
\drop{O Almighty God, who art the giver of all health, and the aid of them that seek to thee for succour: We call upon thee for thy help and goodness mercifully to be shewed upon this thy servant, that being healed of \emph{his} infirmities, \emph{he} may give thanks unto thee in thy holy Church; through Jesus Christ our Lord. \R Amen.}





\begin{leftbar}
    %nonjurors 1718
\pilcrow{If the oil is to be then hallowed, he shall say standing the following prayer.}
\drop{O Almighty Lord God, who hast taught us by thy holy Apostle Saint James to anoint the sick with oil, that they may recover their health and render thanks unto thee for the same; Bless \grealtcross\ this oil, we beseech thee, that whosoever may be anointed therewith, may be delivered from all troubles of body and mind, and from every assault of the powers of evil; through Jesus Christ our Lord. \R Amen.}
\end{leftbar}

%1929
\pilcrow{Then shall the Priest, if the sick person so desire it, proceed to anoint him with oil, saying as followeth:}

%1549
% \pilcrow{If the sick person desire to be anointed, then shall the Priest anoint him upon the forehead or breast only, making the sign of the cross, saying thus,}
\drop{N, I anoint thee with hallowed oil, \grealtcross\ In the Name of the Father, and of the Son, and of the Holy Ghost. \R Amen.}


%greenbook
\centerline{\rubric{He may add the following benediction.}}

\drop{As with this visible oil thy body outwardly is anointed, so may our heavenly Father, God Almighty, grant of his infinite goodness, that thy soul inwardly may be anointed with the Holy Ghost, who is the Spirit of all strength, comfort, relief, and gladness.}
May he % green book
%  and 1549
vouchsafe of his great mercy (if it be his blessed will) to restore unto thee thy bodily health, and strength to serve him joyfully; and send thee release of all thy pains, troubles, and diseases both in body and mind.
\vspace{-1em}
\begin{leftbar}
And howsoever his goodness, by his divine and unsearchable providence, shall dispose of thee: we, his unworthy ministers and servants, humbly beseech the eternal majesty to do with thee according to the multitude of his innumerable mercies, and to pardon thee all thy sins and offences, committted by all thy bodily senses, passions, and carnal affections.
\end{leftbar}

\vspace{-1.7em}

\begin{leftbar}
    May he % green book
% who %1549
also vouchsafe mercifully to grant unto thee ghostly strength by his Holy Spirit to withstand and overcome all temptations and assaults of thine adversary, that in no wise he prevail against thee, but that thou mayest have perfect victory and triumph against the devil, sin, and death;
\end{leftbar}
\vspace{-1em}

Through Christ our Lord, who by his death hath overcome the prince of death; and with the Father and the Holy Ghost everymore liveth and reigneth God, world without end.  \R Amen.

% Usque quo, Domine. Psalm xiii.

% How long wilt thou forget me, (O Lord,) for ever? how long wilt thou hyde thy face from me? How long shall I seke counsell in my soule? and be so vexed in myne herte? how long shall myne enemye triumph over me? Consydre, and heare me, (O lord my God): lighten myne iyes, that I slepe not in death. Leste myne enemy saye: I have prevayled against hym: for yf I be cast downe, they that trouble me will reioyce at it. But my trust is in thy mercy: and my herte is joyfull in thy salvacion. I will sing of the lord, because he hath delte so lovingly with me. Yea, I wyll prayse the name of the Lord the most highest. Glory be to the, \etc As it was in the, \etc

\medskip
\centerline{\rubric{Then shall the Minister say,}}
\drop{The Almighty Lord, who is a most strong tower to all them that put their trust in him, to whom all things in heaven, in earth, and under the earth, do bow and obey, be now and evermore thy defence; and make thee know and feel, that there is none other Name under heaven given to man, in whom, and through whom, thou mayest receive health and salvation, but only the Name of our Lord Jesus Christ. \R Amen.}

\centerline{\pilcrow{And after that he shall say,}}
\drop{Unto God’s gracious mercy and protection we commit thee. The {\scshape Lord} \cross  bless thee, and keep thee. The {\scshape Lord} make his face to shine upon thee, and be gracious unto thee. The {\scshape Lord} lift up his countenance upon thee, and give thee peace, both now and evermore. \R Amen.}


\section{The Communion of the Sick}
\pilcrow{Forasmuch as all mortal men be subject to many sudden perils, diseases, and sicknesses, and ever uncertain what time they shall depart out of this life; therefore, to the intent they may always be in a readiness to die, whensoever it shall please Almighty God to call them, the Curate shall diligently from time to time (but especially in the time of pestilence, or other infectious sickness) exhort their Parishioners to the often receiving of the Holy Communion of the Body and Blood of our Saviour Christ, when it shall be publickly administered in the church; that so doing, they may, in case of sudden visitation, have the less cause to be disquieted for lack of the same.}


% \begin{leftbar}
\pilcrow{The Curate shall also instruct the people concerning the Communion of the Sick, as occasion shall require, that they may not be in ignorance that men can receive the Holy Sacrament in their homes, if they be unable, for any just cause, to come to the church.}
% \end{leftbar}

\pilcrow{But if the sick person be not able to come to the Church, and yet is desirous to receive the Communion in his house; then he must give timely notice to the Priest, signifying also, as far as he may, whether there be some to communicate with him; as is much to be desired.}

\smallskip

\pilcrow{When the consecrated Bread and Wine are taken from the church to the sick person, before the Priest administers the Holy Sacrament, he shall use at least the parts of the \emph{Order of Communion} on \emph{pg.~\pageref{reservedSacrament}} here named: the \emph{General Confession} and \emph{Absolution}, 
% (which may be in the shorter form), 
and the prayer \emph{We do not presume, \etc}, except when extreme sickness shall otherwise require: and after the delivery of the Sacrament of Christ’s Body and Blood with the appointed words, he shall say the \emph{Lord’s Prayer} and the \emph{Blessing}. And immediately thereafter any of the consecrated Elements that remain over shall be reverently consumed, or else taken back to the church.}


% \stylesec{The Celebration}{of the}{Holy Communion for the Sick}
% \smallskip
\pilcrow{And a convenient place in the sick man’s house, together with all things necessary, having been prepared that the Curate may reverently minister, he shall there celebrate the \emph{Order of Communion}, according to the form in this Book prescribed; save only that he may, at his discretion, begin with the Collect, Epistle, and Gospel here following, or else with those of the Day.}

% \drop{O praise the {\scshape Lord}, all ye heathen; praise him, all ye nations. For his merciful kindness is ever more and more towards us; and the truth of the {\scshape Lord} endureth for ever.  Praise the {\scshape Lord}. Glory be to the Father, and to the Son, and to the Holy Ghost;  As it was in the beginning, is now, and ever shall be, world without end. Amen.}

% \medskip
% \centerline{Lord, have mercy upon us.}
% \centerline{\emph{Christ, have mercy upon us.}}
% \centerline{Lord, have mercy upon us.}
% \medskip
% \V The Lord be with you.  \R And with thy spirit.
% \centerline{Let us pray.}
\subsection{\stylesubsec{}{The Collect.}{}}


% \begin{leftbar}

\drop{Almighty and immortal God, giver of life and health: We beseech thee to hear our prayers for this thy servant, that by thy blessing upon \emph{him} and upon those who minister to \emph{him}, \emph{he} may be restored to health of body and mind, and give thanks to thee in thy holy Church; through Jesus Christ our Lord. \R Amen.}

\centerline{\rubric{Or this.}}
\drop{Almighty, everliving God, Maker of mankind, who dost correct those whom thou dost love, and chastise every one whom thou dost receive: We beseech thee to have mercy upon this thy servant visited with thine hand, and to grant that \emph{he} may take \emph{his} sickness patiently, and recover \emph{his} bodily health, (if it be thy gracious will); and whensoever \emph{his} soul shall depart from the body, it may be without spot presented unto thee; through Jesus Christ our Lord. \R Amen.}

% The following collect is one of the Postcommunions, made singular.
\centerline{\rubric{Or this.}}
\drop{Assist us mercifully, O Lord, in these our supplications and prayers, and dispose the way of thy servant towards the attainment of everlasting salvation; that among all the changes and chances of this mortal life, \emph{he} may ever be defended by thy most gracious and ready help; through Jesus Christ our Lord. \R Amen.}
% \end{leftbar}

% \medskip
% \subsection{\stylesubsec{}{The Epistle.}{Hebrews 12.~5.}}
% \drop{My son, despise not thou the chastening of the Lord, nor faint when thou art rebuked of him. For whom the Lord loveth he chasteneth; and scourgeth every son whom he receiveth.}
% % \begin{leftbar}

%     \smallskip
% \centerline{\rubric{Or this.}}
% \vspace{-10pt}
\subsection{\stylesubsec{}{The Epistle.}{2 Corinthians 1.~3.}}
\drop{Blessed be God, even the Father of our Lord Jesus Christ, the Father of mercies, and the God of all comfort; who comforteth us in all our tribulation, that we may be able to comfort them which are in any trouble, by the comfort wherewith we ourselves are comforted of God. For as the sufferings of Christ abound in us, so our consolation also aboundeth by Christ.}
% \end{leftbar}

\medskip
% \subsection{\stylesubsec{}{The Gospel.}{St.~John 5.~24.}}
% \drop{Verily, verily I say unto you, He that heareth my word, and believeth on him that sent me, hath everlasting life, and shall not come into condemnation; but is passed from death unto life.}

% % \begin{leftbar}
%     \smallskip
% \centerline{\rubric{Or this.}}
% \vspace{-10pt}
\subsection{\stylesubsec{}{The Gospel.}{St.~John 10.~14, 15; 27–30.}}
\drop{I am the good shepherd; and I know mine own, and mine own know me, even as the Father knoweth me, and I know the Father; and I lay down my life for the sheep. My sheep hear my voice, and I know them, and they follow me: and I give unto them eternal life; and they shall never perish, and no one shall pluck them out of my hand. My Father, which hath given them unto me, is greater than all; and no one is able to pluck them out of the Father’s hand. I and the Father are one.}
% \end{leftbar}

\pilcrow{After which the Priest shall proceed according to the form before prescribed for the Order of Communion.}
%, beginning at  these words \emph{Ye that do truly,} \etc}
% the {\emph Offertory}, \emph{pg.~\pageref{offertory}}}



% \drop{I will offer to thee the sacrifice of thanksgiving, and will call upon the Name of the {\scshape Lord}.}

% \V The Lord be with you. \R And with thy spirit.

% \V Lift up your hearts.  \R We lift them up unto the Lord.

% \V Let us give thanks unto our Lord God. \R It is meet and right so to do.


% \centerline{\rubric{Then shall the Priest turn to the Lord’s Table, and say,}}

% \drop{It is very meet, right, and our bounden duty, that we should at all times, and in all places, give thanks unto thee, O Lord, Holy Father, Almighty, Everlasting God.  Therefore with Angels and Archangels, and with all the company of heaven, we laud and magnify thy glorious Name; evermore praising thee, and saying,}
% \smallskip

% \drop{Holy, holy, holy, Lord God of hosts, heaven and earth are full of thy glory: Glory be to thee, O Lord most High. \grecross\ Blessed is he that cometh in the Name of the Lord;
% Hosanna in the highest.}

% \bigskip

% \pilcrow{When the Priest, standing before the Table, hath so ordered the Bread and Wine, that he may with the more readiness and decency break the Bread before the people, and take the Cup into his hands, he shall say the Prayer of Consecration, as followeth.}
% \drop{Almighty God, our heavenly Father, who of thy tender mercy didst give thine only Son Jesus Christ to suffer death upon the Cross for our redemption; who made there (by his one oblation of himself once offered) a full, perfect, and sufficient sacrifice, oblation, and satisfaction, for the sins of the whole world; and did institute, and in his holy Gospel command us to continue, a perpetual memory of that his precious death, until his coming again;

% Hear us, O merciful Father, we most humbly beseech thee; and grant that we receiving these thy creatures of bread and wine, according to thy Son our Saviour Jesus Christ’s holy institution, in remembrance of his death and passion, may be partakers of his most blessed \grealtcross\ Body and \grealtcross\ Blood: 

% Who, in the same night that he was betrayed, \footnote{\rubric{Here the Priest is to take the Paten unto his hands:}}took Bread; and, when he had given thanks, \footnote{\rubric{And here to break the Bread:}}he brake it, and gave it to his disciples, saying, Take, eat, \footnote{\rubric{And here to lay his hand upon all the Bread.}}{\scshape this is my Body which is given for you}: Do this in remembrance of me. Likewise after supper he \footnote{\rubric{Here he is to take the Cup into his hand:}}took the Cup; and, when he had given thanks, he gave it to them, saying, Drink ye all of this; \footnote{\rubric{And here to lay his hand upon every vessel (be it Chalice or Flagon) in which there is any Wine to be consecrated.}}{\scshape for this is my Blood of the New Testament, which is shed for you and for many for the remission of sins}: Do this, as oft as ye shall drink it, in remembrance of me.}

% Wherefore, O Lord and heavenly Father, we thy humble servants, having in remembrance the precious death of thy dear Son, his mighty resurrection and glorious ascension, looking also for his coming again, do render unto thee most hearty thanks for the innumerable benefits which he hath procured unto us; and we entirely desire thy fatherly goodness mercifully to accept this our sacrifice of praise and thanksgiving; most humbly beseeching thee to grant, that by the merits and death of thy Son Jesus Christ, and through faith in his blood, we and all thy whole Church may obtain remission of our sins, and all other benefits of his passion.

% And here we offer and present unto thee, O Lord, ourselves, our souls and bodies, to be a reasonable, holy, and lively sacrifice unto thee; 
% and we pray thee of thine almighty goodness to send upon us, and upon these thy gifts, thy holy and blessed Spirit, who is the Sanctifier and the Giver of life; humbly beseeching thee, that all we, who are partakers of this holy Communion, may be fulfilled with thy grace and heavenly \grecross\ benediction. 

% And although we be unworthy, through our manifold sins, to offer unto thee any sacrifice, yet we beseech thee to accept this our bounden duty and service; not weighing our merits, but pardoning our offences;

% Through Jesus Christ our Lord; by whom, and with whom, in the unity of the Holy Ghost, all honour and glory be unto thee, O Father Almighty, world without end. \R Amen.

% \smallskip
% {\centering\footnotesize\rubric{Here shall the people join with the Priest in the Lord’s Prayer, the Priest first saying,}\par}
% As our Saviour Christ hath commanded and taught us we are bold to say,
% \drop{Our Father, which art in heaven, Hallowed be thy Name. Thy kingdom come. Thy will be done, in earth as it is in heaven. Give us this day our daily bread. And forgive us our trespasses, As we forgive them that trespass against us. And lead us not into temptation; But deliver us from evil.  For thine is the kingdom, The power, and the glory, For ever and ever. Amen.}

% \bigskip
% \pilcrow{Then shall the Priest say to them that come to receive the holy Communion,}
% \drop{Draw near with faith, and take this Holy Sacrament to your comfort; and make your humble confession to Almighty God, meekly kneeling upon your knees.}

% \smallskip
% \rubric{Then shall be said by the Minister and people together,}
% \drop{We confess to God Almighty, the Father, the Son, and the Holy Ghost, that we have sinned in thought, word, and deed, through our own grievous fault.  Wherefore we pray God to have mercy upon us.}

% \medskip
% {\centering\footnotesize\rubric{Then shall the Priest standing up, and turning himself to the people, pronounce this Absolution.}\par}
% \drop{Almighty God have mercy upon you, forgive you all your sins, and deliver you from all evil, confirm and strengthen you in all goodness, and bring you to everlasting life; through Jesus Christ our Lord. \R Amen.}

% \centerline{\pilcrow{Then shall the Priest say,}}
% Hear what comfortable words our Saviour Christ saith unto all that truly turn to him.
% \drop{Come unto me all that travail and are heavy laden, and I will refresh you.}\scripture{St.~Matthew xj.~28}

% So God loved the world, that he gave his only-begotten Son, to the end that all that believe in him should not perish, but have everlasting life.\scripture{St.~John iij.~16}

% \centerline{Hear also what Saint Paul saith.}

% This is a true saying, and worthy of all men to be received, That Christ Jesus came into the world to save sinners.\scripture{1 Timothy i.~15.}

% \centerline{Hear also what Saint John saith.}

% If any man sin, we have an Advocate with the Father, Jesus Christ the righteous; and he is the propitiation for our sins.\scripture{1 St.~John ij.~1.}

% \medskip

% {\centering\footnotesize\rubric{Then shall the Priest, kneeling down at the Lord’s Table, say in the name of all them that shall receive the Communion this Prayer following.}\par}
% \drop{We do not presume to come to this thy Table, O merciful Lord, trusting in our own righteousness, but in thy manifold and great mercies. We are not worthy so much as to gather up the crumbs under thy Table. But thou art the same Lord, whose property is always to have mercy: Grant us therefore, gracious Lord, so to eat the flesh of thy dear Son Jesus Christ, and to drink his blood, that our sinful bodies may be made clean by his body, and our souls washed through his most precious blood, and that we may evermore dwell in him, and he in us. Amen.}

\medskip

\pilcrow{At the time of the distribution of the Holy Sacrament, the priest shall first receive the Communion himself, and after minister unto them that are appointed to communicate with the sick, and last of all to the sick person.}
% \medskip

% {\centering\footnotesize\rubric{And, when he delivereth the Bread to any one, he shall say,}\par}
% \drop{The Body of our Lord Jesus Christ, which was given for thee, preserve thy body and soul unto everlasting life.}

% {\centering\footnotesize\rubric{And the Minister that delivereth the Cup to any one shall say,}\par}
% \drop{The Blood of our Lord Jesus Christ, which was shed for thee, preserve thy body and soul unto everlasting life.}

% \medskip

% {\footnotesize\rubric{When all have communicated, the Minister shall return to the Lord’s Table, and reverently place upon it what remaineth of the consecrated Elements, covering the same with a fair linen cloth.}\par}


% After shall be said as followeth.

% \drop{ALMIGHTY and everliving God, we most heartily thank thee, for that thou dost vouchsafe to feed us, who have duly received these holy mysteries, with the spiritual food of the most precious Body and Blood of thy Son our Saviour Jesus Christ; and dost assure us thereby of thy favour and goodness towards us; and that we are very members incorporate in the mystical body of thy Son, which is the blessed company of all faithful people; and are also heirs through hope of thy everlasting kingdom, by the merits of the most precious death and passion of thy dear Son. And we most humbly beseech thee , O heavenly Father, so to assist us with thy grace, that we may continue in that holy fellowship, and do all such good works as thou hast prepared for us to walk in; through Jesus Christ our Lord, to whom, with thee and the Holy Ghost, be all honour and glory, world without end. Amen.}
% \medskip

% {\centering\footnotesize\rubric{Then the Priest shall let them depart with this Blessing.}\par}
% \drop{The peace of God, which passeth all understanding, keep your hearts and minds in the knowledge and love of God, and of his son Jesus Christ our Lord: and the blessing of God Almighty, the Father, \grealtcross\ the Son, and the Holy Ghost, be amongst you and remain with you always. \R Amen.}

% \bigskip

% \begin{leftbar}
% \pilcrow{In case of extreme necessity the Priest may beg in with the Consecration and, immediately after the delivery of the Holy Sacrament to the sick person, end with the Blessing.}
% \end{leftbar}


\medskip

\pilcrow{The Priest shall instruct the people that if any man, by reason of great sickness, or any other just impediment, be not able at any time to receive the Sacrament of Christ’s Body and Blood, yet if he do truly repent him of his sins, and stedfastly believe that Jesus Christ both suffered death upon the Cross for him, and shed his Blood for his redemption, earnestly remembering the benefits he hath thereby, and giving him hearty thanks therefore, he doth eat and drink the Body and Blood of our Saviour Christ profitably to his Soul’s health, although he do not receive the Sacrament with his mouth.}

% When the sick person is visited, and receiveth the holy Communion all at one time, then the Priest, for more expedition, shall cut off the form of the Visitation at the Psalm [In thee, O Lord, have I put my trust \etc] and go straight to the Communion.

% \pilcrow{In the time of the plague, sweat, or such other like contagious times of sickness or diseases, when none of the Parish or neighbors can be gotten to communicate with the sick in their houses, for fear of the infection, upon special request of the diseased, the Minister may only communicate with him.}

\bigskip


\centerline{\rule{0.5\textwidth}{0.5pt}}
\medskip
\pilcrow{When it is desirable to administer both kinds together, the words of administration shall be said thus}

\smallskip

\drop{The Body of our Lord Jesus Christ, which was given for thee, and his Blood which was shed for thee, preserve thy body and soul unto everlasting life.}

\smallskip

{\centering\footnotesize\rubric{Take this in remembrance that Christ died for thee, and feed on him in thy heart by faith with thanksgiving.}\par}


\smallskip

\pilcrow{{\scshape Note}, that the same order shall be observed, with the permission of the Bishop, when it is deemed necessary, through grave danger of infection, to administer both kinds together to certain communicants at the open Communion.}


\section[Special Prayers]{Special Prayers to be Used as Occasion may Serve}
\subseccaption{}{A Litany for the Sick or Dying.}

\drop{O God the Father,}

\qquad\emph{Have mercy.}

O God the Son,

\qquad\emph{Have mercy.}

O God the Holy Ghost,

\qquad\emph{Have mercy.}

O Holy Trinity, one God,

\qquad\emph{Have mercy.}

Remember not, Lord, our offences.

\qquad\emph{Spare us, Good Lord.}

From all evil and sin,

\qquad\emph{Good Lord, deliver \emph{him.}}

From the assaults of the devil,

\qquad\emph{Good Lord, deliver \emph{him.}}

From thy wrath, and from everlasting damnation,

\qquad\emph{Good Lord, deliver \emph{him.}}

In the hour of death,

\qquad\emph{Good Lord, deliver \emph{him.}}

In the day of judgement,

\qquad\emph{Good Lord, deliver \emph{him.}}

By the mystery of thine Incarnation,

\qquad\emph{Save \emph{him}, O Lord.}

By thy Cross and Passion,

\qquad\emph{Save \emph{him.}, O Lord.}

By thy Resurrection and final Triumph,

\qquad\emph{Save \emph{him}, O Lord.}

That it may please thee to grant \emph{him} relief in pain;

\qquad\emph{We beseech thee to hear us.}

To give \emph{him} such health as is agreeable to thy will;

\qquad\emph{We beseech thee to hear us.}

That it may please thee to deliver \emph{his} soul;

\qquad\emph{We beseech thee to hear us.}

To cleanse \emph{him} from \emph{his} sin;

\qquad\emph{We beseech thee to hear us.}
    
That it may please thee to receive \emph{him} to thyself;

\qquad\emph{We beseech thee to hear us.}

To set \emph{him} in a place of light and peace;

\qquad\emph{We beseech thee to hear us.}

To number \emph{him} with thy saints and thine elect;

\qquad\emph{We beseech thee to hear us.}

Son of God;

\qquad\emph{We beseech thee to hear us.}

O Lamb of God;

\qquad\emph{Have mercy upon us.}

O Lamb of God;

\qquad\emph{Grant \emph{him} thy peace.}


\centerline{\rule{0.5\textwidth}{0.5pt}}


\centerline{\pilcrow{The following Prayers may be used as occasion requires.}}
\subseccaption{}{For Healing.}
\drop{O God, who by the might of thy command canst drive away from men’s bodies all sickness and infirmity: Be present in thy goodness with this thy servant, that \emph{his} weakness being banished, and \emph{his} health restored, \emph{he} may live to glorify thy holy Name; through our Lord Jesus Christ. \R Amen.}


\subseccaption{}{For a Sick Child.}
\drop{O Lord Jesus Christ, who didst with joy receive and bless the children brought to thee: Give thy blessing to this thy child; and in thine own time deliver \emph{him} from \emph{his} bodily pain, that \emph{he} may live to serve thee all \emph{his} days. \R Amen.}


\subseccaption{}{For one troubled in Conscience.}
\drop{O blessed Lord, the Father of mercies and the God of all comfort; We beseech thee, look down in pity and compassion on thy servant, whose soul is full of trouble: give \emph{him} a right understanding of \emph{himself}, and also of thy will for \emph{him}, that \emph{he} may neither cast away \emph{his} confidence in thee, nor place it anywhere but in thee; deliver \emph{him} from the fear of evil; lift up the light of thy countenance upon \emph{him}, and give \emph{him} thine everlasting peace; through the merits and mediation of Jesus Christ our Lord. \R Amen.}

%Am1928
\subseccaption{}{For a Person under Affliction.}
\drop{O merciful God, and heavenly Father, who hast taught us in thy holy Word that thou dost not willingly afflict or grieve the children of men; Look with pity, we beseech thee, upon the sorrows of thy servant for whom our prayers are offered. Remember \emph{him}, O Lord, in mercy; endue \emph{his} soul with patience; comfort \emph{him} with a sense of thy goodness; lift up thy countenance upon \emph{him}, and give \emph{him} peace; through Jesus Christ our Lord. \R Amen.}



\subseccaption{}{For a Convalescent.}
\drop{O Lord, whose compassions fail not, and whose mercies are new every morning: We give thee hearty thanks that it hath pleased thee to give to this our \emph{brother} both relief from pain and hope of renewed health; continue, we beseech thee, in \emph{him} the good work that thou hast begun; that, daily increasing in bodily strength, and humbly rejoicing in thy goodness, \emph{he} may so order \emph{his} life and conversation as always to think and do such things as shall please thee; through Jesus Christ our Lord. \R Amen.}


\subseccaption{}{For a Dying Child.}
\drop{O Lord Jesu Christ, the only-begotten Son of God, who for our sakes didst become a babe in Bethlehem: We commit unto thy loving care this child whom thou art calling to thyself. Send thy holy angel to lead \emph{him} gently to those heavenly habitations where the souls of them that sleep in thee have perpetual peace and joy, and fold \emph{him} in the everlasting arms of thine unfailing love; who livest and reignest with the Father and the Holy Ghost, one God world without end. \R Amen.}


\subseccaption{}{Commendatory Prayers.}
\drop{Thou knowest, Lord, the secrets of our hearts; shut not thy merciful ears to our prayer; but spare us, Lord most holy, O God most mighty, O holy and merciful Saviour, thou most worthy Judge eternal, suffer us not at our last hour, for any pains of death, to fall from thee. \R Amen.}

\smallskip

\drop{Unto thee, O Lord, we commend the soul of thy servant \emph{N.}, that, dying to the world, \emph{he} may live to thee; and whatsoever sins \emph{he} has committed through the frailty of earthly life, we beseech thee to do away by thy most loving and merciful forgiveness; through Jesus Christ our Lord. \R Amen.}

\smallskip

\drop{O Almighty God, with whom do live the spirits of just men made perfect, after they are delivered from their earthly prisons: We humbly commend the soul of this thy servant, our dear \emph{brother}, into thy hands, as into the hands of a faithful Creator, and most merciful Saviour; most humbly beseeching thee, that it may be precious in thy sight. Wash it, we pray thee, in the blood of that immaculate Lamb that was slain to take away the sins of the world; that whatsoever defilements it may have contracted in the midst of this miserable and naughty world, through the lusts of the flesh, or the wiles of Satan, being purged and done away, it may be presented pure and without spot before thee; through the merits of Jesus Christ thine only Son our Lord. \R Amen.}

\smallskip

\subseccaption{}{At the Point of Death.}

\drop{Go forth upon thy journey from this world, O Christian soul,}

In the Name of God the Almighty Father who created thee. \R Amen.

In the Name of Jesus Christ who suffered for thee. \R Amen.

In the Name of the Holy Ghost who strengtheneth thee. \R Amen.

In communion with the blessed Saints, and aided by Angels and Archangels,  and all the armies of the heavenly host. \R Amen.

May thy portion this day be in peace, and thy dwelling in the heavenly Jerusalem. \R Amen.

\bigskip
{\footnotesize
{\scshape Note}.— The following prayers and passages of Holy Scripture are suitable for use with the sick person: The Collect in the Communion of the Sick and the Collects appointed for the first, second and fourth Sundays in Advent, the third, fourth, and Sixth Sundays after Epiphany, Ash Wednesday, the second Sunday in Lent, the Sunday next before Easter, the fourth Sunday after Easter, Ascension Day, the Sunday after Ascension, Trinity Sunday, the fourth, sixth, seventh, twelfth, fifteenth, eighteenth, and twenty-first Sundays after Trinity, the Transfiguration, St.~Michael and All Angels, St.~Luke the Evangelist, and All Saints’ Day.

\newcommand{\numberedSuggestion}[3]{{\addfontfeatures{Numbers={Monospaced}}#1.\enspace}{\emph{#2}:\enskip}#3\par}

\numberedSuggestion{1}{Confidence in God}{Psalms 27, 46, 91, 121; Proverbs 3.~11–26; Isaiah 26.~1–9; 40.~1–11; 40.~25 to end; Lamentations 3.~22–41; St.~Matthew 6.~24 to end; Romans 8.~31 to end.}
\numberedSuggestion{2}{Answer to Prayer}{Psalms 30, 34.}
\numberedSuggestion{3}{Prayer for Divine Aid}{Psalms 43, 86, 143; St.~James 5.~10 to end.}
\numberedSuggestion{4}{Penitence}{Psalms 51, 130.}
\numberedSuggestion{5}{Praise and Thanksgiving}{Psalms 103, 146; Isaiah 12.}
\numberedSuggestion{6}{God’s dealing with Man through Affliction}{Job 33.~14–30; Hebrews 12.~1–11.}
\numberedSuggestion{7}{Christ our Example in Suffering}{Isaiah 53; St.~Matthew 26.~36–46; St.~Luke 23.~27–49.}
\numberedSuggestion{8}{God’s call to Repentance and Faith}{Isaiah 55.}
\numberedSuggestion{9}{The Beatitudes}{St.~Matthew 5.~1–12.}
\numberedSuggestion{10}{Watchfulness}{St.~Luke 12.~32–40.}
\numberedSuggestion{11}{Christ the Good Shepherd}{Psalm 23; St.~John 10.~1–18.}
\numberedSuggestion{12}{The Resurrection}{St.~John 20.~1–18; 20.~19 to end; 2 Corinthians 4.~13—5.~9.}
\numberedSuggestion{13}{Redemption}{Romans 5.~1–11; 8.~18 to end; 1 St.~John 1.~1–9.}
\numberedSuggestion{14}{Christian Love}{1 Corinthians 13.}
\numberedSuggestion{15}{Growth in Grace}{Ephesians 3.~13 to end; 6.~10–20; Philippians 3.~7–14.}
\numberedSuggestion{16}{Patience in Suffering}{St.~James 5.~10 to end.}
\numberedSuggestion{17}{God’s Love to Men}{1 St.~John 3.~1–7; 4.~9 to end.}
\numberedSuggestion{18}{The Life of the World to come}{Revelation 7.~9 to end; 21.~1–7; 21.~22 to end; 22.~1–5.}
\numberedSuggestion{19}{Our Lord’s last Discourse before his Passion}{St.~John 14, 15, 16, 17.}
\numberedSuggestion{20}{Christian Hope on the Approach of Death}{Deuteronomy 33.~27; Psalm 16.~9 to end; Psalm 23; St.~John 3.~16; 2 Corinthians 4.~16—5.~1; Revelation 21.~4–7.}
}

\fleuron

% split in two in 1928: for sick, and for dying.
% \subseccaption{}{A Prayer for a sick Child.}
% \drop{O Almighty God, and merciful Father, to whom alone belong the issues of life and death: Look down from heaven, we humbly beseech thee, with the eyes of mercy upon this child now lying upon the bed of sickness: Visit him, O Lord, with thy salvation; deliver him in thy good appointed time from his bodily pain, and save his soul for thy mercies’ sake: That, if it shall be thy pleasure to prolong his days here on earth, he may live to thee, and be an instrument of thy glory, by serving thee faithfully, and doing good in his generation; or else receive him into those heavenly habitations, where the souls of them that sleep in the Lord Jesus enjoy perpetual rest and felicity. Grant this, O Lord, for thy mercies’ sake, in the same thy Son our Lord Jesus Christ, who liveth and reigneth with thee and the Holy Ghost, ever one God, world without end. \R Amen.}

% \subseccaption{}{A Prayer for a sick person, when there appeareth small hope of recovery.}
% \drop{O Father of mercies, and God of all comfort, our only help in time of need: We fly unto thee for succour in behalf of this thy servant, here lying under thy hand in great weakness of body. Look graciously upon him, 0 Lord; and the more the outward man decayeth, strengthen him, we beseech thee, so much the more continually with thy grace and Holy Spirit in the inner man. Give him unfeigned repentance for all the errors of his life past, and stedfast faith in thy Son Jesus; that his sins may be done away by thy mercy, and his pardon sealed in heaven, before he go hence, and be no more seen. We know, 0 Lord, that there is no word impossible with thee; and that, if thou wilt, thou canst even yet raise him up, and grant him a longer continuance amongst us: Yet, forasmuch as in all appearance the time of his dissolution draweth near, so fit and prepare him, we beseech thee, against the hour of death, that after his departure hence in peace, and in thy favour, his soul may be received into thine everlasting kingdom, through the merits and mediation of Jesus Christ, thine only Son, our Lord and Saviour. \R Amen.}

%in the "Commendatory Prayers", slightly shortened.
% \subseccaption{}{A commendatory Prayer for a sick person at the point of departure.}
% \drop{O Almighty God, with whom do live the spirits of just men made perfect, after they are delivered from their earthly prisons: We humbly commend the soul of this thy servant, our dear brother, into thy hands, as into the hands of a faithful Creator, and most merciful Saviour; most humbly beseeching thee, that it may be precious in thy sight. Wash it, we pray thee, in the blood of that immaculate Lamb, that was slain to take away the sins of the world; that whatsoever defilements it may have contracted in the midst of this miserable and naughty world, through the lusts of the flesh, or the wiles of Satan, being purged and done away, it may be presented pure and without spot before thee. And teach us who survive, in this and other like daily spectacles of mortality, to see how frail and un, certain our own condition is; and so to number our days, that we may seriously apply our hearts to that holy and heavenly wisdom, whilst we live here, which may in the end bring us to life everlasting, through the merits of Jesus Christ thine only Son our Lord. \R Amen.}

% \subseccaption{}{A Prayer for persons troubled in mind or in conscience.}
% \drop{O blessed Lord, the Father of mercies, and the God of all comforts: We beseech thee, took down in pity and compassion upon this thy afflicted servant. Thou writest bitter things against him, and makest him to possess his former iniquities; thy wrath lieth hard upon him, and his soul is full of trouble: But, 0 merciful God, who hast written thy holy Word for our learning, that we, through patience and comfort of thy holy Scriptures, might have hope; give him a right understanding of himself, and of thy threats and promises; that he may neither cast away his confidence in thee, nor place it any where but in thee. Give him strength against all his temptations, and heal all his distempers. Break not the bruised reed, nor quench the smoking flax. Shut not up thy tender mercies in displeasure; but make him to hear of joy and gladness, that the bones which thou hast broken may rejoice. Deliver him from fear of the enemy, and lift up the light of thy countenance upon him, and give him peace, through the merits and mediation of Jesus Christ our Lord. \R Amen.}

% \begin{leftbar} %scottish 1912
% \subseccaption{}{A Prayer for the recovery of a sick person.}
% \drop{Almighty and immortal God, giver of life and health; We beseech thee to hear our prayers for thy servant N, for whom we implore thy mercy, that by thy blessing upon him and upon those who minister to him of thy healing gifts, he may be restored, if it be thy gracious will, to health of body and mind, and give thanks to thee in thy holy Church; through Jesus Christ our Lord. \R Amen.}
% \end{leftbar}

% If any question arise as to the manner of doing anything that is here enjoined or permitted, it shall be referred to the Bishop for his decision.

\chapter[The Burial of the Dead]{\stylechapter{The Order for}{The Burial of the Dead}{}}

\pilcrow{Here is to be noted, that the Office ensuing is not to be used for any that die unbaptized, or excommunicate, or have laid violent hands upon themselves.}

% Canada:1918
\pilcrow{Note also, That when this Office is not to be used, the Priest may at the grave read the Sentences beginning \emph{Man that is born,} followed by the Lesser Litany, the Lord’s Prayer, one or more Collects from this Book at his discretion, and \emph{The grace of the Lord \etc}}

\section{The Procession}

\pilcrow{The Minister and Clerks, meeting the Corpse at the entrance of the Church-yard, and going before it, either into the Church, or towards the Grave, shall say, or sing [one or more of the following Sentences; together with one or more of the Penitential Psalms \emph{(6, 32, 38, 51, 102, 130, 143)} if need so require.],}
\drop{I am the resurrection and the life, saith the Lord: he that believeth in me, though he were dead, yet shall he live: and whosoever liveth and believeth in me shall never die.\scripture{St.~John xi.~25, 26.}}

\drop{I know that my Redeemer liveth, and that he shalt stand at the latter day upon the earth. And though after my skin 
worms destroy this body, %KJV
% hath been thus destroyed, %adapted from rsv
% earth: and though this body be destroyed, yet shall I see God: %am1928
yet in my flesh shall I see God: whom I shall see for myself, and mine eyes shall behold, and not another. \scripture{Job xix.~25, 26, 27.}}


\drop{We brought nothing into this world, and it is certain we can carry nothing out. The {\scshape Lord} gave, and the {\scshape Lord} hath taken away; blessed be the Name of the {\scshape Lord}.\scripture{1 Timothy.~vi.~7. Job i.~21.}}

\section{The Service in church}
\pilcrow{After they are come into the Church, shall be sung or said one or more of these Psalms following. Note, that at the end of each of the Psalms the \emph{Gloria Patri} shall be omitted.} %1954


%1928 proposed
% \pilcrow{Before and after any psalm or group of psalms may be said or sung the Anthem following,}

% \drop{O Saviour of the world, who by thy Cross and precious Blood hast redeemed us, Save us and help us, we humbly beseech thee, O Lord.}
{\red\scshape Anthem.} O Saviour of the world, who by thy Cross and precious Blood hast redeemed us, \star\ Save us, and help us, we humbly beseech thee, O Lord.

%1929 
% After they are come into the church, shall be sung or said one or more of these Psalms following.


% Anthem. O Saviour of the world, who by thy Cross and precious Blood hast redeemed us : save us, and help us, we humbly beseech thee, O Lord.

%(1918)(1926)(1928e)(1929)(1954)
\subsection[{Psalm 23}]{\stylesubsec{Psalm 23.}{Dominus regit me.}{}}
\drop{The {\scshape Lord} is my shepherd; \star\ therefore can I lack nothing.}

2\enspace He shall feed me in a green pasture, \star\ and lead me forth beside the waters of comfort.

3\enspace He shall convert my soul, \star\ and bring me forth in the paths of righteousness, for his Name’s sake.

4\enspace Yea, though I walk through the valley of the shadow of death, I will fear no evil; \star\ for thou art with me; thy rod and thy staff comfort me.

5\enspace Thou shalt prepare a table before me against them that trouble me; \star\ thou hast anointed my head with oil, and my cup shall be full.

6\enspace But thy loving-kindness and mercy shall follow me all the days of my life; \star\ and I will dwell in the house of the Lord for ever.

Rest eternal grant unto them, O Lord \star\ and let light perpetual shine upon them.


% Psalm 39.  (1918)(1662)(1926)(1928a)(1929)
\subsection[{Psalm 39}]{\stylesubsec{Psalm 39.}{Dixi, custodiam.}{}}
\drop{I said, I will take heed to my ways, \star\ that I offend not in my tongue.}

2\enspace I will keep my mouth as it were with a bridle, \star\ while the ungodly is in my sight.

3\enspace I held my tongue, and spake nothing: \star\ I kept silence, yea, even from good words; but it was pain and grief to me.

4\enspace My heart was hot within me: and while I was thus musing the fire kindled, \star\ and at the last I spake with my tongue:

5\enspace {\scshape Lord}, let me know mine end, and the number of my days; \star\ that I may be certified how long I have to live.

6\enspace Behold, thou hast made my days as it were a span long, \star\ and mine age is even as nothing in respect of thee; and verily every man living is altogether vanity.

7\enspace For man walketh in a vain shadow, and disquieteth himself in vain; \star\ he heapeth up riches, and cannot tell who shall gather them.

8\enspace And now, Lord, what is my hope? \star\ truly my hope is even in thee.

9\enspace Deliver me from all mine offences; \star\ and make me not a rebuke unto the foolish.

10\enspace I became dumb, and opened not my mouth; \star\ for it was thy doing.

11\enspace Take thy plague away from me: \star\ I am even consumed by the means of thy heavy hand.

12\enspace When thou with rebukes dost chasten man for sin, thou makest his beauty to consume away, like as it were a moth fretting a garment: \star\ every man therefore is but vanity.

13\enspace Hear my prayer, O {\scshape Lord}, and with thine ears consider my calling; \star\ hold not thy peace at my tears;

14\enspace For I am a stranger with thee, and a sojourner, \star\ as all my fathers were.

15\enspace O spare me a little, that I may recover my strength, \star\ before I go hence, and be no more seen.

Rest eternal grant unto them, O Lord \star\ and let light perpetual shine upon them.

% Psalm 90.  (1918)(1662)(1926)(1928a)(1929)(1954)
\subsection[{Psalm 90}]{\stylesubsec{Psalm 90.}{Domine, refugium.}{}}
\drop{Lord, thou hast been our refuge, \star\ from one generation to another.}

2\enspace Before the mountains were brought forth, or ever the earth and the world were made, \star\ thou art God from everlasting, and world without end.

3\enspace Thou turnest man to destruction; \star\ again thou sayest, Come again, ye children of men.

4\enspace For a thousand years in thy sight are but as yesterday, \star\ seeing that is past as a watch in the night.
5\enspace As soon as thou scatterest them they are even as a sleep; \star\ and fade away suddenly like the grass.

6\enspace In the morning it is green, and groweth up; \star\ but in the evening it is cut down, dried up, and withered.

7\enspace For we consume away in thy displeasure, \star\ and are afraid at thy wrathful indignation.

8\enspace Thou hast set our misdeeds before thee; \star\ and our secret sins in the light of thy countenance.

9\enspace For when thou art angry all our days are gone: \star\ we bring our years to an end, as it were a tale that is told.

10\enspace The days of our age are threescore years and ten; and though men be so strong that they come to fourscore years, \star\ yet is their strength then but labour and sorrow; so soon passeth it away, and we are gone.

11\enspace But who regardeth the power of thy wrath? \star\ for even thereafter as a man feareth, so is thy displeasure.

12\enspace So teach us to number our days, \star\ that we may apply our hearts unto wisdom.

13\enspace Turn thee again, O {\scshape Lord}, at the last, \star\ and be gracious unto thy servants.

14\enspace O satisfy us with thy mercy, and that soon: \star\ so shall we rejoice and be glad all the days of our life.

15\enspace Comfort us again now after the time that thou hast plagued us; \star\ and for the years wherein we have suffered adversity.

16\enspace Shew thy servants thy work, \star\ and their children thy glory.

17\enspace And the glorious majesty of the {\scshape Lord} our God be upon us: \star\ prosper thou the work of our hands upon us, O prosper thou our handywork.

Rest eternal grant unto them, O Lord \star\ and let light perpetual shine upon them.


% Psalm 130. (1928a)(1928e)(1929)(1954)
\subsection[{Psalm 130}]{\stylesubsec{Psalm 130.}{De profundis.}{}}

\drop{Out of the deep have I called unto thee, O {\scshape Lord}; \star\ Lord, hear my voice.}

2\enspace O let thine ears consider well \star\ the voice of my complaint.

3\enspace If thou, {\scshape Lord}, wilt be extreme to mark what is done amiss, \star\ O Lord, who may abide it?

4\enspace For there is mercy with thee; \star\ therefore shalt thou be feared.

5\enspace I look for the {\scshape Lord}; my soul doth wait for him; \star\ in his word is my trust.

6\enspace My soul fleeth unto the Lord \star\ before the morning watch, I say, before the morning watch.

7\enspace O Israel, trust in the {\scshape Lord}, for with the {\scshape Lord} there is mercy, \star\ and with him is plenteous redemption.

8\enspace And he shall redeem Ísrael \star\ from all his sins.

Rest eternal grant unto them, O Lord \star\ and let light perpetual shine upon them.


\rubric{Anthem.} O Saviour of the world, who by thy Cross and precious Blood hast redeemed us, \star\ Save us, and help us, we humbly beseech thee, O Lord.


\medskip
\pilcrow{Then shall follow the Lesson, taken out of the fifteenth Chapter of the former Epistle of Saint Paul to the Corinthians.}
\centerline{1 Corinthians 15.~20.}
\drop{Now is Christ risen from the dead, and become the first-fruits of them that slept. For since by man came death, by man came also the resurrection of the dead. For as in Adam all die, even so in Christ shall all be made alive. But every man in his own order: Christ the firstfruits; afterward they that are Christ’s, at his coming. Then cometh the end, when he shall have delivered up the kingdom to God, even the Father; when he shall have put down all rule, and all authority, and power. For he must reign, till he hath put all enemies under his feet. The last enemy that shall be destroyed is death.
\begin{leftbar}
For he hath put all things under his feet. But when he saith, all things are put under him, it is manifest that he is excepted, which did put all things under him. And when all things shall be subdued unto him, then shall the Son also himself be subject unto him that put all things under him, that God may be all in all. Else what shall they do which are baptized for the dead, if the dead rise not at all? Why are they then baptized for the dead? and why stand we in jeopardy every hour? I protest by your rejoicing, which I have in Christ Jesus our Lord, I die daily. If after the manner of men I have fought with beasts at Ephesus, what advantageth it me, if the dead rise not? Let us eat and drink, for to-morrow we die. Be not deceived: evil communications corrupt good manners. Awake to righteousness, and sin not: for some have not the knowledge of God. I speak this to your shame.
\end{leftbar}

But some man will say, How are the dead raised up? and with what body do they come? Thou fool, that which thou sowest is not quickened, except it die. And that which thou sowest, thou sowest not that body that shall be, but bare grain, it may chance of wheat, or of some other grain: But God giveth it a body, as it hath pleased him, and to every seed his own body. All flesh is not the same flesh; but there is one kind of flesh of men, another flesh of beasts, another of fishes, and another of birds. There are also celestial bodies, and bodies terrestrial; but the glory of the celestial is one, and the glory of the terrestrial is another. There is one glory of the sun, and another glory of the moon, and another glory of the stars; for one star differeth from another star in glory. So also is the resurrection of the dead: It is sown in corruption; it is raised in incorruption: It is sown in dishonour; it is raised in glory: It is sown in weakness; it is raised in power: It is sown a natural body; it is raised a spiritual body. There is a natural body, and there is a spiritual body. And so it is written, The first man Adam was made a living soul; the last Adam was made a quickening spirit. Howbeit, that was not first which is spiritual, but that which is natural; and afterward that which is spiritual. The first man is of the earth, earthy: the second man is the Lord from heaven. As is the earthy, such are they that are earthy: and as is the heavenly, such are they also that are heavenly. And as we have borne the image of the earthy, we shall also bear the image of the heavenly. 

Now this I say, brethren, that flesh and blood cannot inherit the kingdom of God; neither doth corruption inherit incorruption. Behold, I shew you a mystery: We shall not all sleep, but we shall all be changed, in a moment, in the twinkling of an eye, at the last trump, (for the trumpet shall sound,) and the dead shall be raised incorruptible, and we shall be changed. For this corruptible must put on incorruption, and this mortal must put on immortality. So when this corruptible shall have put on incorruption, and this mortal shall have put on immortality; then shall be brought to pass the saying that is written, Death is swallowed up in victory. O death, where is thy sting? O grave, where is thy victory? The sting of death is sin, and the strength of sin is the law. But thanks be to God, which giveth us the victory through our Lord Jesus Christ. Therefore, my beloved brethren, be ye stedfast, unmoveable, always abounding in the work of the Lord, forasmuch as ye know that your labour is not in vain in the Lord.}

% \begin{leftbar}
\centerline{\rubric{Or one of the following Lessons:}}
\centerline{2 Corinthians 4.~16-end. and 5.~1-10.}
\drop{For which cause we faint not; but though our outward man perish, yet the inward man is renewed day by day. For our light affliction, which is but for a moment, worketh for us a far more exceeding and eternal weight of glory; While we look not at the things which are seen, but at the things which are not seen: for the things which are seen are temporal; but the things which are not seen are eternal. For we know that if our earthly house of this tabernacle were dissolved, we have a building of God, an house not made with hands, eternal in the heavens. For in this we groan, earnestly desiring to be clothed upon with our house which is from heaven: If so be that being clothed we shall not be found naked. For we that are in this tabernacle do groan, being burdened: not for that we would be unclothed, but clothed upon, that mortality might be swallowed up of life. Now he that hath wrought us for the selfsame thing is God, who also hath given unto us the earnest of the Spirit. Therefore we are always confident, knowing that, whilst we are at home in the body, we are absent from the Lord: (For we walk by faith, not by sight:) We are confident, I say, and willing rather to be absent from the body, and to be present with the Lord. Wherefore we labour, that, whether present or absent, we may be accepted of him. For we must all appear before the judgment seat of Christ; that every one may receive the things done in his body, according to that he hath done, whether it be good or bad.}


\medskip
\centerline{Revelation 7.~9.}%-17  Proofed 12-17-21
\drop{After this I beheld, and, lo, a great multitude, which no man could number, of all nations, and kindreds, and people, and tongues, stood before the throne, and before the Lamb, clothed with white robes, and palms in their hands; and cried with a loud voice, saying, Salvation to our God which sitteth upon the throne, and unto the Lamb. And all the angels stood round about the throne, and about the elders, and the four beasts, and fell before the throne on their faces, and worshipped God, saying, Amen; Blessing, and glory, and wisdom, and thanksgiving, and honour, and power, and might, be unto our God for ever and ever. Amen. And one of the elders answered, saying unto me, What are these which are arrayed in white robes? and whence came they? And I said unto him, Sir, thou knowest. And he said to me, These are they which came out of great tribulation, and have washed their robes, and made them white in the blood of the Lamb. Therefore are they before the throne of God, and serve him day and night in his temple: and he that sitteth on the throne shall dwell among them. They shall hunger no more, neither thirst any more; neither shall the sun light on them, nor any heat. For the Lamb which is in the midst of the throne shall feed them, and shall lead them unto living fountains of waters: and God shall wipe away all tears from their eyes.}


\medskip
\centerline{Revelation 21.~1-7.}
\drop{And I saw a new heaven and a new earth: for the first heaven and the first earth were passed away; and there was no more sea. And I John saw the holy city, new Jerusalem, coming down from God out of heaven, prepared as a bride adorned for her husband. And I heard a great voice out of heaven saying, Behold, the tabernacle of God is with men, and he will dwell with them, and they shall be his people, and God himself shall be with them, and be their God. And God shall wipe away all tears from their eyes; and there shall be no more death, neither sorrow, nor crying, neither shall there be any more pain: for the former things are passed away. And he that sat upon the throne said, Behold, I make all things new. And he said unto me, Write: for these words are true and faithful. And he said unto me, It is done. I am Alpha and Omega, the beginning and the end. I will give unto him that is athirst of the fountain of the water of life freely. He that overcometh shall inherit all things; and I will be his God, and he shall be my son.}
% \end{leftbar}


% Scott has 3 others  more like gospels.


\medskip

\centerline{\rubric{Then the Minister shall say,}}
\centerline{Lord, have mercy upon us.}
\centerline{\emph{Christ, have mercy upon us.}}
\centerline{Lord, have mercy upon us.}

\medskip
\ourFather

\smallskip

\V Enter not into judgement with thy servant, O Lord; \R For in thy sight shall no man living be justified.

\V Grant unto \emph{him} eternal rest; \R And let perpetual light shine upon \emph{him}.

\V We believe verily to see the goodness of the Lord: \R In the land of the living.

\V O Lord, hear our prayer; \R And let our cry come unto thee.

{\centering\pilcrow{THen shall be said one or more of the following Prayers, the Minister first saying,}}

\centerline{Let us pray.}
\drop{Almighty God, with whom do live the spirits of them that depart hence in the Lord, and with whom the souls of the faithful, after they are delivered from the burden of the flesh, are in joy and felicity:
% We give thee hearty thanks, for that it hath pleased thee to deliver this our brother out of the miseries of this sinful world; beseeching thee, that it may please thee, of thy gracious goodness, shortly to accomplish the number of thine elect, and to hasten thy kingdom; that we, with all those that are departed in the true faith of thy holy Name, may have our perfect consummation and bliss, both in body and soul, in thy eternal and everlasting glory; through Jesus Christ our Lord. \R Amen.}
%1549
% \drop{O Lord, with whom do live the spirits of them that be dead; and in whom the souls of them that be elected, after they be delivered from the burden of the flesh, be in joy and felicity: 
Grant unto this thy servant, that the sins which \emph{he} committed in this world be not imputed unto \emph{him}, but that \emph{he}, escaping the gates of hell and the pains of eternal darkness, may ever dwell in the region of light, with Abraham, Isaac, and Jacob, in the place where is no weeping, sorrow, nor heaviness; and when that dreadful day of the general resurrection shall come, make \emph{him} to rise also with the just and righteous, and receive this body again to glory, then made pure and incorruptible: set \emph{him} on the right hand of thy Son Jesus Christ, among thy holy and elect, that then \emph{he} may hear with them these most sweet and comfortable words: Come, ye blessed children of my Father, receive the kingdom prepared for you from the beginning of the world: Grant this, we beseech thee, O merciful Father, through Jesus Christ, our Mediator and Redeemer. \R Amen.}

%This collect from the old mass, moved back with its original ending.
% The Collect
% \drop{O merciful God, the Father of our Lord Jesus Christ, who is the resurrection and the life; ... Resurrection in the last day,}

% we may be found acceptable in thy sight; and receive that blessing, which thy well-beloved Son shall then pronounce to all that love and fear thee, saying, 

% This from the original prayer above:
% Come, ye blessed children of my Father, receive the kingdom prepared for you from the beginning of the world: Grant this, we beseech thee, O merciful Father, through Jesus Christ, our Mediator and Redeemer. \R Amen.

% \begin{leftbar}
% 1923:
% BLESSED Lord, who art the Father of mercies and the God of all consolation: We beseech thee of thy great goodness, to comfort those who by the death of this our brother are sorely bereaved; and teach us so to number our days that, while we live here, we may seriously apply our hearts to that holy and heavenly wisdom, which may in the end bring us to life everlasting, through the merits of Jesus Christ, thine only Son our Lord. Amen.

\smallskip
% 1928
\subsection{\stylesubsec{}{The Collects.}{}}
\drop{O Father of all, we pray to thee for those whom we love, but see no longer. Grant them thy peace; let light perpetual shine upon them; and in thy loving wisdom and almighty power work in them the good purpose of thy perfect will; through Jesus Christ our Lord. \R Amen.}

\smallskip

\drop{Almighty God, Father of all mercies and giver of all comfort: Deal graciously, we pray thee, with those who mourn, that casting every care on thee, they may know the consolation of thy love; through Jesus Christ our Lord. \R Amen.}

\smallskip

\drop{O heavenly Father, who in thy Son Jesus Christ, hast given us a true faith, and a sure hope: Help us, we pray thee, to live as those who believe and trust in the Communion of Saints, the forgiveness of sins, and the resurrection to life everlasting, and strengthen this faith and hope in us all the days of our life: through the love of thy Son, Jesus Christ our Saviour. \R Amen.}

\smallskip

{\centering\footnotesize\rubric{Here may follow the Collect of \emph{All Saints’ Day}, or that of the \emph{Twelfth Sunday after Trinity}, or others from the \emph{Prayers upon Several Occasions.}}\par}
% \end{leftbar}

\medskip
\centerline{\rubric{And then shall be said,}}
\smallskip

\medskip
\theGrace

\smallskip
%Scottish
\drop{May the souls of the faithful departed through the mercy of God rest in peace. \R Amen.}


\section{The Communion}
\pilcrow{When there is a special celebration of the Holy Communion on the day of the Burial, the Priest shall use the Collect appointed in this Order, or the Collect of Easter Even, and for the Epistle}

\subsection{\stylesubsec{}{The Collect.}{}} %Moved back from above.
\drop{O merciful God, the Father of our Lord Jesu Christ, who is the resurrection and the life; in whom whosoever believeth shall live, though he die; and whosoever liveth, and believeth in him, shall not die eternally; who also hath taught us by his holy Apostle Paul, not to be sorry, as men without hope, for them that sleep in him: We meekly beseech thee, O Father, to raise us from the death of sin unto the life of righteousness; that, when we shall depart this life, we may sleep in him, as our hope is this our \emph{brother} doth: and at the general resurrection in the last day both we, and this our \emph{brother} departed, receiving again our bodies, and rising again in thy most gracious favour, may with all thine elect Saints obtain eternal joy.
Grant this, O Lord God, by the means of our Advocate Jesus Christ: which with thee and the Holy Ghost, liveth and reigneth one God for ever. \R Amen.}

\subsection{\stylesubsec{}{The Epistle.}{1 Thessalonians 4.~13}}
\drop{I would not have you to be ignorant, brethren, concerning them which are asleep, that ye sorrow not, even as others which have no hope. For if we believe that Jesus died and rose again, even so them also which sleep in Jesus will God bring with him. For this we say unto you by the word of the Lord, that we which are alive and remain unto the coming of the Lord shall not precede them which are asleep. For the Lord himself shall descend from heaven with a shout, with the voice of the archangel, and with the trump of God: and the dead in Christ shall rise first: then we which are alive and remain shall be caught up together with them in the clouds, to meet the Lord in the air: and so shall we ever be with the Lord. Wherefore comfort one another with these words.}

\begin{leftbar}
\centerline{\rubric{Or this.}}
\centerline{2 Corinthians 4.~16, \emph{and part of chapter 5.}}
‘Though our outward man. . . swallowed up of life’
\end{leftbar}


\subsection{\stylesubsec{}{The Gospel.}{John 6.~37.}}
\drop{Jesus said, All that the Father giveth me shall come to me; and him that cometh to me I will in no wise cast out. For I came down from heaven, not to do mine own will, but the will of him that sent me. And this is the Father's will which hath sent me, that of all which he hath given me I should lose nothing, but should raise it up again at the last day. And this is the will of him that sent me, that every one which seeth the Son, and believeth on him, may have everlasting life: and I will raise him up at the last day.}

\begin{leftbar}
\centerline{\rubric{Or this.}}
\centerline{St.~John 5.~24}
‘Jesus said, Verily, verily, . . . the resurrection of judgement’.
\end{leftbar}

\medskip

%scottish 1929
{\centering\footnotesize\rubric{At Holy Communion in connexion with burials or at memorials of the departed, if the Agnus Dei be sung or said for ‘\emph{have mercy upon us}’, and ‘\emph{grant us thy peace}’, substitute ‘\emph{grant them rest}’, and ‘\emph{grant them rest eternal}’.}\par}




\section{The Burial}
\begin{leftbar}
%1928 proposed
\pilcrow{If the ground be not consecrated, the Priest on coming to the grave may say the prayer following.}

\drop{O God, the Father of our Lord Jesus Christ, vouchsafe, we beseech thee, to \grealtcross\ bless this grave to be the peaceful resting-place of the body of thy servant; through the same thy blessed Son, who is the resurrection and the life, and who liveth and reigneth with thee and the Holy Ghost; one God, world without end. \R Amen.}

Or

scottish, 1929
BENEDICTION OF A GRAVE IN UNCONSECRATED GROUND (also south african)

\pilcrow{When the Priest and people shall have come to the place the Priest shall say,}

\centerline{Let us pray.}
\drop{O Lord Jesu Christ, who wast laid in the new tomb of Joseph, and didst thereby sanctify the grave to be a bed of hope to thy people: Vouchsafe, we beseech thee, to \grealtcross\ bless, \grealtcross\ hallow, and \grealtcross\ consecrate this grave, that it may be a resting-place, peaceful and secure, for the body of thy servant which we are about to commit to thy gracious keeping; who art the resurrection and the life, and who livest and reignest with the Father and the Holy Ghost, one God, world without end. \R Amen.}
\end{leftbar}


\pilcrow{When they come to the Grave, while the Corpse is made ready to be laid into the earth, the Priest shall say, or the Priest and Clerks shall sing:}
\drop{Man that is born of a woman hath but a short time to live, and is full of misery. He cometh up, and is cut down, like a flower; he fleeth as it were a shadow, and never continueth in one stay.}

\drop{In the midst of life we are in death: of whom may we seek for succour, but of thee, O Lord, who for our sins art justly displeased?}

Yet, O Lord God most holy, O Lord most mighty, O holy and most merciful Saviour, deliver us not into the bitter pains of eternal death.

Thou knowest, Lord, the secrets of our hearts; shut not thy merciful ears to our prayer; but spare us, Lord most holy, O God most mighty, O holy and merciful Saviour, thou most worthy judge eternal, suffer us not, at our last hour, for any pains of death, to fall from thee.

\medskip

\pilcrow{Then, while the earth shall be cast upon the Body by some standing by, the Priest shall say,}

% 1549
\drop{I commend thy soul to God the Father Almighty, and thy
% 1662
% \drop{Forasmuch as it hath pleased Almighty God of his great mercy to take unto himself the soul of our dear brother here departed, we therefore commit his}
body to the ground; earth to earth, ashes to ashes, dust to dust; in sure and certain hope of the Resurrection to eternal life, through our Lord Jesus Christ; who shall change our vile body, that it may be like unto his glorious body, according to the mighty working, whereby he is able to subdue all things to himself.}

\pilcrow{When this Order is used at the cremation of the body, in place of the words ‘\emph{commit his body to the ground, earth to earth, ashes to ashes, dust to dust}’ shall be said the words ‘\emph{commit his body to be consumed by fire}’: and in this case it shall suffice to say one or more of the \emph{\scshape Prayers} at the burial of the ashes.}

\rubric{When this Order is used at the burial of the body after cremation, in place of the words ‘\emph{commit his body to the ground, earth to earth, ashes to ashes, dust to dust}’ shall be said the words ‘\emph{commit his ashes to the ground, earth to earth, dust to dust}’, or ‘\emph{commit his ashes to their resting-place}’.}

\centerline{\pilcrow{Then shall be said or sung,}}
\drop{I heard a voice from heaven, saying unto me, Write, From henceforth blessed are the dead which die in the Lord: even so saith the Spirit: for they rest from their labours.}


\centerline{Let us pray.}
\drop{We commend into thy hands of mercy, most merciful Father, the soul of this, our \emph{brother} departed, \emph{N.} And \emph{his} body we commit to the earth, beseeching thine infinite goodness to give us grace to live in thy fear and love, and to die in thy favour: that when the judgment shall come, which thou hast committed to thy well-beloved Son, both this our \emph{brother}, and we, may be found acceptable in thy sight, and receive that blessing, which thy well-beloved Son shall then pronounce to all that love and fear thee, saying, Come, ye blessed children of my Father, receive the kingdom prepared for you from the beginning of the world: Grant this, O merciful Father, for the honour of Jesu Christ, our only Savior, Mediator, and Advocate. \R Amen.}

\centerline{\rubric{This Prayer shall also be added.}}
\drop{Almighty God, we give thee hearty thanks for this thy \emph{servant}, whom thou hast delivered from the miseries of this wretched world, from the body of death and all temptation, and, as we trust, hast brought \emph{his} soul, which \emph{he} committed into thy holy hands, into sure consolation and rest: Grant, we beseech thee, that at the day of judgment \emph{his} soul, and all the souls of thy elect, departed out of this life, may with us, and we with them, fully receive thy promises, and be made perfect altogether, through the glorious resurrection of thy Son Jesus Christ our Lord. \R Amen.}

\medskip
1923/1928
\nowUntoTheKing

%scottish
\begin{leftbar}    
    \theGrace
\end{leftbar}

\fleuron

% \centerline{\rule{0.5\textwidth}{0.5pt}}

\chapter[The Burial of a Child]{\stylechapter{An Order which may be used for}{The Burial of a Child}{}}


% \section{An Order which may be Used for the Burial of A [Baptized] Child}
\pilcrow{The Minister and Clerks meeting the body at the entrance of the church-yard, and going before it either into the church or towards the grave, shall say or sing,}

\section{The Procession}
\drop{I am the resurrection and the life, saith the Lord: he that believeth in me, though he were dead, yet shall he live: and whosoever liveth and believeth in me shall never die.\scripture{St.~John xi.~25, 26.}}

\drop{I know that my Redeemer liveth, and that he shalt stand at the latter day upon the earth. 
% And though after my skin worms destroy this body, %KJV
% hath been thus destroyed, %adapted from rsv
% earth: and though this body be destroyed, yet shall I see God: %am1928
% yet in my flesh shall I see God: 
Whom I shall see for myself, and mine eyes shall behold, and not another. \scripture{Job 19.~25, 27.}}

\drop{We brought nothing into this world, and it is certain we can carry nothing out. The {\scshape Lord} gave, and the {\scshape Lord} hath taken away; blessed be the Name of the {\scshape Lord}.\scripture{1 Timothy.~6.~7. Job 1.~21.}}

\drop{He shall feed his flock like a shepherd: he shall gather the lambs with his arm, and carry them in his bosom. \scripture{Isaiah 40.~11}}


\section{The Service in Church}

\centerline{\pilcrow{After they are come into the church shall be read this Psalm,}}

\subsection[{Psalm 23}]{\stylesubsec{Psalm 23.}{Dominus regit me.}{}}
\drop{The {\scshape Lord} is my shepherd; \star\ therefore can I lack nothing.}

2\enspace He shall feed me in a green pasture, \star\ and lead me forth beside the waters of comfort.

3\enspace He shall convert my soul, \star\ and bring me forth in the paths of righteousness, for his Name’s sake.

4\enspace Yea, though I walk through the valley of the shadow of death, I will fear no evil; \star\ for thou art with me; thy rod and thy staff comfort me.

5\enspace Thou shalt prepare a table before me against them that trouble me; \star\ thou hast anointed my head with oil, and my cup shall be full.

6\enspace But thy loving-kindness and mercy shall follow me all the days of my life; \star\ and I will dwell in the house of the Lord for ever.

Glory be to the Father, and to the Son,\ \star\ and to the Holy Ghost;

As it was in the beginning, is now, and ever shall be,\ \star\ world without end. Amen.

\medskip
\centerline{\pilcrow{Then shall follow this Lesson.}}

\centerline{St.~Mark 10.~13.}
\drop{They brought young children to him, that he should touch them: and his disciples rebuked those that brought them. But when Jesus saw it, he was much displeased, and said unto them, Suffer the little children to come unto me, and forbid them not; for of such is the kingdom of God. Verily I say unto you, Whosoever shall not receive the kingdom of God as a little child, he shall not enter therein. And he took them up in his arms, put his hands upon them, and blessed them.}
 
\medskip

\centerline{\pilcrow{Then the Minister shall say,}}
\centerline{Lord, have mercy upon us.}
\centerline{\emph{Christ, have mercy upon us.}}
\centerline{Lord, have mercy upon us.}

\medskip
\ourFather

\smallskip
\centerline{\pilcrow{The following Versicles and Responses may then be said:}}

\V Grant unto \emph{him} eternal rest; \R And let perpetual light shine upon \emph{him}.

\V We believe verily to see the goodness of the Lord; \R In the land of the living.

\V O Lord, hear our prayer; \R And let our cry come unto thee.

\pilcrow{Then shall be said one or both of the following prayers, the Minister first saying,}

\centerline{Let us pray.}
\drop{O Lord Jesu Christ, who didst take little children into thine arms and bless them:

[1928]Open thou our eyes, we beseech thee, to perceive that it is of thy goodness that thou hast taken this thy

[SA]Grant that in perfect confidence we may commit this 

child into the everlasting arms of thine infinite love; who livest and reignest with the Father and the Holy Spirit, ever one God, world without end. \R Amen.}

\smallskip
\drop{O God, whose ways are hidden and thy works most wonderful, who makest nothing in vain and lovest all that thou hast made: Comfort thou thy servants, whose hearts are sore smitten and oppressed; and grant that they may so love and serve thee in this life, that together with this thy child they may obtain the fulness of thy promises in the world to come; through Jesus Christ our Lord. \R Amen.}

\medskip
\theGrace
 
\section{The Burial}
\pilcrow{When they come to the grave, while the body is made ready to be laid into the earth, the Minister shall say, or the Minister and Clerks shall sing:}

\drop{Man that is born of a woman hath but a short time to live. He cometh up, and is cut down, like a flower; he fleeth as it were a shadow, and never continueth in one stay.}

% (these from SA)
\begin{leftbar}
While the child was yet alive I fasted and wept: for I said, Who can tell whether God will be gracious to me, that the child may live? But now he is dead, wherefore should I fast? Can I bring him back again? I shall go to him, but he shall not return to me.

A voice was heard in Ramah, lamentation, and bitter weeping; Rachel weeping for her children refused to be comforted for her children, because they were not.  Thus saith the Lord; Refrain thy voice from weeping, and thine eyes from tears: for thy work shall be rewarded, saith the Lord; and they shall come again from the land of the enemy.  And there is hope in thine end, saith the Lord, that thy children shall come again to their own border
\end{leftbar}


\medskip

\pilcrow{Then, while the earth shall be cast upon the body by some standing by, the Minister shall say,}
\drop{Forasmuch as it hath pleased Almighty God of his great mercy to take unto himself the soul of this child here departed, we therefore commit \emph{his} body to the ground; earth to earth, ashes to ashes, dust to dust; in sure and certain hope of the resurrection to eternal life, through our Lord Jesus Christ; who shall change the body of our low estate, that it may be like unto his glorious body, according to the mighty working, whereby he is able to subdue all things to himself.}

\centerline{\rubric{Or this.}}
\drop{We commend unto thy hands of mercy, most merciful Father, the soul of this thy child; and we commit \emph{his} body to the ground; earth to earth, ashes to ashes, dust to dust.  And we beseech thine infinite goodness to give us grace to live in thy fear and love, and to die in thy favour, that when the judgement shall come which thou hast comitted to thy well-beloved Son, both this child and we may be found acceptable in thy sight.  Grant this, O merciful Father, for the sake of Jesus Christ, our only Saviour, Mediator, and Advocate. \R Amen.}


\smallskip
\centerline{\pilcrow{Then shall be said or sung,}}

\drop{They shall hunger no more, neither thirst any more; neither shall the sun light on them, nor any heat. For the Lamb which is in the midst of the throne shall feed them, and shall lead them unto living fountains of waters: and God shall wipe away all tears from their eyes.}

\centerline{\pilcrow{Here shall be added by the Minister,}}

\nowUntoTheKing



\centerline{\rule{0.5\textwidth}{0.5pt}}
American 1928:
\drop{It is to be noted that this Office is appropriate to be used only for the faithful departed in Christ, provided that in any other case the Minister may, at his discretion, use such part of this Office, or such devotions taken from other parts of this Book, as may be fitting.}

 
\section{An Order which may be Used when the Prayer Book Service may not Be Used}
\pilcrow{The Priest, meeting the Corpse at the entrance to the Church-yard, and going before it towards the Grave, shall say,}

\pilcrow{The Minister shall await the body at the grave. When the body has been brought to the graveside, he shall say:}
 (Indian)
\subsection[{Psalm 130}]{\stylesubsec{Psalm 130.}{De profundis.}{}}

\drop{Out of the deep have I called unto thee, O {\scshape Lord}; \star\ Lord, hear my voice.}

2\enspace O let thine ears consider well \star\ the voice of my complaint.

3\enspace If thou, {\scshape Lord}, wilt be extreme to mark what is done amiss, \star\ O Lord, who may abide it?

4\enspace For there is mercy with thee; \star\ therefore shalt thou be feared.

5\enspace I look for the {\scshape Lord}; my soul doth wait for him; \star\ in his word is my trust.

6\enspace My soul fleeth unto the Lord \star\ before the morning watch, I say, before the morning watch.

7\enspace O Israel, trust in the {\scshape Lord}, for with the {\scshape Lord} there is mercy, \star\ and with him is plenteous redemption.

8\enspace And he shall redeem Israel \star\ from all his sins.

\medskip
\centerline{\pilcrow{When they come to the Grave shall be said,}}

% Domine, refugium. Psalm 90. 1-12.

% LORD, thou hast been our refuge : from one generation to another.
%     Before the mountains were brought forth, or ever the earth and the world were made : thou art God from everlasting, and world without end.
%     Thou turnest man to destruction : again thou sayest, Come again, ye children of men.
%     For a thousand years in thy sight are but as yesterday: seeing that is past as a watch in the night.
%     As soon as thou scatterest them, they are even as a sleep: and fade away suddenly like the grass.
%     In the morning it is green, and groweth up : but in the evening it is cut down, dried up, and withered.
%     For we consume away in thy displeasure: and are afraid at thy wrathful indignation.
%     Thou hast set our misdeeds before thee : and our secret sins in the light of thy countenance.
%     For when thou art angry all our days are gone : we bring our years to an end, as it were a tale that is told.
%     The days of our age are threescore years and ten; and though men be so strong, that they come to fourscore years : yet is their strength then but labour and sorrow; so soon passeth it away, and we are gone.
%     But who regardeth the power of thy wrath : for even thereafter as a man feareth, so is thy displeasure.
%     So teach us to number our days : that we may apply our hearts unto wisdom.

%     SA and Indian skip here.

\centerline{\pilcrow{Then may be read,}}
\centerline{St.~John 5.~25.}
\drop{Jesus said, Verily, verily, I say unto you, The hour is coming, and now is, when the dead shall hear the voice of the Son of God: and they that hear shall live. For as the Father hath life in himself; so hath he given to the Son to have life in himself; and hath given him authority to execute judgement also, because he is the Son of man. Marvel not at this: for the hour is coming, in the which all that are in the graves shall hear his voice, and shall come forth; they that hath done good, unto the resurrection of life; and they that hath done evil, unto the resurrection of judgement.}

% SA, Indian: Anthems here.
\drop{Man that is born of a woman hath but a short time to live, and is full of misery. He cometh up, and is cut down, like a flower; he fleeth as it were a shadow, and never continueth in one stay.}

\drop{In the midst of life we are in death: of whom may we seek for succour, but of thee, O Lord, who for our sins art justly displeased?}

Yet, O Lord God most holy, O Lord most mighty, O holy and most merciful Saviour, deliver us not into the bitter pains of eternal death.

Thou knowest, Lord, the secrets of our hearts; shut not thy merciful ears to our prayer; but spare us, Lord most holy, O God most mighty, O holy and merciful Saviour, thou most worthy judge eternal, suffer us not, at our last hour, for any pains of death, to fall from thee.

\medskip
\pilcrow{When the Corpse has been laid in the Grave the Priest shall say,}
SA, : 
\drop{We commit the body of our dear \emph{brother} to the ground; earth to earth, ashes to ashes, dust to dust; and we commend \emph{his} soul to the just and merciful judgement of him who alone hath perfect understanding, even Jesus Christ our Lord. \R Amen.}



\centerline{\rubric{Then the Minister shall say,}}
\centerline{Lord, have mercy upon us.}
\centerline{\emph{Christ, have mercy upon us.}}
\centerline{Lord, have mercy upon us.}

\medskip
\ourFather


Ant. Remember not, Lord, our offenses.. (from litany)

\V O Lord, deal not with us after our sins. \R Neither reward us after our iniquities.

\drop{We humbly beseech thee, O Father, mercifully to look upon our infirmities; and for the glory of thy Name turn from us all those evils that we most righteously have deserved; and grant, that in all our troubles we may put our whole trust and confidence in thy mercy, and evermore serve thee in holiness and pureness of living, to thy honour and glory; through our only Media tor and Advocate, Jesus Christ our Lord. Amen.}


Collect: Almighty God, give us grace that we may

\drop{Almighty God, the fountain of all wisdom, who knowest our necessities before we ask, and our ignorance in asking: We beseech thee to have compassion upon our infirmities; and those things, which for our unworthiness we dare not, and for our blindness we cannot ask, vouchsafe to give us, for the worthiness of thy Son Jesus Christ our Lord. \R Amen.}

SA: 
\drop{Almighty God, Father of all mercies and giver of all comfort: Deal graciously, we pray thee, with those who mourn, that casting every care on thee, they may know the consolation of thy love; through Jesus Christ our Lord. \R Amen.}

but indian: 
\drop{O God, whose ways are hidden and thy works past understanding, who makest nothing in vain and lovest all that thou hast made: Deal graciously, we pray thee, with those who mourn because of this bereavement, that casting every care on thee, they may know the consolation of thy love; through Jesus Christ our Saviour. \R Amen.}

Indian: Fountain of all wisdom
Sa: \drop{O Saviour of the world, who by thy Cross and precious Blood hast redeemed us, Save us and help us, we humbly beseech thee, O Lord.}

\medskip
\centerline{\rubric{And then shall be said,}}
\theGrace

 

 
\centerline{\rule{0.5\textwidth}{0.5pt}}

 
\section{An Order for the Burial of an Unbaptized Child}
\pilcrow{On the way to the Grave the following sentences may be said,}

\drop{God made not death: neither delighteth he when the living perish. He created man for incorruption: and made him an image of his own proper being.\scripture{Wisdom of Solomon 1.~13, 2.~23}}

\drop{Despair not then, seeing that thou art far from being able to love his creature more than he. For as his majesty is, so also is his mercy. \scripture{2 Esdras 8.~47; Ecclesiasticus 2.~18}}

\drop{He shall feed his flock like a shepherd, he shall gather the lambs with his arm, and carry them in his bosom. \scripture{Isaiah 40.~11}}

\drop{The Lord gave, and the Lord hath taken away; blessed be the name of the Lord.\scripture{Job 1.~21}}
(SA omits first)
(Indian is only the 4th and 3rd)

\medskip
\centerline{\pilcrow{When they come to the Grave shall be said,}}
% SA: Psalm 121


\subsection[{Psalm 23}]{\stylesubsec{Psalm 23.}{Dominus regit me.}{}}
\drop{The {\scshape Lord} is my shepherd; \star\ therefore can I lack nothing.}

2\enspace He shall feed me in a green pasture, \star\ and lead me forth beside the waters of comfort.

3\enspace He shall convert my soul, \star\ and bring me forth in the paths of righteousness, for his Name’s sake.

4\enspace Yea, though I walk through the valley of the shadow of death, I will fear no evil; \star\ for thou art with me; thy rod and thy staff comfort me.

5\enspace Thou shalt prepare a table before me against them that trouble me; \star\ thou hast anointed my head with oil, and my cup shall be full.

6\enspace But thy loving-kindness and mercy shall follow me all the days of my life; \star\ and I will dwell in the house of the Lord for ever.

(SA: Psalm 121)

\medskip
\centerline{\pilcrow{Then may be read one of the following,}}

Indian: Jeremiah 31:15
Thus saith the Lord: A voice is heard in Ramah...again to their own border.
Or baruch.


\centerline{St.~Matthew 18.~10.}
\drop{Take heed that ye despise not one of these little ones; for I say unto you, That in heaven their angels do always behold the face of my Father which is in heaven. For the Son of man is come to save that which was lost. How think ye? if a man have an hundred sheep, and one of them be gone astray, doth he not leave the ninety and nine, and goeth into the mountains, and seeketh that which is gone astray? And if so be that he find it, verily I say unto you, he rejoiceth more of that sheep, than of the ninety and nine which went not astray. Even so it is not the will of your Father which is in heaven, that one of these little ones should perish.}


\centerline{Baruch 4.~19.}
\drop{Go your way, my children, go your way: for I am left desolate. I have put off the clothing of peace, and put upon me the sack-cloth of my prayer: I will cry unto the Everlasting in my days. Be of good cheer, my children, cry unto the Lord, and he shall deliver you from the power and hand of the enemies. For my hope is in the Everlasting, that he will save you; and joy is come unto me from the Holy One, because of the mercy which shall soon come unto you from the Everlasting our Saviour. For I sent you out with mourning and weeping: but God will give you to me again with joy and gladness for ever.}

\medskip
\centerline{\pilcrow{As the body is being laid in the Grave shall be said,}}
\drop{Unto God’s loving mercy we commit this child, that he may grant \emph{him} a share in the unsearchable riches of the redemption wrought by his Son, our Lord and Saviour Jesus Christ. \R Amen.}


\medskip

\centerline{\rubric{Then the Minister shall say,}}
\centerline{Lord, have mercy upon us.}
\centerline{\emph{Christ, have mercy upon us.}}
\centerline{Lord, have mercy upon us.}

\medskip
\ourFather

\medskip

\centerline{Let us pray.}
\drop{O God, whose ways are hidden and thy works most wonderful, who makest nothing in vain and lovest all that thou hast made: Comfort thou thy servants, whose hearts are sore smitten and oppressed; and grant that they may so love and serve thee in this life, that together with this thy child they may obtain the fulness of thy promises in the world to come; through Jesus Christ our Lord. \R Amen.}

Indian: Numerous collects.
\medskip
\centerline{\rubric{And then shall be said,}}
\theGrace

\chapter[The Churching of Women]{\stylechapter{The Thanksgiving of Women after Child-Birth\\ {\small commonly called}}{The Churching of Women}{}}

\pilcrow{The woman, at the usual time after her delivery, shall come into the Church decently apparelled, and there shall kneel down in some convenient place, as hath been accustomed:%, [or as the Ordinary shall direct]: 
And then the Minister shall say unto her,}


\drop{Forasmuch as it hath pleased Almighty God of his goodness to give you safe deliverance, and hath preserved you in the great danger of child-birth; you shall therefore give hearty thanks unto God, and say,}


\bigskip
% 1549: Levavi oculos. Psalm cxxi., 1559

% 1928 American
\pilcrow{Then shall be said by both of them the following Psalm, the Woman still kneeling.}

%The following psalm has some verses changed from the psalter.
%(Then shall the Priest say the 116th Psalm.)

\subsection[{Psalm 116}]{\stylesubsec{Psalm 116.}{Dilexi, quoniam.}{}}
\drop{I am wéll pléased \star\  that the {\scshape Lord} hath heard the  vóice of mý prayer;}

2\enspace That he hath inclined his éar  unto mé; \star\  therefore will I call upon him as  lóng as Í live.

3\enspace The snares of death compassed me  róund abóut, \star\  and the pains of hell gat  hóld upón me.

4\enspace I \emph{found} trouble and heaviness, and I \emph{called} upon the Náme  of the {\scshape Lórd}; \star\  O {\scshape Lord}, I beseech thee, delíver mý soul.

5\enspace Gracious is the  {\scshape Lórd}, and ríghteous; \star\ yea, our Gód is mérciful.

6\enspace The {\scshape Lord} presérveth the símple: \star\ I was in misery, ánd he hélped me.

7\enspace Turn again then unto thy rést,  O my sóul; \star\ for the {\scshape Lórd} hath rewárded thee.

8\enspace And why? thou hast delivered my  sóul from déath, \star\ mine eyes from tears, and my féet from fálling.

9\enspace I will wálk before the {\scshape Lórd} \star\  in the lánd  of the líving.

10\enspace I believed, and therefore will I speak; but Í  was sore tróubled: \star\  I said in my haste, All mén are líars.

11\enspace What reward shall I gíve unto the {\scshape Lórd} \star\  for all the benefits that hé hath done únto me?

12\enspace I will receive the cúp of salvátion, \star\  and call upon the Náme of thé {\scshape Lord}.

13\enspace I will pay my vows now in the presence of áll his péople: \star\   in the courts of the {\scshape Lord}’s house, even in the midst of thee, O Jerúsalem. Práise the {\scshape Lord}.

Glory be to the Father, and to the Son, \star\  and to the Holy Ghost;

As it was in the beginning, is now, and ever shall be, \star\  world without end. Amen.

\medskip
\centerline{\rubric{Or,}}
\subsection[{Psalm 127}]{\stylesubsec{Psalm 127.}{Nisi Dominus.}{}}
\drop{Except the {\scshape Lord} búild the hóuse,  \star\  their labour is but lóst that búild it.}

2\enspace Except the {\scshape Lord} kéep the cíty,  \star\  the watchman wáketh bút in vain.

3\enspace It is but lost labour that ye haste to rise up early, and so late take rest, and eat the bréad of cárefulness;  \star\  for so he giveth hís belóved sleep.

4\enspace Lo, children and the frúit of the wómb,  \star\  are an heritage and gift that cómeth óf the {\scshape Lord}.

5\enspace Like as the arrows in the hánd of the gíant, * even so are the yöung chíldren.

6\enspace Happy is the man that hath his quíver fúll of them; * they shall not be ashamed when they speak with their én·emies ín the gate.

Glory be to the Father, and to the Son, \star\  and to the Holy Ghost;

As it was in the beginning, is now, and ever shall be, \star\  world without end. Amen.

\medskip
\centerline{\pilcrow{Then the Minister shall say,}}


\centerline{Lord, have mercy upon us.}
\centerline{\emph{Christ, have mercy upon us.}}
\centerline{Lord, have mercy upon us.}

\smallskip
\ourFather

\V O Lord, save this woman thy servant; \R Who putteth her trust in thee.

\V Be thou to her a strong tower;  \R From the face of her enemy.

\V Lord, hear our prayer.  \R And let our cry come unto thee.

\centerline{Let us pray.}
\drop{O Almighty God, we give thee humble thanks for that thou hast vouchsafed to deliver this woman thy servant from the great pain and peril of child-birth; Grant, we beseech thee, most merciful Father, that she, through thy help, may both faithfully live, and walk according to thy will, in this life present; and also may be partaker of everlasting glory in the life to come; through Jesus Christ our Lord. \R Amen.}

% 1923 adds only the following; 1928 adds the following two.
\bigskip
\pilcrow{Then, if there be no Communion at the time of the Churching, shall the Minister say to the Woman,}
\drop{Unto God’s gracious mercy and protection we commit thee. The {\scshape Lord} \cross bless thee, and keep thee. The {\scshape Lord} make his face to shine upon thee, and be gracious unto thee. The {\scshape Lord} lift up his countenance upon thee, and give thee peace, both now and evermore. \R Amen.}




\bigskip
\pilcrow{Prayers which may be used at the discretion of the Priest before the Blessing.}
\drop{O God, our heavenly Father, we thank thee and praise thy glorious name, that thou hast been pleased to bless this woman thy servant, and to bestow upon her the gift of a child: Grant, we beseech thee, most merciful Father, that she [with her husband] may diligently lead her child in the way of righteousness, to their own great blessing and the glory of thy name; through Jesus Christ our Lord. \R Amen.}

\medskip
\drop{O God, whose ways are hidden and thy works most wonderful, who makest nothing in vain, and lovest all that thou hast made: Comfort this thy servant whose heart is sore sitten and oppressed; and grant that she may so love and serve thee in this life, that she may obtain the fulness of thy promises in the world to come; through Jesus Christ our Lord. \R Amen.}

\medskip

\pilcrow{The woman, that cometh to give her thanks, must offer accustomed offerings; and, if there be a Communion, it is convenient that she receive the Holy Communion.}

\medskip
\pilcrow{This Service may also be used by any after an event or procedure in which that person’s survival was in question.}
\medskip

\fleuron

\chapter[A Commination]{\stylechapter{}{A Commination}{or Denouncing of God’s Anger and Judgements against Sinners\\ }}
{\centering\footnotesize\emph{With certain Prayers, to be used on the first Day of Lent, and at other times, as the Ordinary shall appoint}\par}


% "General Sentence" - warning of excommunication, 4 times a year.



\section{The Commination}
\medskip
\pilcrow{After Morning Prayer, the Litany ended according to the accustomed manner, the Priest shall, in the Reading-Pew or Pulpit, say,}
\drop{Brethren, in the Primitive Church there was a godly discipline, that, at the beginning of Lent, such persons as stood convicted of notorious sin were put to open penance, and punished in this world, that their souls might be saved in the day of the Lord; and that others, admonished by their example, might be the more afraid to offend.}

% Instead whereof, until the said discipline may be restored again, (which is much to be wished,) it is thought good, that at this time (in the presence of you all) should be read the general sentences of God’s cursing against impenitent sinners, gathered out of the seven and twentieth Chapter of Deuteronomy, and other places of Scripture; and that ye should answer to every Sentence, \emph{Amen:} To the intent that, being admonished of the great indignation of God against sinners, ye may the rather be moved to earnest and true repentance; and may walk more warily in these dangerous days; fleeing from such vices, for which ye affirm with your own mouths the curse of God to be due.

% \drop{Cursed is the man that maketh any carved or molten image, to worship it.}

% \centerline{\rubric{And the people shall answer and say,}  Amen.}

% Cursed is he that curseth his father or mother. \R Amen.

% Cursed is he that removeth his neighbour’s landmark. \R Amen.

% Cursed is he that maketh the blind to go out of his way. \R Amen.

% Cursed is he that perverteth the judgement of the stranger, the fatherless, and widow. \R Amen.

% Cursed is he that smiteth his neighbour secretly. \R Amen.

% Cursed is he that lieth with his neighbour’s wife. \R Amen.

% Cursed is he that taketh reward to slay the innocent. \R Amen.

% Cursed is he that putteth his trust in man, and taketh man for his defence, and in his heart goeth from the Lord. \R Amen.

% Cursed are the unmerciful, fornicators, and adulterers, covetous persons, idolaters, slanderers, drunkards, and extortioners. \R Amen.
\begin{multicols}{2}{
Wherefore, lest by disuse of the said discipline God’s judgement upon sin be lightly regarded, 

Instead whereof, until the said discipline may be restored again, (which is much to be wished,)
}\end{multicols}
it is thought good that at this time (in the presence of you all)

\begin{multicols}{2}{
it should be declared that God will surely judge them that transgress his holy Commandments; and that ye, imploring his mercy, should answer \emph{Amen} in token that ye assent and submit to his righteous condemnation:

\medskip

should be read the general sentences of God’s cursing against impenitent sinners, gathered out of the seven and twentieth Chapter of Deuteronomy, and other places of Scripture; and that ye should answer to every Sentence, \emph{Amen:}
}\end{multicols}

To the intent that being admonished of the great indignation of God against sinners, ye may the rather be moved to earnest and true repentance; and may walk more warily in these dangerous days; fleeing from such vices for which ye affirm with your own mouths the judgement[curse] of God to be due.

\drop{The Lord our God is one Lord: them that serve other gods, God shall judge.}

\centerline{\pilcrow{And the people shall answer and say,}}

\R Amen.  Lord have mercy upon us.

Idolaters and all them that worship God’s creatures, God shall judge;

\R Amen.  Lord have mercy upon us.

Blasphemers and all them that take God’s name in vain, God shall judge; 

\R Amen.  Lord have mercy upon us.

The Lord’s day is holy; them that profane it, God shall judge;

\R Amen.  Lord have mercy upon us.

Him that honoureth not his father or his mother, and them that are lawless or seditious, God shall judge;

\R Amen.  Lord have mercy upon us.

Murderers and all them that are malicious or cruel, God shall judge;

\R Amen.  Lord have mercy upon us.

Adulterers and fornicators and all unclean persons, God shall judge;

\R Amen.  Lord have mercy upon us.

Robbers and thieves and them that defraud, God shall judge;

\R Amen.  Lord have mercy upon us.

False witnesses and all evil speakers, liars and slanderers, God shall judge;

\R Amen.  Lord have mercy upon us.

Covetous persons and extortioners and them that grind the faces of the poor, God shall judge;

\R Amen. Lord, have mercy upon us, and lay not these sins to our charge.


\centerline{\rubric{Minister.}}

% \drop{Now seeing that all they are accursed (as the prophet David beareth witness) who do err and go astray from the commandments of God; let us (remembering the dreadful judgement hanging over our heads, and always ready to fall upon us) return unto our Lord God, with all contrition and meekness of heart; bewailing and lamenting our sinful life, acknowledging and confessing our offences, and seeking to bring forth worthy fruits of penance. For now is the axe put unto the root of the trees, so that every tree that bringeth not forth good fruit is hewn down, and cast into the fire. It is a fearful thing to fall into the hands of the living God: he shall pour down rain upon the sinners, snares, fire and brimstone, storm and tempest; this shall be their portion to drink. For lo, the Lord is come out of his place to visit the wickedness of such as dwell upon the earth. But who may abide the day of his coming? Who shall be able to endure when he appeareth? His fan is in his hand, and he will purge his floor, and gather his wheat into the bam; but he will burn the chaff with unquenchable fire. The day of the Lord cometh as a thief in the night: and when men shall say, Peace, and all things are safe, then shall sudden destruction come upon them, as sorrow cometh upon a woman travailing with child, and they shall not escape. Then shall appear the wrath of God in the day of vengeance, which obstinate sinners, through the stubbornness of their heart, have heaped unto them, selves; which despised the goodness, patience, and long, sufferance of God, when he calleth them continually to repentance. Then shall they call upon me, (saith the Lord,) but I will not hear; they shall seek me early, but they shall not find me; and that, because they hated knowledge, and received not the fear of the Lord, but abhorred my counsel, and despised my correction. Then shall it be too late to knock when the door shall be shut; and too late to cry for mercy when it is the time of justice. O terrible voice of most just judgement, which shall be pronounced upon them, when it shall be said unto them, Go, ye cursed, into the fire everlasting, which is prepared for the devil and his angels. Therefore, brethren, take we heed betime, while the day of salvation lasteth; for the night cometh, when none can work. But let us, while we have the light, believe in the light, and walk as children of the light; that we be not cast into utter darkness, where is weeping and gnashing of teeth. Let us not abuse the goodness of God, who calleth us mercifully to amendment, and of his endless pity promiseth us forgiveness of that which is past, if with a perfect and true heart we return unto him. For though our sins be as red as scarlet, they shall be made white as snow; and though they be like purple, yet they shall be made white as wool. Turn ye (saith the Lord) from all your wickedness, and your sin shall not be your destruction: Cast away from you all your ungodliness that ye have done: Make you new hearts, and a new spirit: Wherefore will ye die, O ye house of Israel, seeing that I have no pleasure in the death of him that dieth, saith the Lord God? Tom ye then, and ye shall live. Although we have sinned, yet have we an Advocate with the Father, Jesus Christ the righteous; and he is the propitiation for our sins. For he was wounded for our offences, and smitten for our wickedness. Let us therefore return unto him, who is the merciful receiver of all true penitent sinners; assuring ourselves that he is ready to receive us, and most willing to pardon us, if we come unto him with faithful repentance; if we submit ourselves unto him, and from henceforth walk in his ways; if we will take his easy yoke, and light burden upon us, to follow him in lowliness, patience, and charity, and be ordered by the governance of his Holy Spirit; seeking always his glory, and serving him duly in our vocation with thanksgiving: This if we do, Christ will deliver us from the curse of the law, and from the extreme malediction which shall light upon them that shall be set on the left hand; and he will set us on his right hand, and give us the gracious benediction of his Father, commanding us to take possession of his glorious kingdom: Unto which he vouchsafe to bring us all, for his infinite mercy. Amen.}

% \centerline{Here shall follow The Prayers}
%from the proposed 1928:
% \section{An Alternative Commination}
% 


\drop{Now seeing that all they 
are condemned %are accursed (as the prophet David beareth witness)
who do err and go astray from the commandments of God; let us (remembering the dreadful judgement hanging over our heads, and always ready to fall upon us) return unto our Lord God, with all contrition and meekness of heart; bewailing and lamenting our sinful life, acknowledging and confessing our offences, and seeking to bring forth worthy fruits of penance.
For %now is the axe put unto the root of the trees, so that every tree that bringeth not forth good fruit is hewn down, and cast into the fire. 
it is a fearful thing to fall into the hands of the living God%: he shall pour down rain upon the sinners, snares, fire and brimstone, storm and tempest; this shall be their portion to drink. For lo, the Lord is come out of his place to visit the wickedness of such as dwell upon the earth. But who may abide the day of his coming? Who shall be able to endure when he appeareth? His fan is in his hand, and he will purge his floor, and gather his wheat into the bam; but he will burn the chaff with unquenchable fire. The day of the Lord cometh as a thief in the night: and when men shall say, Peace, and all things are safe, then shall sudden destruction come upon them, as sorrow cometh upon a woman travailing with child, and they shall not escape. Then shall appear the wrath of God in the day of vengeance, which obstinate sinners, through the stubbornness of their heart, have heaped unto them, selves; which despised the goodness, patience, and long, sufferance of God, when he calleth them continually to repentance. Then shall they call upon me, (saith the Lord,) but I will not hear; they shall seek me early, but they shall not find me; and that, because they hated knowledge, and received not the fear of the Lord, but abhorred my counsel, and despised my correction. Then shall it be too late to knock when the door shall be shut; and too late to cry for mercy when it is the time of justice. O 
and to hear 
the terrible voice of his most just judgement, which shall be pronounced upon 
obstinate sinners %them,
when it shall be said unto them, Go, ye cursed, into the fire everlasting, which is prepared for the devil and his angels. Therefore, brethren, take we heed betime, while the day of salvation lasteth. %; for the night cometh, when none can work. But let us, while we have the light, believe in the light, and walk as children of the light; that we be not cast into utter darkness, where is weeping and gnashing of teeth. Let us not abuse the goodness of God, who calleth us mercifully to amendment, and of his endless pity promiseth us forgiveness of that which is past, if with a perfect and true heart we return unto him. For though our sins be as red as scarlet, they shall be made white as snow; and though they be like purple, yet they shall be made white as wool. Turn ye (saith the Lord) from all your wickedness, and your sin shall not be your destruction: Cast away from you all your ungodliness that ye have done: Make you new hearts, and a new spirit: Wherefore will ye die, O ye house of Israel, seeing that I have no pleasure in the death of him that dieth, saith the Lord God? Tom ye then, and ye shall live. 
Although we have sinned, yet have we an Advocate with the Father, Jesus Christ the righteous; and he is the propitiation for our sins. For he was wounded for our offences, and smitten for our wickedness. Let us therefore return unto him, who is the merciful receiver of all true penitent sinners; assuring ourselves that he is ready to receive us, and most willing to pardon us, if we come unto him with faithful repentance; if we submit ourselves unto him, and from henceforth walk in his ways; if we will take his easy yoke, and light burden upon us, to follow him in lowliness, patience, and charity, and be ordered by the governance of his Holy Spirit; seeking always his glory, and serving him duly in our vocation with thanksgiving: This if we do, Christ will deliver us from %the curse of the law, and from 
the extreme malediction which shall light upon them that shall be set on the left hand; and he will set us on his right hand, and give us the gracious benediction of his Father, commanding us to take possession of his glorious kingdom: Unto which he vouchsafe to bring us all, for his infinite mercy. Amen.}


\medskip

\section{The Prayers}
\pilcrow{Then shall they all kneel upon their knees, and the Priest and Clerks kneeling (in the place where they are accustomed to say the Litany) shall say this Psalm.}

\subsection{\stylesubsec{Psalm 51.}{Miserere mei, Deus.}{}}
\drop{Have mercy upon me, O God, after thý great goodness;\ \star\ according to the multitude of thy mercies do away mine offences.}

2\enspace Wash me throughly from my wickedness,\ \star\ and cleanse me from my sin.

3\enspace For I knowledge my faults,\ \star\ and my sin is ever before me.

4\enspace Against thee only have I sinned, and done this evil in thy sight;\ \star\ that thou mightest be justified in thy saying, and clear when thou art júdged.

5\enspace Behold, I was shapen in wickedness,\ \star\ and in sin hath my mother conceived me.

6\enspace But lo, thou requirest truth in the inward parts,\ \star\ and shalt make me to understand wisdom secretly.

7\enspace Thou shalt purge me with hyssop, and i shall be clean;\ \star\ thou shalt wash me, and I shall be whiter than snow.

8\enspace Thou shalt make me hear of joy and gladness,\ \star\ that the bones which thou hast broken may rejoice.

9\enspace Turn thy face from my sins,\ \star\ and put out all my misdeeds.

10\enspace Make me a clean heart, O God,\ \star\ and renew a right spirit within me.

11\enspace Cast me not away from thy presence,\ \star\ and take not thy holy Spirit from me.

12\enspace O give me the comfort of thy help again,\ \star\ and stablish me with thý free Spirit.

13\enspace Then shall I teach thy ways únto the wicked,\ \star\ and sinners shall be converted únto thee.

14\enspace Deliver me from blood-guiltiness, O God, thou that art the God of my health;\ \star\ and my tongue shall sing of thy righteousness.

15\enspace Thou shalt open my lips, O Lord,\ \star\ and my mouth shall shew thy praise.

16\enspace For thou desirest no sacrifice, else would I give it thee;\ \star\ but thou delightest not in burnt-offerings.

17\enspace The sacrifice of God is a troubled spirit:\ \star\ a broken and contrite heart, O God, shalt thou not despise.

18\enspace O be favourable and gracious únto Sion;\ \star\ build thou the walls of Hierúsalem.

19\enspace Then shalt thou be pleased with the sacrifice of righteousness, with the burnt-offerings and oblations;\ \star\ then shall they offer young bullocks upon thine altar.

Glory be to the Father, and to the Son,\ \star\ and to the Holy Ghost;

As it was in the beginning, is now, and ever shall be,\ \star\ world without end. Amen.

\medskip

\centerline{Lord, have mercy upon us.}
\centerline{\emph{Christ, have mercy upon us.}}
\centerline{Lord, have mercy upon us.}

\smallskip
\ourFather


\V O Lord, save thy servants; \R That put their trust in thee.

\V Send unto them help from above; \R And evermore mightily defend them.

\V Help us, O God our Saviour. \R And for the glory of thy Name deliver us; be merciful to us sinners, for thy name’s sake.

\V O Lord, hear our prayer; \R And let our cry come unto thee.

\centerline{Let us pray.}
\drop{O Lord, we beseech thee, mercifully hear our prayers, and spare all those who confess their sins unto thee; that they, whose consciences by sin are accused, by thy merciful pardon may be absolved; through Christ our Lord. \R Amen.}

\smallskip

\drop{O most mighty God, and merciful Father, who hast compassion upon all men, and hatest nothing that thou hast made; who wouldest not the death of a sinner, but that he should rather turn from his sin, and be saved: Mercifully forgive us our trespasses; receive and comfort us, who are grieved and wearied with the burden of our sins. Thy property is always to have mercy; to thee only it appertaineth to forgive sins. Spare us therefore, good Lord, spare thy people, whom thou hast redeemed; enter not into judgement with thy servants, who are vile earth, and miserable sinners; but so turn thine anger from us, who meekly acknowledge our vileness, and truly repent us of our faults, and so make haste to help us in this world, that we may ever live with thee in the world to come; through Jesus Christ our Lord. \R Amen.}


\medskip

\pilcrow{Then shall the people say this Anthem that followeth, after the Minister.}
\drop{Turn thou us, O good Lord, and so shall we be turned. Be favourable, O Lord, Be favourable to thy people, Who turn to thee in weeping, fasting, and praying. For thou art a merciful God, Full of compassion, long-suffering, and of great pity. Thou sparest when we deserve punishment, And in thy wrath thinkest upon mercy. Spare thy people, good Lord, spare them, And let not thine heritage be brought to confusion. Hear us, O Lord, for thy mercy is great, And after the multitude of thy mercies look upon us; Through the merits and mediation of thy blessed Son, Jesus Christ our Lord. Amen.}

\medskip

\centerline{\pilcrow{Then the Minister alone shall say,}}
\drop{The {\scshape Lord} bless us, and keep us: the {\scshape Lord} make his face to shine upon us, and be gracious unto us: the {\scshape Lord} lift up the light of his countenance upon us, and give us peace, \grecross\ now and for evermore. \R Amen.}

\fleuron
% \chapter{Blessings}

\section{The order for the Conjuring of Water}

\smallskip
\pilcrow{The Priest prepares the salt as follows,}

\drop{I exorcize thee, creature of salt, by the living \grealtcross\ God, by the holy \grealtcross\ God, by the omnipotent \grealtcross\  God, that thou mayest be purified from all evil influence, in the Name of Him who is Lord of Angels and of men, who filleth the whole earth with his majesty and glory.  \R Amen.}

\drop{We pray thee, O God, in thy boundless lovingkindness to stretch forth the right hand of thy power upon this creature of salt which we \grealtcross\ bless and \grealtcross\ hallow in thy holy Name.  Grant that this salt may make for health of mind and body to all who partake thereof, and that there may be banished from the place where it is used every power of adversity and every illusion or artifice of evil; through Christ our Lord.  \R Amen.}

\smallskip

\pilcrow{The Priest prepares the water as follows,}

\drop{I exorcize thee, creature of water, by the living \grealtcross\ God, by the holy \grealtcross\ God, by the omnipotent \grealtcross\  God, that thou mayest be purified from all evil influence, in the Name of Him who is Lord of Angels and of men, who filleth the whole earth with his majesty and glory.  \R Amen.}

\drop{O God, who for the helping and safeguarding of men dost hallow the water set apart for the service of thy holy Church, send forth thy light and thy power upon this element of water which we \grealtcross\ bless and \grealtcross\ hallow in thy holy Name.  Grant that whosoever uses this water in faithfulness of spirit may be strengthened in all goodness, and that everything sprinkled with it may be made holy and pure and guarded from all assaults of evil; through Christ our Lord.  \R Amen.}

\smallskip

\pilcrow{The Priest casts the salt thrice into the water crosswise, as he says the following,}
\drop{Let salt and water mingle together in the Name of the \grealtcross\ Father, and of the \grealtcross\  Son, and of the Holy \grealtcross\ Ghost.  \R Amen.}


\smallskip

\V The Lord be with you.  \R And with thy spirit.

\drop{O God, the giver of invincible strength and King of irresistible power, whose splendour shines throughout the whole of creation: We pray thee to look upon this thy creature of salt and water, to pour down upon it the radiance of thy \grealtcross\ blessing and to \grealtcross\ hallow it with the dew of thy lovingkindness, that wherever it shall be sprinkled and thy holy Name shall be invoked in prayer, every noble aspiration may be strengthened, every good resolve made firm, and the fellowship of the Holy Spirit vouchsafed to us who place our trust in thee; thou who with the Son livest and reignest in the unity of the same Holy Spirit, God throughout all ages of ages.  \R Amen.}

\pilcrow{The Altar, clergy, and people are then sprinkled, while the following is sung.}

Anthem. Thou shalt purge me, \star\ O Lord, with hyssop, and I shall be clean: thou shalt wash me, and I shall be whiter than snow.

\centerline{\rubric{But from Easter until Whitsunday,}}

Anthem.
\begin{leftbar}
    
\V O Lord, shew thy mercy upon us.  \R And grant us thy salvation. (Alleluya.)


\V Glory be to the Father, and to the Son, and to the Holy Ghost; As it was in the beginning, is now, and ever shall be, world without end. Amen.

\star\ Thou shalt wash me, and I shall be whiter than snow.

\pilcrow{This Anthem is said at the sprinkling of holy water on all Sundays throughout the year, except from Easter to the Feast of the Holy Trinity.  It shall be said even on Passion Sunday and Palm Sunday with \emph{Glory be to the father \etc}}

\pilcrow{From Easter to the Feast of the Holy Trinity the following Anthem should be said at the sprinkling of holy water, the precentor commencing the Anthem.}
\drop{I saw water issuing out of the temple on the right-hand side, alleluya.  And all to whom that water came were made whole, and shall say, Alleluya, alleluya.}

\rubric{Ps.} O give thanks unto the {\scshape Lord}, for he is grácious; because his mercy endureth for ever.

\rubric{Ant.} I saw water, \etc

\V Glory be to the Father, and to the Son, and to the Holy Ghost;As it was in the beginning, is now, and ever shall be, world without end. Amen.

\end{leftbar}


\centerline{Let us pray.}

\drop{Graciously hear us, O Lord, Holy Father, Almighty, everlasting God; and vouchsafe to send thy holy Angel from Heaven to keep, cherish, protect, visit, and defend all who are assembled in this holy habitation.  Through Christ our Saviour.  \R Amen.}

\fleuron


\newcounter{psalmnumber}
\newcommand{\daynumber}{Day 1: M}

\makepagestyle{psalter}
\makeevenhead{psalter}{\daynumber}{\scshape The Psalms}{Psalm \thepsalmnumber}
\makeoddhead{psalter}{Psalm \thepsalmnumber}{\scshape The Psalms}{\daynumber}
\pagestyle{psalter}

\newcommand{\psalterday}[2]{
    \section{Day #1. #2 Prayer}
    \directlua{
        if "#2" == "Morning" then
            tex.print([[\noexpand\renewcommand{\noexpand\daynumber}{Day #1: M.}]])
        else 
            tex.print([[\noexpand\renewcommand{\noexpand\daynumber}{Day #1: E.}]])
        end
    }
}
\newcommand{\psalm}[3][]{
    \subsection[{Psalm #2}]{\stylesubsec{Psalm #2.}{#3}{}}
    \setcounter{psalmnumber}{#2}
    % \directlua{
    %     if [[#1]] \string~= "" then
    %         tex.print([[\noexpand\pilcrow{#1}]])
    %     end
    % }
}

\newcommand{\selah}{
    % \scripture{Selah.}
}
\newcommand{\alleluya}[1][]{
    % \scripture{Alleluya. #1}
}


%
\chapter[The Psalter]{The Psalter, or Psalms of David}
\centerline{\emph{after the translation of the Great Bible}}
\centerline{\emph{pointed as they are to be sung or said in churches.}}
\medskip


\directlua{printPsalter()}


\pagestyle{mystyle}


\chapter{Appendix}
\section{The Great Advent Antiphons.}

Dec.~16. \rubric{O Sapientia.} O Wisdom, which camest out of the mouth of the Most High, and reachest fom one end to another, mightily and sweetly ordering all things: * Come and teach us the way of prudence.

Dec.~17. \rubric{O Adonai.}  O Adonai, and Leader of the house of Israel, who appearedst in the bush to Moses in a flame of fire, and gavest him the Law in Sinai: * Come and deliver us with an outstretched arm.

Dec.~18. \rubric{O Radix Jesse.} O Root of Jesse, which standest for an ensign of the people, at whom kings shall shut their mouths, to whom the Gentiles shall seek: * Come and deliver us, and tarry not.

Dec.~19. \rubric{O Clavis David.}  O Key of David, and Sceptre of the house of Israel; that openest, and no man shutteth, and shuttest, and no man openeth: * Come and bring the prisoner out of the prison-house, and him that sitteth in darkness and the shadow of death.

Dec.~20.  \rubric{O Oriens.}  O Day-spring, Brightness of Light Everlasting, and Sun of Righteousness: * Come and enlighten him that sitteth in darkness and the shadow of death.

Dec.~21.  \rubric{O Rex gentium.}  O King of the Nations, and their desire; the Corner-stone, who makest both one: * Come and save mankind, whom thou formedst of clay.

Dec.~22.  \rubric{O Emmanuel.}  O Emmanuel, our King and Lawgiver, the Desire of all nations, and their Salvation: * Come and save us, O Lord our God.

Dec.~23.  \rubric{O Virgo virginum.}  O Virgin of virgins, how shall this be?  For neither before thee was any like thee, nor shall there be after.  Daughters of Jerusalem, why marvel ye at me? The thing which ye behold is a divine mystery.

\fleuron
\newpage

\section{Ash Wednesday}
\section{Palm Sunday}
\section{Maundy Thursday}
% FROM
\pilcrow{The Eucharist should if possible be sung solemnly on this day, and the organ used; the Sequence \emph{Lauda Sion} may well be included.}

\subsection{The Maundy}
% Postquam surrexit Dominus
\ant After the Lord * had risen from supper, he poured water into a bason, and began to wash the feet of the disciples. So he left them this example. \rubric{Ps.} Blessed are those that are undefiled in the way,\ \star\ and walk in the law of the {\scshape Lord}. \rubric{Ant.} After the Lord, \etc

% Vos vocatis me magister
\ant Ye call me * Lord and Master, and ye say well, for so I am: if I then, your Lord and Master, have washed your feet, ye also ought to wash one another’s feet. \rubric{Ps.} God be merciful unto us, and bless us,\ \star\ and shew us the light of his countenance, and be merciful unto us. \rubric{Ant.} Ye call me \etc

%Mandatum novum
\ant A new commandment I give you, * that ye love one another as I have loved you, saith the Lord. \rubric{Ps.} O hear ye this, all ye people:\ \star\ ponder it with your ears, all ye that dwell in the world. \rubric{Ant.} A new commandment \etc

%Diligamus nos
\ant Let us love * one another, for love is of God; and he that loveth his brother is born of God, and knoweth God. \rubric{Ps.} Behold, how good and joyful a thing it is,\ \star\ brethren, to dwell together in unity. \rubric{Ant.} Let us love \etc

\medskip
\V We wait for thy loving kindess, O God; \R In the midst of thy temple.

\V Thou hast charged; \R That we should diligently keep thy commandments.

\V O Lord, hear our prayer.  \R And let our cry come unto thee.

\V The Lord be with you; \R And with thy spirit.

% Collect of Quinquagesima
\centerline{Let us pray.}
\drop{O Lord, who hast taught us that all our doings without charity are nothing worth; Send thy Holy Ghost, and pour into our hearts that most excellent gift of charity, the very bond of peace and of all virtues, without which whosoever liveth is counted dead before thee: Grant this for thine only Son Jesus Christ’s sake. \R Amen}

\subsection{Stripping and Washing Altars}
\pilcrow{This ceremony may follow Evensong, or ‘The Maundy’, or the Solemn Eucharist. If it follows Evensong, it will be better for the sanctuary to be stripped of carpets and hangings and the lamps extinguished earlier in the day, leaving only the altars to be stripped as a preliminary to the washing.}

\pilcrow{As soon as the Priest and Ministers have taken their place before the altar, the following Respond shall be sung, the \V being sung by the person who begins the \R.}


\fleuron

\section{Good Friday}
\subsection{The Solemn Prayers}
\subsection{The Reproaches}

\section{Holy Saturday}
\subsection{Blessing the New Fire}
\subsection{Blessing the Paschal Candle}
\subsection{The Lessons and Paschal Prayers}

\fleuron

% The Supplement
% to the Indian Book of Common Prayer
\stylesec{The Form for the Blessing of Candles}{on the Festival of}{The Presentation of Christ in the Temple}

\pilcrow{The Priest (who may be vested in a cope) shall bless the candles, which may be placed on a table in the sanctuary, as follows.}

\V The Lord be with you.  \R And with thy spirit.

\centerline{Let us pray.}
\drop{O almighty and everlasting God, the Source and Giver of all light: We humbly thank thee that thou didst send forth into the world thine only begotten Son, the Prince of Glory, to be born of a pure Virgin according to the word of thy holy prophets, and to enlighten those who sat in darkness and the shadow of death. Grant that we, who today shall bear these candles to the praise of thy glory, may evermore rejoice in his unfailing and eternal light; through the same Jesus Christ our Lord. \R Amen.}

\drop{O God, who as on this day didst fulfil the desire of thy holy servant Simeon that he should not see death until he had seen thy Christ; and didst permit him to hold in his arms the world's salvation: We beseech thee to \grealtcross\ bless these candles to our use, so that, as we behold their outward light, our hearts may be enkindled with the fire of thy love. Grant, we pray thee, that all we who have been separated from the darkness of sin, may walk as children of light, and be found worthy to be presented before thee in the temple of thy glory; through the same Jesus Christ our Lord. \R Amen.}

\drop{O Lord, Holy Father, Almighty and everlasting God, whose blessed Son did mightily overthrow the powers of darkness, and has brought us into his kingdom of light: Vouchsafe, we beseech thee, to \grealtcross\ bless these candles to thy service. Mercifully hear the prayers of us thy servants, who desire reverently to bear them in our hands to the honour of thy holy name; and grant that we and all thy faithful people, abiding in his grace, may be fruitful in good works to the glory of thy name; through the same Jesus Christ our Lord. \R Amen.}

\pilcrow{Here may the candles be sprinkled and censed. Then they shall be distributed to the congregation and lighted, after which a procession shall be formed. During the distribution, or during the procession, the following shall be sung:}


\subsection{\stylesubsec{The Song of Symeon}{Nunc dimittis.}{St.~Luke ij.~29.}}

\rubric{Anthem.} A light to lighten the Gentiles: and the glory of thy people Israel.

\drop{Lord, \cross now lettest thou thy servant depart in peace,\ \star\ according to thy word.}

\ant A light, \etc

2\enspace For mine eyes have seen\ \star\ thy salvation,

\ant A light, \etc

3\enspace  Which thou hast prepared\ \star\ before the face of all people;

\ant A light, \etc

4\enspace  To be a light to lighten the Gentiles,\ \star\ and to be the glory of thy people Israel.

\ant A light, \etc

Glory be to the Father, and to the Son,\ \star\ and to the Holy Ghost;

\ant A light, \etc

As it was in the beginning, is now, and ever shall be;\ \star\ world without end. Amen.

\rubric{Anthem.} A light to lighten the Gentiles: and the glory of thy people Israel.


\pilcrow{This service may be said before the Lord’s Super on the Festival of the Presentation, or after the Second Lesson at Evensong.}

\pilcrow{During the Lord’s Supper the candles may again be lit for the reading to the Gospel.}

\fleuron
\newpage

% The Supplement
% to the Indian Book of Common Prayer

\stylesec{The Form for the Blessing of Ashes}{on}{Ash Wednesday}

\pilcrow{Before the Lord’s Supper ashes prepared from the palms blessed the previous Palm Sunday, or other suitable ashes, may be blessed as follows:}

\pilcrow{The ashes shall be placed in a vessel near the holy Table; and the Priest, standing at the Epistle side, shall say,}

\V The Lord be with you.  \R And with thy spirit.

\centerline{Let us pray.}
\drop{O God, our faithful Creator, who wouldest not the death of a sinner, but rather that he should turn from his wickedness, and live: Look with mercy upon the frailty of our mortal nature; and of thy goodness vouchsafe to \grealtcross\ bless these ashes which are now to be set upon our heads as a token of humility and of sorrow for our sins. We acknowledge that we are but dust and ashes, and that, by reason of our offences, unto dust we shall return; yet we beseech thy mercy to grant the forgiveness of all our sins and the pardon which thou hast promised to all who truly repent and believe in thy Son; who with thee and the Holy Spirit, liveth and reigneth, one God, world without end. \R Amen.}

\pilcrow{Here may the ashes be sprinkled and censed,}

\bigskip
\pilcrow{Then shall the Priest put ashes on his own forehead, or if there be another Priest present, he shall put the ashes on the officiant’s forehead; after which the people shall kneel at the Communion rail and the Priest shall put the ashes on their foreheads. During the imposition Psalm 25 may be said or sung, or some suitable Lenten hymn may be sung.}

\medskip

\pilcrow{The Priest shall say to each person, as the ashes are imposed:}

Remember, O man, that dust thou art, and unto dust shalt thou return.

\centerline{\rubric{Or}}

Remember that thou art a sinner, and repent.

\fleuron
\newpage


\chapter[Family Prayer]{Forms of Prayer\\to be used in Families}


\section{Morning Prayer}

\pilcrow{The Master or Mistress having called together as many of the Family as can conveniently be present, let one of them, or any other who may be appointed, say as followeth, all kneeling, and repeating with him the Lord’s Prayer.}

\drop{Our Father, which art in heaven, Hallowed be thy Name. Thy kingdom come. Thy will be done in earth, As it is in heaven. Give us this day our daily bread. And forgive us our trespasses, As we forgive them that trespass against us.  And lead us not into temptation.  But deliver us from evil. Amen.}


\pilcrow{Here may follow the Collect for the day.}


\subsubsection{Acknowledgment of God’s Mercy and Preservation, especially through the Night past.}
\drop{Almighty and everlasting God, in whom we live and move and have our being; We, thy needy creatures, render thee our humble praises, for thy preservation of us from the beginning of our lives to this day, and especially for having delivered us from the dangers of the past night. For these thy mercies, we bless and magnify thy glorious Name; humbly beseeching thee to accept this our morning sacrifice of praise and thanksgiving; for his sake who lay down in the grave, and rose again for us, thy Son our Saviour Jesus Christ. Amen.}


\subsubsection{Dedication of Soul and Body to God’s Service, with a Resolution to be growing daily in Goodness.}
\drop{And since it is of thy mercy, O gracious Father, that another day is added to our lives; We here dedicate both our souls and our bodies to thee and thy service, in a sober, righteous, and godly life: in which resolution, do thou, O merciful God, confirm and strengthen us; that, as we grow in age, we may grow in grace, and in the knowledge of our Lord and Saviour Jesus Christ. Amen.}


\subsubsection{Prayer for Grace to enable us to perform that Resolution.}
\drop{But, O God, who knowest the weakness and corruption of our nature, and the manifold temptations which we daily meet with; We humbly beseech thee to have compassion on our infirmities, and to give us the constant assistance of thy Holy Spirit; that we may be effectually restrained from sin, and incited to our duty. Imprint upon our hearts such a dread of thy judgments, and such a grateful sense of thy goodness to us, as may make us both afraid and ashamed to offend thee. And, above all, keep in our minds a lively remembrance of that great day, in which we must give a strict account of our thoughts, words, and actions to him whom thou hast appointed the Judge of quick and dead, thy Son Jesus Christ our Lord. Amen.}


\subsubsection{For Grace to guide and keep us the following Day, and for God’s Blessing on the business of the Same.}
\drop{In particular, we implore thy grace and protection for the ensuing day. Keep us temperate in all things, and diligent in our several callings. Grant us patience under our afflictions. Give us grace to be just and upright in all our dealings; quiet and peaceable; full of compassion; and ready to do good to all men, according to our abilities and opportunities. Direct us in all our ways. Defend us from all dangers and adversities; and be graciously pleased to take us, and all who are dear to us, under thy fatherly care and protection. These things, and whatever else thou shalt see to be necessary and convenient to us, we humbly beg, through the merits and mediation of thy Son Jesus Christ, our Lord and Saviour. Amen.}

\theGrace
 
 

\section{Evening Prayer}
\pilcrow{The Family being together, a little before bedtime, let the Master or Mistress, or any other who may be appointed, say as followeth, all kneeling, and repeating with him the Lord’s Prayer.}

\drop{Our Father, which art in heaven, Hallowed be thy Name. Thy kingdom come. Thy will be done in earth, As it is in heaven. Give us this day our daily bread. And forgive us our trespasses, As we forgive them that trespass against us.  And lead us not into temptation.  But deliver us from evil. Amen.}

\pilcrow{Here may follow the Collect for the day.}


\subsubsection{Confession of Sins, with a Prayer for Contrition and Pardon.}
\begin{wrapfigure}{r}{0.30\textwidth}
\par{\footnotesize\red\emph{* Here let him who reads make a short pause, that every one may secretly confess the sins and failings of that day.}\par}
\end{wrapfigure}
\wflettrine{M}{ost} merciful God, who art of purer eyes than to behold iniquity, and hast promised forgiveness to all those who confess and forsake their sins; We come before thee in an humble sense of our own unworthiness, acknowledging our manifold transgressions of thy righteous laws.* But, O gracious Father, who desirest not the death of a sinner, look upon us, we beseech thee, in mercy, and forgive us all our transgressions. Make us deeply sensible of the great evil of them; and work in us an hearty contrition; that we may obtain forgiveness at thy hands, who art ever ready to receive humble and penitent sinners; for the sake of thy Son Jesus Christ, our only Saviour and Redeemer. Amen.


\subsubsection{Prayer for Grace to reform and grow Better.}
\drop{And lest, through our own frailty, or the temptations which encompass us, we be drawn again into sin, vouchsafe us, we beseech thee, the direction and assistance of thy Holy Spirit. Reform whatever is amiss in the temper and disposition of our souls; that no unclean thoughts, unlawful designs, or inordinate desires, may rest there. Purge our hearts from envy, hatred, and malice; that we may never suffer the sun to go down upon our wrath; but may always go to our rest in peace, charity, and good-will, with a conscience void of offence towards thee, and towards men; that so we may be preserved pure and blameless, unto the coming of our Lord and Saviour Jesus Christ. Amen.}


\subsubsection{The Intercession.}
\drop{And accept, O Lord, our intercessions for all mankind. Let the light of thy Gospel shine upon all nations; and may as many as have received it, live as becomes it. Be gracious unto thy Church; and grant that every member of the same, in his vocation and ministry, may serve thee faithfully. Bless all in authority over us; and so rule their hearts and strengthen their hands, that they may punish wickedness and vice, and maintain thy true religion and virtue. Send down thy blessings, temporal and spiritual, upon all our relations, friends, and neighbours. Reward all who have done us good, and pardon all those who have done or wish us evil, and give them repentance and better minds. Be merciful to all who are in any trouble; and do thou, the God of pity, administer to them according to their several necessities; for his sake who went about doing good, thy Son our Saviour Jesus Christ. Amen.}


\subsubsection{The Thanksgiving.}
\drop{To our prayers, O Lord, we join our unfeigned thanks for all thy mercies; for our being, our reason, and all other endowments and faculties of soul and body; for our health, friends, food, and raiment, and all the other comforts and conveniences of life. Above all, we adore thy mercy in sending thy only Son into the world, to redeem us from sin and eternal death, and in giving us the knowledge and sense of our duty towards thee. We bless thee for thy pa-tience with us, notwithstanding our many and great provo-cations; for all the directions, assistances, and comforts of thy Holy Spirit; for thy continual care and watchful providence over us through the whole course of our lives; and particularly for the mercies and benefits of the past day; beseeching thee to continue these thy blessings to us, and to give us grace to show our thankfulness in a sincere obedience to his laws, through whose merits and intercession we received them all, thy Son our Saviour Jesus Christ. Amen.}


\subsubsection{Prayer for God’s Protection through the Night following.}
\drop{In particular, we beseech thee to continue thy gracious protection to us this night. Defend us from all dangers and mischiefs, and from the fear of them; that we may enjoy such refreshing sleep as may fit us for the duties of the coming day. And grant us grace always to live in such a state that we may never be afraid to die; so that, living and dying, we may be thine, through the merits and satisfaction of thy Son Christ Jesus, in whose Name we offer up these our imperfect prayers. Amen.}

\theGrace

\pilcrow{On Sundays, and on other days when it may be convenient, it will be proper to begin with a Chapter, or part of a Chapter, from the New Testament.}

 

 

 
A SHORTER FORM.
MORNING.

\pilcrow{After the reading of a brief portion of Holy Scripture, let the Head of the Household, or some other member of the family, say as followeth, all kneeling, and repeating with him the Lord’s Prayer.}

OUR Father, who art in heaven, Hallowed be thy Name. Thy kingdom come. Thy will be done, On earth as it is in heaven. Give us this day our daily bread. And forgive us our trespasses, As we forgive those who trespass against us. And lead us not into temptation, But deliver us from evil. For thine is the kingdom, and the power, and the glory, for ever and ever. Amen.

\drop{O Lord, our heavenly Father, Almighty and everlasting God, who hast safely brought us to the beginning of this day; Defend us in the same with thy mighty power; and grant that this day we fall into no sin, neither run into any kind of danger; but that all our doings, being ordered by thy governance, may be righteous in thy sight; through Jesus Christ our Lord. Amen.}

\pilcrow{Here may be added any special Prayers.}

\theGrace
 
EVENING.

\pilcrow{After the reading of a brief portion of Holy Scripture, let the Head of the Household, or some other member of the family, say as followeth, all kneeling and repeating with him the Lord’s Prayer.}

OUR Father, who art in heaven, Hallowed be thy Name. Thy kingdom come. Thy will be done, On earth as it is in heaven. Give us this day our daily bread. And forgive us our trespasses, As we forgive those who trespass against us. And lead us not into temptation, But deliver us from evil. For thine is the kingdom, and the power, and the glory, for ever and ever. Amen.

\drop{Lighten our darkness, we beseech thee, O Lord; and by thy great mercy defend us from all perils and dangers of this night; for the love of thy only Son, our Saviour, Jesus Christ. Amen.}

\pilcrow{Here may be added any special Prayers.}
\theGrace


\section{Additional Prayers.}

\subsubsection{For the Spirit of Prayer.}
\drop{Almighty God, who pourest out on all who desire it, the spirit of grace and of supplication; Deliver us, when we draw nigh to thee, from coldness of heart and wanderings of mind, that with stedfast thoughts and kindled affections, we may worship thee in spirit and in truth; through Jesus Christ our Lord. Amen.}


\subsubsection{In the Morning.}
\drop{O God, the King eternal, who dividest the day from the darkness, and turnest the shadow of death into the morning; Drive far off from us all wrong desires, incline our hearts to keep thy law, and guide our feet into the way of peace; that having done thy will with cheerfulness while it was day, we may, when the night cometh, rejoice to give thee thanks; through Jesus Christ our Lord. Amen.}


\drop{Almighty God, who alone gavest us the breath of life, and alone canst keep alive in us the holy desires thou dost impart; We beseech thee, for thy compassion’s sake, to sanctify all our thoughts and endeavours; that we may neither begin an action without a pure intention nor continue it without thy blessing. And grant that, having the eyes of the mind opened to behold things invisible and unseen, we may in heart be inspired by thy wisdom, and in work be upheld by thy strength, and in the end be accepted of thee as thy faithful servants; through Jesus Christ our Saviour. Amen.}


\subsubsection{At Night.}
\drop{O Lord, support us all the day long, until the shadows lengthen and the evening comes, and the busy world is hushed, and the fever of life is over, and our work is done. Then in thy mercy grant us a safe lodging, and a holy rest, and peace at the last. Amen.}


\drop{O God, who art the life of mortal men, the light of the faithful, the strength of those who labour, and the repose of the dead; We thank thee for the timely blessings of the day, and humbly supplicate thy merciful protection all this night. Bring us, we beseech thee, in safety to the morning hours; through him who died for us and rose again, thy Son, our Saviour Jesus Christ. Amen.}


\subsubsection{Sunday Morning.}
\drop{O God, who makest us glad with the weekly remembrance of the glorious resurrection of thy Son our Lord; Vouchsafe us this day such blessing through our worship of thee, that the days to come may be spent in thy service; through the same Jesus Christ our Lord. Amen.}


\subsubsection{For Quiet Confidence.}
\drop{O God of peace, who hast taught us that in returning and rest we shall be saved, in quietness and in confidence shall be our strength; By the might of thy Spirit lift us, we pray thee, to thy presence, where we may be still and know that thou art God; through Jesus Christ our Lord. Amen.}

\subsubsection{For Guidance.}
\drop{O God, by whom the meek are guided in judgment, and light riseth up in darkness for the godly; Grant us, in all our doubts and uncertainties, the grace to ask what thou wouldest have us to do, that the Spirit of Wisdom may save us from all false choices, and that in thy light we may see light, and in thy straight path may not stumble; through Jesus Christ our Lord. Amen.}


\subsubsection{For Trustfulness.}
\drop{O most loving Father, who willest us to give thanks for all things, to dread nothing but the loss of thee, and to cast all our care on thee, who carest for us; Preserve us from faithless fears and worldly anxieties, and grant that no clouds of this mortal life may hide from us the light of that love which is immortal, and which thou hast manifested unto us in thy Son, Jesus Christ our Lord. Amen.}

\drop{O heavenly Father, thou understandest all thy children; through thy gift of faith we bring our perplexities to the light of thy wisdom, and receive the blessed encouragement of thy sympathy, and a clearer knowledge of thy will. Glory be to thee for all thy gracious gifts. Amen.}


\subsubsection{For Joy in God’s Creation.}
\drop{O heavenly Father, who hast filled the world with beauty; Open, we beseech thee, our eyes to behold thy gracious hand in all thy works; that rejoicing in thy whole creation, we may learn to serve thee with gladness; for the sake of him by whom all things were made, thy Son, Jesus Christ our Lord. Amen.}


\subsubsection{For the Children.}
\drop{Almighty God, heavenly Father, who hast blessed us with the joy and care of children; Give us light and strength so to train them, that they may love whatsoever things are true and pure and lovely and of good report, following the example of their Saviour Jesus Christ. Amen.}


\subsubsection{For the Absent.}
\drop{O God, whose fatherly care reacheth to the uttermost parts of the earth; We humbly beseech thee graciously to behold and bless those whom we love, now absent from us. Defend them from all dangers of soul and body; and grant that both they and we, drawing nearer to thee, may be bound together by thy love in the communion of thy Holy Spirit, and in the fellowship of thy saints; through Jesus Christ our Lord. Amen.}


\subsubsection{For Those We Love.}
\drop{Almighty God, we entrust all who are dear to us to thy never-failing care and love, for this life and the life to come; knowing that thou art doing for them better things than we can desire or pray for; through Jesus Christ our Lord. Amen.}


\subsubsection{For the Recovery of a Sick Person.}
\drop{O merciful God, giver of life and health; Bless, we pray thee, thy servant, [N.], and those who administer to him of thy healing gifts; that he may be restored to health of body and of mind; through Jesus Christ our Lord. Amen.}


\subsubsection{For One about to undergo an Operation.}
\drop{Almighty God our heavenly Father, we beseech thee graciously to comfort thy servant in his suffering, and to bless the means made use of for his cure. Fill his heart with confidence, that though he be sometime afraid, he yet may put his trust in thee; through Jesus Christ our Lord. Amen.}


\subsubsection{For a Birthday.}
\drop{Watch over thy child, O Lord, as his days increase; bless and guide him wherever he may be, keeping him unspotted from the world. Strengthen him when he stands; comfort him when discouraged or sorrowful; raise him up if he fall; and in his heart may thy peace which passeth understanding abide all the days of his life; through Jesus Christ our Lord. Amen.}


\subsubsection{For an Anniversary of One Departed.}
\drop{Almighty God, we remember this day before thee thy faithful servant [N.], and we pray thee that, having opened to him the gates of larger life, thou wilt receive him more and more into thy joyful service; that he may win, with thee and thy servants everywhere, the eternal victory; through Jesus Christ our Lord. Amen.}


\subsubsection{For Those in Mental Darkness.}
\drop{O heavenly Father, we beseech thee to have mercy upon all thy children who are living in mental darkness. Restore them to strength of mind and cheerfulness of spirit, and give them health and peace; through Jesus Christ our Lord. Amen.}


\subsubsection{For a Blessing on the Families of the Land.}
\drop{Almighty God, our heavenly Father, who settest the solitary in families; We commend to thy continual care the homes in which thy people dwell. Put far from them, we beseech thee, every root of bitterness, the desire of vain-glory, and the pride of life. Fill them with faith, virtue, knowledge, temperance, patience, godliness. Knit together in constant affection those who, in holy wedlock, have been made one flesh; turn the heart of the fathers to the children, and the heart of the children to the fathers; and so enkindle fervent charity among us all, that we be evermore kindly affectioned with brotherly love; through Jesus Christ our Lord. Amen.}


\subsubsection{For all Poor, Homeless, and Neglected Folk.}
\drop{O God, Almighty and merciful, who healest those that are broken in heart, and turnest the sadness of the sorrowful to joy; Let thy fatherly goodness be upon all that thou hast made. Remember in pity such as are this day destitute, homeless, or forgotten of their fellow-men. Bless the congregation of thy poor. Uplift those who are cast down. Mightily befriend innocent sufferers, and sanctify to them the endurance of their wrongs. Cheer with hope all discouraged and unhappy people, and by thy heavenly grace preserve from falling those whose penury tempteth them to sin; though they be troubled on every side, suffer them not to be distressed; though they be perplexed, save them from despair. Grant this, O Lord, for the love of him, who for our sakes became poor, thy Son, our Saviour Jesus Christ. Amen.}


\subsubsection{For Faithfulness in the Use of this World’s Goods.}
\drop{Almight God, whose loving hand hath given us all that we possess; Grant us grace that we may honour thee with our substance, and remembering the account which we must one day give, may be faithful stewards of thy bounty; through Jesus Christ our Lord. Amen.}


\subsubsection{A General Intercession.}
\drop{O God, at whose word man goeth forth to his work and to his labour until the evening; Be merciful to all whose duties are difficult or burdensome, and comfort them concerning their toil. Shield from bodily accident and harm the workmen at their work. Protect the efforts of sober and honest industry, and suffer not the hire of the labourers to be kept back by fraud. Incline the heart of employers and of those whom they employ to mutual forbearance, fairness, and good-will. Give the spirit of governance and of a sound mind to all in places of authority. Bless all those who labour in works of mercy or in schools of good learning. Care for all aged persons, and all little children, the sick and the afflicted, and those who travel by land or by sea. Remember all who by reason of weakness are overtasked, or because of poverty are forgotten. Let the sorrowful sighing of the prisoners come before thee; and according to the greatness of thy power, preserve thou those that are appointed to die. Give ear unto our prayer, O merciful and gracious Father, for the love of thy dear Son, our Saviour Jesus Christ. Amen.}


\section{Grace before Meat.}

\drop{Bless, O Father, thy gifts to our use and us to thy service; for Christ’s sake. \R Amen.}

\drop{Give us grateful hearts, our Father, for all thy mercies, and make us mindful of the needs of others; through Jesus Christ our Lord. \R Amen.}

\fleuron
Appendix:

Service of catechising

"the forme of" or "the order for the conjuring of water"?


Canticles for the weekdays? Maybe with the extra benedicite *there?* (or not.)

Sources:
Collect "For Reconciliation with the Jews" on Good Friday from the Prayer Book Society of Canada; Authorized for use \emph{ad libitum} by the General Synod of the Anglican Church of Canada.


% \chapter{The Kalendar (1662)}

Keep the normal kalendar (this isn't appealing, after all, to the 1662-only folks.)
Keep the saints days, but not the extras, this isn't the real kalendar, and the saints
are only anchors here.
Add in Rev. instead of Acts in december?

\directlua{printKalendar2()} 
% {\tiny\begin{longtabu} to \linewidth 
%   {@{} c @{\hspace{.5em}} c @{\hspace{1em}}X[3,l]
%   |@{\hspace{.3em}}X[1,r]@{\hspace{.3em}}|@{\hspace{.3em}}X[1,r]@{\hspace{.3em}}||@{\hspace{.3em}}X[1,r]@{\hspace{.3em}}|@{\hspace{.3em}}X[1,r]@{\hspace{.3em}}|}

%    &          &          & \multicolumn{2}{c}{\scshape Mattins} & \multicolumn{2}{c}{\scshape Evensong} \\
%    &          &          & 1st        & 2nd        & 1st        & 2nd  \\ 
%    &          &          & Lesson     & Lesson     & Lesson     & Lesson \\ \hline
% 1  & {\red A} & \dub{Circumcision}\dotfill & & & &\\ %of our Lord.
% 2  & b        & \dotfill & Gen.~1     & Matt.~1    & Gen.~2     & Rom.~1 \\
% 3  & c        & \dotfill & \dotfill 3 & \dotfill 2 & \dotfill 4 & \dotfill 2\\
% 4  & d        & \dotfill & \dotfill 5 & \dotfill 3 & \dotfill 6 & \dotfill 3\\
% 5  & e        & \dotfill & \dotfill 7 & \dotfill 4 & \dotfill 8 & \dotfill 4\\
% 6  & f        & \dub{\red Epiphany}\dotfill & & & &\\ %of our Lord}.
% 7  & g        & \dotfill & \dotfill 9 & \dotfill 5 & \dotfill 12 & \dotfill 5 \\ %(Keys of LXX.) 
% 8  & {\red A} & S.~Lucian, B.M.\dotfill&\dotfill 13 & \dotfill 6 &  \dotfill 14 &    \dotfill 6\\
% 9  & b        & \dotfill & \dotfill 15 & \dotfill 7 & \dotfill 16 & \dotfill    7 \\
% 10 & c        & \dotfill & \dotfill 17 & \dotfill 8 & \dotfill 18 &  \dotfill     8 \\
% 11 & d        & \dotfill & \dotfill 19 & \dotfill 9 &  \dotfill 20 & \dotfill    9 \\
% 12 & e        & \dotfill & \dotfill 21 & \dotfill 10 & \dotfill 22 & \dotfill   10\\
% 13 & f        & \mem{S.~Hilary, B.C.}\dotfill&   \dotfill   23 &     \dotfill   11 &   \dotfill   24 &   \dotfill   11\\
% 14 & g        & \dotfill &   \dotfill   25 &   \dotfill     12 & \dotfill     26 &   \dotfill   12\\


% \end{longtabu}}

justus.anglican.org/resources/bcp/copyrights.html:
While it is commonly understood that all books published prior to 1923 are in the public domain, copyright notice or no, it is not so commonly known that many texts published after that date are also in the public domain. Under U. S. copyright law, any book published between 1923 and 1977 with no copyright notice is in the public domain. Also, if the book was published in the US with a copyright notice between 1923 and 1963 but that copyright was not renewed, it is in the public domain.

% \chapter{The Hymnal}

\def\hymn{\obeylines\hymna}
\def\hymnverses{\obeylines\hymna}
\newcommand{\hymna}[1]{
    \directlua{printhymn([[#1]])}
}



\section{Sunday, during Summer}
¶ At Mattins\hfill Nocte surgentes

\hymnverses{
Now, from the slumbers of the night arising,
Chaunt we the holy psalmody of David,
Hymns to our Master, with our best endeavour,
Sweetly intoning.

2. So may our Monarch pitifully hear us,
That we may merit with his Saints to enter
Mansions eternal, therewithal possessing
Joy beatific.

3. This he vouchsafe us, God for ever blessed,
Father eternal, Son, and Holy Spirit,
Whose is the glory, which through all creation
Ever resoundeth. Amen.
}

\rubric{Sundays} \V The Lord is high above all péople.  \R And his glory above the héavens.

\rubric{Weekdays} \V Let thy merciful kindness, O Lord, be upon us.  \R As we do put our trust in thee.

\medskip
¶ At Lauds\hfill Ecce jam noctis

\hymnverses{
Lo! the dim shadows of the night are waning;
Lightsome and blushing, dawn of day returneth;
Fervent in spirit, to the mighty Father
Pray we devoutly.

2. So shall our Maker, of his great compassion,
Banish all sickness, kindly health bestowing;
And may he grant us, of a Father’s goodness,
Mansions in heaven.


3. This he vouchsafe us, God for ever blessed,
Father eternal, Son, and holy Spirit,
Whose is the glory, which through all creation
Ever resoundeth. Amen.
}

\rubric{Sundays} \V The Lord is Kíng.  \R He hath put on glorious apparel, allelúya.

\rubric{Weekdays} \V Have I not thought upon thee when I was waking?  \R Because thou hast been my helper.

¶ At Evensong\hfill Lucis Creator optime

\hymnverses{
O blest Creator of the light,  
Who mak’st the day with radiance bright,  
And o’er the forming world didst call  
The light from chaos first of all:

2. Whose wisdom join’d in meet array
The morn and eve, and named them Day:
Night comes with all its darkling fears,
Regard thy people’s prayers and tears.

3. Lest sunk in sin, and ’whelm’d with strife,
They lose the gift of endless life:
While thinking but the thoughts of time,
They weave new chains of woe and crime.

4. But grant them grace that they may strain
The heav’nly gate and prize to gain:
Each harmful lure aside to cast.
And purge away each error past.

5. O Father, that we ask be done,
Through Jesus Christ, thine only Son:
Who, with the Holy Ghost and thee,
Shall live and reign eternally. Amen.
}

\V Let my prayer be set forth, O Lord.  \R In thy sight as the incense.


\rubric{Monday.} My soul doth magnify \star\ the Lord.

\rubric{Tuesday.} My spirit hath rejoiced \star\ in God my Saviour.

\rubric{Wednesday.} O Lord my God, \star\ thou hast regarded my lowliness.

\rubric{Thursday.} He hath put down \star\ the mighty from their seat; and hath exalted the humble and meek that confess his Christ.

\rubric{Friday.} God hath holpen \star\ his servant Israel, as he promised Abraham and his seed; and he hath exalted the humble for ever.


¶ At Compline.\hfill Te lucis ante terminum

\hymnverses{
To thee, before the close of day,  
Creator of the world, we pray,  
That, with thy wonted favour, thou  
Wouldst be our guard and keeper now.

2. From all ill dreams defend our eyes,
From nightly fears and fantasies:
Tread under foot our ghostly foe,
That no pollution we may know.

3. O Father, that we ask be done.
Through Jesus Christ, thine only Son:
Who, with the holy Ghost and thee,
Shall live and reign eternally. Amen.
}

\V Keep us, O Lórd.  \R As the apple of an eye, hide us under the shadow of thy wíngs.

\bigskip
\section{A Virgin}
¶ At First Evensong \& Mattins.\hfill Virginis Proles

\hymnverses{
Child of a Virgin,   Maker of thy Mother,
Born of a Maiden,  as of Maid conceived,
While we a Virgin’s   triumphs are rehearsing,
Hear our petition.

*2. She, thine own maiden, double blessing winneth,
Striving to vanquish all her nature’s weakness.
E’en by that weakness o’er a world of bloodshed
Victory gaining.

*3. Death and its terrors undismay’d beholding,
Death’s cruel handmaid, torture, she despiseth;
Shedding her life-blood, meet is she to enter
Holiest heaven.

4. God ever-loving, as for us she pleadeth,
Pity our failings, all our sins forgiving:
Thus shall re-echo pure and heart-felt praises
Unto thine honour.

5. Praise to the Father, to the Sole-begotten,
And the blest Spirit, with the twain co-equal,
One only Godhead, who throughout the ages
Reigneth for ever. Amen.
}

¶ Note that if the Virgin be not a Martyr, verses 2 and 3 are ommitted.
¶ For Many Virgins, the hymn and \V is {Ihesu, corona virginum}, as at Lauds.

\rubric{Evensong.} \V Full of grace are thy líps. \R Therefore God hath blessed thee for éver. .

\rubric{Mattins.} \V God shall give her the help of his cóuntenance.  \R God is in the midst of her, therefore shall she not be remóved. .

\bigskip


¶ At Lauds \& Second Evensong.\hfill Ihesu corona virginum

\hymnverses{
Jesu, the Virgins’ Crown, do thou
Accept us as in prayer we bow:
Born of that Virgin, whom alone
The Mother and the Maid we own.

2. Among the lilies thou dost feed,
By Virgin quires accompanied—
With glory deck’d, the spotless brides
Whose bridal gifts thy love provides.

3. They, wheresoe’er thy footsteps bend,
With hymns and praises still attend:
In blessed troops they follow thee,
With dance, and song, and melody.

4. We pray thee therefore to bestow
Upon our senses here below
Thy grace, that so we may endure
From taint of all corruption pure.

Ordinary Doxology
5. All laud to God the Father be,
All praise, eternal Son, to thee:
All glory, as is ever meet,
To God the holy Paraclete. Amen.
}

\V The virgins that be her fellows shall bear her company.  \R And shall be brought unto thee.

¶ For Many Virgins, the hymn and \V is {Rex gloriose martyrum}, as at Lauds for Many Martyrs.




\section{The First Sunday in Advent}

Office. Ad te levavi.
\lettrine{U}{nto} thee, O Lord, lift I up my soul; O my God, in thee have I trusted, let me not • be confóunded; * neither let mine enemies triumph over me; for all they that look for thee shall • not be ashamed. \rubric{Ps.} Shew me • thý ways, Ó Lord * and • teach me thý paths.

\rubric{Grail.} For all they • that lóok for thee * shall not be a•shámed, Ó Lord.  \V Make known to me • thy wáys, O Lord * and • téach me thy paths.  \rubric{Alleluya.} \V Shew us thy mercy, • O Lord * and grant us • thy salvátion.   \rubric{Offertory.} Unto thee, O Lord, lift I up my soul; O my God, in thee have I trusted, let me • not be confóunded * neither let mine enemies triumph over me; for all they that look for thee shall • not be ashamed.  \rubric{Communion.} The Lord shall shew • lóving-kíndness * and our land • shall give her increase.

\section{The Second Sunday of Advent}
Office. Populus Sion.
\lettrine{O}{ people} of Sion, behold, the Lord is nigh at hand to re•déem the nátions: * and in the gladness of your heart the Lord shall cause his glori•ous voice to be heard.  \rubric{Ps.} Hear, O thou • Shépherd of Ísrael: * thou that leadest • Jóseph like a sheep.

\rubric{Grail.} Out of Sion hath God • appéared: * in • pérfect béauty.  \V Gather my saints toge•ther únto me: * those that have made a covenant with • mé with sácrifice.  \rubric{Alleluya.} \V For the powers of heaven shall be • shaken: * and then shall they see the Son of Man coming in a cloud with power • and great glóry. \rubric{Offertory.} Wilt not thou turn again, O God, and quicken us; that thy peo•ple may rejóice in thee: * shew us thy mercy, O Lord; and grant • us thy salvation.  \rubric{Communion.} Jerusalem, haste • thee, and stánd on high: * and behold the joy and gladness, which cometh unto thee • from God thy Saviour.


\section{The Third Sunday of Advent}
Office. Gaudete.
\lettrine{R}{ejoice} ye in the Lord alway, and again I say, rejoice ye;   Let your moderation be known unto all men; the • Lord is át hand. * Be careful for nothing, nor troubled; but in all things, by prayer and supplication, with thanksgiving, let your requests be • made known unto God.  \rubric{Ps.} And the peace of God, which passeth all • únderstánding * shall • keep your héarts and minds.

\rubric{Grail.} Shew thyself, O Lord, thou that sittest upon • the Chérubim: * stir • úp thy stréngth and come.  \V Hear, O thou Shepherd • of Ísrael: * thou that leadest • Jóseph líke a sheep.  \rubric{Alleluya.} \V Stir up thy strength, • O Lord: * and • come and hélp us. \rubric{Offertory.} Lord, thou art become gracious unto thy land; thou hast turned away the capti•vity of Jácob; * thou hast forgiven the of•fence of thy people.  \rubric{Communion.} Say to them that are • of a féarful heart: * Be strong, fear not; behold, your God • will come and save you.


\section{The Fourth Sunday of Advent}
¶ Should the Fourth Sunday fall on {Christmas Eve}, the {Office, Grail, Alleluya, Offertory,} and {Communion} will be those of {Christmas Eve}, below.
Office. Memento nostri.

\lettrine{R}{emember} us, O Lord, with the favour that thou bearest unto thy people; O visit us with • thy salvátion: * that we, beholding the felicity of thy chosen, may rejoice in the gladness of thy people, and may glory with • thine inheritance. \rubric{Ps.} We have sinned • wíth our fáthers: * we have done amiss, • and dealt wíckedly.

\rubric{Grail.} The Lord is nigh unto all them that call • upón him; * yea, all such as call up•ón him fáithfully. \V My mouth shall speak the praise • of thé Lord; * and let all flesh give thanks un•tó his hóly name. \rubric{Alleluia.} \V Come, O Lord, and tar•ry not: * forgive the misdeeds • of thy péople. \rubric{Offertory.} Be strong, fear not; behold, our God will • come with a récompense: * – he • will come and save us. \rubric{Communion.} Behold, a Virgin shall con•ceive and béar a Son, * and his name shall be call•ed Emmanuel.


\section{Christmas Eve}
Office. Hodie scietis.
\lettrine{T}{o-day} shall ye know that the Lord will come to deliver you * and at sunrise shall ye behold his glory. \rubric{Ps.} The earth is the Lord’s, and all that is therein * the compass of the world, and they that dwell therein.

\rubric{Grail.} To-day shall ye know that the Lord will come to deliver you; * and at sunrise shall ye behold his glory. \V Hear, O thou Shepherd of Israel, thou that leadest Joseph like a sheep: * show thyself also, thou that sittest upon the Cherubim, before Ephraim, Benjamin, and Manasses. \rubric{Alleluya.} \V On the morrow the iniquity of the earth shall be blotted out: * and the Saviour of the world shall reign over us. \rubric{Offertory.} Lift up your heads, O ye gates, and be ye lift up, ye everlasting doors: * and the King of Glory shall come in. \rubric{Communion.} The glory of the Lord shall be revealed: * and all flesh shall see the salvation of our God.

\section{The Fifteenth Sunday after Trinity}
\centerline{Office. \emph{Inclina, Domine.}}
\drop{Bów dówn, O Lórd, thine éar to • me, and héar me: \star\ O my Gód, sáve thy sérvant, that trústeth in thée; have mércy upón me, O Lórd, for I have cálled • dáily upón thee. \rubric{Ps.} Cómfort the • sóul of thy sérvant: \star\ for únto thée, O Lórd, do I • líft up my sóul.}

\rubric{Grail.} It is bétter to • trüst in the Lórd: \star\ than to pút any • cónfidence in mán. \V It is bétter to • trüst in the Lórd: \star\ than to pút any cónfi•dence in prínces. \rubric{Allelúya.} \V My héart is réady, O Gód, my héart is • réady: \star\ I will síng, yéa, I will práise thee, with the bést • mémber thát I have. \rubric{Offertory.} I wáited pátiently for the Lórd, and he • inclíned únto me: \star\ hé heard my cálling, and hath pút a néw sóng in my móuth, éven a thanksgív•ing únto our Gód. \rubric{Communion.} Whóso éateth my Flésh, and • drínketh my Blóod: \star\ dwélleth in mé, and I • in hím, sáith the Lórd.

\end{document}
