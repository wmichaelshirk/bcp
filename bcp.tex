\documentclass[foolscapvopaper,10pt,twoside,openany,extrafontsizes,final]{memoir}
% \documentclass[ebook,12pt,twoside,openany,extrafontsizes,final]{memoir}
\usepackage{svg}
\usepackage{fontspec}
\usepackage{lettrine}
\usepackage[normalem]{ulem} % for strike-through \sout{}
\usepackage[autocompile]{gregoriotex}
\usepackage{multicol}
\usepackage{multirow}
\usepackage{wrapfig}
\usepackage{lscape}
% \usepackage[rflt]{floatflt}
\usepackage{longtable,tabu}
\usepackage{xtab}
\usepackage{tabularx}
\usepackage{tabularray}
\usepackage[perpage]{footmisc}
% \usepackage{lua-visual-debug}
% a Font that supplies Astrological glyphs
\newfontfamily\versiculus{DejaVu Sans}

\directlua{dofile("lua-functions.lua")}
\newcommand{\drop}[1]{
    \directlua{drop("\luaescapestring{\unexpanded{#1}}")}
}


\setsecnumdepth{part} % Only number the parts.

% Chapters
\setlength{\beforechapskip}{0\baselineskip}
\setlength{\afterchapskip}{.5\baselineskip}
\renewcommand*{\chaptitlefont}{\sffamily}
\renewcommand*{\printchaptertitle}[1]{\centering\LARGE{\chaptitlefont #1}}

% Sections: Major chapter divisions; Days in the Psalter
\setbeforesecskip{1em plus 2em minus 1ex}
\setsecheadstyle{\scshape\centering}
\setaftersecskip{1pt}

%Subsection: Collects, Canticles, Psalms
%{\schshape Title}{\itshape incipit}{\normal scriptural reference.}
\setbeforesubsecskip{1em plus 0ex minus 1ex}
\setsubsecheadstyle{\centering}
\setaftersubsecskip{1pt}

\newcommand{\stylechapter}[3]{{\normalfont\scshape\normalsize#1\par}#2\par{\normalfont\scshape\normalsize#3}}
\newcommand{\stylesubsec}[3]{\small{\mbox{\scshape#1}} {\mbox{\itshape\red #2}} {\mbox{#3}}}
\newcommand{\stylesec}[3]{\section[#3]{#1\\ {\normalfont\small#2\\ } #3}}
\newcommand{\scripture}[1]{\hspace*{\fill}{\mbox{\small\itshape\red #1}}\par\smallskip}
%Subsubsection
%Descriptive names of prayers, etc.
\setbeforesubsubsecskip{1em plus 0ex minus 1ex}
\setsubsubsecheadstyle{\itshape\footnotesize\red\centering}
\setaftersubsubsecskip{1pt}

\emergencystretch 1.5em


\newcommand \fleuron{{\centering\Large\red❦\par}}

\nouppercaseheads
\makepagestyle{mystyle}
\makeevenhead{mystyle}{}{\scshape\leftmark}{}
\makeoddhead{mystyle}{}{\scshape\rightmark}{}
\makeevenfoot{mystyle}{}{\thepage}{}
\makeoddfoot{mystyle}{}{\thepage}{}
\makepsmarks{mystyle}{%
  \createmark{chapter}{left}{nonumber}{}{}}

\pagestyle{mystyle}



% For the wrapped rubric on the embertide intercession
\setlength{\intextsep}{2pt}
% Get rid of extra space around longtable
\setlength{\LTpre}{.5em}
\setlength{\LTpost}{.5em}

%s"] = "ſ
\directlua{
  fonts.handlers.otf.addfeature{
    name = "calt",
    type = "chainsubstitution",
    lookups = {
      {
        type = "substitution",
        data = {
          ["s"] = "ſ",
        },
      },
    },
    data = {
      rules = {
        {
          before = { { 0xFFFC, "-", "·", "A", "Í", "I", "P", "a", "á", "b", "c", "d", "e", "é", "g", "h", "i", "j", "k", "l", "m", "n", "o", "ó", "p", "q", "r", "s", "ſ", "t", "u", "ú", "v", "w", "x", "y", "z" } },
          after = { { "·", "-", "a", "á", "c", "d", "e", "é", "g", "h", "i", "í", "j", "l", "m", "n", "o", "ó", "p", "q", "r", "s", "t", "u", "ú", "v", "w", "x", "y", "z" } },
          current = { { "s" } },
          lookups = { 1 },
        },
      },
    },
  }
}

\setmainfont[  
    Numbers={OldStyle, Proportional},
    %Ligatures={Rare},%,Historic},
    %CharacterVariant={1:0}
    %CharacterVariant=1,
    %StylisticSet=7,
    RawFeature={+calt}
    % ItalicFeatures={Colour=990000}
]{EBGaramond}
% \setsansfont{UnifrakturMaguntia}[StylisticSet={1},CharacterVariant={11,12,4:1}]
\setsansfont{KJV1611}[RawFeature={+hist,+calt}]

\newcommand \red{\addfontfeature{Color=990000}}
\newcommand \black{\addfontfeature{Color=000000}}

\newcommand \R{{\red℟.\ }}
\newcommand \V{{\red℣.\ }}
\newcommand \ant{{\scshape\small\red Ant.\ }}
\newcommand \etc{\emph{\red\&c.}}
\newcommand \minorheading[2]{{\unexpanded\expandafter{\scshape #1}}\par{\unexpanded\expandafter{\itshape #2}}}
\newcommand \subseccaption[2]{\subsection{#1 {\itshape\small\red #2}}}
\newcommand \prefaceCaption[3]{\subsection{{\small\emph{\red #1} #2 \emph{\red #3}}}}
\newcommand \rubric[1]{\emph{\footnotesize\red #1}}
\newcommand \pilcrow[1]{\par\footnotesize\noindent\makebox[1em][l]{¶ }\hangindent1em\rubric{#1}\par\normalsize}
\newcommand \hangingrubric[1]{\par\footnotesize\noindent\hangindent1em\rubric{#1}\par\normalsize}
\newcommand \centeredrubric[1]{\par{\centering\footnotesize\rubric{#1}\par}\normalsize}

\newcommand \cross {\grecross\ }

\newcommand{\qa}[1]{{\itshape\small\red#1}}

\renewcommand{\thefootnote}{\fnsymbol{footnote}}

\medievalpage
\raggedbottom
\checkandfixthelayout



\newcounter{cnt}\setcounter{cnt}{0}
\def\t{\stepcounter{cnt}\thecnt. cat sat on the mat. }

\newdimen\tttaa
\newdimen\tttbb

\renewcommand\thepage{\the\numexpr(\value{page}+1)/2\relax}

\makeatletter
\def\merge@ps{\afterassignment\merge@ps@\tttbb}
\def\merge@ps@{\afterassignment\merge@ps@@\tttaa}

\def\merge@ps@@{%
\afterassignment\reset@WF@ps\dimen@\WF@ps\valign
%\showthe\count@
\ifnum\count@>\@ne
\advance\count@\m@ne
\expandafter\merge@ps
\fi
}


\def\reset@WF@ps{\afterassignment\reset@WF@ps@\dimen@ii}

\def\reset@WF@ps@#1\valign{%
\edef\new@wf@ps{\new@wf@ps
  \the\dimexpr\dimen@+\tttbb\relax\space
  \the\dimexpr\dimen@ii-\tttbb\relax\space}%
 \def\WF@ps{#1}}


\newcommand\wflettrine[3][]{%
  \setbox\tw@\hbox{\lettrine[#1]{#2}{#3}\global\let\gtmp\L@parshape}%
  \afterassignment\wf@getoffset\count@\gtmp\hoffset
  \setbox\WF@box\hbox{\kern-\dimen@\box\WF@box\kern\dimen@}%
  \noindent\box\tw@
    \def\new@wf@ps{}%
    \afterassignment\merge@ps\count@\gtmp
    \edef\WF@ps{\new@wf@ps\space\WF@ps}%
    \@@parshape\c@WF@wrappedlines\WF@ps\z@\columnwidth}


\def\wf@getoffset{\afterassignment\wf@get@ffset\dimen@}
\def\wf@get@ffset#1\hoffset{}

\makeatother

\makeatletter
\def\tabu@verticalmeasure{\everypar{}%
\unless\ifnum\currentgrouptype=14 \let\tabu@currentgrouptype\currentgrouptype\fi
    \ifnum \tabu@currentgrouptype>12         % 14=semi-simple, 15=math shift group
        \setbox\tabu@box =\hbox\bgroup
            \let\tabu@verticalspacing \tabu@verticalsp@lcr
            \d@llarbegin                % after \hbox ...
    \else
        \edef\tabu@temp{\ifnum\tabu@currentgrouptype=5\vtop
                        \else\ifnum\tabu@currentgrouptype=12\vcenter
                        \else\vbox\fi\fi}%
        \setbox\tabu@box \hbox\bgroup$\tabu@temp \bgroup
            \let\tabu@verticalspacing \tabu@verticalsp@pmb
    \fi
}
\makeatother



\begin{document}
\frontmatter
\title{Book of Common Prayer}
\author{Michael Shirk}
\date{MMXXI}
% \maketitle
\thispagestyle{empty}
{\centering
A 

\smallskip

{\sffamily\HUGE Book of Common Prayer}

\medskip

and {\scshape Administration of the Sacraments }
and {\scshape Other Rites and Ceremonies of the Church}

\smallskip

{\itshape According to the Traditional English Use}

\medskip

\greseparator{3}{15}

\medskip

Together with

{\scshape The Psalter or Psalms of David}

{\itshape Pointed as they are to be Sung or Said in Churches}

\medskip
\grealtcross

\medskip

{\itshape Set forth by authority for use in the}

{Independent Catholic Christian Church}


\bigskip
% \includesvg[width=1in]{Line Version}
\includegraphics[width=1in]{Line Version.png}
\bigskip

{\scshape René Vilatte Press, MMXXII}

}


% argent
% 	a cross gules, a paschal lamb argent shown smaller

% argent, a cross patee gules, a paschal lamb argent shown smaller
% on 
% achievement crest a mitre or shown larger
% motto "Sic, Episcope"


\chapter{Preface}
\drop{It is a most invaluable part of that blessed liberty wherewith Christ hath made us free, that in his worship different forms and usages may without offence be allowed, provided the substance of the Faith be kept entire; and that, in every Church, what cannot be clearly determined to belong to Doctrine must be referred to Discipline and therefore, by common consent and authority, may be altered, abridged, enlarged, amended, or otherwise disposed of, as may seem most convenient for the edification of the people, “according to the various exigencies of times and occasions.”}

The wisdom of our fathers under the good hand of God gave to the Church of England the Book of Common Prayer in English speech. It is, and we believe that it will always be, one of the great books of the world. Nothing save the English version of the Holy Scriptures is enwoven so closely in the language and the deepest thoughts of English speaking people.

There was never any thing by the wit of man so well devised, or so sure established, which in continuance of time hath not been corrupted: As, among other things, it may plainly appear by the Common Prayers in the Church, commonly called Divine Service. The first original and ground whereof if a man would search out by the ancient Fathers, he shall find, that the same was not ordained but of a good purpose, and for a great advancement of godliness. 

And now, this important work being brought to a conclusion, it is hoped the whole will be received and examined by every true member of our Church, and every sincere Christian, with a meek, candid, and charitable frame of mind; without prejudice or prepossessions; seriously considering what Christianity is, and what the truths of the Gospel are; and earnestly beseeching Almighty God to accompany with his blessing every endeavour for promulgating them to mankind in the clearest, plainest, most affecting and majestic manner, for the sake of Jesus Christ, our blessed Lord and Saviour.  

In all things we have set before our eyes the duty of faithfulness to the teaching of Scripture and the godly and decent order of the ancient Fathers, and we pray that by God's blessing upon our work those who use this book may be enabled to keep the unity of the Spirit in the bond of peace.



\settocdepth{chapter}
\tableofcontents*

% Of Ceremonies

The Order how the Psalter is appointed to be read.
The Order how the Rest of Holy Scripture is appointed to be read
Proper Lessons
Proper Psalms


kalendar

Tables and rules
Table to find Easter Day
Rules to Order the service
  Every service begins "in the name of ... cross"

\mainmatter

\newcommand \Sol[1]{{\versiculus☉}~in~{\versiculus#1}}
\newcommand \dub[1]{{\footnotesize\sffamily #1}}		% Double Feasts
\newcommand \mem[1]{\emph{#1}} % Memorials

%\renewcommand \dub[1]{{\sffamily #1}}		% Double Feasts
%\renewcommand \Rul[1]{{\scshape #1}}		% Ruled Feasts
%\markright{Kalendar}\part{Kalendar}

\chapter{The Kalendar}
\directlua{printKalendar(false)}


\newpage

\section*{A Table to Find Easter-Day}
{\begin{multicols}{2}
{\tiny
\noindent\begin{tabular} { @{}c@{\hspace{.15cm}} c@{\hspace{.2cm}} r@{\hspace{.2cm}} l@{\hspace{.2cm}} }
Golden & Month & Day & Sunday \\
Number &    &   & Letter \\
\hline
xiv & March & 22 & d \\
iij & ” & 23 & e \\
    & ” & 24 & f \\
xj  & ” & 25 & g \\
    & ” & 26 & A \\
xix & ” & 27 & b \\
viij & ” & 28 & c \\
    & ” & 29 & d \\
xvj & ” & 30 & e \\
v   & ” & 31 & f \\
    & April & 1 & g \\
xiij & ” & 2 & A\\
ij & ” & 3 & b\\
  & ” & 4 & c \\
x &	”	&5&	d\\
&”	&6&	e\\
xviij	&”&	7&	f\\
vij	&”&	8&	g\\
&”	&9&	A\\
xv	&”&	10&	b\\
iv	&”&	11&	c\\
&”	&12&	d\\
xij	&”&	13&	e\\
j	&”&	14&	f\\
&”	&15&	g\\
ix	&”&	16&	A\\
xvij	&”&	17&	b\\
vj	&”&	18&	c\\
&”	&19&	d\\
&”	&20&	e\\
&”	&21&	f\\
&”	&22&	g\\
&”	&23&	A\\
&”	&24&	b\\
&”	&25&	c\\

\end{tabular}}
%\vfill
%\columnbreak
\footnotesize
\vspace{6pt}

This Table contains so much of the Calendar as is necessary for the determining of Easter; to find which, look for the Golden Number of the year in the first Column of the Table, against which stands the day of the Paschal Full Moon; then look in the third Column for the Sunday Letter, next after the day of the Full Moon, and the day of the Month standing against that Sunday Letter is Easter Day. If the Full Moon happens upon a Sunday, then (according to the first rule) the next Sunday after is Easter-Day.

To find the Golden Number, or Prime, add one to the Year of our Lord, and then divide by 19; the remainder, if any, is the Golden Number; but if nothing remaineth, then 19 is the Golden Number.

To find the Dominical or Sunday Letter, according to the Calendar, until the Year 2099 inclusive, add to the Year of our Lord its Fourth Part, omitting Fractions; and also the Number 6: Divide the sum by 7; and if there is no remainder, the A is the Sunday Letter: But if any number remaineth, then the Letter standing against that number in the small annexed Table is the Sunday Letter.
\vspace{6pt}

{\centering\setlength{\extrarowheight}{2pt}
{\tiny\begin{tabular} { | c c c c c c c | }
\hline
0 & 1 & 2 & 3 & 4 & 5 & 6 \\
\hline
A & G & F & E & D & C & B \\
\hline
\end{tabular}}\par}

\vspace{2pt}
For the next Century, that is, from the year 2100 till the year 2199 inclusive, add to the current year its fourth part, and also the number 5, and then divide by 7, and proceed as in the last Rule.

Note, that in all Bissextile or Leap-Years, the Letter found as above will be the Sunday Letter, from the intercalated day exclusive to the end of the year.

The Golden Numbers in the foregoing Calendar will point out the Days of the Paschal Full Moons from the Year 1900, to the Year 2199 inclusive.
\vspace{12pt}
\end{multicols}}

\section{How the Psalter is appointed to be read}

{\scriptsize
% American and Scottish are more similar to each other;
% Maybe better? Probably tend toward the scotteish for sundays
% and to the american for feasts (longer for the 6, and a couple
% extras.?
\drop{Psalms to be read at Morning and at Evening Prayer are appointed
for every Sunday in the year, and for certain other Holy-days.
Otherwise the Psalter will be read through in order once every
month as is appointed.}

Whensoever Proper Psalms are appointed, then the Psalms of
ordinary course for the day of the month shall be omitted.

On week days (unless Proper Psalms are provided) shall be read the
Psalms for the day of the month, as they are appointed, for
Morning and Evening Prayer.

And, whereas January, March, May, July, August, October, and
December have one-and-thirty days apiece; It is ordered, that on
the last day of any one of the said months being an ordinary week
day shall be read the Psalms assigned to the 30th day, or else
the Psalms of the monthly course omitted on one of the Sundays in
that month; So that the Psalter may begin again the first day of
the next month ensuing.

And, whereas the 119th Psalm is divided into twenty-two portions,
and is over-long to be read at one time; It is so ordered, that
at one time shall not be read above four or five of the said
portions.

And at the end of every Psalm, and of every such part of the
119th Psalm, shall be repeated this Hymn,


{\normalsize
Glory be to the Father, and to the Son : and to the Holy Ghost;

%\emph{Answer}. 
As it was in the beginning, is now, and ever shall be : world without end. Amen.
}


Note, that the Psalter followeth the Division of the Hebrews, and
the Translation of the great English Bible, set forth and used in
the time of King \emph{Henry} the Eighth, and \emph{Edward} the Sixth.

% [Psalms have also been selected for use on various occasions,
% and on such occasions one or more at the discretion of the Minister
% may be read at Morning and Evening Prayer in place of the Psalms
% of the Day.

% Upon occasions to be approved by the Bishop, other Psalms may,
% with his consent, be substituted for the Psalms of the Day or for
% those which are appointed in the Table of Proper Psalms.]
}
\medskip
\section[Proper Psalms for Certain Days]{Table of Proper Psalms for Certain Days}
{\scriptsize

\SetTblrTemplate{head}{empty}
\SetTblrTemplate{foot}{empty}
\begin{longtblr}[
    entry=none,
    label=none,
    caption=none
]{
    rowhead=2,
    rowfoot=0,
    rows={rowsep=0pt},
    row{1}={rowsep=2pt},
    row{2}={ht=2pt},
    colspec={X|c|c},
    column{1}={leftsep=0pt},
    column{3}={rightsep=0pt}
}
\hline
 & {\scshape Mattins} & {\scshape Evensong}\\
\hline\\

First Sunday in Advent\dotfill      & 1, 7          & 46, 48\\
Second Sunday in Advent\dotfill     & 9, 11         & 50, 67\\
Third Sunday in Advent\dotfill      & 73            & 75, 76, 82 \\
Fourth Sunday in Advent\dotfill     & 94            & 96, 97, 98 \\
Christmas Eve\dotfill               & —             & 89 (1–36) \\
Christmas Day\dotfill               & 19, 85        & 132 \\
1st Sunday after Christmas\dotfill  & 2, 8          & 45, 110, 113\\
New Year’s Eve \dotfill             & —                 & 90, 133, 134\\
Circumcision\dotfill                & 119 (1–32)        & 91, 121 \\
2nd Sunday after Christmas\dotfill  & 103            & 104 \\
Eve of Epiphany\dotfill             & —                 & 19, 87 \\
Epiphany\dotfill                    & 72            & 96, 97, 117 \\
1st Sunday after Epiphany\dotfill   & 46, 47, 67        & 18 \\
2nd Sunday after Epiphany\dotfill   & 27, 36        & 68 \\
3rd Sunday after Epiphany\dotfill   & 42, 43            & 33, 34 \\
4th Sunday after Epiphany\dotfill   & 60, 63            & 74 \\
5th Sunday after Epiphany\dotfill   & 99, 112           & 106 \\
6th Sunday after Epiphany\dotfill   & 80, 81            & 78 \\
Septuagesima\dotfill                & 104               & 147, 148 \\
Sexagesima\dotfill                  & 139               & 25, 26\\
Quinquagesima\dotfill               & 15, 20, 23        & 30, 31\\
Ash Wednesday\dotfill               & 6, 32, 38         & 102, 130, 143 \\
1st Sunday in Lent\dotfill          & 51                & 6, 32, 143 \\
2nd Sunday in Lent\dotfill          & 119 (1–32)        & 119 (33–72) \\
3rd Sunday in Lent\dotfill          & 119 (73–104)      & 119 (105–144) \\
4th Sunday in Lent\dotfill          & 119 (145–176)     & 39, 40 \\
5th Sunday in Lent\dotfill          & 22                & 51 \\
6th Sunday in Lent\dotfill          & 61, 62            & 86, 130 \\
Monday in Holy Week\dotfill         & 13, 25            & 26, 27, 28 \\
Tuesday in Holy Week\dotfill        & 31                & 88 \\
Wednesday in Holy Week\dotfill      & 41, 42, 43        & 54, 55\\
Thursday in Holy Week\dotfill       & 56, 64            & 23, 109 \\
Good Friday\dotfill                 & 22                & 40, 69\\
Easter Even\dotfill                 & 23, 30, 142       & 115, 116, 117 \\
Easter Day\dotfill                  & 2, 16, 111        & 113, 114, 118 \\
1st Sunday after Easter\dotfill     & 3, 57              & 103\\
2nd Sunday after Easter\dotfill     & 120, 121, 122, 123 & 65, 66 \\
3rd Sunday after Easter\dotfill     & 124, 125, 126, 127 & 81, 84 \\
4th Sunday after Easter\dotfill     & 128, 129, 130, 131 & 145, 146 \\
5th Sunday after Easter\dotfill     & 132, 133, 134      & 107 \\
\emph{Rogation Monday}\dotfill      & 34, 127            & 62, 63\\
\emph{Rogation Tuesday}\dotfill     & 65, 66, 67         & 102\\
\emph{Rogation Wednesday}\dotfill   & 121, 144           &\\
Eve of Ascension\dotfill            & —                  & 15, 97, 99 \\
Ascension Day\dotfill               & 8, 21              & 24, 47, 110 \\
Sunday after Ascension Day\dotfill  & 93, 96            & 148, 149, 150\\
Eve of Whitsunday\dotfill           & —                 & 48, 145 \\
Whitsunday\dotfill                  & 68                & 104\\
Trinity Sunday\dotfill              & 29, 33            & 93, 99, 115\\
1st Sunday after Trinity\dotfill    & 1, 3, 5           & 4, 7, 8\\
2nd Sunday after Trinity\dotfill    & 10, 12, 13        & 15, 16, 17\\
3rd Sunday after Trinity\dotfill    & 18                & 19, 20, 21\\
4th Sunday after Trinity\dotfill    & 24, 25            & 22, 23\\
5th Sunday after Trinity\dotfill    & 26,28             & 27, 29, 30\\
6th Sunday after Trinity\dotfill    & 31, 32            & 33, 36\\
7th Sunday after Trinity\dotfill    & 34                & 37\\
8th Sunday after Trinity\dotfill    & 39, 40           & 41, 42, 43\\
9th Sunday after Trinity\dotfill    & 46, 47, 48       & 44, 45\\
10th Sunday after Trinity\dotfill   & 50, 53           & 51, 54\\
11th Sunday after Trinity\dotfill   & 56, 57           & 61, 62, 63\\
12th Sunday after Trinity\dotfill   & 65, 66           & 68\\
13th Sunday after Trinity\dotfill   & 71                & 67, 72\\
14th Sunday after Trinity\dotfill   & 75, 76           & 73, 77\\
15th Sunday after Trinity\dotfill   & 84, 85           & 89\\
16th Sunday after Trinity\dotfill   & 86, 87           & 90, 91\\
17th Sunday after Trinity\dotfill   & 92, 93           & 100, 101, 102\\
18th Sunday after Trinity\dotfill   & 103                & 107\\
19th Sunday after Trinity\dotfill   & 111, 112, 113      & 120, 121, 122, 123\\
20th Sunday after Trinity\dotfill   & 114, 115           & 124, 125, 126, 127\\
21st Sunday after Trinity\dotfill   & 116, 117           & 128, 129, 130, 131\\
22nd Sunday after Trinity\dotfill   & 118                & 132, 133, 134\\
23rd Sunday after Trinity\dotfill   & 110, 135           & 137, 138, 139\\
24th Sunday after Trinity\dotfill   & 136                & 140, 141, 142\\
25th Sunday after Trinity\dotfill   & 49                 & 79, 83\\
26th Sunday after Trinity\dotfill   & 84, 144            & 105\\
Sunday next before Advent\dotfill   & 145, 146           & 147, 148, 149, 150\\
&&\\

Michaelmas Eve\dotfill            & —                  & 91 \\
Michaelmas\dotfill                & 34, 103            & 148 \\
All Hallows’s Eve\dotfill         & —                  & 146, 148\\
All Saints’\dotfill               & 1, 15              & 145 \\
Eve of a Greater Feast\dotfill    & —                  & 1, 30 \\
A Greater Feast\dotfill            & 111, 112           & 148, 149 \\
Eve of the Dedication\dotfill        & —                  & 84, 87\\
Feast of the Dedication\dotfill     & 132                & 122, 133, 134\\
Harvest Thanksgiving\dotfill      & 103                & 65, 67\\
                                  & 104                & 147, 150
\end{longtblr}
}


\medskip
\section{Psalms for Special Occasions}
{\scriptsize
One or more of the following Psalms may be used on the occasions specified:—

\begin{hangparas}{.25in}{1}
Eves of Holy-days and Holy-days.—1, 15, 24, 30, 34, 42, 43, 84, 91, 103, 111, 112, 113, 116, portions of 119, 131, 132, 138, 145, 146, 148, 149.

Feast of Dedication or Patronal Feast.—24, 48, 84, 122, 132, 133, 134.

Thanksgiving for Harvest.—65, 67, 103, 104, 144, 145, 147, 148, 150.

For Home Missions and Missions beyond the Seas.—2, 45, 46, 47, 48, 67, 72, 85, 87, 96, 97, 100, 117.

Times of trouble or anxiety.—23, 25, 46, 77, 80, 86, 90, 130.

Occasions of thanksgiving.—30, 33, 65, 107, 111, 115, 138, 145, 146, 148, 150.

\end{hangparas}
% Propers and rogation days from "The Churchpeople's Prayer Book"(1935)
}

\fleuron
\chapter{Table of Lessons}
(The Table of 1922, as revised in 1928)

To be read in course throughout the year

This Table is arranged according to the weeks of the ecclesiastical year, beginning with the First Sunday in Advent.  The Lessons for the Immoveable Feasts not given in this Table are to be found in the Calendar following the Table.

Except on Septuagesima Sunday, and the Sunday next before Advent, on ever Sunday on which Lessons from the Gospels are provided both for Mattins and Evensong, one of such Lessons shall always be read.

It is convenient that, when alternative Lessons are provided, choice be exercised according to some scheme of consecutive reading.

\newcommand \one {{\addfontfeature{Numbers=Proportional}(1)}\ }
\newcommand \two {{\addfontfeature{Numbers=Proportional}(2)}\ }

{\scriptsize
\begin{longtabu} to \linewidth { @{}X | X  | X@{} }
\hline 
    & {\scshape Mattins} & {\scshape Evensong}\\
\hline
\endfirsthead
%---------------------------------------------------------------%
\hline
    & {\scshape Mattins} & {\scshape Evensong}\\
\hline
\endhead
\scshape{Advent Sunday} &
    \one Isaiah \textbf{1,} 1–20
    \two John \textbf{3,} 1–21, \emph{or} 
         1~Thess.~\textbf{4,} 13—\textbf{5,} 11 &
    \one Isaiah \textbf{2,} \emph{or} Isaiah \textbf{1,} 18–end \par
    \two Matthew \textbf{24,} 1–28, \emph{or} Revelation \textbf{14,} 13—\textbf{15,} 4\\

\emph{Monday} &
    \one Isaiah \textbf{3,} 1–15  \par \two Mark \textbf{1,} 1–20 &
    \one Isaiah \textbf{4,} 2–end \par \two James \textbf{1}\\
    
\hline 
\scshape{Sunday next before Advent} & \one Ecclesiastes \textbf{11} and \textbf{12} \par \two John \textbf{19,} 13–end, \emph{or} Hebrews \textbf{11,} 1–16 & \one Haggai \textbf{2,} 1–9, \emph{or} Malachi \textbf{3} and \textbf{4} \par \two John \textbf{20,} \emph{or} Hebrews \textbf{11,} 17—\textbf{12,} 2, \emph{or} Luke \textbf{15,} 11–end\\
\emph{Monday} & \one Wisdom \textbf{1} \par \two Revelation \textbf{1} & \one Wisdom \textbf{2} \par \two Revelation \textbf{2}\\

\emph{Tuesday} & \one Wisdom \textbf{3,} 1–9 \par \two Revelation \textbf{3} & \one Wisdom \textbf{4,} 7–end \par \two Revelation \textbf{4}\\

\emph{Wednesday} & \one Wisdom \textbf{5,} 1–16 \par \two Revelation \textbf{5} & \one Wisdom \textbf{6,} 1–21 \par \two Revelation \textbf{6}\\

\emph{Thursday} & \one Wisdom \textbf{7,} 15—\textbf{8,} 4 \par \two Revelation \textbf{7} & \one Wisdom \textbf{8,} 5–18 \par \two Revelation \textbf{10} and \textbf{11,} 1–14\\

\emph{Friday} & \one Wisdom \textbf{8,} 21—\textbf{9} end \par \two Revelation \textbf{11,} 15—\textbf{12} end & \one Wisdom \textbf{10,} 15—\textbf{11,} 10 \par \two Revelation \textbf{14,} 1–13\\

\emph{Saturday} & \one Wisdom \textbf{11,} 21—\textbf{12,} 2 \par \two Revelation \textbf{18} & \one Wisdom \textbf{12,} 12–21 \par \two Revelation \textbf{19,} 1–16\\


\end{longtabu}}

Proper Lessons
Feast of Dedication or Patronal Feast

Thanksgiving for Harvest
% When this Talbe of Lessons... etcc.

% \def\tcase#1{%
%   $\left\lbrace\,
%     \begin{tabular}{@{}c @{}l@{}}
%       #1
%     \end{tabular}
%   \right.$%
% }
% \newpage
% \section{Table of Lessons}
% {\tiny
% \begin{longtabu} to \linewidth{|X[2]X[-1,r]X[2,l]|X[-1,r]l}
%     % X| l r@{} l | l r@{} l ||  l r@{} l |  l r@{} l  }
% \hline
% \multicolumn3{|c}{\scshape Mattins} & \multicolumn2{c}{\scshape Evensong} \\
% \tabuphantomline
% \hline
% Third Sunday in Advent & &
% \tcase{(1) & Isaiah {\bfseries 25,} 1–9 \\ \\
%         (2) & Luke {\bfseries 3,} 1–17, \emph{or} \\
%         & 1 Timothy {\bfseries 1,} 12—{\bfseries 2,} 7} &  &
% \tcase{(1) & Isaiah {\bfseries 26,} \emph{or} \\& Isaiah {\bfseries 28,} 1–22 \\ 
%     (2) & Matthew {\bfseries 25,} 1–30, \emph{or} \\& Revelation {\bfseries 21,} 9—{\bfseries 22,} 5}
%     \\

%     & M. & \tcase{(1) & Isaiah {\bfseries 30,} 19–end \\ (2) & Mark {\bfseries 7,} 1–23}
%     & M. & \tcase{(1) & Isaiah {\bfseries 31} \\ (2) & 1 John {\bfseries 1,} 1—{\bfseries 2,} 5} \\
%     & Tu. & \tcase{(1) & Isaiah {\bfseries 38,} 1–20 \\ (2) & Mark {\bfseries 7,} 24—{\bfseries 8,} 10} 
%     & Tu. & \tcase{(1) & Isaiah {\bfseries 40,} 1–40 \\ (2) & 1 John {\bfseries 2,} 7–end} \\
% Ember day & W. & \tcase{(1) & Isaiah {\bfseries 40,} 12–end \\ (2) & Mark {\bfseries 8,} 11—{\bfseries 9,} 1} 
%     & W. & \tcase{(1) & Isaiah {\bfseries 41} \\ (2) & 1 John {\bfseries 3}} 


% \end{longtabu}
% }







% \begin{landscape}
% \section{Table of Lessons}
% {\footnotesize
% \begin{longtabu} to \linewidth {X| l r@{} l | l r@{} l ||  l r@{} l |  l r@{} l  }
% \hline
% & \multicolumn6{c}{\scshape Mattins} & \multicolumn6{c}{\scshape Evensong} \\
% \hline
% \multirow{2}{*}{Advent Sunday\dotfill  }& \multirow{2}{*}{Isa.} & \multirow{2}{*}{\bfseries 1,} & \multirow{2}{*}{1–20} & John & {\bfseries 3,} & 1–21 & Isa. & {\bfseries 2} & & Matt. & {\bfseries 24,} & 1–28 \\
% & & & & 1 Thess. & {\bfseries 4,} & 13—{\bfseries 5,} 11 & „ & {\bfseries 1,} & 18–end & Rev. & {\bfseries 14,} & 13—{\bfseries 15,} 4 \\
% \hline

   
% \end{longtabu}

% \begin{longtabu} to \linewidth {X| l r@{} l | l r@{} l | l r@{} l }\hline
   
%  & \multicolumn3{c|}{\scshape First Evensong} & \multicolumn3{c|}{\scshape Mattins} & \multicolumn3{c}{\scshape Second Evensong} \\
%  \hline 
% St.\ Matthew\dotfill & 1 Kings  & {\bfseries 19, } & 15–end & Prov. & {\bfseries 3, } & 1–18  & 1 Chron. & {\bfseries 29, } & 9–17\\
%                      & Matt.    & {\bfseries 6, } & 19–end & Matt. & {\bfseries 19, } & 16–end & 1 Tim.   & {\bfseries 6, }  & 6–19 \\ 
% \hline

% \end{longtabu}
% }
% \end{landscape}
% \part{Service Music}

% Primarily from "A Prayer Book Revised", 1913
\chapter[Certain Notes]{Certain Notes for the more Plain Explanation and Decent Ministration of things Contained in this Book}


% ! The changeable portions of the Mass for [[Sundays]] and \Dub{Greater Feasts} will be found in the Propers through the Year.  At Mattins and Evensong, the same Collect is used as at Mass and proper Lessons are appointed, but the Invitatory and Hymns are found in the Hymnal in the Season or class of Saint.
% On \Dub{Principle Feasts}, there is furthermore a proper Preface for the Mass.

% On Lesser Feasts, the common Mass, Invitatory, and Hymns for the class of Saint will generally serve, unless propers be provided.
% The Mass of a Sunday shall serve throughout the week, unless another be appointed.

\drop{The word ‘Minister’ in this Book includes bishops, priests, and deacons. When the word ‘Bishop’ is used, none but a bishop may say the words there appointed; when the word ‘Priest’, then may none but a bishop or priest use the words; when the word ‘Deacon’ is used, then shall the words appointed to the deacon be said by one who is in that office, or by a bishop or priest executing that office for the occasion, or by the priest himself when there is no other minister.}

A Clerk is any person appointed to lead in the singing, or to serve the minister and to lead in the responses; the clerk may also read the Lessons and the Epistle.

When one service follows upon another, opportunity shall be given for people to come and go between the services, whether by the singing of a hymn or by a pause. And none shall go out of church during any service or sermon except in case of necessity.

A sermon shall be preached every Sunday at the time appointed. On Sundays and Holy-days in general, a lecture or sermon on a catechetical topic may be delivered after the Second Lesson at Evensong, or the Priest, or one chosen by him for this purpose as Catechist, may instruct the young people of his parish.

The \emph{Gloria} is always to be added to the Psalms, and to the Canticles specified in the rubric, except from Morning Prayer on Maundy Thursday until Evening Prayer on Easter Even; and also it is omitted at all Funeral and Memorial services.

When any minister or reader says a prayer or other form together with the people, he that reads shall say alone the opening words (as, \emph{Our Father}, \emph{I believe in God}, \emph{Glory be to God on high}, and in other places as far as the comma); and the clerks and people shall take up the following words with him.

The full ending of a Collect may be used on any occasion, whether it be printed or not; except that when more than two Collects are said together, without any intermediate bidding, the first and the last shall have the full ending (and the people shall say \emph{Amen}), and the intermediate Collects shall have no ending. The normal full ending is, \emph{Through Jesus Christ thy Son our Lord, who liveth and reigneth with thee in the unity of the Holy Ghost, ever one God, world without end; \emph{or, if our Lord has been already mentioned in the Collect,} Through the same thy Son Jesus Christ, \etc; \emph{or if the Holy Spirit has been already mentioned,} who liveth and reigneth with thee and the same Spirit \etc}


When Anthems are appointed, they are to be sung in full before the Psalm, and to be repeated at the end of the \emph{Gloria} (or of the Psalm itself, when there is no \emph{Gloria} said); but in any Procession the Anthem may be repeated after each verse, if necessity require.

Saying is to be taken to include singing; and words that are appointed to be sung may be said, if need be. But words which are directed to be said in a humble voice should be said without any musical note or inflection.

% When it is desired to use music composed for them, other authorized liturgical texts may be used in place of the corresponding texts in this book.

To avoid a continual repetition of rubrics, let it here also be said that a minister who is reading the service is not included in a general direction to kneel. He stands to read, unless it be expressly stated that he is to kneel down. All others present kneel during prayers, unless it be otherwise stated, except any who are helping the priest in his ministration.

Whenever any passage from the Scripture is read, he that reads shall stand and turn towards the people, who may sit; except that when the Liturgical Gospel is read, they also shall stand, and turn towards the minister who reads. And whenever the priest speaks to the people, as in absolutions and benedictions, he shall turn to them. All are to stand when Canticles are sung; but during the singing of the Psalms it is lawful to sit.

% If more specific details on the ceremonial of the services are desired, {A Directory of Ceremonial} by René Vilatte Press is recommended.

And since there must of necessity be many things not mentioned in these Notes, we may well, for the rest, observe that golden rule of the venerable Council of Nicæa, “Let ancient customs prevail,” till reason plainly requires the contrary.


\fleuron

\chapter[An Introduction]{An Introduction to Morning or Evening Prayer}

% This rite is made optional/occasional; and so needs none of the shortenings
% provided in the various books.

\bigskip

% Only penitential sentences are provided, since this is a penitential opening,
% not the "general" beginning of mattins.
\pilcrow{The Minister shall read with a loud voice some one or more of these Sentences of the Scriptures that follow.}

\drop{When the wicked man turneth away from his wickedness that he hath committed, and doeth that which is lawful and right, he shall save his soul alive.}
\scripture{Ezekiel xviii.~27.}

I acknowledge my transgressions, and my sin is ever before me.\scripture{Psalm lj.~3.}

Hide thy face from my sins, and blot out all mine iniquities.\scripture{Psalm lj.~9.}

The sacrifices of God are a broken spirit: a broken and a contrite heart, O God, thou wilt not despise.\scripture{Psalm lj.~17.}

Rend your heart, and not your garments, and turn unto the Lord your God: for he is gracious and merciful, slow to anger, and of great kindness, and repenteth him of the evil.\scripture{Joel ij.~13.}

To the Lord our God belong mercies and forgivenesses, though we have rebelled against him; neither have we obeyed the voice of the Lord our God, to walk in his laws which he set before us.\scripture{Daniel ix.~9, 10.}

O Lord, correct me, but with judgement; not in thine anger, lest thou bring me to nothing.\par\scripture{Jeremiah x.~24, Psalm vj.~1.}

Repent ye; for the Kingdom of Heaven is at hand.\scripture{St.~Matthew iij.~2.}

I will arise and go to my father, and will say unto him, Father, I have sinned against heaven, and before thee, and am no more worthy to be called thy son.\scripture{St.~Luke xv.~18, 19.}

Enter not into judgement with thy servant, O Lord; for in thy sight shall no man living be justified.\scripture{Psalm cxliij.~2.}

If we say that we have no sin, we deceive ourselves, and the truth is not in us; but if we confess our sins, God is faithful and just to forgive us our sins, and to cleanse us from all unrighteousness.\scripture{1 St.~John i.~8, 9.}

\bigskip


\centerline{\pilcrow{Then the Minister shall say,}}

\drop{Dearly beloved brethren, the Scripture moveth us in sundry places to acknowledge and confess our manifold sins and wickedness; and that we should not dissemble nor cloak them before the face of Almighty God our heavenly Father; but confess them with an humble, lowly, penitent, and obedient heart; to the end that we may obtain forgiveness of the same, by his infinite goodness and mercy. And although we ought at all times humbly to acknowledge our sins before God; yet ought we most chiefly so to do, when we assemble and meet together to render thanks for the great benefits that we have received at his hands, to set forth his most worthy praise, to hear his most holy Word, and to ask those things which are requisite and necessary, as well for the body as the soul. Wherefore I pray and beseech you, as many as are here present, to accompany me with a pure heart, and humble voice, unto the throne of the heavenly grace, saying after me;} %proofread vs 1928 proposed


\bigskip

{\centering\pilcrow{A general Confession to be said of the whole Congregation after the Minister, all kneeling.}}

\drop{Almighty and most merciful Father; \ We have erred, and strayed from thy ways like lost sheep. We have followed too much the devices and desires of our own hearts. We have offended against thy holy laws. We have left undone those things which we ought to have done; \ And we have done those things which we ought not to have done; \ And there is no health in us. But thou, O Lord, have mercy upon us, miserable offenders. Spare thou them, O God, which confess their faults. Restore thou them that are penitent; \ According to thy promises declared unto mankind in Christ Jesu our Lord. And grant, \ O most merciful Father, for his sake; \ That we may hereafter live a godly, righteous, and sober life, \ To the glory of thy holy Name. Amen.} %proofread vs 1928 proposed

\medskip

{\centering\pilcrow{The Absolution, or Remission of sins, to be pronounced by the Priest alone, standing; the people still kneeling.}}

\drop{Almighty God, the Father of our Lord Jesus Christ, who desireth not the death of a sinner, but rather that he may turn from his wickedness, and live; and hath given power, and commandment, to his Ministers, to declare and pronounce to his people, being penitent, the Absolution and Remission of their sins: He pardoneth \grealtcross\ and absolveth all them that truly repent, and unfeignedly believe his holy Gospel. Wherefore let us beseech him to grant us true repentance, and his Holy Spirit, that those things may please him, which we do at this present; and that the rest of our life hereafter may be pure, and holy; so that at the last we may come to his eternal joy; through Jesus Christ our Lord.} %proofread vs 1928 proposed
\centerline{\rubric{The people shall answer here, and at the end of all other prayers,}}
\centerline{\R Amen.}

\medskip

\pilcrow{If no priest be present the person saying the service shall read the Collect for the Twenty-First Sunday after Trinity, that person and the people still kneeling.}


\fleuron

\chapter[The Order for Mattins]{\stylechapter{The Order for}{Mattins}{Daily Throughout the Year.}}

\bigskip

\pilcrow{The Minister shall kneel, and say the Lord's Prayer with an audible voice; the people also kneeling, and repeating it with him, both here, and wheresoever else it is used in Divine Service.}
\newcommand\ourFather{\drop{Our Father, which art in heaven, \ Hallowed be thy Name; \ Thy kingdom come; \ Thy will be done; \ In earth as it is in heaven. Give us this day our daily bread. And forgive us our trespasses, \ As we forgive them that trespass against us. And lead us not into temptation; \ But deliver us from evil. Amen.}}
\ourFather


\medskip
\pilcrow{Here all standing up, the Priest shall say,}

\V O Lord, open thou my lips.  \R And my mouth shall shew forth thy praise.

\drop{O God, \cross make speed to save me.  \R O Lord, make haste to help me.}

\V Glory be to the Father, and to the Son, and to the Holy Ghost;  \R As it was in the beginning, is now, and ever shall be, world without end. Amen.

\centerline{Alleluya.}

\centeredrubric{From First Evensong of Septuagesima until Easter, instead of \emph{\black Alleluya} is said:}

\V Praise ye the Lord.  \R The Lord’s Name be praised.

\bigskip

\pilcrow{Then shall be said or sung this Psalm following; Except on Easter Day, upon which another Anthem is appointed; and on the nineteenth day of every month it is not to be read here, but in the ordinary course of the Psalms.}

\pilcrow{On the days hereafter named, before this Psalm and after the \emph{Gloria Patri} which follows it, may be sung or said the Invitatory:}

\rubric{On the Sundays in Advent.—}Our King and Saviour draweth nigh \star\ O come, let us adore him.

\rubric{On Christmas Day and until the Epiphany.—}Alleluya.  Unto us a child is born \star\ O come, let us adore him.  Alleluya.

\rubric{On the Epiphany and seven days after, and on the Feast of the Transfiguration.—}The Lord hath manifested forth his glory \star\ O come, let us adore him.

\rubric{On the Purification and the Annunciation.—}The Word was made flesh \star\ O come, let us adore him.

\rubric{On the Monday following the first Sunday after Easter, and until Ascension Day.—}Alleluya. The Lord is risen indeed \star\ O come, let us adore him.  Alleluya.

\rubric{On Ascension Day and until Whitsunday.—}Alleluya.  Christ the Lord ascendeth into heaven \star\ O come, let us adore him.  Alleluya.

\rubric{On Whitsunday and six days after.—}Alleluya. The Spirit of the Lord filleth the world \star\ O come, let us adore him.  Alleluya.

\rubric{On Trinity Sunday.—}One God in Trinity, and Trinity in Unity \star\ O come, let us adore him.

\rubric{On other Festivals for which a proper Epistle and Gospel are appointed.—}The Lord is glorious in his saints \star\ O come, let us adore him.



\subsection{\stylesubsec{}{Venite, exultemus Domino.}{Psalm xcv.}}
\drop{O come, let us sing unto the {\scshape Lord};\ \star\ let us heartily rejoice in the strength of our salvation.}

2\enspace Let us come before his presence with thanksgiving;\ \star\ and shew ourselves glad in him with psalms.

3\enspace For the {\scshape Lord} is a great God;\ \star\ and a great King above all gods.

4\enspace In his hand are all the corners of the earth;\ \star\ and the strength of the hills is his also.

5\enspace  The sea is his, and he made it;\ \star\ and his hands prepared the dry land.

6\enspace  {\red ‡} O come, let us worship and fall down,\ \star\ and kneel before the {\scshape Lord} our Maker.

7\enspace  For he is the Lord our God;\ \star\ and we are the people of his pasture, and the sheep of his hand.

8\enspace  To day if ye will hear his voice, harden not your hearts\ \star\ as in the provocation, and as in the day of temptation in the wilderness;

9\enspace  When your fathers tempted me,\ \star\ proved me, and saw my works.

10\enspace  Forty years long was I grieved with this generation, and said,\ \star\ It is a people that do err in their hearts, for they have not known my ways.

11\enspace  Unto whom I sware in my wrath,\ \star\ that they should not enter into my rest.

Glory be to the Father, and to the Son,\ \star\ and to the Holy Ghost;

As it was in the beginning, is now, and ever shall be,\ \star\ world without end. Amen.

\bigskip
\pilcrow{Then shall follow the Psalms in order as they be appointed. And at the end of every Psalm throughout the year, and likewise at the end of \emph{Benedicite, Benedictus, Magnificat,} and \emph{Nunc dimittis}, shall be repeated,}
Glory be to the Father, and to the Son,\ \star\ and to the Holy Ghost;

As it was in the beginning, is now, and ever shall be,\ \star\ world without end. Amen.

\bigskip
\pilcrow{Then shall be read distinctly with an audible voice the First Lesson, taken out of the Old Testament, as is appointed in the Table of Lessons, except there be proper Lessons assigned for that day: He that readeth so standing and turning himself, as he may best be heard of all such as are present.}

\pilcrow{Note, that before every Lesson the Minister shall say,}
Here beginneth such a Chapter, \rubric{or} Verse of such a Chapter, of such a Book: \rubric{And after every Lesson,} Here endeth the First, \rubric{or} the Second Lesson.

\medskip

\pilcrow{And after that, shall be said or sung, in English, the Hymn called \emph{Te Deum Laudamus}, except from Septuagesima until Easter: during which time shall be used instead \emph{Benedicite, omnia opera}, in English, as followeth:}

\subsection{\stylesubsec{}{Te Deum Laudamus.}{}}
\drop{We praise thee, O God;\ \star\ we acknowledge thee to be the Lord.}

2\enspace All the earth doth worship thee,\ \star\ the Father everlasting.

3\enspace To thee all Angels cry aloud;\ \star\ the Heavens, and all the Powers therein.

4\enspace To thee Cherubin and Seraphin\ \star\ continually do cry,

5\enspace {\red †} Holy, Holy, Holy,\ \star\ Lord God of Sabaoth;

6\enspace Heaven and earth are full of the Majesty\ \star\ of thy glory.

7\enspace The glorious company of the Apostles\ \star\ praise thee.

8\enspace The goodly fellowship of the Prophets\ \star\ praise thee.

9\enspace The noble army of Martyrs\ \star\ praise thee.

10\enspace The holy Church throughout all the world\ \star\ doth acknowledge thee;

11\enspace The Father,\ \star\ of an infinite Majesty;

12\enspace Thine honourable, true,\ \star\ and only Son;

13\enspace Also the Holy Ghost,\ \star\ the Comforter.


\drop{Thou art the King of Glory,\ \star\ O Christ.}

15\enspace Thou art the everlasting Son\ \star\ of the Father.

16\enspace {\red †} When thou tookest upon thee to deliver man,\ \star\ thou didst not abhor the Virgin’s womb.

17\enspace When thou hadst overcome the sharpness of death,\ \star\ thou didst open the Kingdom of Heaven to all believers.

18\enspace Thou sittest at the right hand of God,\ \star\ in the glory of the Father.

19\enspace We believe that thou shalt come\ \star\ to be our Judge.

20\enspace {\red †} We therefore pray thee, help thy servants,\ \star\ whom thou hast redeemed with thy precious blood.

21\enspace Make them to be numbered with thy Saints,\ \star\ in glory everlasting.


\drop{O Lord, save thy people,\ \star\ and bless thine heritage.}\\
23\enspace Govern them,\ \star\ and lift them up for ever.

24\enspace Day by day\ \star\ we magnify thee;

25\enspace And we worship thy Name\ \star\ ever, world without end.

26\enspace Vouchsafe, O Lord,\ \star\ to keep us this day without sin.

27\enspace O Lord, have mercy upon us,\ \star\ have mercy upon us.

28\enspace O Lord, let thy mercy lighten upon us,\ \star\ as our trust is in thee.

29\enspace O Lord, in thee have I trusted;\ \star\ let me never be confounded.

\subsubsection{Or this Canticle,}

\subsection{\stylesubsec{The Song of the Three Children}{Benedicite, omnia opera.}{vv.~35-66.}}

\drop{O all ye Works of the Lord, bless ye the Lord; * praise him, and magnify him for ever.}

2 O ye Angels of the Lord, bless ye the Lord; * O ye Heavens, bless ye the Lord.

3 O ye Waters that be above the firmament, bless ye the Lord; * O all ye Powers of the Lord, bless ye the Lord.

4 O ye Sun and Moon, bless ye the Lord; * O ye Stars of heaven, bless ye the Lord.

5 O ye Showers and Dew, bless ye the Lord; * O ye Winds of God, bless ye the Lord.

6 O ye Fire and Heat, bless ye the Lord; * O ye Winter and Summer, bless ye the Lord.

7 O ye Dews and Frosts, bless ye the Lord; * O ye Frost and Cold, bless ye the Lord.

8 O ye Ice and Snow, bless ye the Lord; * O ye Nights and Days, bless ye the Lord.

9 O ye Light and Darkness, bless ye the Lord; * O ye Lightnings and Clouds, bless ye the Lord.

\drop{O let the Earth bless the Lord; * yea, let it praise him, and magnify him for ever.}

11 O ye Mountains and Hills, bless ye the Lord; * O all ye Green Things upon the earth, bless ye the Lord.

12 O ye Wells, bless ye the Lord; * O ye Seas and Floods, bless ye the Lord.

13 O ye Whales, and all that move in the waters, bless ye the Lord; * O all ye Fowls of the air, bless ye the Lord.

14 O all ye Beasts and Cattle, bless ye the Lord; * O ye Children of Men, bless ye the Lord.

\drop{O let Israel bless the Lord; * praise him, and magnify him for ever.}

16 O ye Priests of the Lord, bless ye the Lord; * O ye Servants of the Lord, bless ye the Lord.

17 O ye Spirits and Souls of the Righteous, bless ye the Lord; * O ye holy and humble Men of heart, bless ye the Lord.

18 O Ananias, Azarias, and Misael, bless ye the Lord; * praise him, and magnify him for ever.

% Composite: 1928 (America), 1929 (scottish), Anglican Breviary, Monastic Diurnal, Sarum Diurnal.
Let us bless the Father, and the Son, and the Holy Ghost; * let us praise him and magnify him for ever.

Blessed art thou, O Lord, in the firmament of heaven; * and to be praised and exalted above all for ever.


\medskip


{\centering\footnotesize{\rubric{On week days, the following Canticle \emph{Benedictus es Domine} may be said or sung instead; or, from Septuagesima until Easter, \emph{Psalm 51}, pg.~\pageref{psalm51}}\par}}


% \subsubsection{Or during Lent, \emph{Psalm 51, pg.~\pageref{psalm51}},}

% \medskip

% \subsubsection{Or this Canticle,}

\subsection{\stylesubsec{The Song of the Three Children}{Benedictus es Domine.}{vv.~29-34}}
\drop{Blessed art thou, O Lord God of our fathers;\ \star\ praised and exalted above all for ever.}

2\enspace Blessed art thou for the Name of thy Majesty;\ \star\ praised and exalted above all for ever.

3\enspace Blessed art thou in the temple of thy holiness;\ \star\ praised and exalted above all for ever.

4\enspace Blessed art thou that beholdest the depths, and dwellest between the Cherubim:\ \star\ praised and exalted above all for ever.

5\enspace Blessed art thou on the glorious throne of thy Kingdom:\ \star\ praised and exalted above all for ever.

6\enspace Blessed art thou in the firmament of heaven:\ \star\ praised and exalted above all for ever.

% American 1928, Scottish 1929
Glory be to the Father, and to the Son,\ \star\ and to the Holy Ghost;

As it was in the beginning, is now, and ever shall be,\ \star\ world without end. Amen.

\subsubsection{}

\pilcrow{Then shall be read in like manner the Second Lesson, taken out of the New Testament. And after that, the Hymn following.}

\subsection{\stylesubsec{The Song of Zachary}{Benedictus.}{St.~Luke i.~68.}}

\drop{Blessed \cross be the Lord God of Israel;\ \star\ for he hath visited and redeemed his people;}

2\enspace And hath raised up a mighty salvation for us,\ \star\ in the house of his servant David;

3\enspace As he spake by the mouth of his holy Prophets,\ \star\ which have been since the world began;

4\enspace That we should be saved from our enemies,\ \star\ and from the hand of all that hate us.

5\enspace To perform the mercy promised to our forefathers,\ \star\ and to remember his holy Covenant;

6\enspace To perform the oath which he sware to our forefather Abraham,\ \star\ that he would give us;

7\enspace That we being delivered out of the hand of our enemies\ \star\ might serve him without fear;

8\enspace In holiness and righteousness before him,\ \star\ all the days of our life.

9\enspace And thou, Child, shalt be called the Prophet of the Highest:\ \star\ for thou shalt go before the face of the Lord to prepare his ways;

10\enspace To give knowledge of salvation unto his people\ \star\ for the remission of their sins,

11\enspace Through the tender mercy of our God;\ \star\ whereby the day-spring from on high hath visited us;

12\enspace To give light to them that sit in darkness, and in the shadow of death,\ \star\ and to guide our feet into the way of peace.

Glory be to the Father, and to the Son,\ \star\ and to the Holy Ghost;

As it was in the beginning, is now, and ever shall be,\ \star\ world without end. Amen.

\bigskip
\pilcrow{Then shall be sung or said the Apostle's Creed, by the Minister and the people standing: Except only such days as the Creed of Saint Athanasius is appointed to be read.}

\drop{I believe in God the Father Almighty, \ Maker of heaven and earth:}

And in Jesus Christ his only Son our Lord: \ Who was conceived by the Holy Ghost, \ Born of the Virgin Mary: \ Suffered under Pontius Pilate, \ Was crucified, dead, and buried: \ He descended into hell; \ The third day he rose again from the dead: \ He ascended into heaven, \ And sitteth on the right hand of God the Father Almighty: \ From thence he shall come to judge the quick and the dead.

I believe in the Holy Ghost: \ The holy Catholick Church; \ The Communion of Saints: \ The Forgiveness of sins: \ The Resurrection of the body, \ And the Life everlasting. Amen.


\bigskip
\pilcrow{And after that, these Prayers following, all devoutly kneeling,}

\centerline{Lord, have mercy upon us.}
\centerline{\emph{Christ, have mercy upon us.}}
\centerline{Lord, have mercy upon us.}

\medskip
{\centering\footnotesize\rubric{Then the Minister, Clerks, and people shall say the Lord's Prayer with a loud voice.}\par}
\ourFather

\smallskip

\centerline{\rubric{Then the Minister standing up shall say,}}

\V O Lord, shew thy mercy upon us.  \R And grant us thy salvation.

\V O Lord, save the \emph{State}.  \R And mercifully hear us when we call upon thee.

\V Endue thy Ministers with righteousness.  \R And make thy chosen people joyful.

\V  O Lord, save thy people.  \R And bless thine inheritance.

\V Give peace in our time, O Lord.  \R Because there is none other that fighteth for us, but only thou, O God.

\V O God, make clean our hearts within us.  \R And take not thy Holy Spirit from us.

\V The Lord be with you.  \R And with thy spirit.

\centerline{Let us pray.}


\pilcrow{Then shall follow three Collects; the first of the day, which shall be the same that is appointed at the Communion; The second for Peace; The third for Grace to live well. And the two last Collects shall never alter, but daily be said at Morning Prayer throughout all the year, as followeth, all kneeling.}

\subsection{\stylesubsec{The second Collect, }{for Peace.}{}}
\drop{O God, who art the author of peace and lover of concord, in knowledge of whom standeth our eternal life, whose service is perfect freedom; Defend us thy humble servants in all assaults of our enemies; that we, surely trusting in thy defence, may not fear the power of any adversaries, through the might of Jesus Christ our Lord. \R Amen.}

\subsection{\stylesubsec{The third Collect, }{for Grace.}{}}
\drop{O Lord, our heavenly Father, Almighty and everlasting God, who hast safely brought us to the beginning of this day; Defend us in the same with thy mighty power; and grant that this day we fall into no sin, neither run into any kind of danger; but that all our doings may be ordered by thy governance, to do always that is righteous in thy sight; through Jesus Christ our Lord. \R Amen.}
\bigskip

\V The Lord be with you.  \R And with thy spirit.

\V Let us bless the Lord.  \R Thanks be to God.
\bigskip

\pilcrow{Here may follow any of the Occasional Prayers and Thanksgivings, as need may require, ending with one of the Conclusions.}

\subsubsection{Here endeth the Order of Morning Prayer throughout the Year.}

\fleuron
\chapter{The Order for Evensong}
\subsubsection{Daily Throughout the Year.}

\pilcrow{Here all standing up, the Priest shall say,}

\drop{O God, \cross make speed to save \emph{me}.  \R O Lord, make haste to help \emph{me}.}

\V Glory be to the Father, and to the Son, and to the Holy Ghost;  \R As it was in the beginning, is now, and ever shall be, world without end. Amen.

\V Praise ye the Lord.  \R The Lord's Name be praised.

\medskip

\pilcrow{Then shall be said or sung the Psalms in order as they be appointed. Then a Lesson of the Old Testament, as is appointed. And after that \emph{Magnificat} (or the Song of the blessed Virgin Mary) in English, as followeth.}

\subsection{\stylesubsec{The Song of the Blessed Virgin Mary}{Magnificat.}{St.~Luke i.~46.}}

\drop{My \cross soul doth magnify the Lord : and my spirit hath rejoiced in God my Saviour.}

2 For he hath regarded : the lowliness of his handmaiden.

3 For behold, from henceforth : all generations shall call me blessed.

4 For he that is mighty hath magnified me : and holy is his Name.

5 And his mercy is on them that fear him : throughout all generations.

6 He hath shewed strength with his arm : he hath scattered the proud in the imagination of their hearts.

7 He hath put down the mighty from their seat : and hath exalted the humble and meek.

8 He hath filled the hungry with good things : and the rich he hath sent empty away.

9 He remembering his mercy hath holpen his servant Israel : as he promised to our forefathers, Abraham and his seed, for ever.

Glory be to the Father, and to the Son : and to the Holy Ghost;

As it was in the beginning, is now, and ever shall be : world without end. Amen.

\medskip

\pilcrow{Then a Lesson of the New Testament, as it is appointed. And after that \emph{Nunc dimittis} (or the Song of Symeon) in English, as followeth.}
\subsection{\stylesubsec{The Song of Symeon}{Nunc dimittis.}{St.~Luke ij.~29.}}

\drop{Lord, \cross now lettest thou thy servant depart in peace : according to thy word.}

2 For mine eyes have seen : thy salvation,

3 Which thou hast prepared : before the face of all people;

4 To be a light to lighten the Gentiles : and to be the glory of thy people Israel.

Glory be to the Father, and to the Son : and to the Holy Ghost;

As it was in the beginning, is now, and ever shall be : world without end. Amen.

\medskip

\pilcrow{Then shall be sung or said the Apostles' Creed, by the Minister and the people standing.}
\drop{I believe in God the Father Almighty, Maker of heaven and earth :}

And in Jesus Christ his only Son our Lord: Who was conceived by the Holy Ghost, Born of the Virgin Mary: Suffered under Pontius Pilate, Was crucified, dead, and buried: He descended into hell; The third day he rose again from the dead: He ascended into heaven, And sitteth on the right hand of God the Father Almighty: From thence he shall come to judge the quick and the dead.

I believe in the Holy Ghost: The holy Catholick Church; The Communion of Saints: The Forgiveness of sins: The Resurrection of the body, And the Life everlasting. Amen.

\medskip

\pilcrow{And after that, these Prayers following, all devoutly kneeling,}

\centerline{Lord, have mercy upon us.}
\centerline{\emph{Christ, have mercy upon us.}}
\centerline{Lord, have mercy upon us.}

\pilcrow{Then the Minister, Clerks, and people shall say the Lord's Prayer with a loud voice.}
\drop{Our Father, which art in heaven, Hallowed be thy Name. Thy kingdom come. Thy will be done in earth, As it is in heaven. Give us this day our daily bread. And forgive us our trespasses, As we forgive them that trespass against us. And lead us not into temptation, But deliver us from evil. Amen.}

\pilcrow{Then the Priest standing up shall say,}

\V O Lord, shew thy mercy upon us.  \R And grant us thy salvation.

\V O Lord, save the \emph{State}.  \R And mercifully hear us when we call upon thee.

\V Endue thy Ministers with righteousness.  \R And make thy chosen people joyful.

\V  O Lord, save thy people.  \R And bless thine inheritance.

\V Give peace in our time, O Lord.  \R Because there is none other that fighteth for us, but only thou, O God.

\V O God, make clean our hearts within us.  \R And take not thy Holy Spirit from us.

\V The Lord be with you.  \R And with thy spirit.

\centerline{Let us pray.}


\pilcrow{Then shall follow three Collects ; the first of the day; The second for Peace ; The third for Aid against all Perils, as hereafter followeth : which two last Collects shall be daily said at Evening Prayer without alteration.}

\subsection{\stylesubsec{The second Collect, }{for Peace.}{}}
\drop{O God, from whom all holy desires, all good counsels, and all just works do proceed; Give unto thy servants that peace which the world cannot give; that both our hearts may be set to obey thy commandments, and also that by thee, we, being defended from the fear of our enemies, may pass our time in rest and quietness; through the merits of Jesus Christ our Saviour. \R Amen.}

\subsection{\stylesubsec{The third Collect, }{for  Aid against all Perils.}{}}
\drop{Lighten our darkness, we beseech thee, O Lord; and by thy great mercy defend us from all perils and dangers of this night; for the love of thy only Son, our Saviour, Jesus Christ. \R Amen.}

\V The Lord be with you.  \R And with thy spirit.

\V Let us bless the Lord.  \R Thanks be to God.

\medskip

\pilcrow{In Quires and Places where they sing here followeth the Anthem.}

\pilcrow{Here may follow any of the Occasional Prayers and Thanksgivings, as need may require, ending with one of the Conclusions.}

\subsubsection{Here endeth the Order of Evening Prayer throughout the Year.}

\fleuron
\chapter{Quicunque vult} %proofed 12-26-21 vs anglican breviary
\pilcrow{Upon these Feasts; Christmas Day, the Epiphany, Easter Day, Ascension Day, Whitsunday, and upon Trinity Sunday, shall be sung or said at Morning Prayer, instead of the Apostles' Creed, this Confession of our Christian Faith, commonly called the Creed of Athanasius, by the Minister and people standing.}
\pilcrow{\emph{Quicunque vult} may also be sung or said at Mattins upon these Feasts; Saint Matthias, Saint John Baptist, Saint James, Saint Bartholomew, Saint Matthew, Saint Simon and Saint Jude, and Saint Andrew}

\subsubsection{Quicunque vult.}
\drop{Whosoever will be saved,\ \star\ before all things it is necessary that he hold the Catholick Faith.}

2\enspace Which Faith except every one do keep whole and undefiled,\ \star\ without doubt he shall perish everlastingly.

\drop{And the Catholick Faith is this:\ \star\  That we worship one God in Trinity, and Trinity in Unity;}

4\enspace Neither confounding the Persons,\ \star\ nor dividing the Substance.

5\enspace For there is one Person of the Father, another of the Son,\ \star\ and another of the Holy Ghost.

6\enspace But the Godhead of the Father, of the Son, and of the Holy Ghost, is all one;\ \star\ the Glory equal, the Majesty co-eternal.

7\enspace Such as the Father is, such is the Son,\ \star\ and such is the Holy Ghost.

8\enspace The Father uncreate, the Son uncreate,\ \star\ and the Holy Ghost uncreate.

9\enspace The Father incomprehensible, the Son incomprehensible,\ \star\ and the Holy Ghost incomprehensible.

10\enspace The Father eternal, the Son eternal,\ \star\ and the Holy Ghost eternal.

11\enspace And yet they are not three eternals,\ \star\ but one eternal.

12\enspace As also there are not three incomprehensibles, nor three uncreated;\ \star\ but one uncreated, and one incomprehensible.

13\enspace So likewise the Father is Almighty, the Son Almighty,\ \star\ and the Holy Ghost Almighty.

14\enspace And yet they are not three Almighties,\ \star\ but one Almighty.

15\enspace So the Father is God, the Son is God,\ \star\ and the Holy Ghost is God.

16\enspace And yet they are not three Gods,\ \star\ but one God.

17\enspace So likewise the Father is Lord, the Son Lord,\ \star\ and the Holy Ghost Lord.

18\enspace And yet not three Lords,\ \star\ but one Lord.

19\enspace For like as we are compelled by the Christian verity\ \star\ to acknowledge every Person by himself to be both God and Lord;

20\enspace So are we forbidden by the Catholick Religion\ \star\ to say, There be three Gods, or three Lords.

21\enspace The Father is made of none;\ \star\ neither created, nor begotten.

22\enspace The Son is of the Father alone;\ \star\ not made, nor created, but begotten.

23\enspace The Holy Ghost is of the Father and of the Son;\ \star\ neither made, nor created, nor begotten, but proceeding.

24\enspace So there is one Father, not three Fathers; one Son, not three Sons;\ \star\ one Holy Ghost, not three Holy Ghosts.

25\enspace And in this Trinity none is afore, or after other;\ \star\ none is greater, or less than another;

26\enspace But the whole three Persons are co-eternal together,\ \star\ and co-equal.

27\enspace So that in all things, as is aforesaid,\ \star\ the Unity in Trinity and the Trinity in Unity is to be worshipped.

28\enspace He therefore that will be saved\ \star\ must think thus of the Trinity.

\drop{Furthermore, it is necessary to everlasting salvation\ \star\ that he also believe rightly the Incarnation of our Lord Jesus Christ.}

30\enspace For the right Faith is, that we believe and confess\ \star\ that our Lord Jesus Christ, the Son of God, is God and Man;

31\enspace God, of the substance of the Father, begotten before the worlds;\ \star\ and Man of the substance of his Mother, born in the world;

32\enspace Perfect God and perfect Man:\ \star\ of a reasonable soul and human flesh subsisting.

33\enspace Equal to the Father, as touching his Godhead,\ \star\ and inferior to the Father, as touching his manhood;

34\enspace Who, although he be God and Man,\ \star\ yet he is not two, but one Christ;

35\enspace One, not by conversion of the Godhead into flesh,\ \star\ but by taking of the Manhood into God;

36\enspace One altogether; not by confusion of Substance,\ \star\ but by unity of Person.

37\enspace For as the reasonable soul and flesh is one man,\ \star\ so God and Man is one Christ;

38\enspace Who suffered for our salvation:\ \star\ descended into hell, rose again the third day from the dead.

39\enspace He ascended into heaven, he sitteth at the right hand of the Father, God Almighty:\ \star\ from whence he will come to judge the quick and the dead.

40\enspace At whose coming all men will rise again with their bodies,\ \star\ and shall give account for their own works.

41\enspace And they that have done good shall go into life everlasting;\ \star\ and they that have done evil into everlasting fire.

\drop{This is the Catholick Faith,\ \star\ which except a man believe faithfully, he cannot be saved.}

Glory be to the Father, and to the Son,\ \star\ and to the Holy Ghost;

As it was in the beginning, is now, and ever shall be,\ \star\ world without end. Amen.

\fleuron


\chapter{The Litany}

\pilcrow{Here followeth the \emph{Litany}, or General Supplication, to be sung or said upon Sundays, Wednesdays, and Fridays (except on Christmas-day, Easter-day, and Whit Sunday), on the Rogation Days, and at other times when it shall be commanded by the Ordinary.}

\subsubsection{The Invocations.}

% \chant{1}{}{chants/litany/oGodtheFather}
\drop{O God the Father, of héaven : have mercy upon us.}

\emph{O God the Father, of héaven : have mercy upon us.}

O God the Son, Redeemer of the wórld : have mercy upon us.

\emph{O God the Son, Redeemer of the wórld : have mercy upon us.}

O God the Holy Ghost, proceeding from the Father and the Són : have mercy upon us.

\emph{O God the Holy Ghost, proceeding from the Father and the Són : have mercy upon us.}

O holy, blessed, and glorious Trinity, three Persons and one Gód : Have mercy upon us.

\emph{O holy, blessed, and glorious Trinity, three Persons and one Gód : Have mercy upon us.}


\drop{Holy Virgin Mary, Mother of God our Saviour Jesus Chríst:}

\centerline{\emph{Pray for us.}}
% \chant{0}{}{chants/litany/oraProNobis}

All holy Angels and Archangels and all holy orders of blessed Spírits:

\centerline{\emph{Pray for us.}}

All holy Patriarchs and Prophets, Apostles, Martyrs, Confessors, and Virgins, and all the blessed company of Héaven:

\centerline{\emph{Pray for us.}}


\subsubsection{The Deprecations.}

\drop{Remember not, Lord, our offences, nor the offences of our forefathers; neither take thou vengeance of our sins: Spare us, good Lord, spare thy people, whom thou hast redeemed with thy most precious blood, and be not angry with · us for éver:}
% \chant{1}{}{chants/litany/rememberNot}

\centerline{\emph{Spare us, good Lord.}}

From all evil and mischief, from sin, from the crafts and assaults of the devil, from thy wrath, and from everlast·ing damnátion,

\centerline{\emph{Good Lord, deliver us.}}

% \chant{0}{}{chants/litany/liberaNos}
From all blindness of heart; from pride, vainglory, and hypocrisy; from envy, hatred, and malice, and · all uncháritableness,

\centerline{\emph{Good Lord, deliver us.}}

From fornication, and all other deadly sin; and from all the deceits of the world, the flesh, · and the dévil,

\centerline{\emph{Good Lord, deliver us.}}

From lightning and tempest; from earthquake, fire, and flood; from plague, pestilence, and famine; from battle and murder, and from · sudden déath,

\centerline{\emph{Good Lord, deliver us.}}

From all sedition, privy conspiracy, and rebellion; from all false doctrine, heresy, and schism; from hardness of heart, and contempt of thy Word · and Commándment,

\centerline{\emph{Good Lord, deliver us.}}



\subsubsection{The Obsecrations.}
\drop{By the mystery of thy holy Incarnation; by thy holy Nativity and Circumcision; by thy Baptism, Fasting, · and Temptátion,}

\centerline{\emph{Good Lord, deliver us.}}

By thine Agony and Bloody Sweat; by thy Cross and Passion; by thy precious Death and Burial; by thy glorious Resurrection and Ascension, and by the Coming of the · Holy Ghóst,

\centerline{\emph{Good Lord, deliver us.}}

In all time of our tribulation; in all time of our wealth; in the hour of death, and in the · day of júdgment,

\centerline{\emph{Good Lord, deliver us.}}


\subsubsection{The Intercessions.}
\drop{We sinners do beseech thee to hear us, O Lord God; and that it may please thee to rule and govern thy holy Church universal in · the right wáy;}
% \chant{0}{}{chants/litany/teRogamus}

\centerline{\emph{We beseech thee to hear us, good Lord.}}

That it may please thee so to rule the heart of thy \emph{servant}, The President of the United States, that \emph{he} may above all things seek thy ho·nour and glóry;

\centerline{\emph{We beseech thee to hear us, good Lord.}}

That it may please thee to bless and preserve all Rulers and Magistrates, giving them grace to execute justice, and to · maintain trúth;

\centerline{\emph{We beseech thee to hear us, good Lord.}}

That it may please thee to illuminate all Bishops, Priests, and Deacons, with true knowledge and understanding of thy Word; and that both by their preaching and living they may set it forth, and shew · it accórdingly;

\centerline{\emph{We beseech thee to hear us, good Lord.}}

\begin{leftbar}
That it may please thee to bless thy \emph{servants and handmaidens} at this time \begin{wrapfigure}{r}{0.46\textwidth}\rubric{To be used in the Ember Weeks, and on the day of an Ordination.}\end{wrapfigure} [to be] admitted to the Order of Deacons or of Priests, and to pour thy grace upon them; that they may duly execute their office to the edifying of thy Church, and to the glory of thy holy name; 
\centerline{\emph{We beseech thee to hear us, good Lord.}}
\end{leftbar}
That it may please thee to send forth labourers in·to thy hárvest;

\centerline{\emph{We beseech thee to hear us, good Lord.}}

That it may please thee to bless and keep · all thy péople;

\centerline{\emph{We beseech thee to hear us, good Lord.}}

That it may please thee to give to all nations unity, · peace, and cóncord;

\centerline{\emph{We beseech thee to hear us, good Lord.}}

That it may please thee to give us an heart to love and dread thee, and diligently to live after · thy commándments;

\centerline{\emph{We beseech thee to hear us, good Lord.}}

That it may please thee to give to all thy people increase of grace to hear meekly thy Word, and to receive it with pure affection, and to bring forth the fruits · of the Spírit;

\centerline{\emph{We beseech thee to hear us, good Lord.}}

That it may please thee to bring into the way of truth all such as have erred, and · are decéived;

\centerline{\emph{We beseech thee to hear us, good Lord.}}

That it may please thee to strengthen such as do stand; and to comfort and help the weak-hearted; and to raise up them that fall; and finally to beat down Satan un·der our féet;

\centerline{\emph{We beseech thee to hear us, good Lord.}}

That it may please thee to succour, help, and comfort, all that are in danger, necessity, and · tribulátion;

\centerline{\emph{We beseech thee to hear us, good Lord.}}

That it may please thee to preserve all who travel by land, by water, or by air; all women labouring of child, all sick persons, and young children; and to shew thy pity upon all prison·ers and cáptives; 

\centerline{\emph{We beseech thee to hear us, good Lord.}}

That it may please thee to defend, and provide for, the fatherless children, and widows, and all that are desolate · and oppréssed;

\centerline{\emph{We beseech thee to hear us, good Lord.}}

That it may please thee to have mercy · upon áll men;

\centerline{\emph{We beseech thee to hear us, good Lord.}}

That it may please thee to forgive our enemies, persecutors, and slanderers, and to · turn their héarts;

\centerline{\emph{We beseech thee to hear us, good Lord.}}

That it may please thee to give and preserve to our use the kindly fruits of the earth, so that in due time we · may enjóy them;

\centerline{\emph{We beseech thee to hear us, good Lord.}}

That it may please thee to give us true repentance; to forgive us all our sins, negligences, and ignorances; and to endue us with the grace of thy Holy Spirit to amend our lives according to thy · holy Wórd;

\centerline{\emph{We beseech thee to hear us, good Lord.}}


\subsubsection{The Conclusion.}
% \chant{1}{}{chants/litany/sonOfGod}\par%\end

\drop{Son of God : we beseech thee to hear us.}

\centerline{\emph{Son of God : we beseech thee to hear us.}}

O Lamb of God : that takest away the sins of the world;

\centerline{\emph{Grant us thy peace.}}

O Lamb of God : that takest away the sins of the world;

\centerline{\emph{Have mercy upon us.}}

\centerline{O Christ, hear us.}
\centerline{\emph{O Christ, hear us.}}

\centerline{Lord, have mercy upon us.}
\centerline{\emph{Lord, have mercy upon us.}}
\centerline{Christ, have mercy upon us.}
\centerline{\emph{Christ, have mercy upon us.}}
\centerline{Lord, have mercy upon us.}
\centerline{\emph{Lord, have mercy upon us.}}


% \chant{1}{}{chants/litany/agnusDei2}
\noindent
\pilcrow{Then shall the Priest, and the people with him, say the Lord’s Prayer.}
\drop{Our Father, which art in heaven, Hallowed be thy Name. Thy kingdom come. Thy will be done in earth, As it is in heaven. Give us this day our daily bread. And forgive us our trespasses, As we forgive them that trespass against us.  And lead us not into temptation.  But deliver us from evil. Amen.}


\pilcrow{On ordinary days, the \emph{Litany} continues at \emph{The Station} on pg.\ \emph{\pageref{litanyStation}.}}

\section{A Supplication}
\pilcrow{On Rogation Days, during penitential seasons, and in times of trouble, the Litany may continue thus,}

\V O Lord, deal not with us after our síns.  \R Neither reward us according to our iníquities.

\centerline{Let us pray.}
\drop{O God, merciful Father, that despisest not the sighing of a contrite heart, nor the desire of such as be sorrowful; Mercifully assist our prayers which we make before thee in all our troubles and adversities, whensoever they oppress us; and graciously hear us, that those evils which the craft and subtilty of the devil or man worketh against us be brought to nought; and by the providence of thy goodness they may be dispersed; that we thy servants, being hurt by no persecutions, may evermore give thanks unto thee in thy holy Church; through Jesus Christ our Lord.  \R Amen.}


% ¶ The following is then sung, usually proceeding to the Altar (when sung in procession).  When the procession is long, as on Rogation Days, other verses of {Psalm 44} may be sung as well.

% \chant{1}{ij.}{chants/litany/oLordArise}
\ant O Lord, arise, help us, and deliver us for thy name’s sake.

\V O God, we have heard with our ears, and our fathers have • declared únto us, * the noble works that thou didst in their days, and in the • old time before them.

\ant O Lord, arise, help us, and deliver us for thine honour.

\V Glory be to the Father, and to the Son, • and to the Hóly Ghost; † As it was in the beginning, is now, • and ever sháll be, * world • without end. Amen.

\ant O Lord, arise, help us, and deliver us for thy name’s sake.

% \chant{1}{\V}{chants/litany/suffrages}

\V From our enemies defend us, O Chríst.  \R Graciously look upon our afflíctions.

\V Pitifully behold the sorrows of our héarts.  \R Mercifully forgive the sins of thy péople.

\V Favourably with mercy hear our práyers.  \R O Son of David, have mercy upon ús.

\V Both now and ever vouchsafe to hear us, O Chríst.  \R Graciously hear us, O Christ; graciously hear us, O Lord Chríst.


\section{The Station}
\label{litanyStation}

\V O Lord, let thy mercy be shewed upon ús; \R As we do put our trust in thée.

\centerline{Let us pray.}

\drop{We humbly beseech thee, O Father, mercifully to look upon our infirmities; and, for the glory of thy Name, turn from us all those evils that we most righteously have deserved; and grant, that in all our troubles we may put our whole trust and confidence in thy mercy, and evermore serve thee in holiness and pureness of living, to thy honour and glory; through our only Mediator and Advocate, Jesus Christ our Lord. \R Amen.}


\pilcrow{The following two prayers, \emph{For the State} and \emph{For the Church}, may be replaced on Wednesdays and Fridays with \emph{Other Prayers} below.}


\subsubsection{A Prayer for \emph{The President of the United States}, and all in Civil Authority.}


\drop{O Lord, our heavenly Father, the high and mighty Ruler of the universe, who dost from thy throne behold all the dwellers upon earth; Most heartily we beseech thee with thy favour to behold and bless thy \emph{servant} The President of the United States, and all others in authority; and so replenish them with the grace of thy Holy Spirit, that they may always incline to thy will, and walk in thy way. Endue them plenteously with heavenly gifts; grant them in health and prosperity long to live; and finally, after this life, to attain everlasting joy and felicity; through Jesus Christ our Lord. \R Amen.}


\subsubsection{A Prayer for the Clergy and People.}
\drop{Almighty and everlasting God, who alone workest great marvels; Send down upon our Bishops, and Clergy, and all Congregations committed to their charge, the healthful Spirit of thy grace; and that they may truly please thee, pour upon them the continual dew of thy blessing. Grant this, O Lord, for the honour of our Advocate and Mediator, Jesus Christ. \R Amen.}


\pilcrow{Other approved prayers may be included, \emph{ad libitum.}}


\pilcrow{During Ember Weeks, the prayer \emph{9. For them that are to be admitted into Holy Orders} shall be said here.}

\subsubsection{A Prayer for Mercy.}
\drop{O God, whose nature and property is ever to have mercy and to forgive, receive our humble petitions; and though we be tied and bound with the chain of our sins, yet let the pitifulness of thy great mercy loose us; for the honour of Jesus Christ, our Mediator and Advocate. \R Amen.}

\label{chrysostom}
\subsubsection{A Prayer of St. Chrysostom.}
\drop{Amighty God, who hast given us grace at this time with one accord to make our common supplications unto thee; and dost promise, that when two or three are gathered together in thy Name thou wilt grant their requests; Fulfil now, O Lord, the desires and petitions of thy servants, as may be most expedient for them; granting us in this world knowledge of thy truth, and in the world to come life everlasting. \R Amen.}

\subsubsection{II Corinthians xiij.}
\drop{The grace \cross of our Lord Jesus Christ, and the love of God, and the fellowship of the Holy Ghost, be with us all evermore. \R Amen.}


\fleuron

\chapter[Prayers and Thanksgivings]{Prayers and Thanksgivings upon several Occasions}
\subsubsection{To be used before the final two prayers at the Litany, or after Mattins or Evensong; or at other times.}
\label{prayers}


\subsection{For the State}
%Am1928
\subseccaption{1.}{A Prayer for Congress, to be used during their Session.}
\drop{Most gracious God, we humbly beseech thee, as for the people of these United States in general, so especially for their Senate and Representatives in Congress assembled; that thou wouldest be pleased to direct and prosper all their consultations, to the advancement of thy glory, the good of thy Church, the safety, honour, and welfare of thy people; that all things may be so ordered and settled by their endeavours, upon the best and surest foundations, that peace and happiness, truth and justice, religion and piety, may be established among us for all generations. These and all other necessaries, for them, for us, and thy whole Church, we humbly beg in the Name and mediation of Jesus Christ, our most blessed Lord and Saviour. \R Amen.}


%Am1928
\subseccaption{2.}{For Courts of Justice.}
\drop{Almighty God, who sittest in the throne judging right; We humbly beseech thee to bless the courts of justice and the magistrates in all this land; and give unto them the spirit of wisdom and understanding, that they may discern the truth and impartially administer the law in the fear of thee alone; through him who shall come to be our judge, thy Son, our Saviour, Jesus Christ. \R Amen.}


%Am1928
\subseccaption{3.}{For a State Legislature.}
\drop{O God, the fountain of wisdom, whose statutes are good and gracious and whose law is truth; We beseech thee so to guide and bless the Legislature of this State, that it may ordain for our governance only such things as please thee, to the glory of thy Name and the welfare of the people; through Jesus Christ, thy Son, our Lord. \R Amen.}

%Am1928
\subseccaption{4.}{For Our Country.}
\drop{Almighty God, who hast given us this good land for our heritage; We humbly beseech thee that we may always prove ourselves a people mindful of thy favour and glad to do thy will. Bless our land with honourable industry, sound learning, and pure manners. Save us from violence, discord, and confusion; from pride and arrogancy, and from every evil way. Defend our liberties, and fashion into one united people the multitudes brought hither out of many kindreds and tongues. Endue with the spirit of wisdom those to whom in thy Name we entrust the authority of government, that there may be justice and peace at home, and that, through obedience to thy law, we may shew forth thy praise among the nations of the earth. In the time of prosperity, fill our hearts with thankfulness, and in the day of trouble, suffer not our trust in thee to fail; all which we ask through Jesus Christ our Lord. \R Amen.}


\subsection{For the Church}
%Am1928
\subseccaption{5.}{For the Church.}
\drop{O Gracious Father, we humbly beseech thee for thy holy Catholic Church; that thou wouldst be pleased to fill it with all truth, in all peace. Where it is corrupt, purify it; where it is in error, direct it; where in anything it is amiss, reform it. Where it is right, establish it; where it is in want, provide for it; where it is divided, reunite it; for the sake of him who died and rose again, and ever liveth to make intercession for us, Jesus Christ, thy Son, our Lord. \R Amen.}


%Eng1923
\subseccaption{6.}{For Unity.}
\drop{O Lord Jesus Christ, who didst say to thine Apostles, Peace I leave with you, my peace I give unto you: Regard not our sins but the faith of thy Church, and grant it that peace and unity which is agreeable to thy will; who livest and reignest with the Father and the Holy Spirit, one God, world without end.  \R Amen.}

%Eng1923, Am1928
\subseccaption{7.}{For the Increase of the Sacred Ministry.}
\drop{O almighty God, look mercifully upon the world which thou hast redeemed by the blood of thy dear Son, and incline the hearts of many to the ministry of thy Church, so that by their labours thy light may shine in the darkness, and the kingdom of thy Son be hastened by the perfecting of thine elect; through Jesus Christ our Lord. \R Amen.}


%Eng1923
\subseccaption{8.}{For Candidates for Confirmation.}
\drop{O God, who through the teaching of thy Son Jesus Christ didst prepare the disciples for the coming of the Comforter; Make ready, we beseech thee, the hearts and minds of thy \emph{servants} who at this time are seeking the gifts of the Holy Ghost through the laying on of hands, that, drawing near with penitent and faithful hearts, they may be filled with the power of his divine indwelling; through the same Jesus Christ our Lord. \R Amen.}


%Eng1662
\subseccaption{9.}{In the Ember Weeks, to be said every day, for them that are to be admitted into Holy Orders.}
\drop{Almighty God, our heavenly Father, who hast purchased to thyself an universal Church by the precious blood of thy dear Son; Mercifully look upon the same, and at this time so guide and govern the minds of thy servants the Bishops and Pastors of thy flock, that they may lay hands suddenly on no man, but faithfully and wisely make choice of fit persons to serve in the sacred Ministry of thy Church. And to those which shall be ordained to any holy function give thy grace and heavenly benediction; that both by their life and doctrine they may set forth thy glory, and set forward the salvation of all men; through Jesus Christ our Lord. \R Amen.}


%Sc1912
\subseccaption{10.}{For Synods and Chapters of the Church.}
\drop{O Eternal God, the fountain of all wisdom, who didst send thy Holy Spirit to lead the disciples into all the truth; Vouchsafe that he being present with thy servants and handmaidens, the Bishops [\emph{or} Bishop] and Presbyters about to assemble [\emph{or} now assembled] in the Synod of this jurisdiction, may so rule their hearts and guide their counsels that in all things they may seek only thy glory and the good of thy holy Church; through Jesus Christ our Lord. \R Amen.}


%En1662
\subsection{For the Natural Order}
\subseccaption{11.}{For Rain.}
\drop{O God, heavenly Father, who by thy Son Jesus Christ hast promised to all them that seek thy kingdom, and the righteousness thereof, all things necessary to their bodily sustenance; Send us, we beseech thee, in this our necessity, such moderate rain and showers, that we may receive the fruits of the earth to our comfort, and to thy honour; through Jesus Christ our Lord. \R Amen.}


%Sc1912
\subseccaption{12.}{For Fair Weather.}
\drop{Almighty God, our Heavenly Father, who art the author and giver of all good things; Look, we beseech thee, in thy loving-kindness upon us thine unworthy servants, and grant to us at this time such fair weather that we may receive the fruits of the earth in their season, to our comfort and the glory of thy holy Name, through Jesus Christ, our Mediator and Advocate.  \R Amen.}

%Eng1923
\subseccaption{13.}{In the time of Dearth and Famine.}
\drop{O God, our heavenly Father, who by thy blessed Son hast taught us to ask our daily bread of thee; Behold, we beseech thee, the affliction of thy people, and send us a seasonable relief in this our necessity.  Increase the fruits of the earth by thy heavenly benediction; and grant that we, receiving with thankfulness thy gracious gifts, may use the same to thy glory, the relief of those that are needy, and our own comfort; through Jesus Christ our Lord. \R Amen.}


%Eng1923
\subseccaption{14.}{In the time of any common Plague or Sickness.}
\drop{Grant, we beseech thee, merciful Lord, help and deliverance unto us, who are visited with grievous mortality and sickness. Sanctify to us this our sore distress, and prosper with thy continual blessing those who labour to devise for mankind protection against disease and pain; through him who both healed and glorified pain, thy Son Jesus Christ our Lord. \R Amen.}



%Am1928
\subseccaption{15.}{In Time of Calamity.}
\drop{O God, merciful and compassionate, who art ever ready to hear the prayers of those who put their trust in thee; Graciously hearken to us who call upon thee, and grant us thy help in this our need; through Jesus Christ our Lord. \R Amen.}

\subseccaption{16.}{On the Rogation Days}
%En1923
\drop{Almighty God, who hast blessed the earth that it should be fruitful and bring forth abundantly whatsoever is needful for the life of man: Prosper, we beseech thee, the labours of the husbandman, and grant such seasonable weather that we may gather in the fruits of the earth and ever rejoice in thy goodness, to the praise of thy holy Name; through Jesus Christ our Lord. \R Amen.}


%En1923
\drop{O almighty God, who hast made the sea and all that moveth therein: Bestow thy blessing on the harvest of the waters, that it may be abundant in its season, and on our fishermen and mariners, that they may be safe in every peril of the deep; so that we all with thankful hearts may acknowledge thee who art the Lord of the sea and of the dry land; through Jesus Christ our Lord. \R Amen.}


%En1923
\drop{Almighty Father, who by thy Son Jesus Christ hast sanctified labour to the welfare of mankind: Prosper, we pray thee, the industries of this land and all those who are engaged therein; that shielded in all their temptations and dangers, and receiving a rich reward of their labours, they may praise thee by living according to thy will; through Jesus Christ our Lord. \R Amen.}


\subsection{For the Social Order}
%Am1928
\subseccaption{17.}{For Children.}
\drop{O Lord, Jesus Christ, who dost embrace children with the arms of thy mercy, and dost make them living members of thy Church; Give them grace, we pray thee, to stand fast in thy faith, to obey thy word, and to abide in thy love; that being made strong by thy Holy Spirit they may resist temptation and overcome evil; and may rejoice in the life that now is, and dwell with thee in the life that is to come; through thy merits, O merciful Saviour, who with the Father and the Holy Ghost livest and reignest one God, world without end. \R Amen.}


%Am1928
\subseccaption{18.}{For Every Man in his Work.}
\drop{Almighty God, our heavenly Father, who declarest thy glory and shewest forth thy handiwork in the heavens and in the earth; Deliver us, we beseech thee, in our several callings, from the service of mammon, that we may do the work which thou givest us to do, in truth, in beauty, and in righteousness, with singleness of heart as thy servants, and to the benefit of our fellow men; for the sake of him who came among us as one that serveth, thy Son, Jesus Christ our Lord. \R Amen.}


%Am1928
\subseccaption{19.}{For a Person under Affliction.}
\drop{O merciful God, and heavenly Father, who hast taught us in thy holy Word that thou dost not willingly afflict or grieve the children of men; Look with pity, we beseech thee, upon the sorrows of thy \emph{servant} for whom our prayers are offered. Remember \emph{him}, O Lord, in mercy; endue \emph{his} soul with patience; comfort \emph{him} with a sense of thy goodness; lift up thy countenance upon \emph{him}, and give \emph{him} peace; through Jesus Christ our Lord. \R Amen.}


%Am1928
\subseccaption{20.}{For Christian Service.}
\drop{O Lord our heavenly Father, whose blessed Son came not to be ministered unto, but to minister; We beseech thee to bless all who, following in his steps, give themselves to the service of their fellow men. Endue them with wisdom, patience, and courage, that they may strengthen the weak and raise up those who fall; and, being inspired by thy love, may worthily minister in thy Name to the suffering, the friendless, and the needy; for the sake of him who laid down his life for us, the same thy Son, our Saviour, Jesus Christ. \R Amen.}


%Am1928
\subseccaption{21.}{For Social Justice.}
\drop{Almighty God, who hast created man in thine own image; Grant us grace fearlessly to contend against evil, and to make no peace with oppression; and, that we may reverently use our freedom, help us to employ it in the maintenance of justice among men and nations, to the glory of thy holy Name; through Jesus Christ our Lord. \R Amen.}


%Am1928
\subseccaption{22.}{For the Family of Nations.}
\drop{Almighty God, our heavenly Father, guide, we beseech thee, the Nations of the world into the way of justice and truth, and establish among them that peace which is the fruit of righteousness, that they may become the Kingdom of our Lord and Saviour Jesus Christ. \R Amen.}



%Eng1923
\subseccaption{23.}{In the time of War and Tumults.}
\drop{O almighty Lord, who art a most strong tower to all them that put their trust in thee: Be now and evermore our defence: look in pity upon the wounded and the prisoners; cheer the anxious; comfort the bereaved; succour the dying; have mercy on the fallen; and hasten the time when war shall cease in all the world; through Jesus Christ our Lord. \R Amen.}


%Am1928
\subseccaption{24.}{For Soldiers.}
\drop{O Lord God of Hosts, stretch forth, we pray thee, thine almighty arm to strengthen and protect the soldiers of our country; Support them in the day of battle, and in the time of peace keep them safe from all evil; endue them with courage and loyalty; and grant that in all things they may serve without reproach; through Jesus Christ our Lord. \R Amen.}


%Am1928
\subseccaption{25.}{Memorial Days.}
\drop{Almighty God, our heavenly Father, in whose hands are the living and the dead; We give thee thanks for all those thy servants who have laid down their lives in the service of our country. Grant to them thy mercy and the light of thy presence, that the good work which thou hast begun in them may be perfected; through Jesus Christ thy Son our Lord. \R Amen.}


%Sc1919
\subseccaption{26.}{In Commemoration of the Faithful Departed.}
\drop{O almighty God, the God of the spirits of all flesh, who by a voice from heaven didst proclaim, Blessed are the dead who die in the Lord: Multiply, we beseech thee, to those who rest in Jesus, the manifold blessings of thy love, that the good work which thou didst begin in them may be perfected unto the day of Jesus Christ. And of thy mercy, O heavenly Father, vouchsafe that we, who now serve thee here on earth, may at the last, together with them, be found meet to be partakers of the inheritance of the saints in light; for the sake of the same thy Son, Jesus Christ, our Lord and Saviour.  \R Amen.}



\subseccaption{27.}{A Collect or Prayer for all Conditions of Men, to be used at such times when the Litany is not appointed to be said.}
\lettrine{O}{ God,} the Creator and Preserver of all mankind, we humbly beseech thee for all sorts and conditions of men: that thou wouldest be pleased to make thy ways known unto them, thy saving health unto all nations. More especially, we pray for the good estate of the Catholick Church; that it may be so guided and governed by thy good Spirit, that all who profess and call themselves Christians may be led into the way of truth, and hold the faith in unity of spirit, in the bond of peace,
% \begin{wrapfigure}{r}{0.46\textwidth}\rubric{* This to be said when any desire the Prayers of the Congregation.}\end{wrapfigure} 
and in righteousness of life. Finally, we commend to thy fatherly goodness all those, who are any ways afflicted, or distressed, in
% \begin{floatingfigure}{0.46\textwidth}\rubric{* This to be said when any desire the Prayers of the Congregation.}\end{floatingfigure}
mind, body, or estate; [*\emph{especially those for whom our prayers are desired;}] that it may please thee to comfort and relieve them, according to their several necessities, giving them patience under their sufferings, and a happy issue out of all their afflictions. And this we beg for Jesus Christ his sake. \R Amen.
\bigskip

\section{Thanksgivings}
\subseccaption{1.}{A General Thanksgiving.}
% \begin{wrapfigure}{r}{0.46\textwidth}\rubric{* This is to be said when any that have been prayed for desire to return praise.}\end{wrapfigure}
\lettrine{A}{lmighty} God, Father of all mercies, we thine unworthy servants do give thee most humble and hearty thanks for all thy goodness and loving-kindness to us, and to all men;  [*\emph{particularly to those who desire now to offer up their praises and thanksgivings for thy late mercies vouchsafed unto them.}] We bless thee for our creation, preservation, and all the blessings of this life; but above all, for thine inestimable love in the redemption of the world by our Lord Jesus Christ; for the means of grace, and for the hope of glory. And, we beseech thee, give us that due sense of all thy mercies, that our hearts may be unfeignedly thankful, and that we shew forth thy praise, not only with our lips, but in our lives; by giving up ourselves to thy service, and by walking before thee in holiness and righteousness all our days; through Jesus Christ our Lord, to whom with thee and the Holy Ghost be all honour and glory, world without end. \R Amen.


%Am1928
\subseccaption{2.}{For Rain}
\drop{O God our heavenly Father, who by thy gracious providence dost cause the former and the latter rain to descend upon the earth, that it may bring forth fruit for the use of man; We give thee humble thanks that it hath pleased thee to send us rain to our great comfort, and to the glory of thy holy Name; through thy mercies in Jesus Christ our Lord. \R Amen.}


%Eng1923
\subseccaption{3.}{For Seasonable Weather.}
\drop{O Lord God, who hast in thy mercy relieved and comforted thy servants by this seasonable change of weather: We yield thee hearty thanks for this thy goodness towards us, beseeching thee to give us grace to use all thy mercies to the honour and glory of thy holy Name; through Jesus Christ our Lord. \R Amen.}


%Eng1923
\subseccaption{4.}{For the Blessings of Harvest.}
\drop{O Lord God Almighty, the Creator and Father of all, we yield thee hearty thanks that thou hast ordained for mankind both seedtime and harvest, and dost now bestow upon us thy children the fruits of the earth in their season. For these and all other thy mercies we laud and magnify thy glorious Name; through Jesus Christ our Lord, to whom, with thee and the Holy Ghost, be all honour and glory, now and for evermore. \R Amen.}


%Eng1662
\subseccaption{5.}{For Plenty.}
\drop{O most merciful Father, who of thy gracious goodness hast heard the devout prayers of thy Church, and turned our dearth and scarcity into cheapness and plenty; We give thee humble thanks for this thy special bounty; beseeching thee to continue thy loving-kindness unto us, that our land may yield us her fruits of increase, to thy glory and our comfort; through Jesus Christ our Lord. \R Amen.}



%Eng1662, Am1928
\subseccaption{6.}{For Peace, and Deliverance from our Enemies.}
\drop{O almighty God, who art a strong tower of defence unto thy servants against the face of their enemies; We yield thee praise and thanksgiving for our deliverance from those great and apparent dangers wherewith we were compassed. We acknowledge it thy goodness that we were not delivered over as a prey unto them; beseeching thee still to continue such thy mercies towards us, that all the world may know that thou art our Saviour and mighty Deliverer; through Jesus Christ our Lord. \R Amen.}


%Am1928, Eng1662
\subseccaption{7.}{For restoring Publick Peace at Home.}
\drop{O eternal God, our heavenly Father, Who alone makest men to be of one mind in a house, and stillest the outrage of a violent and unruly people; We bless thy holy Name, that it hath pleased thee to appease the seditious tumults which have been lately raised up amongst us; most humbly beseeching thee to grant to all of us grace, that we may henceforth obediently walk in thy holy commandments; and, leading a quiet and peaceable life in all godliness and honesty, may continually offer unto thee our sacrifice of praise and thanksgiving for these thy mercies towards us; through Jesus Christ our Lord. \R Amen.}


\subseccaption{7.}{For Deliverance from Common Sickness.}
\drop{O Lord God, who dost not willingly afflict the children of men: We most heartily thank thee that in thy mercy thou hast delivered us from sickness and affliction, and with grateful hearts we desire to offer unto thy fatherly goodness ourselves, our souls and bodies, to be a living sacrifice unto thee, always praising and magnifying thy loving-kindness in the midst of thy Church; through Jesus Christ our Lord. \R Amen.}

\section{Conclusions}

\subseccaption{1.}{II Corinthians xiij.}
\drop{The grace \cross of our Lord Jesus Christ, and the love of God, and the fellowship of the Holy Ghost, be with us all evermore. \R Amen.}


\subseccaption{2.}{Numbers vj.}
\drop{The Lord bless us and keep us. The Lord make his face to shine upon us, and be gracious unto us. The Lord lift up the light of his countenance upon us and give us peace,  \cross now and for evermore. \R Amen.}


\subseccaption{3.}{At Night.}
\drop{The almighty and merciful God bless \cross us and keep us this night and for evermore. \R Amen.}


\subseccaption{4.}{For the Departed.}
\drop{May the souls \cross of the faithful, through the mercy of God, rest in peace. \R Amen.}

%Am1928; conclusion Sc1912
\section{A Bidding Prayer}
\subsubsection{To be used before Sermons, or on Special Occasions.}

\pilcrow{And NOTE, That the Minister, in his discretion, may omit any of the clauses in this Prayer, or may add others, as occasion may require.}

\drop{Good Christian People, I bid your prayers for Christ’s holy Catholic Church, the blessed company of all faithful people; that it may please God to confirm and strengthen it in purity of faith, in holiness of life, and in perfectness of love, and to restore to it the witness of visible unity; and more especially for that branch of the same planted by God in this land, whereof we are members; that in all things it may work according to God’s will, serve him faithfully, and worship him acceptably.}

Ye shall pray for the President of these United States, and for the Governor of this State, and for all that are in authority; that all, and every one of them, may serve truly in their several callings to the glory of God, and the edifying and well-governing of the people, remembering the account they shall be called upon to give at the last great day.

Ye shall also pray for the ministers of God’s Holy Word and Sacraments; for Bishops (and herein more especially for our Bishop), that they may minister faithfully and wisely the discipline of Christ; likewise for all Priests and Deacons, that they may shine as lights in the world, and in all things may adorn the doctrine of God our Saviour.

And ye shall pray for a due supply of persons fitted to serve God in the Ministry and in the State; and to that end, as well as for the good education of all the youth of this land, ye shall pray for all schools, colleges, and seminaries of sound and godly learning, and for all whose hands are open for their maintenance; that whatsoever tends to the advancement of true religion and useful learning may for ever flourish and abound.

Ye shall pray for all the people of these United States, that they may live in the true faith and fear of God, and in brotherly charity one towards another.

Ye shall pray also for all who travel by land, sea, or air; for all prisoners and captives; for all who are in sickness or in sorrow; for all who have fallen into grievous sin; for all who, through temptation, ignorance, helplessness, grief, trouble, dread, or the near approach of death, especially need our prayers.

Ye shall also praise God for rain and sunshine; for the fruits of the earth; for the products of all honest industry; and for all his good gifts, temporal and spiritual, to us and to all men.

Finally, ye shall yield unto God most high praise and hearty thanks for the wonderful grace and virtue declared in all his saints, who have been the choice vessels of his grace and the lights of the world in their several generations; and pray unto God, that we may have grace to direct our lives after their good examples; that, this life ended, we may be made partakers with them of the glorious resurrection, and the life everlasting.

All which things let us humbly ask in the words which Christ himself hath taught us, saying:

\drop{Our Father, which art in heaven, Hallowed be thy Name. Thy kingdom come. Thy will be done in earth, As it is in heaven. Give us this day our daily bread. And forgive us our trespasses, As we forgive them that trespass against us.  And lead us not into temptation.  But deliver us from evil. Amen.}
 

\fleuron

O merciful God, ........ of the Lord; and ..... may indeed be ...... , one God, world....
Merciful Lord, hear the prayers of thy servants who commemorate the Nativity of the Mother of God; and grant that by the incarnation of thy dear Son, we may be indeed made nigh into him, who liveth and reigneth with thee and the Holy Ghost, ever one God, world without end.  Amen.
The Priest's Book of Private Devotion - 1899
The Day-hours of the Church of England - 1891



\chapter{A Devotion}

{\centering\rubric{which may be said by the Priest and people immediately before the celebration of the Holy Communion.}\par}


\pilcrow{The Priest, standing at God’s Board, shall say with the Ministers and the people, all kneeling, as follows.}

\drop{In the name of the Father, \grecross\ and of the Son, and of the Holy Ghost. Amen.}

\rubric{Anthem.} I will go unto the altar of God, \star\ even unto the God of my joy and gladness.

\subsection{\stylesubsec{Psalm 43.}{Judica me, Deus.}{}}

\drop{Give sentence with me, O God, and defend my cause against the ungódly péople; \star\  O deliver me from the decéitful and wícked man.}

2\enspace For thou art the God of my strength, why hast thou pút me fróm thee? \star\  and why go I so heavily, while the enemý opprésseth me?

3\enspace O send out thy light and thy truth, that théy may léad me, \star\  and bring me unto thy holy hill, and tó thy dwélling.

4\enspace And that I may go unto the altar of God, even unto the God of my jóy and gládness; \star\  and upon the harp will I give thanks unto thée, O Gód, my God.

5\enspace Why art thou so héavy, O my sóul? \star\  and why art thou so disquietéd withín me?

6\enspace O pút thy trust in Gód; \star\  for I will yet give him thanks, which is the help of my cóuntenance, ánd my God.

Glory be to the Father, and to the Son, \star\  and to the Holy Ghost;

As it was in the beginning, is now, and ever shall be, \star\  world without end. Amen.

\rubric{Anthem.} I will go unto the altar of God, even unto the God of my joy and gladness.

\medskip

\V Our help standeth in the name of the Lord;  \R Who hath made heaven and earth.

\medskip

\rubric{If desired, the Confession and Absolution may be said here, and ommitted in the Order of Communion.}

\medskip

\V Wilt thou not turn again and quicken us;  \R That thy people may rejoice in thee?

\V O Lord, show thy mercy upon us;  \R And grant us thy salvation.

\V O Lord, hear our prayer;  \V And let our cry come unto thee.

\V The Lord be with you; \V And with thy spirit.

\centerline{Let us pray.}

\pilcrow{Then shall the Priest proceed with the celebration of the Holy Communion}

\chapter[Holy Communion]{\stylechapter{The Order for the Administration of the Lord's Supper, or}{Holy Communion}{Commonly Called the Mass.}}


\pilcrow{It is an ancient and laudable custom of the Church to receive this Holy Sacrament fasting. Yet for the avoidance of all scruple it is hereby declared that such preparation may be used or not used, according to every man’s conscience in the sight of God.}

\medskip
% \pilcrow{So many as intend to be partakers of the holy Communion shall signify their names to the Curate at least some time the day before.}

% And if any of those be an open and notorious evil liver, or have done any wrong to his neighbours by word or deed, so that the Congregation be thereby offended; the Curate having knowledge thereof, shall call him and advertise him, that in any wise he presume not to come to the Lord's Table, until he have openly declared himself to have truly repented and amended his former naughty life, that the Congregation may thereby be satisfied, which before were offended; and that he have recompensed the parties, to whom he hath done wrong; or at least declare himself to be in full purpose so to do, as soon as he conveniently may.

% The same order shall the Curate use with those betwixt whom he perceiveth malice and hatred to reign; not suffering them to be partakers of the Lord's Table. until he know them to be reconciled. And if one of the parties so at variance be content to forgive from the bottom of his heart all that the other hath trespassed against him, and to make amends for that he himself hath offended; and the other party will not be persuaded to a godly unity, but remain still in his frowardness and malice: the Minister in that case ought to admit the penitent person to the holy Communion, and not him that is obstinate.

% And when any person is warned as in the two precedent paragraphs not to come to the Lord's Table, the Minister shall inform him that the case shall be laid before the Bishop of the Diocese without delay, and that pending the judgement of the Bishop he cannot be admitted to the Holy Communion.

% And on every such occasion as is set forth in the three precedent paragraphs, the Minister shall immediately give an account of the case to the Bishop and shall await his directions. And if occasion require, the Ordinary shall proceed against the offending person according to the Canon.

% ¶ The Priest shall say the Service following in a distinct and audible voice.

\pilcrow{The Holy Table, having at the Communion time a fair white linen cloth upon it, with other decent furniture meet for the high Mysteries there to be celebrated, shall stand at the uppermost part of the Chancel or Church. And the Priest, standing at the Holy Table, shall say the Lord’s Prayer, with the collect following for due preparation, the people kneeling.} % 1912 Scott (this only) 


% 1549: Upon the date and at the tyme appoincted for the ministracion of the holy Communion, the Priest that shal execute the holy ministery, shall put upon hym the vesture appoincted for that ministracion, that is to saye: a white Albe plain, with a vestement or Cope. And where there be many Priestes, or Decons, there so many shalbe ready to helpe the Priest, in the ministracion, as shalbee requisite: And shall have upon them lykewise the vestures appointed for their ministery, that is to saye, Albes with tunacles. Then shall the Clerkes syng in Englishe for the office, or Introite, (as they call it,) a Psalme appointed for that daie.

% 1549: The Priest standing humbly afore the middes of the Altar, shall saie the Lordes praier, with this Collect.

\section{The Introduction}

\ourFather

\subsection{\stylesubsec{}{The Collect.}{}}
\drop{Almighty God, unto whom all hearts be open, all desires known, and from whom no secrets are hid; Cleanse the thoughts of our hearts by the inspiration of thy Holy Spirit, that we may perfectly love thee, and worthily magnify thy holy Name; through Christ our Lord. \R Amen.}

\bigskip
%location from green book; text from am28

\pilcrow{Here may be sung a Hymn or an Anthem.}
% he saie a Psalme appointed for the introite: whiche Psalme ended the Priest shall saye, or els the Clerkes shal syng,}

\bigskip

% iii. Lorde have mercie upon us.
% iii. Christ have mercie upon us.
% iii. Lorde have mercie upon us.

% Then the Prieste standyng at Goddes borde shall begin,

% Glory be to God on high.


\pilcrow{Then shall the Priest, turning to the people, rehearse distinctly all the TEN COMMANDMENTS; and the people, still kneeling, shall after every Commandment ask God mercy for their transgression of every duty therein (either according to the letter or according to the spiritual import thereof) for the time past, and grace to keep the same for the time to come, as followeth.} %1923, includes 1912 Scott

\smallskip

\centerline{God spake these words, and said,}
\drop{I am the Lord thy God: Thou shalt have none other gods but me.}

\R Lord, have mercy upon us, and incline our hearts to keep this law.

II. Thou shalt not make to thyself any graven image, nor the likeness of any thing that is in heaven above, or in the earth beneath, or in the water under the earth. Thou shalt not bow down to them, nor worship them.%: for I the Lord thy God am a jealous God, and visit the sins of the fathers upon the children unto the third and fourth generation of them that hate me, and shew mercy unto thousands in them that love me, and keep my commandments.

\R Lord, have mercy upon us, and incline our hearts to keep this law.

III. Thou shalt not take the Name of the Lord thy God in vain%: for the Lord will not hold him guiltless, that taketh his Name in vain.
.

\R Lord, have mercy upon us, and incline our hearts to keep this law.

IV. Remember that thou keep holy the Sabbath-day. Six days shalt thou labour, and do all that thou hast to do; but the seventh day is the Sabbath of the Lord thy God.% [In it thou shalt do no manner of work, thou, and thy son, and thy daughter, thy man-servant, and thy maid-servant, thy cattle, and the stranger that is within thy gates. For in six days the Lord made heaven and earth, the sea, and ail that in them is, and rested the seventh day: wherefore the Lord blessed the seventh day, and hallowed it.]

\R Lord, have mercy upon us, and incline our hearts to keep this law.

V. Honour thy father and thy mother.%; that thy days may be long in the land which the Lord thy God giveth thee.


\R Lord, have mercy upon us, and incline our hearts to keep this law.

VI. Thou shalt do no murder.

\R Lord, have mercy upon us, and incline our hearts to keep this law.

VII. Thou shalt not commit adultery.
    
\R Lord, have mercy upon us, and incline our hearts to keep this law.

VIII. Thou shalt not steal.

\R Lord, have mercy upon us, and incline our hearts to keep this law.

IX. Thou shalt not bear false witness% against thy neighbour.
.

\R Lord, have mercy upon us, and incline our hearts to keep this law.

X. Thou shalt not covet% thy neighbour's house, thou shalt not covet thy neighbour's wife, nor his servant, nor his maid, nor his ox, nor his ass, nor any thing that is his.
.

\R Lord, have mercy upon us, and write all these thy laws in our hearts, we beseech thee.

\medskip

%scottish 1912
\subsubsection{Or the Priest may rehearse, instead of the Ten Commandments, the Summary of the Law as followeth:}

%Irish, Canadian, American
\centerline{Hear what our Lord Jesus Christ saith.}
 
% \drop{Hear, O Israel, the Lord our God is one Lord: and thou shalt love the Lord thy God with all thy heart, and with all thy soul, and with all thy mind, with all thy strength: This is the first commandment; And the second is like, namely this, Thou shalt love thy neighbour as thyself: there is none other commandment greater than these.}\scripture{St.~Mark xij.~29.}
% On these two commandments hang all the Law and the Prophets.

\drop{Thou shalt love the Lord thy God with all thy heart, and with all thy soul, and with all thy mind.  This is the first and great commandment.  And the second is like unto it, Thou shalt love thy neighbour as thyself.  On these two commandments hang all the law and the prophets.}\scripture{St.~Matthew xxij.~37.}

\R Lord, have mercy upon us, and write these thy laws in our hearts, we beseech thee.


%\rubric{Or else, instead of the Ten Commandments or the Summary of the Law, may be sung or said on week-days, not being Great Festivals, as followeth:}

\medskip

\pilcrow{Here, if the Decalogue hath been omitted, shall be said or sung,}

\smallskip

\centerline{Lord, have mercy upon us. \rubric{iij.}}
\centerline{Christ, have mercy upon us. \rubric{iij.}}
\centerline{Lord, have mercy upon us. \rubric{iij.}}
\subsubsection{or}

\centerline{Kyrie eleison. \rubric{iij.}}
\centerline{Christe, eleison. \rubric{iij.}}
\centerline{Kyrie eleison. \rubric{iij.}}

\bigskip

\pilcrow{Then, on Sundays and on Feast days (except in Advent and from Septuagesima to Palm Sunday inclusive), shall be said or sung as follows:}

\drop{Glory be to God on high, and in earth peace, good will towards men. We praise thee, we bless thee, \footnote{Bow} we worship thee, we glorify thee, we give thanks to thee for thy great glory, O Lord God, heavenly King, God the Father Almighty.}
%\emph{and to thee, O God, the only begotten Son Jesu Christ; and to thee, O God, the Holy Ghost. - Scottish}

O Lord, the only-begotten Son Jesu Christ; O Lord God, Lamb of God, Son of the Father, that takest away the sins of the world, have mercy upon us. 
%\emph{Thou that takest away the sins of the world, have mercy upon us. - English} 
Thou that takest away the sins of the world, * receive our prayer. Thou that sittest at the right hand of God the Father, have mercy upon us.

For thou only art holy; thou only art the Lord; thou only, * O {Jesu} Christ, with the Holy Ghost, art most high \grecross\ in the glory of God the Father. Amen.

% Rubric from "The Orange Book" - 1923
{\footnotesize\rubric{This hymn may be omitted here, and sung instead at the end of this Order after the Thanksgiving after Communion.}\par}

\bigskip
%1549, 1912 Scottish:
\pilcrow {Then the priest shall turn him to the people and say,}
\V The Lord be with you. \R And with thy spirit.

% \pilcrow{Then shall follow one of these two Collects for the Queen, the Priest standing as before, and saying,}
\centerline{Let us pray.}
% \drop{Almighty God, whose kingdom is everlasting, and power infinite: Have mercy upon the whole Church; and so rule the heart of thy chosen servant ELIZABETH, our Queen and Governor, that she (knowing whose minister she is) may above all things seek thy honour and glory; and that we, and all her subjects (duly considering whose authority she hath) may faithfully serve, honour, and humbly obey her, in thee, and for thee, according to thy blessed Word and ordinance; through Jesus Christ our Lord, who with thee and the Holy Ghost liveth and reigneth, ever one God, world without end. \R Amen.}

% Or,

% \drop{Almighty and everlasting God, we are taught by thy Holy Word, that the hearts of Kings are in thy rule and governance, and that thou dost dispose and turn them as it seemeth best to thy godly wisdom: We humbly beseech thee so to dispose and govern the heart of ELIZABETH thy Servant, our Queen and Governor, that, in all her thoughts, words, and works, she may ever seek thy honour and glory, and study to preserve thy people committed to her charge, in wealth, peace, and godliness: Grant this, O merciful Father, for thy dear Son's sake, Jesus Christ our Lord. \R Amen.}

% \pilcrow{Then the Priest, turning to the Holy Table, shall say the \emph{Collect} or Collects.}
%En28
\medskip
\pilcrow{And turning to the Holy Table he shall say the Collect of the Day. Other collects contained in this Book or authorized by the Bishop may follow.}

\medskip

\section{The Ministry of the Word}

\pilcrow{Immediately thereafter he that readeth the Epistle shall say,}
The Epistle [\rubric{or,} The portion of Scripture appointed for the Epistle] is written in the --- chapter of --- beginning at the --- verse. 
\rubric{And the Epistle ended, he shall say,} Here endeth the Epistle.

\medskip
% From the Am28
\pilcrow{Here may be sung a Hymn or an Anthem.}
\medskip

\pilcrow{Then the Deacon or Priest that readeth the Gospel (the people all standing up) shall say,}
\V The Lord be with you. \R And with thy Spirit. 

\drop{The \grealtcross\ Holy Gospel is written in the — chapter of — beginning at the — verse. \R Glory \grecross\ be to thee, O Lord.}
% \pilcrow{He that readeth the Epistle or the Gospel shall so stand and turn himself as he may best be heard of the people.}
% \rubric{And after the Gospel the people may in like manner say or sing,}
% Thanks be to thee, O Lord, for this thy glorious Gospel. %1912: 

\centerline{\rubric{The Gospel ended, there may be said,}}
\centerline{Praise be to thee, O Christ.}
%Am28, Green, En28
% ¶ And after the Gospel may be said,
% Praise be to thee, O Christ.

\bigskip

\pilcrow{Then shall be sung or said the Creed following, the people still standing as before: except that at the discretion of the Minister it may be omitted on any day not being a Sunday or a Holy-day.} %Eng28

\drop{I believe in one God the Father Almighty, Maker of heaven and earth, And of all things visible and invisible:}

And in one Lord Jesus Christ, the only-begotten son of God, Begotten of his Father before all worlds, God of God, Light of Light, Very God of very God, Begotten, not made, Being of one substance with the Father, By whom all things were made: Who for us men, and for our salvation came down from heaven, * And was incarnate by the Holy Ghost of the Virgin Mary, * And was made man, * And was crucified also for us under Pontius Pilate. He suffered and was buried, And the third day he rose again according to the Scriptures, And ascended into heaven, And sitteth on the right hand of the Father. And he shall come again with glory to judge both the quick and the dead: Whose kingdom shall have no end.

And I believe in the Holy Ghost, The Lord and giver of life, Who proceedeth from the Father and the Son, Who with the Father and the Son together is worshipped and glorified, Who spake by the Prophets. And I believe One {Holy} Catholick and Apostolick Church. I acknowledge one Baptism for the remission of sins. And I look for the Resurrection of the dead, * And the life of the world to come. Amen.

\bigskip
\pilcrow{Then the Curate shall declare unto the people what Holy-days, or Fasting-days, are in the week following to be observed. And then also (if occasion be) shall notice be given of the Holy Communion, or of other services; Banns of matrimony may be published, and Briefs, Citations, and Excommunications shall be read, and Bidding of Prayers may be made. And nothing shall be proclaimed or published in the Church during the time of Divine Service, but by the Minister: nor by him any thing, but what is prescribed in the rules of this Book, or enjoined by the %Queen, or by the 
Ordinary of the place.}% adapted toward En1928


\smallskip


\pilcrow{Then may follow the Sermon, or one of the Homilies already set forth, or hereafter to be set forth, by authority.}


\smallskip


%scottish 1912
\pilcrow{When the Priest giveth warning of the Holy Communion he may, at his discretion, use the first or the second of the Exhortations appended to this Liturgy.}


\smallskip


% \pilcrow{The third Exhortation appended to this Liturgy may be used at the discretion of the Presbyter before the Offertory, the people standing.}
\medskip

\pilcrow{At the time of the celebration of the Holy Communion, the communicants being conveniently placed for the receiving of the Holy Sacrament, the Priest may say this Exhortation.}

\drop{Dearly beloved in the Lord, ye that mind to come to the holy Communion of the Body and Blood of our Saviour Christ, must consider how Saint Paul exhorteth all persons diligently to try and examine themselves, before they presume to eat of that Bread, and drink of that Cup. For as the benefit is great, if with a true penitent heart and lively faith we receive that holy Sacrament; \emph{(for then we spiritually eat the flesh of Christ, and drink his blood; then we dwell in Christ, and Christ in us; we are one with Christ, and Christ with us;)} so is the danger great, if we receive the same unworthily.}
\rubric{For then we are guilty of the Body and Blood of Christ our Saviour; we eat and drink our own damnation, not considering the Lord's Body; we kindle God's wrath against us; we provoke him to plague us with divers diseases, and sundry kinds of death.}

Judge therefore yourselves, brethren, that ye be not judged of the Lord; repent you truly for your sins past; have a lively and stedfast faith in Christ our Saviour; amend your lives, and be in perfect charity with all men; so shall ye be meet partakers of those holy mysteries. And above all things ye must give most humble and hearty thanks to God, the Father, the Son, and the Holy Ghost, for the redemption of the world by the death and passion of our Saviour Christ, both God and man; who did humble himself, even to the death upon the Cross, for us, miserable sinners, who lay in darkness and the shadow of death; that he might make us the children of God, and exalt us to everlasting life.

And to the end that we should alway remember the exceeding great love of our Master, and only Saviour, Jesus Christ, thus dying for us, and the innumerable benefits which by his precious blood-shedding he hath obtained to us; he hath instituted and ordained holy mysteries, as pledges of his love, and for a continual remembrance of his death, to our great and endless comfort. To him therefore, with the Father and the Holy Ghost, let us give (as we are most bounden) continual thanks; submitting ourselves wholly to his holy will and pleasure, and studying to serve him in true holiness and righteousness all the days of our life. Amen.


\bigskip

\section{The Offertory}
\label{offertory}
% From the "Green Book"
\pilcrow{Then shall the Priest return to the Lord's Table, and begin the \emph{Offertory}. The Priest shall say, or the Clerks shall sing, one of these Sentences following, or some other convenient sentence taken out of Holy Scripture. A Hymn may follow.}
% \pilcrow{1549 : ¶ Then shall folowe for the Offertory, one or mo, of these Sentences of holy scripture, to bee song whiles the People doo offer, or els one of them to bee saied by the minister, immediatly afore the offeryng.}

% \pilcrow{Then shall the Priest return to the Lord's Table, and begin the Offertory, saying one or more of these Sentences following, as he thinketh most convenient in his discretion.}

\drop{Let your light so shine before men, that they may see your good works, and glorify your Father which is in heaven.}\scripture{St.~Matthew v.~16.}

Lay not up for yourselves treasure upon the earth; where the rust and moth doth corrupt, and where thieves break through and steal: but lay up for yourselves treasures in heaven; where neither rust nor moth doth corrupt, and where thieves do not break through and steal.\scripture{St.~Matthew vj.~19.}

Whatsoever ye would that men should do unto you, even so do unto them; for this is the Law and the Prophets.\scripture{St.~Matthew vij.~12.}

Not every one that saith unto me, Lord, Lord, shall enter into the kingdom of heaven; but he that doeth the will of my Father which is in heaven.\scripture{St.~Matthew vij.~21.}

Remember the words of the Lord Jesus, how he said, It is more blessed to give than to receive.\scripture{Acts xx.~35.}

Godliness is great riches, if a man be content with that he hath: for we brought nothing into the world, neither may we carry any thing out.\scripture{1 Timothy vj.~6.}

Be merciful after thy power. If thou hast much, give plenteously: if thou hast little, do thy diligence gladly to give of that little: for so gatherest thou thyself a good reward in the day of necessity.\scripture{Tobit iv.~8.}

%\emph{Zacchæus stood forth, and said unto the Lord, Behold, Lord, the half of my goods I give to the poor; and if I have done any wrong to any man, I restore four-fold.\scripture{St.~Luke xix.}} Omitted in En28

%\emph{Who goeth a warfare at any time of his own cost? Who planteth a vineyard, and eateth not of the fruit thereof? Or who feedeth a flock, and eateth not of the milk of the flock?\scripture{1 Corithians ix}} Omitted in En28

If we have sown unto you spiritual things, is it a great matter if we shall reap your worldly things?\scripture{1 Corithians ix.~11.}

Do ye not know, that they who minister about holy things live of the sacrifice; and they who wait at the altar are partakers with the altar? Even so hath the Lord also ordained, that they who preach the Gospel should live of the Gospel.\scripture{1 Corithians ix.~13.}

He that soweth little shall reap little; and he that soweth plenteously shall reap plenteously. Let every man do according as he is disposed in his heart, not grudging, or of necessity; for God loveth a cheerful giver.\scripture{2 Corithians ix.~6.}

Let him that is taught in the Word minister unto him that teacheth, in all good things. Be not deceived, God is not mocked: for whatsoever a man soweth that shall he reap.\scripture{Galatians vj.~6.}

While we have time, let us do good unto all men; and specially unto them that are of the household of faith.\scripture{Galatians vj.~10.}

God is not unrighteous, that he will forget your works, and labour that proceedeth of love; which love ye have shewed for his Name’s sake, who have ministered unto the saints, and yet do minister.\scripture{Hebrews vj.~10.}

Lift up your eyes and look upon the fields; for they are white already to harvest.\scripture{St.~John iv.~35.}

Charge them who are rich in this world, that they be ready to give, and glad to distribute; laying up in store for themselves a good foundation against the time to come, that they may attain eternal life.\scripture{1 Timothy vj.~17.}

Whoso hath this world’s good, and seeth his brother have need, and shutteth up his compassion from him, how dwelleth the love of God in him?\scripture{1 St.~John iij.~17}

Blessed be the man that provideth for the sick and needy: the Lord shall deliver him in the time of trouble.\scripture{Psalm xli.~1.}

To do good, and to distribute, forget not; for with such sacrifices God is pleased.\scripture{Hebrews xiij.~16.}

Offer unto God thanksgiving, and pay thy vows unto the most Highest.\scripture{Psalm l.~14.}

I will offer in his dwelling an oblation with great gladness: I will sing and speak praises unto the Lord.\scripture{Psalm xxvij.~6.}

Melchizedek king of Salem brought forth bread and wine; and he was the priest of the most high God.\scripture{Genesis xiv.~18.}

%\emph{Give alms of thy goods, and never turn thy face from any poor man; and then the face of the Lord shall not be turned away from thee. \scripture{Tobit iv.}} Omitted in En28

%\emph{He that hath pity upon the poor lendeth unto the Lord: and look, what he layeth out, it shall be paid him again.\scripture{Proverbs xix.}} Omitted in En28

% \pilcrow{1549 : Where there be Clerkes, thei shall syng one, or many of the sentences above written, accordyng to the length and shortenesse of the tyme, that the people be offeryng.}

% \pilcrow{In the meane time, whyles the Clerkes do syng the Offertory, so many as are disposed, shall offer unto the poore mennes boxe every one accordynge to his habilitie and charitable mynde. And at the offeryng daies appoynted, every manne and woman shall paie to the Curate, the due and accustomed offerynges.}


% \pilcrow{Then so manye as shalbe partakers of the holy Communion, shall tary still in the quire, or in some convenient place nigh the quire, the men on the one side, and the women on the other syde. All other (that mynde not to receive the said holy Communion) shall departe out of the quire, except the ministers and Clerkes.}

% \pilcrow{Than shall the minister take so muche Bread and Wine, as shalt suffice for the persons appoynted to receive the holy Communion, laiyng the breade upon the corporas, or els in the paten, or in some other comely thyng, prepared for that purpose. And puttyng ye wyne into the Chalice, or els in some faire or convenient cup, prepared for that use (if the Chalice will not serve), puttyng thereto a litle pure and cleane water: And settyng both the breade and wyne upon the Alter: then the Priest shall saye.}

\pilcrow{Whilst these Sentences are said or sung, the Deacons, Church-wardens, or other fit person appointed for that purpose, shall receive the alms for the poor, or other devotions of the people,
% in a decent basin to be provided by the Parish for that purpose; 
and reverently bring them to the Priest, who shall humbly present and place them upon the Holy Table in a decent bason to be provided for that purpose.}

\bigskip
% \pilcrow{1912: While the Presbyter distinctly pronounceth one or more of these sentences for the Offertory, the Deacon, or (if no such be present) some other fit person, shall receive the devotions of the people there present, in a bason provided for that purpose. And when all have offered, her shall reverently bring the said bason, with the offerings therein, and deliver it to the Presbyter; who shall humbly present it before the Lord, and set it upon the Holy Table.}

% \pilcrow{And when there is a Communion, the Priest shall then place upon the Table so much Bread and Wine, as he shall think sufficient. After which done, the Priest shall say,}
%1912
\pilcrow{{And when there is a Communion,} the Priest shall then offer up, and place the bread and wine prepared for the Sacrament upon the Lord’s Table; and shall say,}

\drop{Blessed be thou, O {\scshape Lord} God, for ever and ever. Thine, O {\scshape Lord}, is the greatness, and the glory, and the victory, and the majesty: for all that is in the heaven and in the earth, is thine: thine is the kingdom, O {\scshape Lord}, and thou art exalted as head above all: both riches and honour come of thee, and of thine own do we give unto thee. \R Amen.}
\scripture{1 Chronicles xxix.~10}

\bigskip
\centerline{Let us pray for the whole state of Christ's Church.} % militant here in earth.

% \begin{wrapfigure}[lineheight]{position}[overhang]{width}

\drop{Almighty and everliving God, who by thy holy Apostle hast taught us to make prayers, and supplications, and to give thanks for all men;}
% \begin{wrapfigure}[5]{r}{0.40\textwidth}{\footnotesize
% \rubric{* If there be no alms or oblations, then the words \emph{[to accept our alms and oblations]} be left out unsaid.}\par
% }\end{wrapfigure}\noindent
We humbly beseech thee most mercifully [\footnote{\rubric{If there be no alms or oblations, then the words \emph{[to accept our alms and oblations]} be left out unsaid.}}\emph{to accept our alms and \grealtcross\ oblations, and}] to receive these our prayers, which we offer unto thy Divine Majesty; beseeching thee to inspire continually the Universal Church with the spirit of truth, unity, and concord: And grant, that all they that do confess thy holy Name may agree in the truth of thy holy Word, and live in unity, and godly love.



% "M & SA" We beseech thee also to lead all nations in the way of righteousness and peace, so directing all Kings, Presidents, and ruling Authorities, that under them the world may bc godly and quietly governed. And grant unto thy servant George, our King, his Ministers and Parliaments, and all who are set in authority over us, that they may truly and impartially minister justice, to the removing of all wickedness and vice, and the maintenance of thy true religion and virtue.


% We beseech thee also to save and defend all Christian Kings, Princes, and Governours; and specially thy Servant ELIZABETH our Queen; that under her we may be godly and quietly governed: And grant unto her whole Council, and to all that are put in authority under her, that they may truly and impartially minister justice, to the punishment of wickedness and vice, and to the maintenance of thy true religion, and virtue.

%orange book
% We beseech thee also to lead all nations in the way of peace and righteousness, so directing all Kings and ruling authorities that under them the world may be godly and quietly governed.
%SA1954
We beseech thee also to lead all nations into the way of righteousness and peace, and so to direct all 
% Kings, Presidents and Rulers
ruling authorities,
that under them the world may be godly and quietly governed. 
And grant unto all that are put in authority,
% under them, 
% American
% [We beseech thee also, so to direct and dispose the hearts of all Christian Rulers,] 
that they may truly and indifferently minister justice, to the punishment of wickedness and vice, and to the maintenance of thy true religion, and virtue.

Give grace, O heavenly Father, to all Bishops, Priests, and Deacons, 
% 1912 Scot
% and Curates (1662)
% and other Ministers, (1928 Amer)
especially to thy servant \emph{N.} our bishop, %En28
that they may both by their life and doctrine set forth thy true and lively Word, and rightly and duly administer thy holy Sacraments.

And to all thy people give thy heavenly grace,
%; and specially to this congregation here present; 
that with meek heart and due reverence, they may hear and receive thy holy Word; truly serving thee in holiness and righteousness all the days of their life.
And especially we commend unto thy merciful goodness this congregation which is here assembled in thy Name, to celebrate the Commemoration of the most glorious death of thy Son.

And we most humbly beseech thee, of thy goodness, O Lord, to comfort and succour all them, who in this transitory life are in trouble, sorrow, need, sickness, or any other adversity.

%And we also bless thy holy Name for all thy servants departed this life in thy faith and fear; beseeching thee to give us grace so to follow their good examples, that with them we may be partakers of thy heavenly kingFdom.

% {\begin{multicols}{2}
% And here we do give unto thee most high praise and hearty thanks for the wonderful grace and virtue declared in all thy saints from the beginning of the world: And chiefly in the glorious and most blessed Virgin Mary, mother of thy Son Jesus Christ our Lord and God, and in the holy Patriarchs, Prophets, Apostles, and Martyrs, whose examples (O Lord) and stedfastness in the faith, and keeping thy holy commandments, grant us to follow.

% We commend unto thy mercy all other thy servants which are departed from us with the sign of faith, and now to rest in the sleep of peace: Grant unto them, we beseech thee, thy mercy and everlasting peace, and that at the day of the general resurrection, we and all they which be of the mystical body of thy Son, may all together be set on his right hand, and hear that his most joyful voice: Come ye blessed of my Father, and possess the kingdom which is prepared for you from the beginning of the world.



And here we do give unto thee most high praise and hearty thanks for the wonderful grace and virtue declared in all thy Saints, from the beginning of the world: And chiefly in the glorious and most blessed Virgin Mary, Mother of thy Son Jesus Christ our Lord and God, [in \emph{N.},] and in the holy Patriarchs, Prophets, Apostles and Martyrs, whose examples, O Lord, and stedfastness in thy faith, and keeping thy holy commandments, grant us to follow.

We commend unto thy mercy, O Lord, all other thy servants which are departed hence from us, with the sign of faith, and now do rest in the sleep of peace.  Grant unto them, we beseech thee, thy mercy, and everlasting peace, and that, at the day of the general resurrection, we and all thy servants which be of the mystical body of thy Son, may altogether be set on his right hand, and hear that his most joyful voice: Come unto me, O ye that be blessed of my Father, and possess the kingdom, which is prepared for you from the beginning of the world. 


% \end{multicols}}

Grant this, O Father, for Jesus Christ's sake, our only Mediator and Advocate. \R Amen.

\section{The Consecration}

\pilcrow{Turning himself to the people the Priest shall say,}
\V The Lord be with you. \R And with thy spirit.

\V Lift up your hearts.  \R We lift them up unto the Lord.

\V Let us give thanks unto our Lord God. \R It is meet and right so to do.

\centerline{\rubric{Then shall the Priest turn to the Lord's Table, and say,}}

\drop{It is very meet, right, and our bounden duty, that we should at all times, and in all places, give thanks unto thee, O Lord, Holy Father, Almighty, Everlasting God.}
% \rubric{* These words} [Holy Father] \rubric{must be omitted on Trinity Sunday.}

{\footnotesize\centering\rubric{Here shall follow the proper Preface, according to the time, if there be any specially appointed: or else immediately shall follow, \emph{Therefore with Angels \etc}}\par}


% \drop{Therefore with Angels and Archangels, and with all the company of heaven, we laud and magnify thy glorious Name; evermore praising thee, and saying,}
% \drop{Holy, holy, holy, Lord God of hosts, heaven and earth are full of thy glory: Glory be to thee, O Lord most High.}
% Blessed is he that cometh in the Name of the Lord;
% Hosanna in the highest.

\section*{Proper Prefaces}
\prefaceCaption{Upon}{Christmas Day,}{and seven days after.}
\drop{Because thou didst give Jesus Christ thine only Son to be born as at this time for us; who, by the operation of the Holy Ghost, was made very man of the substance of the Virgin Mary his mother; and that without spot of sin, to make us clean from all sin. Therefore with Angels, \etc}

\prefaceCaption{Upon the}{Epiphany,}{and seven days after.}
\drop{Through Jesus Christ our Lord: Who in substance of our mortal flesh manifested forth his glory: That he might bring us out of darkness into his own glorious light.  Therefore with Angels, \etc}

% (sE) Maundy Thursday

\prefaceCaption{Upon}{Easter Day,}{and seven days after.}
\drop{But chiefly are we bound to praise thee for the glorious Resurrection of thy Son Jesus Christ our Lord: for he is the very Paschal Lamb, which was offered for us, and hath taken away the sin of the world; who by his death hath destroyed death, and by his rising to life again hath restored to us everlasting life. Therefore with Angels, \etc}


\prefaceCaption{Upon}{Ascension Day,}{and seven days after.}
\drop{Through thy most dearly beloved Son Jesus Christ our Lord; who after his most glorious Resurrection manifestly appeared to all his Apostles, and in their sight ascended up into heaven to prepare a place for us; that where he is, thither we might also ascend, and reign with him in glory. Therefore with Angels, \etc}


\prefaceCaption{Upon}{Whitsunday,}{and six days after.}
\drop{Through Jesus Christ our Lord; according to whose most true promise, the Holy Ghost came down as at this time from heaven with a sudden great sound, as it had been a mighty wind in the likeness of fiery tongues, lighting upon the Apostles, to teach them, and to lead them to all truth; giving them both the gift of divers languages, and also boldness with fervent zeal constantly to preach the Gospel unto all nations; whereby we have been brought out of darkness and error into the clear light and true knowledge of thee, and of thy Son Jesus Christ. Therefore with Angels, \etc}

\prefaceCaption{Upon the Feast of}{Trinity}{only.}
\drop{Who 
with thine only-begotton Son and the Holy Ghost %AmEn1928
art one God, one Lord,
in Trinity of Persons and Unity of Substance. %AmEn1928
% ; not one only Person, but three Persons in one Substance.
For that which we believe of 
% the glory of the Father, 
thy glory, O Father, %AmEn1928
the same we believe of the Son, and of the Holy Ghost, without any difference or inequality. Therefore with Angels, \etc}


\prefaceCaption{Upon the}{Purification, Annunciation, \rubric{and} Transfiguration}{.}
\drop{Because in the Mystery of the Word made flesh, thou hast caused a new light to shine in our hearts, to give the knowledge of thy glory in the face of thy Son, Jesus Christ our Lord. Therefore with Angels, \etc} %am1928

\prefaceCaption{Upon}{All Saints’ Day}{and the Feasts of Apostles, Evangelists, and \emph{St.\ John Baptist’s Nativity}, except when the Proper Preface of any Principal Feast is appointed.} 
\drop{Who, in the multitude of thy saints, hast compassed us about with so great a cloud of witnesses that we, rejoicing in their fellowship, may run with patience the race that is set before us, and together with them may receive the crown of glory that fadeth not away. Therefore with Angels, \etc }


\subsubsection{After each of which Prefaces shall immediately be sung or said,}

\drop{Therefore with Angels and Archangels, and with all the company of heaven, we laud and magnify thy glorious Name; evermore praising thee, and saying,}
\smallskip
\drop{Holy, holy, holy, Lord God of hosts, heaven and earth are full of thy glory: Glory be to thee, O Lord most High.
\grecross\ Blessed is he that cometh in the Name of the Lord;
Hosanna in the highest.}

\bigskip

\pilcrow{When the Priest, standing before the Table, hath so ordered the Bread and Wine, that he may with the more readiness and decency break the Bread before the people, and take the Cup into his hands, he shall say the Prayer of Consecration, as followeth.}

\drop{All glory be to thee, Almighty God, our heavenly Father, for that thou of thy tender mercy didst give thine only Son Jesus Christ to suffer death upon the Cross for our redemption; who made there (by his one oblation of himself once offered) a full, perfect, and sufficient sacrifice, oblation, and satisfaction, for the sins of the whole world; and did institute, and in his holy Gospel command us to continue, a perpetual memory of that his precious death 
and sacrifice, %scottish
until his coming again;}

% 1549: Hear us, O most merciful Father, and with thy Holy Spirit and word vousafe to bless \grecross and sanctify \grecross these thy gifts and creatures of bread and wine that they may be unto us the body and blood of thy most dearly beloved Son Jesus Christ.

%1637 Heare us, O mercifull Father, we most humbly beseech thee, and of thy almighty goodnesse vouchsafe so to blesse and sanctifie with thy word and holy Spirit these thy gifts and creatures of bread and wine, that they may bee unto us the body and bloud of thy most dearly beloved Son; so that wee receiving them according to thy Sonne our Saviour Jesus Christs holy institution, in remembrance of his death and passion, may be partakers of the same his most precious body and bloud
%English: Hear us, O merciful Father, we most humbly beseech thee; and grant that we receiving these thy creatures of bread and wine, according to thy Son our Saviour Jesus Christ's holy institution, in remembrance of his death and passion, may be partakers of his most blessed Body and Blood: 


% [For, in the night (S)  (a) he took]
\drop{Who, in the same night that he was betrayed, \footnote{\rubric{Here the Priest is to take the Bread into his hands:}}took Bread; and, when he had given thanks, \footnote{\rubric{And here to touch or break the Bread:}}he brake it, and gave it to his disciples, saying, Take, eat, \footnote{\rubric{And here to lay his hand upon all the Bread.}}}

{\centering{\scshape this is my Body which is given for you:} 
Do this in remembrance of me.\par}

\medskip

Likewise after supper he \footnote{\rubric{Here he is to take the Cup into his hand:}}took the Cup; and, when he had given thanks, he gave it to them, saying, Drink ye all of this; \footnote{\rubric{And here to lay his hand upon every vessel (be it Chalice or Flagon) in which there is any Wine to be consecrated.}}

{\centering{\scshape for this is my Blood of the New Testament, which is shed for you and for many for the remission of sins:}

Do this, as oft as ye shall drink it, in remembrance of me.\par}


\centerline{\rubric{The Oblation}}
\drop{Wherefore, O Lord, and heavenly Father, according to the institution of thy dearly beloved Son our Saviour Jesus Christ, we thy humble servants do celebrate and make here before thy divine Majesty, with these thy holy gifts, {\scshape which we now offer unto thee,} the memorial thy Son hath commanded us to make; having in remembrance his blessed passion, and precious death, his mighty resurrection and glorious ascension; rendering unto thee most hearty thanks for the innumerable benefits which he hath procured unto us by the same, 
and looking for his coming again with power and great glory.} %late scottish

% ["invocation" - 1928, 1928 scottish (all) ]
% 1549: Hear us, O most merciful Father, and with thy Holy Spirit and word vousafe to bless \grecross and sanctify \grecross these thy gifts and creatures of bread and wine that they may be unto us the body and blood of thy most dearly beloved Son Jesus Christ.

%1637 Heare us, O mercifull Father, we most humbly beseech thee, and of thy almighty goodnesse vouchsafe so to blesse and sanctifie with thy word and holy Spirit these thy gifts and creatures of bread and wine, that they may bee unto us the body and bloud of thy most dearly beloved Son; so that wee receiving them according to thy Sonne our Saviour Jesus Christs holy institution, in remembrance of his death and passion, may be partakers of the same his most precious body and bloud
%English 1928: Hear us, O Merciful Father, we most humbly beseech thee, and with thy Holy and Life-giving Spirit vouchsafe to bless and sanctify both us and these thy gifts of Bread and Wine, that they may be unto us the Body and Blood of thy Son, our Saviour, Jesus Christ, to the end that, receiving the same, we may be strengthened and refreshed both in body and soul.
%English: Hear us, O merciful Father, we most humbly beseech thee; and grant that we receiving these thy creatures of bread and wine, according to thy Son our Saviour Jesus Christ's holy institution, in remembrance of his death and passion, may be partakers of his most blessed Body and Blood: 

% "Scottish office" - newer than 1637; older than 1912.  The 1637 is actually similar but goes on with the english ending.
\centerline{\rubric{The Invocation}}
\drop{Hear us, O Merciful Father, we most humbly beseech thee, % ENglish
% And we most humbly beseech thee, O merciful Father, to hear us, Scottish
and of thy almighty goodness vouchsafe to \grealtcross\  bless and \grealtcross\  sanctify, with thy Word and Holy Spirit, these thy gifts and creatures of bread and wine, that they may become the Body and Blood of thy most dearly beloved Son, }
to the end that all who shall receive the same may be sanctified both in body and soul, and preserved unto everlasting life. % (late Scottish)]


% (scottish)

\drop{And we entirely %english
% [earnestly]  %scottish
desire thy fatherly goodness mercifully to accept this our sacrifice of praise and thanksgiving; most humbly beseeching thee to grant, that by the merits and death of thy Son Jesus Christ, and through faith in his blood, we and all thy whole Church may obtain remission of our sins, and all other benefits of his passion.}

And here we %[humbly] scottish
offer and present unto thee, O Lord, our selves, our souls and bodies, to be a reasonable, holy, and lively sacrifice unto thee; 
% humbly beseeching thee, that whosoever shall be partakers 
% humbly beseeching thee, that all we, who are partakers 
humbly beseeching thee, that all we, who shall be partakers
% that we, and all others who shall be partakers
of this holy Communion, may
worthily receive the most precious Body and Blood of thy Son Jesus Christ, and
be fulfilled with thy grace and heavenly \grecross\  benediction,
and made one body with him, that he may dwell in us, and we in him.

And although we be unworthy, through our manifold sins, to offer unto thee any sacrifice, yet we beseech thee to accept this our bounden duty and service, and command these our prayers and supplications by the ministry of thy holy Angels to be brought up into thy holy Tabernacle before the sight of thy divine Majesty; not weighing our merits, but pardoning our offences,

Through Jesus Christ our Lord; by whom, and with whom, in the unity of the Holy Ghost, all honour and glory be unto thee, O Father Almighty, world without end. \R Amen.

% \centerline{Let us pray.}
% [And now,]
{\centering\rubric{Here shall the people join with the Priest in the Lord’s Prayer, the Priest first saying,}\par}
As our Saviour Christ hath commanded and taught us, we are bold to say,

\smallskip
% \pilcrow{Then shall the Priest say the Lord's Prayer, the people repeating after him every Petition.}
\drop{Our Father, which art in heaven, Hallowed be thy Name. Thy kingdom come. Thy will be done, in earth as it is in heaven. Give us this day our daily bread. And forgive us our trespasses, As we forgive them that trespass against us. And lead us not into temptation; But deliver us from evil.}

For thine is the kingdom, the power, and the glory, For ever and ever. Amen.

\centerline{\rubric{Here the Priest is to break the consecrated bread.}}

\medskip

\pilcrow{Then shall the Priest say or sing,}
\drop{The Peace of the Lord be always with you.  \R And with thy spirit.}

\medskip

\drop{Christ, our Paschal Lamb, is offered up for us, once for all, when he bare our sins on his body upon the Cross; for he is the very Lamb of God that taketh away the sins of the world, wherefore let us keep a joyful and holy feast unto the Lord.}


\section{The Communion}

\pilcrow{Then shall the Minister say to them that come to receive the Holy Communion,}

\drop{Ye that do truly and earnestly repent you of your sins, and are in love and charity with your neighbours, and intend to lead a new life, following the commandments of God, and walking from henceforth in his holy ways; Draw near with faith, and take this holy Sacrament to your comfort; and make your humble confession to Almighty God, meekly kneeling upon your knees.}

\medskip
%1662
% \pilcrow{Then shall this general Confession be made, in the name of all those that are minded to receive the holy Communion, by one of the Ministers; both he and all the people kneeling humbly upon their knees, and saying,}
%Scottish 1912
% Then shall this general confession be made by the people, along with the Presbyter; he first kneeling down.
%american 1928
\pilcrow{Then shall this \emph{General Confession be made}, by the Priest and all those who are minded to receive the Holy Communion, humbly kneeling.}

\drop{Almighty God, Father of our Lord Jesus Christ, Maker of all things, judge of all men; We acknowledge and bewail our manifold sins and wickedness, Which we, from time to time, most grievously have committed, By thought, word, and deed, Against thy Divine Majesty, Provoking most justly thy wrath and indignation against us. We do earnestly repent, And are heartily sorry for these our misdoings; The remembrance of them is grievous unto us; The burden of them is intolerable. Have mercy upon us, Have mercy upon us, most merciful Father; For thy Son our Lord Jesus Christ's sake, Forgive us all that is past; And grant that we may ever hereafter Serve and please thee In newness of life, To the honour and glory of thy Name; Through Jesus Christ our Lord. Amen.}


\medskip


\pilcrow{Then shall the Priest (or the Bishop, being present,) standing up, and turning himself to the people, pronounce this Absolution.}
\drop{Almighty God, our heavenly Father, who of his great mercy hath promised forgiveness of sins to all them that with hearty repentance and true faith turn unto him; Have mercy upon you; pardon  \grealtcross\  and deliver you from all your sins; confirm and strengthen you in all goodness; and bring you to everlasting life; through Jesus Christ our Lord. \R Amen.}

\medskip

\centerline{\pilcrow{Then shall the Priest say,}}
Hear what comfortable words our Saviour Christ saith unto all that truly turn to him.
\drop{Come unto me all that travail and are heavy laden, and I will refresh you.}\scripture{St.~Matthew xj.~28}

So God loved the world, that he gave his only-begotten Son, to the end that all that believe in him should not perish, but have everlasting life.\scripture{St.~John iij.~16}

\centerline{Hear also what Saint Paul saith.}

This is a true saying, and worthy of all men to be received, That Christ Jesus came into the world to save sinners.\scripture{1 Timothy i.~15.}

\centerline{Hear also what Saint John saith.}

If any man sin, we have an Advocate with the Father, Jesus Christ the righteous; and he is the propitiation for our sins.\scripture{1 St.~John ij.~1.}

\medskip


{\centering\footnotesize\rubric{Then shall the Priest, turning him to the Altar, kneel down, and say, in the name of all them that shall communicate, this Collect of humble access to the Holy Communion, as followeth:}\par} % Scottish rubric

\drop{We do not presume to come to this thy Table, O merciful Lord, trusting in our own righteousness, but in thy manifold and great mercies. We are not worthy so much as to gather up the crumbs under thy Table. But thou art the same Lord, whose property is always to have mercy: Grant us therefore, gracious Lord, so to eat the flesh of thy dear Son Jesus Christ, and to drink his blood, that our sinful bodies may be made clean by his body, and our souls washed through his most precious blood, and that we may evermore dwell in him, and he in us. \R Amen.}


\medskip
% Scottish Position:
\pilcrow{Here may be sung or said:} %scottish rubric
% \pilcrow{In the Communion time the choir and people may say or sing the following, beginning so soon as the Priest doth receive the holy Communion.}

\drop{O Lamb of God, that takest away the sins of the world: have mercy upon us.

O Lamb of God that, takest away the sins of the world: have mercy upon us.

O Lamb of God, that takest away the sins of the world: grant us thy peace.}
\bigskip

\pilcrow{Then shall the Minister first receive the Communion in both kinds himself, and then proceed to deliver the same to the Bishops, Priests, and Deacons, in like manner, (if any be present,) and after that to the people also in order, 
into their hands, % Percy: This has always been the custom with us.
all meekly kneeling. And, when he delivereth the Bread to any one, he shall say,}
\drop{The Body of our Lord Jesus Christ, which was given for thee, preserve thy body and soul unto everlasting life. \R Amen.}

\smallskip

{\centering\footnotesize\rubric{Take and eat this in remembrance that Christ died for thee, and feed on him in thy heart by faith with thanksgiving.}\par}

\bigskip

{\centering\footnotesize\rubric{And the Minister that delivereth the Cup to any one shall say,}\par}

\drop{The Blood of our Lord Jesus Christ, which was shed for thee, preserve thy body and soul unto everlasting life. \R Amen.}

\smallskip

{\centering\footnotesize\rubric{Drink this in remembrance that Christ's Blood was shed for thee, and be thankful.}\par}

\bigskip


% \pilcrow{If the consecrated Bread or Wine be all spent before all have communicated, the Priest is to consecrate more cording to the Form before prescribed: Beginning at [Our Saviour Christ in the same night, etc.] for the blessing of the Bread ; and at [Likewise after Supper, etc.] for the blessing of the Cup.}

\pilcrow{When all have communicated, the Minister shall return to the Lord's Table, and reverently place upon it what remaineth of the consecrated Elements, covering the same with a fair linen cloth.}

\medskip
% Green book.
\pilcrow{Here may follow an Anthem or Hymn.}

\section{The Thanksgiving}

\pilcrow{Then shall the Priest give thanks to God in the name of all them that have communicated, turning him first to the people, and saying,}

%1923 (1929 option, along with 1549)
\V O give thanks unto the Lord, for he is gracious:  \R And his mercy endureth for ever.

%1549
\V The Lord be with you. \R And with thy spirit.

\centerline{Let us pray.}

%proposed 1928, shortened from Scottish.
% Having now 
% by faith <- added! 
% received the precious Body and Blood of Christ, let us give thanks unto our Lord God,

\drop{Almighty and everliving God, we most heartily thank thee, for that thou dost vouchsafe to feed us, who have duly received these holy mysteries, with the spiritual food of the most precious Body and Blood of thy Son our Saviour Jesus Christ; and dost assure us thereby of thy favour and goodness towards us; and that we are very members incorporate in the mystical body of thy Son, which is the blessed company of all faithful people; and are also heirs through hope of thy everlasting kingdom, by the merits of the most precious death and passion of thy dear Son. And we most humbly beseech thee, O heavenly Father, so to assist us with thy grace, that we may continue in that holy fellowship, and do all such good works as thou hast prepared for us to walk in; through Jesus Christ our Lord, to whom, with thee and the Holy Ghost, be all honour and glory, world without end. Amen.}

\bigskip
% Rubric from "The Orange Book" - 1923
\pilcrow{The \emph{Gloria in excelsis} may be omitted after the \emph{Kyrie eleison}, and sung here instead: provided that it be always said or sung in one or other position on Holy-days and on all Sundays except those in Advent and from Septuagesima to Palm Sunday inclusive.}
\bigskip

\centerline{\rubric{For the Post-Communions see pages X-X}} %scottish
\bigskip

\V The Lord be with you. \R And with thy spirit.

\V Depart in peace, \rubric{or} Let us bless the Lord.

\R Thanks be to God.

\medskip
\pilcrow{Then the Priest (or Bishop if he be present) shall let them depart with this Blessing.}
\drop{The peace of God, which passeth all understanding, keep your hearts and minds in the knowledge and love of God, and of his son Jesus Christ our Lord: and the blessing of God Almighty, the Father, \grealtcross\ the Son, and the Holy Ghost, be amongst you and remain with you always. \R Amen.}

\medskip

\fleuron

\bigskip

\section{Collects}
\pilcrow{To be said after the Offertory, when there is no Communion, every such day one or more; and the same may be said also, as often as occasion shall serve, after the Collects either of Morning or Evening Prayer, or the Litany, or immediately before the Blessing at Holy Communion, by the discretion of the Minister.}

% Scottish:
\drop{O Almighty Father, wellspring of life to all things that have being, from amid the unwearied praises of Cherubim and Seraphim who stand about thy throne of light which no man can approach unto, give ear, we humbly beseech thee, to the supplications of thy people who put their sure trust in thy mercy, through Jesus Christ our Lord.  \R Amen.}
% [From the Book of Deer.]}

\smallskip


\drop{O Lord Jesus Christ, before whose judgement-seat we must all appear and give account of the things done in the body: Grant, we beseech thee, that when the books are opened in that day, the faces of thy servants may not be ashamed, through thy merits, O blessed Saviour, who livest and reignest with the Father and the Holy Spirit, one God, world without end.  \R Amen.}
% [From the Altus of St Columba]}

\smallskip

\drop{Assist us mercifully, O Lord, in these our supplications and prayers, and dispose the way of thy servants towards the attainment of everlasting salvation; that, among all the changes and chances of this mortal life, they may ever be defended by thy most gracious and ready help; through Jesus Christ our Lord. \R Amen.}

\smallskip

\drop{O Almighty Lord, and everlasting God, vouchsafe, we beseech thee, to direct, sanctify, and govern, both our hearts and bodies, in the ways of thy laws, and in the works of thy commandments; that through thy most mighty protection, both here and ever, we may be pre- served in body and soul; through our Lord and Saviour Jesus Christ. \R Amen.}

\smallskip


\drop{Grant, we beseech thee, Almighty God, that the words, which we have heard this day with our outward ears, may through thy grace be so grafted inwardly in our hearts, that they may bring forth in us the fruit of good living, to the honour and praise of thy Name; through Jesus Christ our Lord. \R Amen.}

\smallskip

\drop{Prevent us O Lord, in all our doings with thy most gracious favour, and further us with thy continual help; that in all our works begun, continued, and ended in thee, we may glorify thy holy Name, and finally by thy mercy obtain everlasting life; through Jesus Christ our Lord. \R Amen.}

\smallskip

\drop{Almighty God, the fountain of all wisdom, who knowest our necessities before we ask, and our ignorance in asking; We beseech thee to have compassion upon our infirmities; and those things, which for our unworthiness we dare not, and for our blindness we cannot ask, vouchsafe to give us, for the worthiness of thy Son Jesus Christ our Lord. \R Amen.}

\smallskip

% The two following collects may be said before the Blessing.

\drop{O Lord, our God, thou Saviour of the world, through whom we have celebrated these sacred mysteries: Receive our humble thanksgiving, and of thy great mercy vouchsafe to sanctify us evermore in body and soul, who livest and reignest with the Father and the Holy Spirit, one God, world without end.  \R Amen.}

\smallskip

\drop{Almighty God, who hast promised to hear the petitions of them that ask in thy Son's Name; We beseech thee mercifully to incline thine ears to us that have made now our prayers and supplications unto thee; and grant, that those things, which we have faithfully asked according to thy will, may effectually be obtained, to the relief of our necessity, and to the setting forth of thy glory; through Jesus Christ our Lord. \R Amen.}

    %Scottish 1912
\subsection{Prayers for Certain Festivals and Seasons}

\prefaceCaption{}{Advent.}{} % Proofed vs 1929 on 11-29-21
\drop{Grant, O Almighty God, that as thy blessed Son Jesus Christ at his first advent came to seek and to save that which was lost, so at his second and glorious appearing he may find in us the fruits of the redemption which he wrought; who liveth and reigneth, with thee and the Holy Spirit, one God, world without end. Amen.}

\prefaceCaption{}{Christmas Day,}{and seven days after.}
\drop{O God, who hast given us grace at this time to celebrate the birth of our Saviour, Jesus Christ: We laud and magnify thy glorious Name for the countless blessings which he hath brought unto us; and we beseech thee to grant that We may ever set forth thy praise in joyful obedience to thy will; through the same Jesus Christ our Lord. \R Amen.}

%1928 prop
\prefaceCaption{}{New Year’s Day}{}
\drop{O eternal Lord God, who hast brought thy servants to the beginning of another year: Pardon, we humbly beseech thee, our transgressions in the past, and graciously abide with us all the days of our life; through Jesus Christ our Lord.  \R Amen}

\prefaceCaption{}{Epiphany,}{and seven days after.}
\drop{Almighty God, who at the baptism of thy blessed Son Jesus Christ in the river Jordan didst manifest his glorious Godhead: Grant, we beseech thee, that the brightness of his presence may shine in our hearts, and his glory be set forth in our lives; through the same Jesus Christ our Lord. \R Amen.}

\prefaceCaption{}{Easter Day,}{and seven days after.}
\drop{O Lord God Almighty, whose blessed Son, our Saviour, Jesus Christ, did on the third day rise triumphant over death: Raise us, we beseech thee, from the death of sin unto the life of righteousness, that we may seek those things which are above, where he sitteth on thy right hand in glory; and this we beg for the sake of the same, thy Son, Jesus Christ our Lord. \R Amen.}

\prefaceCaption{}{Ascension Day,}{and seven days after.}
\drop{Almighty God, whose blessed Son, our Saviour, Jesus Christ, ascended far above all heavens that he might fill all things: Mercifully give us faith to perceive that according to his promise he abideth with his Church on earth, even unto the end of the world; through the same Jesus Christ our Lord. \R Amen.}

\prefaceCaption{}{Whitsunday,}{and six days after.}
\drop{O Almighty God, who on the day of Pentecost didst send the Holy Ghost the Comforter to abide in thy Church unto the end: Bestow upon us and upon all thy faithful people his manifold gifts of grace, that with minds enlightened by his truth and hearts purified by his presence, we may day by day be strengthened with power in the inward man; through Jesus Christ our Lord, who with thee and the same Spirit liveth and reigneth, one God, world without end. \R Amen.}


\prefaceCaption{}{Trinity Sunday.}{}
\drop{O Lord God Almighty, Eternal, Immortal, Invisible, the mysteries of whose being are unsearchable: Accept, we beseech thee, our praises for the revelation which thou hast made of thyself, Father, Son, and Holy Ghost, three Persons, and one God; and mercifully grant, that ever holding fast this faith, we may magnify thy glorious Name; who livest and reignest, one God, world without end. \R Amen.}




\section{Exhortations}

\pilcrow{When the Minister giveth warning for the celebration of the holy Communion, (which he shall always do upon the Sunday, or some Holy-day, immediately preceding,) after the Sermon or Homily ended, he shall read this Exhortation following.}
\drop{Dearly beloved, on — I purpose, through God’s assistance, to administer to all such as shall be religiously and devoutly disposed the most comfortable Sacrament of the Body and Blood of Christ; to be by them received in remembrance of his meritorious Cross and Passion; whereby alone we obtain remission of our sins, and are make partakers of the Kingdom of heaven.} 

Wherefore it is our duty to render most humble and hearty thanks to Almighty God our heavenly Father, for that he hath given his Son our Saviour Jesus Christ, not only to die for us, but also to be our spiritual food and sustenance in that holy Sacrament.

Which being so divine and comfortable a thing to them who receive it worthily, and so dangerous to them that will presume to receive it unworthily; my duty is to exhort you in the mean season to consider the dignity of that holy mystery, and the great peril of the unworthy receiving thereof; and so to search and examine your own consciences, (and that nor lightly, and after the manner of dissemblers with God; but so) that ye may come holy and clean to such a heavenly Feast, in the marriage-garment required by God in holy Scripture, and be received as worthy partakers of that holy Table.

The way and means thereto is; First, to examine your lives and conversations by the rule of God’s commandments; and wherein soever ye shall perceive yourselves to have offended, either by will, word, or deed, there to bewail your own sinfulness, and to confess yourselves to Almighty God, with full purpose of amendment of life. 

And if ye shall perceive your offences to be such as are not only against God, but also against your neighbours; then ye shall reconcile yourselves unto them; being ready to make restitution and satisfaction, according to the uttermost of your powers, for all injuries and wrongs done by you to any other; and being likewise ready to forgive others that have offended you, as ye would have forgiveness of your offences at God’s hand: \emph{[for otherwise the receiving of the holy Communion doth nothing else but increase your \emph{damnation} [guilt, condemnation].]}

Therefore if any of you be a blasphemer of God, an hinderer or slanderer of his Word, an adulterer, or be in malice, or envy, or in any other grievous crime, repent you of your sins, or else come not to that holy Table\rubric{; lest, after the taking of that holy Sacrament, the devil enter into you, as he entered into Judas, and fill you full of all iniquities, and bring you to destruction both of body and soul}.

And because it is requisite, that no man should come to the holy Communion, but with a full trust in God's mercy, and with a quiet conscience; therefore if there be any of you, who by this means cannot quiet his own conscience herein, but requireth further comfort or counsel, let him come to me, or to some other discreet and learned Minister of God's Word, and open his grief; that by the ministry of God's holy Word he may receive the benefit of absolution, together with ghostly counsel and advice, to the quieting of his conscience, and avoiding of all scruple and doubtfulness.

% Rubric moved to Confession.

\medskip
%a new form, composed apparently by Peter Martyr at the instance of Bucer. (i)

{\centering\rubric{Or, in case he shall see the people negligent to come to the Holy Communion, instead of the former, he may use this Exhortation.}\par}


\drop{Dearly beloved brethren, on — I intend, by God’s grace, to celebrate the Lord’s Supper: unto which, in God’s behalf, I bid you all that are here present; and beseech you, for the Lord Jesus Christ’s sake, that ye will not refuse to come thereto, being so lovingly called and bidden by God himself.}

Ye know how grievous and unkind a thing it is, when a man hath prepared a rich feast, decked his table with all kind of provision, so that there lacketh nothing but the guests to sit down; and yet they who are called (without any cause) most unthankfully refuse to come. Which of you in such a case would not be moved? Who would not think a great injury and wrong done unto him? Wherefore, most dearly beloved in Christ, take ye good heed, lest ye, withdrawing yourselves from this holy Supper, provoke God’s indignation against you. It is an easy matter for a man to say, I will not communicate, because I am otherwise hindered with worldly business. But such excuses are not so easily accepted and allowed before God. If any man say, I am a grievous sinner, and therefore am afraid to come: wherefore then do ye not repent and amend? When God calleth you, are ye not ashamed to say ye will not come? When ye should return to God, will ye excuse yourselves, and say ye are not ready? Consider earnestly with yourselves how little such feigned excuses will avail before God. They that refused the feast in the Gospel, because they had bought a farm, or would try their yokes of oxen, or because they were married, were not so excused, but counted unworthy of the heavenly feast.

I, for my part, shall be ready; and, according to mine office, I bid you in the Name of God, I call you in Christ’s behalf, I exhort you, as ye love your own salvation, that ye will be partakers of this Holy Communion. And as the Son of God did vouchsafe to yield up his soul by death upon the Cross for your salvation; so it is your duty to receive the Communion in remembrance of the sacrifice of his death, as he himself hath commanded: which if ye shall neglect to do, consider with yourselves how great injury ye do unto God, and how sore punishment hangeth over your heads for the same; when ye wilfully abstain from the Lord’s Table, and separate from your brethren, who come to feed on the banquet of that most heavenly food.

These things if ye earnestly consider, ye will by God’s grace return to a better mind: for the obtaining whereof we shall not cease to make our humble petitions unto Almighty God our heavenly Father.


\section{General Rubricks}

\pilcrow{Upon the Sundays and other Holy-days (if there be no Communion) shall be said all that is appointed at the Communion, until the end of the general Prayer \emph{For the whole state of Christ’s Church} together with one or more of these Collects last before rehearsed, concluding with the Blessing.}


\pilcrow{And whenever this Service is used, Collects, contained in this Book, or sanctioned by the Bishop, may be said after \emph{The Intercession}, or before the Blessing.}

\medskip

% 1928 Rubrics:
\pilcrow{If any of the consecrated Bread and Wine remain, apart from that which may be reserved,
% for the Communion of the sick,
it shall not be carried out of the church; but the Priest, and such other of the communicants as he shall call unto him, shall, immediately after the Blessing, reverently eat and drink the same.}

% \pilcrow{And there shall be no celebration of the Lord's Supper, except there be a convenient number to communicate with the Priest, according to his discretion.}

% \pilcrow{And if there be not above twenty persons in the Parish of discretion to receive the Communion: yet there shall be no Communion, except four (or three at the least) communicate with the Priest.}

% \pilcrow{And in Cathedral and Collegiate Churches, and Colleges, where there are many Priests and Deacons, they shall all receive the Communion with the Priest every Sunday at the least, except they have a reasonable cause to the contrary.}

% \pilcrow{And to take away all occasion of dissension, and superstition, which any person hath or might have concerning the Bread and Wine, it shall suffice that the Bread be such as is usual to be eaten; but the best and purest Wheat Bread that conveniently may be gotten.}

% \pilcrow{And if any of the Bread and Wine remain unconsecrated, the Curate shall have it to his own use: but if any remain of that which was consecrated, it shall not be carried out of the Church, but the Priest, and such other of the Communicants as he shall then call unto him, shall, immediately after the Blessing, reverently eat and drink the same.}

\pilcrow{The Bread and Wine for the Communion shall be provided by the Curate and the Church-wardens at the charges of the Parish.}


\pilcrow{After the Divine Service ended, the money given at the Offertory shall be disposed of to such pious and charitable uses, as the Minister and Church-wardens shall think fit. Wherein if they disagree, it shall be disposed of as the Ordinary shall appoint.}

\medskip

% all following from 1928Eng
\pilcrow{And note, that every confirmed member of the Church shall communicate at the least three times in the year, of which Easter to be one.} 


\pilcrow{If any be an open and notorious evil liver, or have done any wrong to his neighbours by word or deed, so that the Congregation be thereby offended; the Curate, having knowledge thereof, shall call him and advertise him, that in any wise he presume not to come to the Lord’s Table, until he hath openly declared himself to have truly repented and amended his former naughty life, that the Congregation may thereby be satisfied, which before were offended; and that he hath recompensed the parties, to whom he hath done wrong; or at least declare himself to be in full purpose so to do, as soon as he conveniently may.}


\pilcrow{The same order shall the Curate use with those betwixt whom he perceiveth malice and hatred to reign; not suffering them to be partakers of the Lord’s Table, until he known them to be reconciled. And if one of the parties so at variance be content to forgive from the bottom of his heart all that the other hath trespassed against him, and to make amends for that he himself hath offended, and the other party will not be persuaded to a godly unity, but remain still in his frowardness and malice: the Minister in that case ought to admit the penitent person to the holy Communion, and not him that is obstinate.}


\pilcrow{Provided that every Minister so advertising repelling any, as is specified in the two precedent paragraphs, shall be obliged forthwith to give an account of the same to the Bishop, and therein to obey his order and direction.}

\fleuron

%All the following from the Indian Supplement
\chapter[Order of Communion]{\stylechapter{The}{Order of Communion}{with the Reserved Sacrament.}}
\label{reservedSacrament}
\pilcrow{In the absence of a Priest, a Deacon may administer from the Reserved Sacrament, but he shall use the Collect for the Twenty-first Sunday after Trinity instead of the Absolution, and \emph{The Grace of our Lord Jesus Christ  \etc} instead of the Blessing.}

\pilcrow{When this Order is used for those unable to be present at a celebration of the Lord’s Supper but who are not sick, it shall be used with the Collect, Epistle and Gospel of the Day. At the discretion of the Minister other parts of the Order for the Lord’s Supper (except the setting apart of the Bread and Wine and the Prayer of Consecration) may be added.}

\pilcrow{When this Order is used for the sick, it is fitting that a table be prepared at a convenient place in the sick person’s house. The table shall be covered with a clean white cloth, whereon is to be placed a vessel containing a little water. The Priest shall come at the appointed time, bearing the Reserved Sacrament, and shall place it upon the table.}

% If the sick person is not too weak, the Priest may begin with the Collect, Epistle and Gospel of the Day, or else with the following:

% The Collett

% ALMIGHTY Father, giver of life and health: Look mercifully, we beseech thee, on this thy servant, that by thy blessing upon him and upon those who minister to him, he may speedily be restored to health, if it be thy gracious will, and live to the glory of thy holy name; through Jesus Christ our Lord. Amen.

% Or this,

% ASSIST us mercifully, O Lord, in these our supplications and prayers, and dispose the way of thy servant towards the attainment of everlasting salvation; that among all the changes and chances of this mortal life, he may ever be defended by thy most gracious and ready help; through Jesus Christ our Lord. Amen.

% The Epistle. 2 Corinthians 1. 3

% BLESSED be God, even the Father of our Lord Jesus Christ, the Father of mercies, and the God of all comfort; who comforteth us in all our tribulation, that we may be able to comfort them which are in any trouble, by the comfort wherewith we ourselves are comforted of God. For as the sufferings of Christ abound in us, so our consolation also aboundeth by Christ.

% The Gospel. St John 10. 15, 27-30

% IAM the good shepherd; and I know mine own, and mine own know me, even as the Father knoweth me, and I know the Father; and I lay down my life for the sheep. My sheep hear my voice, and I know them, and they follow me: and I give unto them eternal life; and they shall never perish, and no one shall pluck them out of my hand. My Father, which hath given them unto me, is greater than all; and no one is able to pluck them out of the Father's hand. I and the Father are one.
 

 
\pilcrow{Then the Priest and those present shall say the \emph{General Confession} from the Order for the Lord’s Supper, the Priest adding the \emph{Absolution} and the \emph{Comfortable Words}, or else the following shortened Confession and Absolution may be used:}
\drop{Draw near with faith, and take this Holy Sacrament to your comfort; and make your humble confession to Almighty God.}

\medskip
\drop{We confess to God Almighty, the Father, the Son, and the Holy Ghost, that we have sinned in thought, word, and deed, through our own grievous fault. Wherefore we pray God to have mercy upon us.}

\medskip
{\centering\footnotesize\rubric{After which the Priest shall pronounce this Absolution,}\par} 
\drop{Almighty God have mercy upon you, forgive \grealtcross\ you all your sins, and deliver you from all evil, confirm and strengthen you in all goodness, and bring you to life everlasting; through Jesus Christ our Lord. \R Amen.}

\medskip
{\centering\footnotesize\rubric{Then shall the Priest and those who are to receive the Sacrament say together:}\par} 
\drop{We do not presume to come to this thy Table, O merciful Lord, trusting in our own righteousness, but in thy manifold and great mercies. We are not worthy so much as to gather up the crumbs under thy Table. But thou art the same Lord, whose property is always to have mercy: Grant us therefore, gracious Lord, so to eat the flesh of thy dear Son Jesus Christ, and to drink his blood, that our sinful bodies may be made clean by his body, and our souls washed through his most precious blood, and that we may evermore dwell in him, and he in us. \R Amen.}


\medskip
% Scottish Position:
\pilcrow{Then may be sung or said:} %scottish rubric

\drop{O Lamb of God, that takest away the sins of the world: have mercy upon us.

O Lamb of God that, takest away the sins of the world: have mercy upon us.

O Lamb of God, that takest away the sins of the world: grant us thy peace.}
\bigskip

\medskip
\rubric{After a short space of silence the Priest shall deliver the Sacrament with the customary words of administration.}


\medskip

{\centering\footnotesize\rubric{After a short silence the Priest and all present shall repeat together the Lord's Prayer,}\par} 
\ourFather


\medskip
{\centering\footnotesize\rubric{Then the Priest shall say,}\par} 
\drop{Almighty and everliving God, we most heartily thank thee, for that thou dost vouchsafe to feed us, who have duly received these holy mysteries, with the spiritual food of the most precious Body and Blood of thy Son our Saviour Jesus Christ; and dost assure us thereby of thy favour and goodness towards us; and that we are very members incorporate in the mystical body of thy Son, which is the blessed company of all faithful people; and are also heirs through hope of thy everlasting kingdom, by the merits of the most precious death and passion of thy dear Son. And we most humbly beseech thee, O heavenly Father, so to assist us with thy grace, that we may continue in that holy fellowship, and do all such good works as thou hast prepared for us to walk in; through Jesus Christ our Lord, to whom, with thee and the Holy Ghost, be all honour and glory, world without end. Amen.}

\medskip
{\centering\footnotesize\rubric{And the Blessing,}\par}
\drop{The peace of God, which passeth all understanding, keep your hearts and minds in the knowledge and love of God, and of his son Jesus Christ our Lord: and the blessing of God Almighty, the Father, \grealtcross\ the Son, and the Holy Ghost, be amongst you and remain with you always. \R Amen.}


\fleuron


% \chapter{ An Alternative Order of the Communion}
% \ourFather

% \subsection{\stylesubsec{}{The Collect.}{}}
% \drop{Almighty God, unto whom all hearts be open, all desires known, and from whom no secrets are hid; Cleanse the thoughts of our hearts by the inspiration of thy Holy Spirit, that we may perfectly love thee, and worthily magnify thy holy Name; through Christ our Lord. \R Amen.}

% \medskip
% \centerline{\pilcrow{Here may be sung a Hymn or an Anthem.}}

% \medskip

% \centerline{Lord, have mercy upon us. \rubric{iij.}}
% \centerline{Christ, have mercy upon us. \rubric{iij.}}
% \centerline{Lord, have mercy upon us. \rubric{iij.}}


% \bigskip
% \pilcrow {Then the priest shall turn him to the people and say,}
% \V The Lord be with you. \R And with thy spirit.

% \centerline{Let us pray.}

% \medskip
% \pilcrow{And turning to the Holy Table he shall say the Collect of the Day. Other collects contained in this Book or authorized by the Bishop may follow.}

% \medskip

% \section{The Ministry of the Word}

% \pilcrow{Immediately thereafter he that readeth the Epistle shall say,}
% The Epistle [\rubric{or,} The portion of Scripture appointed for the Epistle] is written in the --- chapter of --- beginning at the --- verse. 
% \rubric{And the Epistle ended, he shall say,} Here endeth the Epistle.

% \medskip
% \centerline{\pilcrow{Here may be sung a Hymn or an Anthem.}}
% \medskip

% \pilcrow{Then the Deacon or Priest that readeth the Gospel (the people all standing up) shall say,}
% \V The Lord be with you. \R And with thy Spirit. 

% \drop{The \grealtcross\ Holy Gospel is written in the — chapter of — beginning at the — verse. \R Glory \grecross\ be to thee, O Lord.}

% \centerline{\rubric{The Gospel ended, there may be said,}}
% \centerline{Praise be to thee, O Christ.}


% \bigskip

% \pilcrow{Then shall be sung or said the Creed following, the people still standing as before: except that at the discretion of the Minister it may be omitted on any day not being a Sunday or a Holy-day.} 

% \drop{I believe in one God the Father Almighty, Maker of heaven and earth, And of all things visible and invisible:}

% And in one Lord Jesus Christ, the only-begotten son of God, Begotten of his Father before all worlds, God of God, Light of Light, Very God of very God, Begotten, not made, Being of one substance with the Father, By whom all things were made: Who for us men, and for our salvation came down from heaven, * And was incarnate by the Holy Ghost of the Virgin Mary, * And was made man, * And was crucified also for us under Pontius Pilate. He suffered and was buried, And the third day he rose again according to the Scriptures, And ascended into heaven, And sitteth on the right hand of the Father. And he shall come again with glory to judge both the quick and the dead: Whose kingdom shall have no end.

% And I believe in the Holy Ghost, The Lord and giver of life, Who proceedeth from the Father and the Son, Who with the Father and the Son together is worshipped and glorified, Who spake by the Prophets. And I believe One {Holy} Catholick and Apostolick Church. I acknowledge one Baptism for the remission of sins. And I look for the Resurrection of the dead, * And the life of the world to come. Amen.

% \bigskip
% \pilcrow{Then the Curate shall declare unto the people what Holy-days, or Fasting-days, are in the week following to be observed. And then also (if occasion be) shall notice be given of the Holy Communion, or of other services; Banns of matrimony may be published, and Briefs, Citations, and Excommunications shall be read, and Bidding of Prayers may be made. And nothing shall be proclaimed or published in the Church during the time of Divine Service, but by the Minister: nor by him any thing, but what is prescribed in the rules of this Book, or enjoined by the %Queen, or by the 
% Ordinary of the place.}% adapted toward En1928

% \smallskip

% \pilcrow{Then may follow the Sermon, or one of the Homilies already set forth, or hereafter to be set forth, by authority.}


% \bigskip

% \section{The Offertory}

% \pilcrow{Then shall the Priest return to the Lord's Table, and begin the \emph{Offertory}. The Priest shall say, or the Clerks shall sing, one of these Sentences following, or some other convenient sentence taken out of Holy Scripture. A Hymn may follow.}

% \drop{To do good, and to distribute, forget not; for with such sacrifices God is pleased.\scripture{Hebrews xiij.~16.}}

% Lay not up for yourselves treasure upon the earth; where the rust and moth doth corrupt, and where thieves break through and steal: but lay up for yourselves treasures in heaven; where neither rust nor moth doth corrupt, and where thieves do not break through and steal.\scripture{St.~Matthew vj.~19.}

% Not every one that saith unto me, Lord, Lord, shall enter into the kingdom of heaven; but he that doeth the will of my Father which is in heaven.\scripture{St.~Matthew vij.~21.}

% If we have sown unto you spiritual things, is it a great matter if we shall reap your worldly things?\scripture{1 Corithians ix.~11.}

% Do ye not know, that they who minister about holy things live of the sacrifice; and they who wait at the altar are partakers with the altar? Even so hath the Lord also ordained, that they who preach the Gospel should live of the Gospel.\scripture{1 Corithians ix.~13.}

% He that soweth little shall reap little; and he that soweth plenteously shall reap plenteously. Let every man do according as he is disposed in his heart, not grudging, or of necessity; for God loveth a cheerful giver.\scripture{2 Corithians ix.~6.}

% Let him that is taught in the Word minister unto him that teacheth, in all good things. Be not deceived, God is not mocked: for whatsoever a man soweth that shall he reap.\scripture{Galatians vj.~6.}

% While we have time, let us do good unto all men; and specially unto them that are of the household of faith.\scripture{Galatians vj.~10.}

% God is not unrighteous, that he will forget your works, and labour that proceedeth of love; which love ye have shewed for his Name’s sake, who have ministered unto the saints, and yet do minister.\scripture{Hebrews vj.~10.}

% Charge them who are rich in this world, that they be ready to give, and glad to distribute; laying up in store for themselves a good foundation against the time to come, that they may attain eternal life.\scripture{1 Timothy vj.~17.}

% Whoso hath this world’s good, and seeth his brother have need, and shutteth up his compassion from him, how dwelleth the love of God in him?\scripture{1 St.~John iij.~17}



% \pilcrow{Whilst these Sentences are said or sung, the Deacons, Church-wardens, or other fit person appointed for that purpose, shall receive the alms for the poor, or other devotions of the people, and reverently bring them to the Priest, who shall humbly present and place them upon the Holy Table in a decent bason to be provided for that purpose.}

% \bigskip

% \pilcrow{{And when there is a Communion,} the Priest shall then offer up, and place the bread and wine prepared for the Sacrament upon the Lord’s Table; and shall say,}

% \drop{Blessed be thou, O {\scshape Lord} God, for ever and ever. Thine, O {\scshape Lord}, is the greatness, and the glory, and the victory, and the majesty: for all that is in the heaven and in the earth, is thine: thine is the kingdom, O {\scshape Lord}, and thou art exalted as head above all: both riches and honour come of thee, and of thine own do we give unto thee. \R Amen.}
% \scripture{1 Chronicles xxix.~10}

% \bigskip
% \centerline{Let us pray for the whole state of Christ's Church.}

% \drop{Almighty and everliving God, who by thy holy Apostle hast taught us to make prayers, and supplications, and to give thanks for all men;}
% We humbly beseech thee most mercifully [\footnote{\rubric{If there be no alms or oblations, then the words \emph{[to accept our alms and oblations]} be left out unsaid.}}\emph{to accept our alms and \grealtcross\ oblations, and}] to receive these our prayers, which we offer unto thy Divine Majesty; beseeching thee to inspire continually the Universal Church with the spirit of truth, unity, and concord: And grant, that all they that do confess thy holy Name may agree in the truth of thy holy Word, and live in unity, and godly love.

% We beseech thee also to lead all nations into the way of righteousness and peace, and so to direct all 
% ruling authorities, that under them the world may be godly and quietly governed. 
% And grant unto all that are put in authority, 
% that they may truly and indifferently minister justice, to the punishment of wickedness and vice, and to the maintenance of thy true religion, and virtue.

% Give grace, O heavenly Father, to all Bishops, Priests, and Deacons, especially to thy servant \emph{N.} our bishop, that they may both by their life and doctrine set forth thy true and lively Word, and rightly and duly administer thy holy Sacraments.

% And to all thy people give thy heavenly grace; and specially to this congregation here present; that with meek heart and due reverence, they may hear and receive thy holy Word; truly serving thee in holiness and righteousness all the days of their life.

% And we most humbly beseech thee, of thy goodness, O Lord, to comfort and succour all them, who in this transitory life are in trouble, sorrow, need, sickness, or any other adversity.

% And we commend to thy gracious keeeping, O Lord, all thy servants departed this life in thy faith and fear, beseeching them to grant them everlasting light and peace.

% And here we give thee most high praise and hearty thanks for all thy Saints, who have been the chosen vessels of thy grace, and lights of the world in their several generations; and we pray, that rejoicing in their fellowship, and following their good examples, we may be partakers with them of thy heavenly kingdom.

% Grant this, O Father, for Jesus Christ's sake, our only Mediator and Advocate. \R Amen.

% \section{The Consecration}

% \pilcrow{Turning himself to the people the Priest shall say,}

% \V The Lord be with you. \R And with thy spirit.

% \V Lift up your hearts.  \R We lift them up unto the Lord.

% \V Let us give thanks unto our Lord God. \R It is meet and right so to do.


% \centerline{\rubric{Then shall the Priest turn to the Lord’s Table, and say,}}

% \drop{It is very meet, right, and our bounden duty, that we should at all times, and in all places, give thanks unto thee, O Lord, Holy Father, Almighty, Everlasting God.


% {\centering\rubric{Here shall follow the Proper Preface, according to the time, if there be any specially appointed, or else immediately shall follow,}\par}


% Therefore with Angels and Archangels, and with all the company of heaven, we laud and magnify thy glorious Name; evermore praising thee, and saying,}
% \smallskip

% \drop{Holy, holy, holy, Lord God of hosts, heaven and earth are full of thy glory: Glory be to thee, O Lord most High.}
% \begin{leftbar}
%     \grecross\ Blessed is he that cometh in the Name of the Lord; Hosanna in the highest.
% \end{leftbar}

% \centerline{\rubric{Then shall the priest continue thus.}}
% \drop{All glory be to thee, Almighty God, our heavenly Father, for that thou of thy tender mercy didst give thine only Son Jesus Christ to suffer death upon the Cross for our redemption; who made there (by his one oblation of himself once offered) a full, perfect, and sufficient sacrifice, oblation, and satisfaction, for the sins of the whole world; and did institute, and in his holy Gospel command us to continue, a perpetual memory of that his precious death, until his coming again;

% Hear us, O merciful Father, we most humbly beseech thee; and grant that we receiving these thy creatures of bread and wine, according to thy Son our Saviour Jesus Christ’s holy institution, in remembrance of his death and passion, may be partakers of his most blessed \grealtcross\ Body and \grealtcross\ Blood: 

% Who, in the same night that he was betrayed, \footnote{\rubric{Here the Priest is to take the Paten unto his hands:}}took Bread; and, when he had given thanks, \footnote{\rubric{And here to break the Bread:}}he brake it, and gave it to his disciples, saying, Take, eat, \footnote{\rubric{And here to lay his hand upon all the Bread.}}{\scshape this is my Body which is given for you}: Do this in remembrance of me. Likewise after supper he \footnote{\rubric{Here he is to take the Cup into his hand:}}took the Cup; and, when he had given thanks, he gave it to them, saying, Drink ye all of this; \footnote{\rubric{And here to lay his hand upon every vessel (be it Chalice or Flagon) in which there is any Wine to be consecrated.}}{\scshape for this is my Blood of the New Testament, which is shed for you and for many for the remission of sins}: Do this, as oft as ye shall drink it, in remembrance of me.}

% Wherefore, O Lord and heavenly Father, we thy humble servants, having in remembrance the precious death of thy dear Son, his mighty resurrection and glorious ascension, looking also for his coming again, do render unto thee most hearty thanks for the innumerable benefits which he hath procured unto us; and we entirely desire thy fatherly goodness mercifully to accept this our sacrifice of praise and thanksgiving; most humbly beseeching thee to grant, that by the merits and death of thy Son Jesus Christ, and through faith in his blood, we and all thy whole Church may obtain remission of our sins, and all other benefits of his passion.

% And here we offer and present unto thee, O Lord, ourselves, our souls and bodies, to be a reasonable, holy, and lively sacrifice unto thee; 
% and we pray thee of thine almighty goodness to send upon us, and upon these thy gifts, thy holy and blessed Spirit, who is the Sanctifier and the Giver of life; humbly beseeching thee, that all we, who are partakers of this holy Communion, may be fulfilled with thy grace and heavenly \grecross\ benediction. 

% And although we be unworthy, through our manifold sins, to offer unto thee any sacrifice, yet we beseech thee to accept this our bounden duty and service; not weighing our merits, but pardoning our offences;

% Through Jesus Christ our Lord; by whom, and with whom, in the unity of the Holy Ghost, all honour and glory be unto thee, O Father Almighty, world without end. \R Amen.

% \smallskip
% {\centering\footnotesize\rubric{Here shall the people join with the Priest in the Lord’s Prayer, the Priest first saying,}\par}
% As our Saviour Christ hath commanded and taught us we are bold to say,
% \drop{Our Father, which art in heaven, Hallowed be thy Name. Thy kingdom come. Thy will be done, in earth as it is in heaven. Give us this day our daily bread. And forgive us our trespasses, As we forgive them that trespass against us. And lead us not into temptation; But deliver us from evil.}
% For thine is the kingdom, the power, and the glory, For ever and ever. Amen.

% \medskip
% \begin{leftbar}
%     \pilcrow{Then may the Priest say,}
%     \drop{The Peace of the Lord be always with you.  \R And with thy spirit.}
% \end{leftbar}

% \bigskip
% \pilcrow{Then shall the Priest say to them that come to receive the holy Communion,}
% \drop{Draw near with faith, and take this Holy Sacrament to your comfort; and make your humble confession to Almighty God, meekly kneeling upon your knees.}

% \smallskip
% \rubric{Then shall be said by the Minister and people together,}
% \drop{We confess to God Almighty, the Father, the Son, and the Holy Ghost, that we have sinned in thought, word, and deed, through our own grievous fault.  Wherefore we pray God to have mercy upon us.}

% \medskip
% {\centering\footnotesize\rubric{Then shall the Priest standing up, and turning himself to the people, pronounce this Absolution.}\par}
% \drop{Almighty God have mercy upon you, \grealtcross\ forgive you all your sins, and deliver you from all evil, confirm and strengthen you in all goodness, and bring you to everlasting life; through Jesus Christ our Lord. \R Amen.}

% \centerline{\pilcrow{Then shall the Priest say,}}
% Hear what comfortable words our Saviour Christ saith unto all that truly turn to him.
% \drop{Come unto me all that travail and are heavy laden, and I will refresh you.}\scripture{St.~Matthew xj.~28}

% So God loved the world, that he gave his only-begotten Son, to the end that all that believe in him should not perish, but have everlasting life.\scripture{St.~John iij.~16}

% \centerline{Hear also what Saint Paul saith.}

% This is a true saying, and worthy of all men to be received, That Christ Jesus came into the world to save sinners.\scripture{1 Timothy i.~15.}

% \centerline{Hear also what Saint John saith.}

% If any man sin, we have an Advocate with the Father, Jesus Christ the righteous; and he is the propitiation for our sins.\scripture{1 St.~John ij.~1.}

% \medskip

% {\centering\footnotesize\rubric{Then shall the Priest, kneeling down at the Lord’s Table, say in the name of all them that shall receive the Communion this Prayer following.}\par}
% \drop{We do not presume to come to this thy Table, O merciful Lord, trusting in our own righteousness, but in thy manifold and great mercies. We are not worthy so much as to gather up the crumbs under thy Table. But thou art the same Lord, whose property is always to have mercy: Grant us therefore, gracious Lord, so to eat the flesh of thy dear Son Jesus Christ, and to drink his blood, that our sinful bodies may be made clean by his body, and our souls washed through his most precious blood, and that we may evermore dwell in him, and he in us. Amen.}

% \medskip

% \pilcrow{Then shall the Minister first receive the Communion in both kinds himself, and then proceed to deliver the same to the Bishops, Priests, and Deacons, in like manner, (if any be present,) and after that to the people also in order, into their hands, all meekly kneeling. And, when he delivereth the Bread to any one, he shall say,}
% \drop{The Body of our Lord Jesus Christ, which was given for thee, preserve thy body and soul unto everlasting life. \R Amen.}

% \smallskip
% {\centering\footnotesize\rubric{And the Minister that delivereth the Cup to any one shall say,}\par}
% \drop{The Blood of our Lord Jesus Christ, which was shed for thee, preserve thy body and soul unto everlasting life.}

% \medskip
% \centerline{\pilcrow{Here may be sung a Hymn or an Anthem.}}
% \medskip

% {\footnotesize\rubric{When all have communicated, the Minister shall return to the Lord’s Table, and reverently place upon it what remaineth of the consecrated Elements, covering the same with a fair linen cloth.}\par}

% \medskip
% {\centering\footnotesize\rubric{Then the Priest shall say,}\par} 
% \drop{Almighty and everliving God, we most heartily thank thee, for that thou dost vouchsafe to feed us, who have duly received these holy mysteries, with the spiritual food of the most precious Body and Blood of thy Son our Saviour Jesus Christ; and dost assure us thereby of thy favour and goodness towards us; and that we are very members incorporate in the mystical body of thy Son, which is the blessed company of all faithful people; and are also heirs through hope of thy everlasting kingdom, by the merits of the most precious death and passion of thy dear Son. And we most humbly beseech thee, O heavenly Father, so to assist us with thy grace, that we may continue in that holy fellowship, and do all such good works as thou hast prepared for us to walk in; through Jesus Christ our Lord, to whom, with thee and the Holy Ghost, be all honour and glory, world without end. Amen.}

% \medskip
% \pilcrow{Then, on Sundays and on Feast days (except in Advent and from Septuagesima to Palm Sunday inclusive), shall be said or sung as follows:}

% \drop{Glory be to God on high, and in earth peace, good will towards men. We praise thee, we bless thee, \footnote{Bow} we worship thee, we glorify thee, we give thanks to thee for thy great glory, O Lord God, heavenly King, God the Father Almighty.}

% O Lord, the only-begotten Son Jesu Christ; O Lord God, Lamb of God, Son of the Father, that takest away the sins of the world, have mercy upon us. 
% Thou that takest away the sins of the world, * receive our prayer. Thou that sittest at the right hand of God the Father, have mercy upon us.

% For thou only art holy; thou only art the Lord; thou only, * O {Jesu} Christ, with the Holy Ghost, art most high \grecross\ in the glory of God the Father. Amen.


% {\centering\rubric{This hymn may be omitted here, and sung instead at the beginning of this Order after the \emph{Kyrie eleison}.}\par}


% \bigskip
% {\centering\rubric{Then the Priest shall let them depart with this Blessing.}\par}
% \drop{The peace of God, which passeth all understanding, keep your hearts and minds in the knowledge and love of God, and of his son Jesus Christ our Lord: and the blessing of God Almighty, the Father, \grealtcross\ the Son, and the Holy Ghost, be amongst you and remain with you always. \R Amen.}

% {\centering\rubric{Or, if there be no congregation,}\par}
% \drop{In the Name of the Father, \grealtcross\ and of the Son, and of the Holy Ghost. \R Amen.}



\chapter[The Order of Baptism]{\stylechapter{The Ministration of}{Holy Baptism}{To be Used in the Church }}

% 1549
\pilcrow{It appeareth by ancient writers that the Sacrament of Bapitism in the old time was not commonly ministered but at two times in the year, at Easter and Whitsuntide, at which times it was only ministered in presence of all the congregation: which custom now being grown out of use, although it cannot for many considerations be well restored again, yet it is thought good to follow the same, as near as conveniently may be: wherefore the people are to be admonished, that it is most convenient that Baptism should not be ministered but upon Sundays and other Holy Days, when the most number of people may come together, as well for that the Congregation there present may testify the receiving of them that be newly baptized into the number of Christ’s Church, as also because in the Baptism of Infants, every man present may be put in remembrance of his own profession made to God in his Baptism. Neverthelesse (if necessity so require) children may at all times be baptized at home.}

\smallskip

\pilcrow{Due notice, normally of at least a week, shall be given before a child is brought to the church to be baptized.}

\medskip


\pilcrow{At the time appointed the godfathers and godmothers and [the parents or guardians with] the \emph{child} must be ready at the church door, either immediately before the last Canticle at Mattins or else immediately before the last Canticle at Evensong, as the Curate by his discretion shall appoint.  And the Priest standing there shall proceed as follows.}

\drop{Hath this Child [\emph{or} Person] been already baptized or no?}

\smallskip

\centerline{\pilcrow{If they answer, \emph{No}: Then shall the Priest proceed as followeth.}}



\section{Admission to the Catechuminate}
\drop{Dearly beloved, forasmuch as all men are conceived and born in sin: and that our Saviour Christ saith, None can enter into the kingdom of God, except he be regenerate and born anew of Water and of the Holy Ghost: I beseech you to call upon God the Father, through our Lord Jesus Christ, that of his bounteous mercy he will grant to \emph{these Children} [\rubric{or} \emph{these persons}] that thing which by nature \emph{they} cannot have; that \emph{they} may be baptized with Water and the Holy Ghost, and received into Christ’s holy Church, and be made \emph{lively members} of the same.}

\centerline{\pilcrow{Then shall the Priest say,}}
\centerline{Let us pray.}
\drop{Almighty and everlasting God, who of thy great mercy didst save Noah and his family in the ark from perishing by water; and also didst safely lead the children of Israel thy people through the Red Sea, figuring thereby thy holy Baptism; and by the Baptism of thy well-beloved Son Jesus Christ, in the river Jordan, didst sanctify Water to the mystical washing away of sin: We beseech thee, for thine infinite mercies, that thou wilt mercifully look upon \emph{these Children} [\rubric{or} \emph{these} thy \emph{Servants and Handmaidens}]; wash \emph{them} and sanctify \emph{them} with the Holy Ghost; that \emph{they}, being delivered from thy wrath, may be received into the ark of Christ’s Church; and being stedfast in faith, joyful through hope, and rooted in charity, may so pass the waves of this troublesome world, that finally \emph{they} may come to the land of everlasting life, there to reign with thee world without end; through Jesus Christ our Lord. \R Amen.}

\medskip

%% <1549>
\pilcrow{Here shall the priest ask what shall be the name of the \emph{child}, [and when the Godfathers and Godmothers have told the name,] then shall he make a cross upon the \emph{child’s} forehead and breast, saying,}

\lettrine{\emph{N.}}{ receive} the sign of the holy Cross, both in thy \grealtcross\ forehead, and in thy \grealtcross\ breast, in token that thou shalt not be ashamed to confess thy faith in Christ crucified, and manfully to fight under his banner, against sin, the world, and the devil; and to continue Christ’s faithful soldier and servant unto thy life’s end. \R Amen.

{\footnotesize\rubric{And this he shall do and say to as many as be presented to be baptized, one after another.}\par}

\medskip


\centerline{Let us pray.}
%% </1549>
\drop{Almighty and immortal God, the aid of all that need, the helper of all that flee to thee for succour, the life of them that believe, and the resurrection of the dead: We call upon thee for \emph{these Infants} [\rubric{or} \emph{these Persons}], that \emph{they}, coming to thy holy Baptism, may receive remission of \emph{their} sins by spiritual regeneration. Receive \emph{them}, O Lord, as thou hast promised by thy well-beloved Son, saying, Ask, and ye shall have; seek, and ye shall find; knock, and it shall be opened unto you: So give now unto us that ask; let us that seek find; open the gate unto us that knock; that \emph{these Infants} [\rubric{or} \emph{these Persons}] may enjoy the everlasting benediction of thy heavenly washing, and may come to the eternal kingdom which thou hast promised by Christ our Lord. \R Amen.}


\medskip


%% <1549>
\pilcrow{Then let the priest, looking upon the \emph{Children}, say,} 

\drop{I command thee, unclean spirit, in the name of the Father, of the Son, and of the Holy Ghost, that thou come out, and depart from \emph{these Infants} [\rubric{or} \emph{these Persons}], whom our Lord Jesus Christ hath vouchsafed to call to his holy Baptism, to be made \emph{members} of his body, and of his holy congregation. Therefore thou cursed spirit, remember thy sentence, remember thy judgment, remember the day to be at hand wherein thou shalt burn in fire everlasting, prepared for thee and thy angels. And presume not hereafter to exercise any tyranny towards \emph{these Infants} [\rubric{or} \emph{these Persons}], whom Christ hath bought with his precious blood, and by this his holy Baptism, called to be of his flock, In the name of the same our Lord Jesus Christ, who shall come to judge the quick and the dead and the world by fire. \R Amen.}
%% </1549>


\medskip


\pilcrow{Then, if all candidates be children, the Priest shall say the Gospel and Exhortation as follow,}
\V The Lord be with you.  \R And with thy spirit.

\drop{Hear the words of the \grecross\ Gospel, written by Saint Mark, in the tenth chapter, at the thirteenth verse. \R Glory be to thee, O Lord.}

\smallskip

\drop{They brought young children to Christ, that he should touch them; and his disciples rebuked those that brought them. But when Jesus saw it, he was much displeased, and said unto them, Suffer the little children to come unto me, and forbid them not; for of such is the kingdom of God. Verily I say unto you, Whosoever shall not receive the kingdom of God as a little child, he shall not enter therein. And he took them up in his arms, put his hands upon them, and blessed them.}

\centerline{\R Praise be to thee, O Christ.}

\medskip
\pilcrow{After the Gospel is read, the Minister shall make this brief Exhortation upon the words of the Gospel.}

\drop{Beloved, ye hear in this Gospel the words of our Saviour Christ, that he commanded the children to be brought unto him; how he blamed those that would have kept them from him; how he exhorteth all men to follow their innocency. Ye perceive how by his outward gesture and deed he declared his good will toward them; for he embraced them in his arms, he laid his hands upon them, and blessed them. Doubt ye not therefore, but earnestly believe, that he will likewise favourably receive \emph{these} present \emph{Infants}; that he will embrace \emph{them} with the arms of his mercy; that he will give unto \emph{them} the blessing of eternal life, and make \emph{them} partaker\emph{s} of his everlasting kingdom.

Wherefore we being thus persuaded of the good will of our heavenly Father towards \emph{these Infants}, declared by his Son Jesus Christ; and nothing doubting but that he favourably alloweth this charitable work of ours in bringing \emph{these Infants} to his holy Baptism; let us faithfully and devoutly give thanks unto him, and say the prayer which the Lord himself taught. And in declaration of our faith, let us also recite the articles contained in our Creed.} %and say,}


\begin{leftbar}
\centerline{\pilcrow{Or, for those of Riper Years,}}
\V The Lord be with you.  \R And with thy spirit.

\drop{Hear the words of the \grecross\ Gospel, written by Saint John, in the third chapter, beginning at the first verse.  \R Glory be to thee, O Lord.}

\smallskip

\drop{There was a man of the Pharisees, named Nicodemus, a ruler of the Jews. The same came to Jesus by night, and said unto him, Rabbi, we know that thou art a teacher come from God; for no man can do these miracles that thou doest, except God be with him. Jesus answered and said unto him, Verily, verily I say unto thee, except a man be born again, he cannot see the kingdom of God. Nicodemus saith unto him, How can a man be born when he is old? Can he enter the second time into his mother’s womb, and be born? Jesus answered, Verily, verily I say unto thee, except a man be born of water and of the Spirit, he cannot enter into the kingdom of God. That which is born of the flesh is flesh; and that which is born of the Spirit is spirit. Marvel not that I said unto thee, Ye must be born again. The wind bloweth where it listeth, and thou hearest the sound thereof; but canst not tell whence it cometh, and whither it goeth: so is every one that is born of the Spirit.}

\centerline{\R Praise be to thee, O Christ.}

\medskip

\pilcrow{After which he shall say this Exhortation following.}
\drop{Beloved, ye hear in this Gospel the express words of our Saviour Christ, that except a man be born of water and of the Spirit, he cannot enter into the kingdom of God. Whereby ye may perceive the great necessity of this Sacrament, where it may be had. Likewise, immediately before his ascension into heaven, (as we read in the last chapter of Saint Mark’s Gospel,) he gave command to his disciples, saying, Go ye into all the world, and preach the Gospel to every creature. He that believeth and is baptized shall be saved; but he that believeth not shall be condemned. Which also sheweth unto us the great benefit we reap thereby. For which cause Saint Peter the Apostle, when upon his first preaching of the Gospel many were pricked at the heart, and said to him and the rest of the Apostles, Men and brethren, what shall we do? replied and said unto them, Repent, and be baptized every one of you for the remission of sins, and ye shall receive the gift of the Holy Ghost. For the promise is to you and your children, and to all that are afar off, even as many as the Lord our God shall call. \footnote{\rubric{These words may be omitted.}}[And with many other words exhorted he them, saying, Save yourselves from this untoward generation. For (as the same Apostle testifieth in another place) even Baptism doth also now save us, (not the putting away of the filth of the flesh, but the answer of a good conscience towards God,) by the resurrection of Jesus Christ.] Doubt ye not therefore, but earnestly believe, that he will favourably receive these present persons, truly repenting, and coming unto him by faith; that he will grant them remission of their sins, and bestow upon them the Holy Ghost; that he will give them the blessing of eternal life, and make them partakers of his everlasting kingdom.}

Wherefore we being thus persuaded of the good will of our heavenly Father towards \emph{these persons}, declared by his Son Jesus Christ; let us faithfully and devoutly give thanks to him, and say the prayer which the Lord himself taught. And in declaration of our faith, let us also recite the articles contained in our Creed.
\end{leftbar}


\pilcrow{Here the minister with the Godfathers, Godmothers, and people present, shall say,}

\drop{Our Father, which art in heaven, Hallowed be thy Name. Thy kingdom come. Thy will be done, in earth as it is in heaven. Give us this day our daily bread. And forgive us our trespasses, As we forgive them that trespass against us. And lead us not into temptation; But deliver us from evil. Amen.}
 

% And then shall say openly.
\medskip

\drop{I believe in God the Father Almighty, Maker of heaven and earth:}

And in Jesus Christ his only Son our Lord: Who was conceived by the Holy Ghost, Born of the Virgin Mary: Suffered under Pontius Pilate, Was crucified, dead, and buried: He descended into hell; The third day he rose again from the dead: He ascended into heaven, And sitteth on the right hand of God the Father Almighty: From thence he shall come to judge the quick and the dead.

I believe in the Holy Ghost: The holy Catholick Church; The Communion of Saints: The Forgiveness of sins: The Resurrection of the body, And the Life everlasting. Amen.

\smallskip

\centerline{\rubric{The priest shall add also this prayer.}}
\drop{Almighty and everlasting God, heavenly Father, we give thee humble thanks, for that thou hast vouchsafed to call us to the knowledge of thy grace, and faith in thee: Increase this knowledge, and confirm this faith in us evermore. Give thy Holy Spirit to \emph{these Infants} [\rubric{or} \emph{these Persons}], that \emph{they} may be born again, and be made \emph{an heir} of everlasting salvation; through our Lord Jesus Christ, who liveth and reigneth with thee and the Holy Spirit, now and for ever. \R Amen.}

\medskip

\pilcrow{Then let the priest take [one of] the \emph{children} by the right hand, [the others being brought after him]. And coming into the Church say,}

\drop{The Lord vouchsafe to receive \emph{you} into his holy household, and to keep and govern \emph{you} alway in the same, that \emph{ye may} have everlasting life. Amen.}




% It consisted of three acts. First the closing exorcism — the Bishop stretching his hands over them as they knelt facing eastwards, prayed for the last time for the ejection of the evil spirit from them; secondly, the exsufflation — he breathed in their faces; thirdly, the Effeta — he touched each candidate on the mouth, ears, \etc, with spittle or oil, after the example of our Lord’s action in; healing the deaf and dumb man.9

% sarum:
% Exorcism, Gospel, Effeta, LordsPrayer/creed. Signing on right hand + blessing, introduction to church.




\section{The Promises}

\pilcrow{Here may be sung a Hymn or an Anthem, \emph{or} the Invocations and Conclusion from the Litany.}

\medskip

\pilcrow{Then, standing at the Font, shall the Priest speak unto the Godfathers and Godmothers on this wise.}

\drop{Dearly beloved, ye have brought \emph{these Children} here to be baptized, ye have prayed that our Lord Jesus Christ would vouchsafe to receive \emph{them}, to release \emph{them} of \emph{their} sins, to sanctify \emph{them} with the Holy Ghost, to give \emph{them} the kingdom of heaven, and everlasting life. Ye have heard also that our Lord Jesus Christ hath promised in his Gospel to grant all these things that ye have prayed for: which promise he, for his part, will most surely keep and perform. 

Wherefore, after this promise made by Christ, \emph{these Infants} must also faithfully, for \emph{their} part, promise by you that are \emph{their} sureties, (until \emph{they} come of age to take it upon \emph{themselves},) that \emph{they} will renounce the devil and all his works, and constantly believe God’s holy Word, and obediently keep his commandments.}


\begin{leftbar}
\centerline{\rubric{Or, for those of Riper Years,}}

\pilcrow{Then the Priest shall speak to the persons to be baptized on this wise}

\drop{Well-beloved, who are come hither desiring to receive Holy Baptism, \emph{ye have} heard how the congregation hath prayed, that our Lord Jesus Christ would vouchsafe to receive \emph{you} and bless \emph{you}, to release \emph{you} of \emph{your} sins, to give \emph{you} the kingdom of heaven, and everlasting life. \emph{Ye have} heard also, that our Lord Jesus Christ hath promised in his holy Word to grant all those things that we have prayed for; which promise he, for his part, will most surely keep and perform.}

Wherefore, after this promise made by Christ, \emph{ye} must also faithfully, for \emph{your} part, promise in the presence of these \emph{your} Witnesses, and this whole congregation, that \emph{ye will} renounce the devil and all his works, and constantly believe God’s holy Word, and obediently keep his commandments.
\end{leftbar}

\medskip
\pilcrow{Then shall the Priest demand of each of the \emph{Children} to be baptized, severally, these questions following:}

\smallskip
\centerline{\rubric{Question.}}
\centerline{I demand therefore,}

\drop{Dost thou %, in the name of this Child, 
renounce the devil and all his works?\\
\qa{Answer.}  I renounce him.

\smallskip
\centerline{\rubric{Question.}}
Dost thou renounce the vain pomp and glory of the world, with all covetous desires of the same?

\qa{Answer.}  I renounce them.

\smallskip
\centerline{\rubric{Question.}}

Does thou renounce the sinful desires of the flesh, so that thou wilt not follow, nor be led by them?}

\qa{Answer.}  I renounce them all.

\medskip

% [sarum: Unction on breast and back)]


\centerline{\rubric{Question.}}
\drop{Dost thou believe in God the Father Almighty, Maker of heaven and earth?}

\qa{Answer.}  This I believe.

\smallskip
\centerline{\rubric{Question.}}
%And
Dost thou believe 
in Jesus Christ his only-begotten Son our Lord? And that he was conceived by the Holy Ghost; born of the Virgin Mary; that he suffered under Pontius Pilate, was crucified, dead, and buried; that he went down into hell, and also did rise again the third day; that he ascended into heaven, and sitteth at the right hand of God the Father Almighty; and from thence shall come again at the end of the world, to judge the quick and the dead?

\qa{Answer.}  This I believe.

\smallskip
\centerline{\rubric{Question.}}

And dost thou believe in the Holy Ghost; the holy Catholick Church; the Communion of Saints; the Remission of sins; the Resurrection of the flesh; and everlasting life after death?

\qa{Answer.} All this I stedfastly believe.

\medskip
\centerline{\rubric{Question.}}

\drop{Wilt thou be baptized in this faith?\\
\qa{Answer.} That is my desire.}

\smallskip
\centerline{\rubric{Question.}}

\drop{Wilt thou then obediently keep God’s holy will and commandments, and walk in the same all the days of thy life?}

\qa{Answer.} I will endeavour so to do, God being my helper. % 1928En

% \qa{Answer.} I will. 1662



% \begin{leftbar}

% \centerline{\rubric{Question.}}

% \drop{Dost thou renounce the devil and all his works, the vain pomp and glory of the world, with all covetous desires of the same, and the sinful desires of the flesh, so that thou wilt not follow, nor be led by them?}

% \qa{Answer.} I renounce them all.

% \centerline{\rubric{Question.}}

% Dost thou profess the Christian faith?

% \qa{Answer.} I do.

% \rubric{Then shall be said by the Candidates with the Priest and the Witnesses as followeth}

% \drop{I believe in God the Father Almighty, Maker of heaven and earth:}

% And in Jesus Christ his only Son our Lord, Who was conceived by the Holy Ghost, Born of the Virgin Mary, Suffered under Pontius Pilate, Was crucified, dead, and buried, He descended into hell; The third day he rose again from the dead, He ascended into heaven, And sitteth on the right hand of God the Father Almighty; From thence he shall come to judge the quick and the dead.

% I believe in the Holy Ghost; The holy Catholick Church; The Communion of Saints ; The Forgiveness of sins ; The Resurrection of the body; And the Life everlasting. Amen.

% \centerline{\rubric{Question.}}

% \drop{Wilt thou be baptized in this faith?\\
% \qa{Answer.} That is my desire.}

% \centerline{\rubric{Question.}}

% \drop{Wilt thou then obediently keep God’s holy will and commandments, and walk in the same all the days of thy life?}

% \qa{Answer.} I will endeavour so to do, God being my helper.
% \end{leftbar}








\section{The Blessing of the Water}


\centerline{\pilcrow{After which the Priest shall proceed, saying,}}
% The water in the fonte shalbe chaunged every moneth once at the lest, and afore any child be Baptized in the water so chaunged, the priest shall say at the font these prayers folowing.

\V The Lord be with you.  \R And with thy spirit.


\centerline{Let us pray.}
\drop{O most merciful God our Saviour Jesu Christ, who hast ordained the element of water for the regeneration of thy faithful people, Upon whom, being baptized in the river of Jordan, the Holy Ghost came down in likeness of a dove: Send down, we beseech thee, the same thy Holy Spirit to assist us, and to be present at this our invocation of thy Holy Name: Sanctify \grealtcross\ this fountain of baptism, thou that art the Sanctifier of all things, that by the power of thy Word all those that shall be baptized therein may be spiritually regenerated, and made the children of everlasting adoption. \R Amen.}

\smallskip
\centerline{\pilcrow{Then shall the Priest say,}}
\drop{O merciful God, grant that the old Adam in \emph{these Children} [\rubric{or} \emph{these Persons}] may be so buried, that the new man may be raised up in \emph{them}.  \R Amen.}

Grant that all %carnal
sinful %20th c. books
affections may die in \emph{them}, and that all things belonging to the Spirit may live and grow in \emph{them}. \R Amen.

Grant that \emph{they} may have power and strength to have victory, and to triumph, against the devil, the world, and the flesh. \R Amen.

%<1549>
Whosoever shall confess thee, O Lord, recognise him also in thy kingdom. \R Amen.

Grant that all sin and vice here may be so extinct, that they never have power to reign in thy servants. \R Amen.

Grant that whosoever here shall begin to be of thy flock, may evermore continue in the same. \R Amen.

Grant that all they which for thy sake in this life do deny and forsake themselves, may win and purchase thee, O Lord, which art everlasting treasure. \R Amen.
%</1549>

Grant that whosoever is here dedicated to thee by our office and ministry may also be endued with heavenly virtues, and everlastingly rewarded, through thy mercy, O blessed Lord God, who dost live, and govern all things, world without end. \R Amen.

\medskip
%1549
\V The Lord be with you.  \R And with thy spirit.

% 1928En, and other modern:
\V Lift up your hearts.  \R We lift them up unto the Lord.

\V Let us give thanks unto our Lord God. \R It is meet and right so to do.
\drop{It is very meet, right, and our bounden duty, that we should % [at all times, and in all places,] (all sources omit)
give thanks unto thee, O Lord, Holy Father, Almighty, Everlasting God, for that thy
% \drop{Almighty, everliving God, whose most * 
dearly beloved Son Jesus Christ, for the forgiveness of our sins, did shed out of his most precious side both water and blood; and gave commandment to his disciples, that they should go teach all nations, and baptize them In the Name of the Father, and of the Son, and of the Holy Ghost: Regard, we beseech thee, the supplications of thy congregation;}
%1662
Sanctify \grealtcross\ this Water to the mystical washing away of sin; and grant that \emph{these Children} [\rubric{or} \emph{these} thy \emph{Servants}], now to be baptized therein,
% Or. 1549
% and grant that all thy servants which shall be baptized in this water, prepared for the ministration of thy holy sacrament, 
may receive the fulness of thy grace, and ever remain in the number of thy faithful and elect children;
%  through Jesus Christ our Lord. \R Amen.}
% 1928Am:
through the same Jesus Christ our Lord, to whom, with thee, in the unity of the Holy Spirit, be all honour and glory, now and evermore. \R Amen.


\section{The Baptism}

\pilcrow{Then the Minister shall take the Child into his hands, and shall say to the Godfathers and Godmothers,}
\centerline{Name this Child.}

{\footnotesize\rubric{And then naming it after them (if they shall certify him that the Child may well endure it) % he shall dip it in the Water discreetly and warily, saying,
he shall dip it in the Water thrice. First dipping the right side, second the left side, the third time dipping the face towards the font, so it be discreetly and warily done, saying,}\par}

% And then naming it after them, he shall dip it in the water, or pour water upon it, saying,
% he shall dip it in the Water discreetly and warily, saying,

\lettrine{\emph{N.}}{ I} baptize thee in the Name of the Father, and of the Son, and of the Holy Ghost. Amen.

{\footnotesize\rubric{But if they certify that the Child is weak, it shall suffice to pour Water upon it, saying the foresaid words, N. \emph{I baptize thee \etc}}\par}

\smallskip
\begin{leftbar}
\pilcrow{But NOTE, That if the Person to be baptized be of Riper Years, the Minister shall take \emph{him} by the right hand, and placing \emph{him} conveniently by the Font, according to his discretion, shall ask the Godfathers and Godmothers the Name; and then shall dip \emph{him} in the Water, or pour Water upon \emph{him}, using the same form of words.}
\end{leftbar}

\medskip

\pilcrow{Then the Priest shall annoint the \emph{Infant} upon the head with chrism, saying,}
    
\drop{Almighty God, the Father of our Lord Jesus Christ, who hath regenerated thee by water and the Holy Ghost, and hath given unto thee remission of all thy sins, may he vouchsafe to anoint thee with the unction \grealtcross\ of his Holy Spirit, and bring thee to the inheritance of everlasting life. \R Amen.}


% \centerline{\rubric{Then the Priest shall say,}}
% \drop{We receive this \emph{Child} into the congregation of Christ’s flock, \grealtcross\ and do sign \emph{him} with the sign of the Cross, in token that hereafter \emph{he} shall not be ashamed to confess the faith of Christ crucified, and manfully to fight under his banner, against sin, the world, and the devil; and to continue Christ’s faithful soldier and servant unto \emph{his} life’s end. \R Amen.}
% * here the Priest shall make a Cross upon the Child's forehead.

\medskip

%1549
\pilcrow{Then the Godfathers and Godmothers shall take and lay their hands upon the \emph{Child}, and the Priest shall put upon \emph{him his} white vesture, commonly called the chrysom; and say}

\drop{Take this white vesture as a token of the innocency which, by God’s grace, in this holy sacrament of Baptism, is given unto thee; and for a sign whereby thou art admonished, so long as thou livest, to give thyself to innocency of living: that after this transitory life thou mayest be partaker of the life everlasting.}
     
\medskip

%india supplement.
\pilcrow{Then the Minister may deliver to each candidate a burning light, saying,}
% \drop{Receive this burning light, and walk in the light ft by faith in Jesus Christ. Amen.}
% or, South Africa % and Alcuin
\drop{Receive the light of Christ, that when the Bridegroom cometh thou mayest go forth with all the saints to meet him; and see that thou keep the grace of thy baptism.}

\medskip

\pilcrow{When there are many to be baptized, this order of baptizing, annointing, putting on the chrysom, and delivering the light, shall be used severally with every \emph{Child}. Those that be first baptized departing from the font, and remainyng in some convenient place within the Church until all be baptized.}


\section{The Thanksgiving}
\centerline{\pilcrow{Then shall the Priest say,}}
\drop{Seeing now, dearly beloved brethren, that \emph{these Children} [\rubric{or} \emph{these Persons}] \emph{are} regenerate, and grafted into the body of Christ’s Church, let us give thanks unto Almighty God for these benefits; and with one accord make our prayers unto him, that \emph{these Children} [\rubric{or} \emph{these Persons}] may lead the rest of \emph{their} life according to this beginning.}


% Then shall be said, all kneeling;
% \drop{Our Father}
\smallskip

\centerline{\rubric{Then shall the Priest say,}}
\drop{We yield thee hearty thanks, most merciful Father, that it hath pleased thee to regenerate \emph{these Infants} [\rubric{or} \emph{these Persons}] with thy Holy Spirit, to receive \emph{them} for thine own \emph{Children} by adoption, and to incorporate \emph{them} into thy holy Church. And humbly we beseech thee to grant, that \emph{they}, being dead unto sin, and living unto righteousness, and being buried with Christ in his death, may crucify the old man, and utterly abolish the whole body of sin; and that, as \emph{they are} made partaker of the death of thy Son, \emph{they} may also be partaker of his resurrection; so that finally, with the residue of thy holy Church, \emph{they} may be \emph{inheritors} of thine everlasting kingdom; through Christ our Lord. \R Amen.}


\bigskip

\section{The Duties of the Godfathers and Godmothers}


\pilcrow{Then the Priest shall say to the Godfathers and Godmothers and Parents this Exhortation following.}
\drop{Forasmuch as \emph{these Children have} promised by you \emph{their} sureties to renounce the devil and all his works, to believe in God, and to serve him: ye must remember, that it is your parts and duties to see that \emph{these Infants} be taught, so soon as \emph{they} shall be able to learn, what a solemn vow, promise, and profession, \emph{they have} here made by you. And that \emph{they} may know these things the better, ye shall call upon \emph{them} to hear Sermons; and chiefly ye shall provide, that \emph{them} may learn the Creed, the Lord’s Prayer, and the Ten Commandments, in the vulgar tongue, and all other things which a Christian ought to know and believe to his soul’s health; and that \emph{these Children} may be virtuously brought up to lead a godly and a Christian life.}


Will you help \emph{them} to learn and to do all these things?

\qa{Answer.}  I will, the Lord being my helper.

Remember always that Baptism doth represent unto us our profession; which is, to follow the example of our Saviour Christ, and to be made like unto him; that, as he died, and rose again for us, so should we, who are baptized, die from sin, and rise again unto righteousness; continually mortifying all our evil and corrupt affections and daily proceeding in all virtue and godliness of living.

\medskip
% the following is a repetition of the previous exhortation, and the following rubric; and doesn't survive into the modern books.
% \centerline{\rubric{Then shall he add and say,}}
% \drop{Ye are to take care that this Child be brought to the Bishop to be confirmed by him, so soon as he can say the Creed, the Lord's Prayer, and the Ten Commandments, in the vulgar tongue, and be further instructed in the Church-Catechism set forth for that purpose.}
% It is certain by God's Word, that children which are baptized, dying before they commit actual sin, are undoubtedly saved.


\pilcrow{The Minister shall command that the chrysoms be brought to the church, and delivered to the Priests after the accustomed manner, at the purification of the mother of every child: and that the children be brought to the Bishop to be confirmed of him, so soon as they can say, in their vulgar tongue, the Articles of the Faith, the Lord’s Prayer, and the Ten Commandments, and further be instructed in the Catechism set forth for that purpose accordingly as it is there expressed.}

% And so let the congregation depart in the name of the Lord.
\begin{leftbar}
\section{The Duties of the Witnesses and of the New Baptized}
\pilcrow{Then, all standing up, the Priest shall use this exhortation following; speaking to the Godfathers and Godmothers first.}
\drop{Forasmuch as \emph{these persons have} promised in your presence to renounce the devil and all his works, to believe in God, and to serve him: ye must remember, that it is your part and duty to put \emph{them} in mind, what a solemn vow, promise, and profession \emph{they have} now made before this congregation, and especially before you \emph{their} chosen witnesses. And ye are also to call upon \emph{them} to use all diligence to be rightly instructed in God’s holy Word; that so \emph{they} may grow in grace, and in the knowledge of our Lord Jesus Christ, and live godly, righteously, and soberly in this present world.}


{\footnotesize\rubric{(And then, speaking to the new baptized \emph{persons}, he shall proceed, and say,)}\par}


\drop{And as for \emph{you}, who \emph{have} now by Baptism put on Christ, it is \emph{your} part and duty also, being made \emph{the children} of God and of the light by faith in Jesus Christ, to walk answerably to \emph{your} Christian calling, and as becometh the children of light; remembering always that Baptism representeth unto us our profession; which is, to follow the example of our Saviour Christ, and to be made like unto him; that as he died, and rose again for us; so should we, who are baptized, die from sin, and rise again unto righteousness; continually mortifying all our evil and corrupt affections, and daily proceeding in all virtue and godliness of living.}

\medskip
\pilcrow{It is expedient that every person, thus baptized, should be confirmed by the Bishop so soon after his Baptism as conveniently may be.}
\end{leftbar}

\fleuron


\section{Private Baptism}
\pilcrow{When, in consideration of extreme sickness, necessity may require, then the following form shall suffice:}

\pilcrow{The Child (or Person) being named by some one who is present, the Minister shall pour Water upon him, saying these words:}

\lettrine{\emph{N.}}{ I} baptize thee In the Name of the Father, and of the Son, and of the Holy Ghost. Amen.

\pilcrow{After which shall be said the Lord’s Prayer, and the Thanksgiving from the Office, beginning, \emph{We yield thee hearty thanks, \etc}}

\begin{leftbar}
    \pilcrow{But NOTE, That in the case of an Adult, the Minister shall first ask the questions provided in this Office for the Baptism of Adults.}
\end{leftbar}
\pilcrow{In cases of extreme sickness, or any imminent peril, if a Minister cannot be procured, then any person present may administer holy Baptism, using the foregoing form. Such Baptism shall be promptly reported to the authorities.}

\pilcrow{And let them not doubt, but that the \emph{Child} so baptized is lawfully and sufficiently baptized, and ought not to be baptized again.}

\section{The Receiving of one Privately Baptized}
\pilcrow{It is expedient that a Child or Person so baptized be afterward brought to the Church, at which time these parts of the foregoing service shall be used:}

{\footnotesize\rubric{The Gospel, the Questions (omitting the question \emph{Wilt thou be baptized in this Faith?} and the answer thereto), the Annointing, and the remainder of the Office.}\par}



%Am1928
\section{Conditional Baptism}
\pilcrow{If there be reasonable doubt whether any Person were baptized with Water, \emph{In the Name of the Father, and of the Son, and of the Holy Ghost} (which are essential parts of Baptism), such Person may be baptized in the manner herein appointed; saving that, at the immersion or the pouring of water, the Minister shall use this Form of words.}

\drop{If thou art not already baptized, \emph{N.}, I baptize thee In the Name of the Father, and of the Son, and of the Holy Ghost. Amen.}



\section{General Rubrics}

\pilcrow{For every child to be baptized there shall be not fewer than three godparents, of whom at least two shall be of the same sex as the child and of whom at least one shall be of the opposite sex; save that, when three cannot be conveniently had, one godparent shall suffice. Parents may be godparents for their own children provided that the child shall have at least one other godparent. The godparents shall be persons who have been baptized and confirmed and will faithfully fulfil their responsibilities both by their care for the child committed to their charge and by the example of their own godly living. Nevertheless the Minister shall have power to dispense with the requirement of confirmation in any case in which in his judgement need so requires.}

\pilcrow{The Minister shall instruct the parents or guardians of an infant to be admitted to Holy Baptism that the same responsibilities rest on them as are in the service of Holy Baptism required of the godparents.}

\pilcrow{No Minister shall refuse or, save for the purpose of preparing or instructing the parents or guardians or godparents, delay to baptize any infant within his cure that is brought to the church to be baptized, provided that due notice has been given and the provisions relating to godparents are observed. If the Minister shall refuse or unduly delay to baptize any such infant, the parents or guardians may apply to the Bishop of the diocese who shall, after consultation with the Minister, give such directions as he thinks fit.}

\pilcrow{The Minister, before proceeding to the Baptism, shall have satisfied himself that the child presented to him has not already been baptized.}
\begin{leftbar}    
\pilcrow{When any such persons, as are of riper years, are to be baptized, timely notice shall be given to the Bishop, or whom he shall appoint for that purpose, a week before at the least, by the Minister of the Parish, the parents, or some other discreet persons; that so due care may be taken for their examination, whether they be sufficiently instructed in the principles of the Christian Religion; and that they may be exhorted to prepare themselves with prayers and fasting for the receiving of this holy Sacrament.}


\pilcrow{And if they shall be found fit, they shall each choose three, or at the least one, to be their Witnesses, who shall be ready to present them at the Font, immediately after the Second Lesson, either at Morning or Evening Prayer, or (if need so require) at such other time as the Minister in his discretion shall think fit.}

\smallskip

\pilcrow{It is convenient that the Admission to the Catechuminate should precede the Baptism by a week or more.}
\end{leftbar}



\fleuron

\include{catechumens.tex}
% Form of admitting Catechumens (SA)
% The Ministration of Baptism for those of Riper Years


\chapter[A Catechism]{\stylechapter{}{A Catechism}{{\footnotesize That is to say}\\ an Instruction to be Learned of Every Person before he be Brought to be Confirmed by the Bishop\\ }}





\centerline{\qa{Question.}}
\drop{What is your Name?\\
\qa{Answer.} \emph{N.} or \emph{M.}}

\qa{Question.} Who gave you this Name?

\qa{Answer.} My Godfathers and Godmothers in my Baptism; wherein I was made a member of Christ, the child of God, and an inheritor of the kingdom of heaven.

\qa{Question.} What did your Godfathers and Godmothers then for you?

\qa{Answer.} They did promise and vow three things in my name. First, that I should renounce the devil and all his works, the pomps and vanity of this wicked world, and all the sinful lusts of the flesh. Secondly, that I should believe all the Articles of the Christian Faith. And thirdly, that I should keep God’s holy will and commandments, and walk in the same all the days of my life.

\qa{Question.} Dost thou not think that thou art bound to believe, and to do, as they have promised for thee?

\qa{Answer.} Yes verily: and by God’s help so I will. And I heartily thank our heavenly Father, that he hath called me to this state of salvation, through Jesus Christ our Saviour. And I pray unto God to give me his grace, that I may continue in the same unto my life’s end.


\medskip
\centerline{\qa{Catechist.}}
\centerline{Rehearse the Articles of thy Belief.}


\centerline{\qa{Answer.}}
\drop{I believe in God the Father Almighty, Maker of heaven and earth:}

And in Jesus Christ his only Son our Lord, Who was conceived by the Holy Ghost, Born of the Virgin Mary, Suffered under Pontius Pilate, Was crucified, dead, and buried, He descended into hell; The third day he rose again from the dead, He ascended into heaven, And sitteth at the right hand of God the Father Almighty; From thence he shall come to judge the quick and the dead.

I believe in the Holy Ghost; The holy Catholick Church; The Communion of Saints; The Forgiveness of sins; The Resurrection of the body; And the Life everlasting. Amen.

\centerline{\qa{Question.}}
What dost thou chiefly learn in these Articles of thy Belief?

\qa{Answer.} First, I learn to believe in God the Father, who hath made me, and all the world.

Secondly, in God the Son, who hath redeemed me, and all mankind.

Thirdly, in God the Holy Ghost, who sanctifieth me, and all the elect people of God.

\medskip
\centerline{\qa{Question.}}
You said, that your Godfathers and Godmothers did promise for you, that you should keep God’s commandments. Tell me how many there be?

\qa{Answer.} Ten.

\qa{Question.} Which be they?

\centerline{\qa{Answer.}}
\drop{The same which God spake in the twentieth Chapter of Exodus, saying, I am the Lord thy God, who brought thee out of the land of Egypt, out of the house of bondage.}

I. Thou shalt have none other gods but me.

II. Thou shalt not make to thyself any graven image, nor the likeness of any thing that is in heaven above, or in the earth beneath, or in the water under the earth. Thou shalt not bow down to them, nor worship them%: for I the Lord thy God am a jealous God, and visit the sins of the fathers upon the children unto the third and fourth generation of them that hate me, and shew mercy unto thousands in them that love me, and keep my commandments.
.

III. Thou shalt not take the Name of the Lord thy God in vain%: for the Lord will not hold him guiltless that taketh his Name in vain.
.

IV. Remember that thou keep holy the Sabbath-day. Six days shalt thou labour, and do all that thou hast to do; but the seventh day is the Sabbath of the Lord thy God. %In it thou shalt do no manner of work, thou, and thy son, and thy daughter, thy man-servant, and thy maid-servant, thy cattle, and the stranger that is within thy gates. For in six days the Lord made heaven and earth, the sea, and all that in them is, and rested the seventh day; wherefore the Lord blessed the seventh day, and hallowed it.

V. Honour thy father and thy mother%, that thy days may be long in the land which the Lord thy God giveth thee.
.

VI. Thou shalt do no murder.

VII. Thou shalt not commit adultery.

VIII. Thou shalt not steal.

IX. Thou shalt not bear false witness% against thy neighbour.
.

X. Thou shalt not covet% thy neighbour's house, thou shalt not covet thy neighbour's wife, nor his servant, nor his maid, nor his ox, nor his ass, nor any thing that is his.
.

\medskip
\centerline{\qa{Question.}}
What dost thou chiefly learn by these Commandments?

\qa{Answer.} I learn two things: my duty towards God, and my duty towards my neighbour.

\qa{Question.} What is thy duty towards God?

\qa{Answer.} My duty towards God, is to believe in him, to fear him, and to love him with all my heart, with all my mind, with all my soul, and with all my strength; to worship him, to give him thanks, to put my whole trust in him, to call upon him, to honour his holy name and his Word, and to serve him truly all the days of my life.

\qa{Question.} What is thy duty towards thy Neighbour?

\qa{Answer.} My duty towards my neighbour, is to love him as myself, and to do to all men, as I would they should do unto me: To love, honour, and succour my father and mother: To honour and obey the \emph{civil authority}%,  and all that are put in authority under her
: To submit myself to all my governors, teachers, spiritual pastors and masters: To order myself lowly and reverently to all my betters: To hurt no body by word nor deed: To be true and just in all my dealing: To bear no malice nor hatred in my heart: To keep my hands from picking and stealing, and my tongue from evil-speaking, lying, and slandering: To keep my body in temperance, soberness, and chastity: Not to covet nor desire other men’s goods; but to learn and labour truly to get mine own living, and to do my duty in that state of life, unto which it shall please God to call me.

\medskip
\centerline{\qa{Catechist.}}
My good child, know this, that thou art not able to do these things of thyself, nor to walk in the Commandments of God, and to serve him, without his special grace; which thou must learn at all times to call for by diligent prayer. Let me hear therefore, if thou canst say the Lord’s Prayer.

\centerline{\qa{Answer.}}
\drop{Our Father, which art in heaven, Hallowed be thy Name. Thy kingdom come. Thy will be done, in earth as it is in heaven. Give us this day our daily bread. And forgive us our trespasses, As we forgive them that trespass against us. And lead us not into temptation; But deliver us from evil. Amen.}

\qa{Question.} What desirest thou of God in this Prayer?

\qa{Answer.} I desire my Lord God our heavenly Father, who is the giver of all goodness, to send his grace unto me, and to all people; that we may worship him, serve him, and obey him, as we ought to do. And I pray unto God, that he will send us all things that be needful both for our souls and bodies; and that he will be merciful unto us, and forgive us our sins; and that it will please him to save and defend us in all dangers ghostly and bodily; and that he will keep us from all sin and wickedness, and from our ghostly enemy, and from everlasting death. And this I trust he will do of his mercy and goodness, through our Lord Jesus Christ. And therefore I say, Amen, So be it.

% \medskip
% \begin{leftbar}
% \centerline{\qa{Question.}}
% \drop{How many Sacraments hath Christ ordained in his Church?}

% \qa{Answer.} Two only, as generally necessary to salvation, that is to say, Baptism, and the Supper of the Lord.

% \qa{Question.} What meanest thou by this word Sacrament?

% \qa{Answer.} I mean an outward and visible sign of an inward and spiritual grace given unto us, ordained by Christ himself, as a means whereby we receive the same, and a pledge to assure us thereof.

% \qa{Question.} How many parts are there in a Sacrament?

% \qa{Answer.} Two; the outward visible sign, and the inward spiritual grace.

% \qa{Question.} What is the outward visible sign or form in Baptism?

% \qa{Answer.} Water; wherein the person is baptized \emph{In the Name of the Father, and of the Son, and of the Holy Ghost.}

% \qa{Question.} What is the inward and spiritual grace?

% \qa{Answer.} A death unto sin, and a new birth unto righteousness: for being by nature born in sin, and the children of wrath, we are hereby made the children of grace.

% \qa{Question.} What is required of persons to be baptized?

% \qa{Answer.} Repentance, whereby they forsake sin; and Faith, whereby they stedfastly believe the promises of God made to them in that Sacrament.

% \qa{Question.} Why then are Infants baptized, when by reason of their tender age they cannot perform them?

% \qa{Answer.} Because they promise them both by their sureties; which promise, when they come to age, themselves are bound to perform.

% \qa{Question.} Why was the Sacrament of the Lord's Supper ordained?

% \qa{Answer.} For the continual remembrance of the sacrifice of the death of Christ and of the benefits which we receive thereby.


% \qa{Question.} What is the outward part or sign of the Lord's Supper?


% \qa{Answer.} Bread and Wine, which the Lord hath commanded to be received.


% \qa{Question.} What is the inward part, or thing signified?

% \qa{Answer.} The Body and Blood of Christ, which are verily and indeed taken and received by the faithful in the Lord's Supper.

% \qa{Question.} What are the benefits whereof we are partakers thereby?

% \qa{Answer.} The strengthening and refreshing of our souls by the Body and Blood of Christ, as our bodies are by the Bread and Wine.

% \qa{Question.} What is required of them who come to the Lord's Supper?

% \qa{Answer.} To examine themselves, whether they repent them truly of their former sins, stedfastly purposing to lead a new life; have a lively faith in God's mercy through Christ, with a thankful remembrance of his death; and be in charity with all men.
% \end{leftbar}

\medskip

\pilcrow{The Curate of every Parish shall diligently upon Sundays and Holy-days, %after the second Lesson at Evening Prayer,
half an hour before Evensong, % 1549
openly in the Church instruct and examine so many Children of his Parish sent unto him, as he shall think convenient, in some Part of this Catechism.}

\pilcrow{And all Fathers, Mothers, Masters, and Dames, shall cause their Children, Servants, and Prentices (which have not learned their Catechism,) to come to the Church at the time appointed, and obediently to hear, and be ordered by the Curate, until such time as they have learned all that is here appointed for them to learn.}

\pilcrow{So soon as children are come to a competent age, and can say, in their mother tongue, the Creed, the Lord’s Prayer, and the Ten Commandments; and also can answer to the other questions of this short Catechism; they shall be brought to the Bishop.}

\pilcrow{It is convenient that every one shall have a Godfather, or a Godmother, as a witness of their Confirmation.}

\pilcrow{And whensoever the Bishop shall give knowledge for Children to be brought unto him for their Confirmation, the Curate of every Parish shall either bring, or send in writing, with his hand subscribed thereunto, the names of all such persons within his Parish, as he shall think fit to be presented to the Bishop to be confirmed. And, if the Bishop approve of them, he shall confirm them in manner following.}

\fleuron

\chapter[Confirmation]{\stylechapter{}{The Order of Confirmation}{Or Laying on of Hands upon Those That Are Baptized and Come to Years of Discretion\\ }}



\pilcrow{Upon the day appointed, all that are to be then confirmed, being placed, and standing in order, before the Bishop; he (or some other Minister appointed by him) shall read this Preface following.}

% 1549 rubric, then become a preface.
% \drop{To the end that Confirmation may be ministered to the more edifying of such as shall receive it, the Church hath thought good to order, That none hereafter shall be Confirmed, but such as can say the Creed, the Lord's Prayer, and the Ten Commandments; and can also answer to such other Questions, as in the short Catechism are contained; which order is very convenient to be observed; to the end, that children, being now come to the years of discretion, and having learned what their Godfathers and Godmothers promised for them in Baptism, they may themselves, with their own mouth and consent, openly before the Church, ratify and confirm the same; and also promise, that by the grace of God they will evermore endeavour themselves faithfully to observe such things, as they, by their own confession, have assented unto.}
%En28, Sc29

\drop{Dearly beloved in the Lord, in ministering Confirmation the Church doth follow the example of the Apostles of Christ. For in the eighth chapter of the Acts of the Apostles we thus read:—}

They therefore that were scattered abroad went about preaching the word. And Philip went down to the city of Samaria, and proclaimed unto them the Christ. When they believed Philip preaching good tidings concerning the kingdom of God and the name of Jesus Christ, they were baptized, both men and women. Now when the Apostles which were at Jerusalem heard that Samaria had received the word of God, they sent unto them Peter and John; who, when they were come down, prayed for them, that they might receive the Holy Ghost: for as yet he was fallen upon none of them; only they had been baptized into the Name of the Lord Jesus. Then laid they their hands on them, and they received the Holy Ghost.

The Scripture here teacheth us that a special gift of the Holy Spirit is bestowed through laying on of hands with prayer. And forasmuch as this gift cometh from God alone, let us that are here present pray to Almighty God, that he will strengthen with his Holy Spirit in Confirmation those who in Baptism were made his children.

You, then, who are to be confirmed must now declare before this congregation that you are stedfastly purposed, with the help of this gift, to lead your life in the faith of Christ and in obedience to God’s will and commandments; and must openly acknowledge yourselves bound to fulfil the Christian duties to which your Baptism hath pledged you.

\section{The Renewal of Baptismal Vows}

\centerline{\pilcrow{Then shall the Bishop say,}}

\lettrine{D}{\emph{o}} \emph{ye} here, in the presence of God, and of this congregation, renew the solemn promise and vow that was made in \emph{your} name at \emph{your} Baptism; ratifying and confirming the same in \emph{your} own persons, and acknowledging \emph{yourselves} bound to believe, and to do, all those things, which \emph{your} Godfathers and Godmothers then undertook for \emph{you}?

\centerline{\rubric{And every one shall audibly answer,}}
\centerline{I do.}

\medskip

\centerline{\rubric{Or else the Bishop shall say,}}
%En1928
% \drop{Do ye here, in the presence of God, and of this congregation, renounce the devil and all his works, the pomps and vanity of this wicked world, and all the sinful lusts of the flesh, so that ye will not follow nor be led by them?}
% Scottish / 28Proposed
\lettrine{D}{\emph{o}} \emph{ye} here, in the presence of God, and of this congregation, renounce the devil and all his works, the vain pomp and glory of the world, with all covetous desires of the same, and the sinful desires of the flesh, so that \emph{ye will} not follow, nor be led by them?

\R I do.

\lettrine{D}{\emph{o}} \emph{ye} believe all the Articles of the Christian Faith as contained in the Apostles’ Creed?

\R I do.

\lettrine{D}{\emph{o}} \emph{ye} promise that \emph{ye will} endeavour to keep God’s holy will and commandments, and to walk in the same all the days of \emph{your} life?

% \drop{Will ye endeavour to keep God’s holy will and commandments, and to walk in the same all the days of your life?}

\R I do.

\medskip

\section{The Confirmation}

\centerline{\pilcrow{The Bishop.}}

\drop{Our \grecross\ help is in the Name of the Lord;  \R Who hath made heaven and earth.}

\V Blessed be the Name of the Lord;  \R Henceforth, world without end.

\V Lord, hear our prayers.  \R And let our cry come unto thee. %1662

\V The Lord be with you.  \R And with thy spirit. %1549

\centerline{Let us pray.}
\drop{Almighty and everliving God, who hast vouchsafed to regenerate these thy servants by Water and the Holy Ghost, and hast given unto them forgiveness of all their sins: Strengthen them, we beseech thee, O Lord, with the Holy Ghost the Comforter, and daily increase in them thy manifold gifts of grace;

The Spirit of wisdom and understanding;

The Spirit of counsel and ghostly strength;

The Spirit of knowledge and true godliness;

and fill them, O Lord, with the Spirit of thy holy fear, now and for ever. \R Amen.}

\smallskip

% 1549 & 1929
\drop{Sign them, O Lord, and mark them to be thine for ever by the virtue of the Holy Cross; mercifully confirm them with the inward unction of the Holy Ghost, that they may attain unto everlasting life.  \R Amen.}

\medskip

\pilcrow{Then all of them in order kneeling before the Bishop, he shall lay his hand upon the head of every one severally, saying}

%1912, 1549
\lettrine{\emph{N.}}{ I} sign thee with the sign of the \grealtcross\ Cross\footnote{\rubric{Here the Bishop shall sign the person with the sign of the Cross on the forehead with the holy Chrism.}} and I lay my hands [\rubric{or} hand] upon thee, in the Name of the Father, and of the Son, and of the Holy Ghost.

% Sarum
% and I confirm thee with the chrism of salvation.  In the name...
\smallskip

% The following from 1552, etc.
\drop{Defend, O Lord, this thy Child [\rubric{or} this thy Servant \rubric{or} Handmaiden] with thy heavenly grace, that \emph{he} may continue thine for ever; and daily increase in thy Holy Spirit more and more, until \emph{he} come unto thy everlasting kingdom. \R Amen.}

\medskip
\centerline{\pilcrow{Then shall the Bishop say,}}

\V Peace be with you.  \R And with thy spirit. %1549; 1552 on it was "The Lord be with you.

\centerline{\rubric{And (all kneeling down) the Bishop shall add,}}

\centerline{Let us pray.}

\begin{leftbar} % added in 1662
    \ourFather

\centerline{\rubric{And this Collect.}}
\end{leftbar}


\drop{Almighty and everliving God, who makest us both to will and to do those things that be good and acceptable unto thy divine Majesty; We make our humble supplications unto thee for these thy servants, upon whom (after the example of thy holy Apostles) we have now laid our hands, to certify them (by this sign) of thy favour and gracious goodness towards them. Let thy fatherly hand, we beseech thee, ever be over them; let thy Holy Spirit ever be with them; and so lead them in the knowledge and obedience of thy Word, that in the end they may obtain everlasting life; through our Lord Jesus Christ, who with thee and the Holy Ghost liveth and reigneth, ever one God, world without end. \R Amen.}

\section{The Conclusion}
\drop{O almighty Lord, and everlasting God, vouchsafe, we beseech thee, to direct, sanctify, and govern, both our hearts and bodies, in the ways of thy laws, and in the works of thy commandments; that, through thy most mighty protection both here and ever, we may be preserved in body and soul; through our Lord and Saviour Jesus Christ. \R Amen.}

\medskip
\centerline{\pilcrow{Then the Bishop shall bless them, saying thus,}}
% En1928 only
\begin{leftbar}
    
\drop{Go forth into the world in peace; be of good courage; hold fast that which is good; render to no man evil for evil; strengthen the fainthearted; support the weak; help the afflicted; honour all men; love and serve the Lord, rejoicing in the power of the Holy Spirit.}
\end{leftbar}

And the Blessing of God Almighty, the \grealtcross\ Father, the \grealtcross\ Son, and the Holy \grealtcross\ Ghost, be upon you, and remain with you for ever. \R Amen.

\medskip

\pilcrow{And there shall none be admitted to the holy Communion, until such time as he be confirmed, or be ready and desirous to be confirmed.}

\fleuron
\chapter[Confession]{\stylechapter{The Order for \\ the Reconciliation of a Penitent\\ {\small commonly called}}{Confession}{}}

\pilcrow{Note. Such as shall be satisfied with a general Confession should not be offended with them that do use, to their further satisfying, confession to the Priest; and those also which think needful or convenient, for the quietness of their own consciences, particularly to open their sins to the Priest, should not be offended with them that are satisfied with their humble confession to God, and the general Confession to the Church. But in all things everyone should follow and keep the rule of charity, and be satisfied with his own conscience, not judging other men’s minds and consciences, whereas he hath no warrant of God’s word to the same.} %1929 Scottish; slightly altered from 1549
\bigskip 

\centerline{\pilcrow{The Penitent, kneeling, begins,}}
\centerline{Bless me, for I have sinned.}

\subsubsection{The Priest gives the blessing,}
\drop{The Lord be in thy heart and upon thy lips, that so thou mayest worthily and rightly confess all thy sins, \grealtcross\ in the Name of the Father, and of the Son, and of the Holy Ghost. Amen.}

\medskip
\pilcrow{The Penitent then makes \emph{his} confession, saying,}

\drop{I confess to God Almighty, the Father, the Son, and the Holy Ghost, that I have sinned in thought, word, and deed, through my own grievous fault.} % Wherefore I pray God to have mercy upon me.}

And especially I have sinned in these ways . . .

% \rubric{The penitent then states the specific sins he can remember, and should end with the following:}

For these and all other sins which I cannot now remember, I am truly sorry. I pray God to have mercy on me. I firmly intend amendment of life, and I humbly beg forgiveness of God and his Church, and ask thee for penance, counsel, and absolution.
\medskip
\pilcrow{After the confession, the Priest may find it helpful to question the penitent, so that advice about possible reparation, or restitution, or how to face the future more successfully may be given.}

\pilcrow{Then some form of penance is given. This is \emph{not} a \emph{penalty} but some useful act which aids the penitent to make outward embodiment of his contrite purpose.}

%    Here the Priest may offer counsel, direction, and comfort.
\medskip
\pilcrow{The Priest then pronounces this absolution:}

\drop{Our Lord Jesus Christ, who hath left power to his Church to absolve all sinners who truly repent and believe in him, of his great mercy forgive thee thine offences: And by his authority committed to me, I absolve thee from all thy sins, \grealtcross\  In the Name of the Father, and of the Son, and of the Holy Ghost. \R Amen.}

\smallskip
The Lord hath put away all thy sins. \R Thanks be to God.

\subsubsection{The Priest concludes,}

\centerline{Go in peace, and pray for me, a sinner.}

\fleuron

% Book of Common Prayer (1662) Absolution
% Book of Common Prayer (proposed, 1923, Visitation of the Sick) Confession
% The Armed Forces Prayer Book (1951) - Rubrics & confession (parts)
% Book of Common Prayer (1979) - traditionalized.
% Anglican Service Book (some tweaks)



% ¶ Then let him tell his sins, which being ended the Priest shall say—God Almighty have mercy, and The Almighty and merciful Lord, as in the Ordinary of the Mass.

% {Almighty God have mercy upon thee, forgive thee thy sins, and bring thee to everlasting life.  Amen.}
% {May the Almighty and Merciful Lord grant thee pardon, absolution, and remission of thy sins.  Amen.}

% The Passion of our Lord Jesus Christ, the merits of the Blessed Virgin Mary, and of all the Saints, whatsoever good thou hast done, or evil thou hast endured, be to the for the remission of sins, the increase of grace, and the reward of eternal life.  Amen.

% ¶ Here let him enjoin the Penance, saying—
% And for a special Penance thou shalt say or do this or that.
% ¶ Then let him absolve him, and say—









\pilcrow{First, the Banns of all that are to be married together, must be published in the Church three several Sundays, or Holy-days in the time of Divine Service, immediately before the the sentences for the Offertory: the Curate saying after the accustomed manner.} % Parson's Handbook

I publish the Banns of Marriage between \emph{N.} of this Parish and \emph{N.} of this Parish. If any of you know cause, or just impediment, why these two persons should not be joined together in holy Matrimony, ye are to declare it.  This is the \emph{first} time of asking.% Parson's Handbook

% \chapter{The Blessing of Civil Marriage}
% SA/Indian suppliment
% Churching moved here?

\chapter[The Visitation of the Sick]{\stylechapter{The Order for}{The Visitation of the Sick}{and the Communion of the Sick}}

\section{Visitation}
%These rubrics from 1923,1928,1929
\pilcrow{When any person is sick, notice shall be given thereof to the Minister,
%  of the Parish; 
who shall minister to the sick person after the form following, or in like manner.}
\smallskip
\pilcrow{When he cometh into the sick person’s house, he shall say,}
\drop{Peace be to this house, and to all that dwell in it.}

\bigskip
{\centering\pilcrow{When he cometh into the sick person’s presence he shall say, kneeling down,}}


% \centerline{\rubric{Then the Minister shall say,}}
% Let us pray.

\centerline{Lord, have mercy upon us.}
\centerline{\emph{Christ, have mercy upon us.}}
\centerline{Lord, have mercy upon us.}

\medskip
\ourFather


\V O Lord, save thy servant; \R Which putteth \emph{his} trust in thee.

\V Send \emph{him} help from thy holy place; \R  And evermore mightily defend \emph{him}.

\V Let the enemy have no advantage of \emph{him};  \R Nor the wicked approach to hurt \emph{him}.

\V Be unto \emph{him}, O Lord, a strong tower. \R  From the face of \emph{his} enemy.

\V O Lord, hear our prayers.  \R And let our cry come unto thee.

\centerline{Let us pray.}
\drop{O Lord, look down from heaven, behold, visit, and relieve this thy servant. Look upon \emph{him} with the eyes of thy mercy, give \emph{him} comfort and sure confidence in thee, defend \emph{him} from the danger of the enemy, and keep \emph{him} in perpetual peace and safety; through Jesus Christ our Lord. \R Amen.}

\smallskip

Hear us, Almighty and most merciful God and Saviour; extend thy accustomed goodness to this thy servant who is grieved with sickness.  \R Amen. % The following divisions are from 1928 proposed.
\smallskip

% \begin{leftbar}
% restored from 1549
Visit \emph{him}, O Lord, as thou didst visit Peter’s wife’s mother, and the captain’s servant. And as thou preservedst Toby and Sarah by thy Angel from danger, so restore unto this sick person \emph{his} former health, if it be thy will.  \R Amen.
% \end{leftbar}
\smallskip

Sanctify this trial unto \emph{him}; that the sense of \emph{his} weakness may add strength to \emph{his} faith, and seriousness to \emph{his} repentance. \R Amen.

\smallskip

May it be thy good pleasure to restore \emph{him} to \emph{his} former health, that so \emph{he} may lead the residue of \emph{his} life in thy fear, and to thy glory. \R Amen.

\smallskip

And whatsoever the issue that thou shalt ordain for \emph{him}, give \emph{him} grace so to be conformed to thy will, that \emph{he} may be made meet to dwell with thee in life everlasting; through Jesus Christ our Lord. \R Amen.

\section{Exhortation to Faith and Prayer}

{\centering\pilcrow{Then shall the Minister exhort the sick person upon such subjects as the following:}}

\drop{Our Heavenly Father, in his love for all men, uses sickness as a gracious means whereby to correct his children.}

Our Lord Jesus Christ, ever present with us, is ready to impart to us spiritual strength to use sickness well to the glory of God.

Our Lord, manifested in the Gospel as the healer of disease, is still ready to minister grace for the healing of the body.

Our Lord himself, though sinless, was made perfect through sufferings; and sinful man needs discipline in order to correct and amend in him whatever is amiss in the eyes of our heavenly Father.

The aim of the Christian, whether in health or in sickness, is that God may be glorified in him through Jesus Christ.

There is great honour in suffering if our pain be conformed to the spirit of Jesus Christ; for in the bearing of pain God manifested his will to redeem the world.

In sickness as in health we are to seek constantly the inspiration of God the Holy Ghost, the Spirit of Christ.


% \medskip
% \rubric{Or the Minister may exhort the sick person after this form,}
% \drop{Dearly beloved, know this, that Almighty God is the Lord of life and death, and of all things to them pertaining, as youth, strength, health, age, weakness, and sickness. Wherefore, whatsoever your sickness is, know you certainly, that it is God’s visitation. And for what cause soever this sickness is sent unto you; whether it be to try your patience for the example of others, and that your faith may be found in the day of the Lord laudable, glorious, and honourable, to the increase of glory and endless felicity; or else it be sent unto you to correct and amend in you whatsoever doth offend the eyes of your heavenly Father; know you certainly, that if you truly repent you of your sins, and bear your sickness patiently, trusting in God’s mercy, for his dear Son Jesus Christ’s sake, and render unto him humble thanks for his fatherly visitation, submitting yourself wholly unto his will, it shall turn to your profit, and help you forward in the right way that leadeth unto everlasting life.}

% Take therefore in good part the chastisement of the Lord: For (as Saint Paul saith in the twelfth Chapter to the Hebrews) whom the Lord loveth he chasteneth, and scourgeth every son whom he receiveth. If ye endure chastening, God dealeth with you as with sons; for what son is he whom the father chasteneth not? But if ye be without chastisement, whereof all are partakers, then are ye bastards, and not sons. Furthermore, we have had fathers of our flesh, which corrected us, and we gave them reverence: shall we not much rather be in subjection unto the Father of spirits, and live? For they verily for a few days chastened us after their own pleasure; but he for our profit, that we might be partakers of his holiness. These words, good \emph{brother}, are written in holy Scripture for our comfort and instruction; that we should patiently, and with thanksgiving, bear our heavenly Father’s correction, whensoever by any manner of adversity it shall please his gracious goodness to visit us. And there should be no greater comfort to Christian persons, than to be made like unto Christ, by suffering patiently adversities, troubles, and sicknesses. For he himself went not up to joy, but first he suffered pain; he entered not into his glory before he was crucified. So truly our way to eternal joy is to suffer here with Christ; and our door to enter into eternal life is gladly to die with Christ; that we may rise again from death, and dwell with him in everlasting life. Now therefore, taking your sickness, which is thus profitable for you, patiently, I exhort you, in the Name of God, to remember the profession which you made unto God in your Baptism. And forasmuch as after this life there is an account to be given unto the righteous judge, by whom all must be judged, without respect of persons, I require you to examine yourself and your estate, both toward God and man; so that, accusing and condemning yourself for your own faults, you may find mercy at our heavenly Father’s hand for Christ’s sake, and not be accused and condemned in that fearful judgement. Therefore I shall rehearse to you the Articles of our Faith, that you may know whether you do believe as a Christian man should, or no.

\medskip
\centeredrubric{Or if need require he shall explain to him some part of the Christian faith. Which explanation ended, he shall say,}

\drop{I exhort you in the name of God to remember the profession of faith which you made unto God in your baptism, and therefore I shall rehearse to you the Articles of our Faith, that you may shew whether you do believe as a Christian man should.}
\medskip

\bigskip

{\centering\pilcrow{Here the Minister shall rehearse the Articles of the Faith, saying thus,}\par}

\drop{Dost thou believe in God the Father Almighty, Maker of heaven and earth?}


And in Jesus Christ his only-begotten Son our Lord? And that he was conceived by the Holy Ghost, born of the Virgin Mary; that he suffered under Pontius Pilate, was crucified, dead, and buried; that he went down into hell, and also did rise again the third day; that he ascended into heaven, and sitteth at the right hand of God the Father Almighty; and from thence shall come again at the end of the world, to judge the quick and the dead?

And dost thou believe in the Holy Ghost; the holy Catholick Church; the Communion of Saints; the Remission of sins; the Resurrection of the flesh; and everlasting life after death?

\centerline{\rubric{The sick person shall answer,}}
\centerline{All this I stedfastly believe.}

\medskip

%1928 proposed:
\pilcrow{Thereafter, as occasion serves, the Minister shall instruct the sick person so to order his rule of prayer, for himself and others, that his days of sickness may be a time of faithful and loving intercourse with God.}


\section{Exortation to Repentance}
\pilcrow{The Minister shall examine the sick person, whether he repent him truly of his sins, and be in charity with all the world; exhorting him to forgive, from the bottom of his heart, all persons that have offended him; and if he hath offended any other, to ask them forgiveness; and where he hath done injury or wrong to any man, that he make amends to the uttermost of his power.}

\smallskip

\pilcrow{And if he have not before disposed of his goods, let him then be admonished to make his Will, and to declare his Debts, what he oweth, and what is owing unto him; for the better discharging of his conscience, and the quietness of his Executors. But men should often be put in remembrance to take order for the settling of their temporal estates, whilst they are in health.}

% These next two omitted in 1928:
% \pilcrow{These words before rehearsed may be said before the Minister begin his Prayer, as he shall see cause.}

\pilcrow{The Minister should not omit earnestly to move such sick persons as are of ability to be liberal to the poor.}

\medskip
\centerline{\rubric{Then shall the Priest say,}}
%1928, 1929
\drop{Forasmuch as after this life there is an account to be given unto the righteous Judge, by whom all must be judged, without respect of persons, I require you to examine yourself and your state, both toward God and man; so that accusing and condemning yourself for your own faults, you may find mercy at our heavenly Father’s hand for Christ’s sake.}

\medskip

\centeredrubric{After such examination he shall say,}
\drop{Remember not, Lord, our offences, neither take thou vengeance of our sins; spare us, good Lord, spare thy people, whom thou hast redeemed with thy most precious blood, and be not angry with us for ever.}

\R Spare us, good Lord.


\medskip

\pilcrow{Here shall the sick person be moved to make a special confession of his sins, if he feel his conscience troubled with any weighty matter,
in this or other like form.%1928 proposed
}
\drop{I confess to God Almighty, the Father, the Son, and the Holy Ghost, that I have sinned in thought, word, and deed, through my own grievous fault; wherefore I pray God to have mercy on me. And especially I have sinned in these ways....}

\medskip

\centeredrubric{After which confession, the Priest shall absolve him (if he humbly and heartily desire it) after this sort.}

\drop{Our Lord Jesus Christ, who hath left power to his Church to absolve all sinners who truly repent and believe in him, of his great mercy forgive thee thine offences: And by his authority committed to me, I absolve thee from all thy sins, \grealtcross\  In the Name of the Father, and of the Son, and of the Holy Ghost. \R Amen.}

\medskip
\centerline{\rubric{And then he shall say the Collect following.}}
\centerline{Let us pray.}
\drop{O most merciful God, who, according to the multitude of thy mercies, dost so put away the sins of those who truly repent, that thou rememberest them no more: Open thine eye of mercy upon this thy servant, who most earnestly desireth pardon and forgiveness.
\vspace{-1em}
\begin{leftbar}
Renew in \emph{him}, most loving Father, whatsoever hath been decayed by the fraud and malice of the devil, or by \emph{his} own carnal will and frailness; preserve and continue this sick member in the unity of the Church; consider \emph{his} contrition, accept \emph{his} tears, asswage \emph{his} pain, as shall seem to thee most expedient for \emph{him}.
\end{leftbar}
\vspace{-1em}

And forasmuch as he putteth \emph{his} full trust only in thy mercy, impute not unto \emph{him} \emph{his} former sins, but strengthen \emph{him} with thy blessed Spirit; and, when thou art pleased to take \emph{him} hence, take \emph{him} unto
% thy favour,
thine everlasting favour;
through the merits of thy most dearly beloved Son Jesus Christ our Lord. 
\R Amen.}

%green book
\section{The Blessing and Anointing of the Sick}
% 1928 An Act of Prayer and Blessing
% 1929 ANOINTING, AND LAYING ON OF HANDS
% SA THE ANOINTING OF THE SICK 


\rubric{Anthem.} O Saviour of the world, who by thy Cross and precious Blood hast redeemed us, save us, and help us, we humbly beseech thee, O Lord.

% \subsection[{Psalm 71}]{\stylesubsec{Psalm 71.}{In te, Domine, speravi.}{}}
% \drop{In thee, O {\scshape Lord}, have I put my trust; let me never be put to confusion, \star\ but rid me and deliver me in thy righteousness; incline thine ear unto me, and save me.}

% 2\enspace Be thou my strong hold, whereunto I may alway resort: \star\ thou hast promised to help me, for thou art my house of defence and my castle.

% 3\enspace Deliver me, O my God, out of the hand of the ungodly, \star\ out of the hand of the unrighteous and cruel man.

% 4\enspace For thou, O Lord {\scshape God}, art the thing that I long for: \star\ thou art my hope, even from my youth.

% 5\enspace Through thee have I been holden up ever since I was born: \star\ thou art he that took me out of my mother’s womb: my praise shall be always of thee.

% 6\enspace I am become as it were a monster unto many, \star\ but my sure trust is in thee.

% 7\enspace O let my mouth be filled with thy praise, \star\ that I may sing of thy glory and honour all the day long.

% 8\enspace Cast me not away in the time of age; \star\ forsake me not when my strength faileth me.

% 9\enspace For mine enemies speak against me; and they that lay wait for my soul take their counsel together, saying, \star\ God hath forsaken him; persecute him, and take him, for there is none to deliver him.

% 10\enspace Go not far from me, O God; \star\ my God, haste thee to help me.

% 11\enspace Let them be confounded and perish that are against my soul; \star\ let them be covered with shame and dishonour that seek to do me evil.

% 12\enspace As for me, I will patiently abide alway, \star\ and will praise thee more and more.

% 13\enspace My mouth shall daily speak of thy righteousness and salvation; \star\ for I know no end thereof.

% 14\enspace I will go forth in the strength of the Lord {\scshape God}, \star\ and will make mention of thy righteousness only.

% 15\enspace Thou, O God, hast taught me from my youth up until now; \star\ therefore will I tell of thy wondrous works.

% 16\enspace Forsake me not, O God, in mine old age, when I am gray-headed, \star\ until I have shewed thy strength unto this generation, and thy power to all them that are yet for to come.

% 17\enspace Thy righteousness, O God, is very high, \star\ and great things are they that thou hast done: O God, who is like unto thee!

\subsection[{Psalm 121}]{\stylesubsec{Psalm 121.}{Levavi oculus.}{}}
% \drop{In thee, O {\scshape Lord}, have I put my trust; let me never be put to confusion, \star\ but rid me and deliver me in thy righteousness; incline thine ear unto me, and save me.}


\drop{I will lift up mine eyes unto the hills; \star\ from whence cometh my help?}

2\enspace My help cometh even from the {\scshape Lord}, \star\ who hath made heaven and earth.

3\enspace He will not suffer thy foot to be moved; \star\ and he that keepeth thee will not sleep.

4\enspace Behold, he that keepeth israel \star\ shall neither slumber nor sleep.

5\enspace The {\scshape Lord} himself is thy keeper; \star\ the {\scshape Lord} is thy defence upon thy right hand;

6\enspace So that the sun shall not burn thee by day, \star\ neither the moon by night.

7\enspace The {\scshape Lord} shall preserve thee from all evil; \star\ yea, it is even he that shall keep thy soul.

8\enspace The {\scshape Lord} shall preserve thy going out, and thy coming in, \star\ from this time forth for evermore.

Glory be to the Father, and to the Son, \star\  and to the Holy Ghost;

As it was in the beginning, is now, and ever shall be, \star\  world without end. Amen.

\smallskip
\centeredrubric{Or any other Psalm, such as the following: \emph{23, 27, 43, 71 (\emph{verses} 1–17), 77, 86, 91, 103, 130, 142, 146.}}

\smallskip
\rubric{Anthem.} O Saviour of the world, who by thy Cross and precious Blood hast redeemed us, save us, and help us, we humbly beseech thee, O Lord.

\medskip

\pilcrow{Then shall the Minister say (laying his hands upon the sick person if desired),} %1928
\drop{O Almighty God, who art the giver of all health, and the aid of them that seek to thee for succour: We call upon thee for thy help and goodness mercifully to be shewed upon this thy servant, that being healed of \emph{his} infirmities, \emph{he} may give thanks unto thee in thy holy Church; through Jesus Christ our Lord. \R Amen.}





\begin{leftbar}
    %nonjurors 1718
\pilcrow{If the oil is to be then hallowed, he shall say standing the following prayer.}
\drop{O Almighty Lord God, who hast taught us by thy holy Apostle Saint James to anoint the sick with oil, that they may recover their health and render thanks unto thee for the same; Bless \grealtcross\ this oil, we beseech thee, that whosoever may be anointed therewith, may be delivered from all troubles of body and mind, and from every assault of the powers of evil; through Jesus Christ our Lord. \R Amen.}
\end{leftbar}

%1929
\pilcrow{Then shall the Priest, if the sick person so desire it, proceed to anoint him with oil, saying as followeth:}

%1549
% \pilcrow{If the sick person desire to be anointed, then shall the Priest anoint him upon the forehead or breast only, making the sign of the cross, saying thus,}
\drop{N, I anoint thee with hallowed oil, \grealtcross\ In the Name of the Father, and of the Son, and of the Holy Ghost. \R Amen.}


%greenbook
\centerline{\rubric{He may add the following benediction.}}

\drop{As with this visible oil thy body outwardly is anointed, so may our heavenly Father, God Almighty, grant of his infinite goodness, that thy soul inwardly may be anointed with the Holy Ghost, who is the Spirit of all strength, comfort, relief, and gladness.}
May he % green book
%  and 1549
vouchsafe of his great mercy (if it be his blessed will) to restore unto thee thy bodily health, and strength to serve him joyfully; and send thee release of all thy pains, troubles, and diseases both in body and mind.
\vspace{-1em}
\begin{leftbar}
And howsoever his goodness, by his divine and unsearchable providence, shall dispose of thee: we, his unworthy ministers and servants, humbly beseech the eternal majesty to do with thee according to the multitude of his innumerable mercies, and to pardon thee all thy sins and offences, committted by all thy bodily senses, passions, and carnal affections.
\end{leftbar}

\vspace{-1.7em}

\begin{leftbar}
    May he % green book
% who %1549
also vouchsafe mercifully to grant unto thee ghostly strength by his Holy Spirit to withstand and overcome all temptations and assaults of thine adversary, that in no wise he prevail against thee, but that thou mayest have perfect victory and triumph against the devil, sin, and death;
\end{leftbar}
\vspace{-1em}

Through Christ our Lord, who by his death hath overcome the prince of death; and with the Father and the Holy Ghost everymore liveth and reigneth God, world without end.  \R Amen.

% Usque quo, Domine. Psalm xiii.

% How long wilt thou forget me, (O Lord,) for ever? how long wilt thou hyde thy face from me? How long shall I seke counsell in my soule? and be so vexed in myne herte? how long shall myne enemye triumph over me? Consydre, and heare me, (O lord my God): lighten myne iyes, that I slepe not in death. Leste myne enemy saye: I have prevayled against hym: for yf I be cast downe, they that trouble me will reioyce at it. But my trust is in thy mercy: and my herte is joyfull in thy salvacion. I will sing of the lord, because he hath delte so lovingly with me. Yea, I wyll prayse the name of the Lord the most highest. Glory be to the, \etc As it was in the, \etc

\medskip
\centerline{\rubric{Then shall the Minister say,}}
\drop{The Almighty Lord, who is a most strong tower to all them that put their trust in him, to whom all things in heaven, in earth, and under the earth, do bow and obey, be now and evermore thy defence; and make thee know and feel, that there is none other Name under heaven given to man, in whom, and through whom, thou mayest receive health and salvation, but only the Name of our Lord Jesus Christ. \R Amen.}

\centerline{\pilcrow{And after that he shall say,}}
\drop{Unto God’s gracious mercy and protection we commit thee. The {\scshape Lord} \cross  bless thee, and keep thee. The {\scshape Lord} make his face to shine upon thee, and be gracious unto thee. The {\scshape Lord} lift up his countenance upon thee, and give thee peace, both now and evermore. \R Amen.}


\section{The Communion of the Sick}
\pilcrow{Forasmuch as all mortal men be subject to many sudden perils, diseases, and sicknesses, and ever uncertain what time they shall depart out of this life; therefore, to the intent they may always be in a readiness to die, whensoever it shall please Almighty God to call them, the Curate shall diligently from time to time (but especially in the time of pestilence, or other infectious sickness) exhort their Parishioners to the often receiving of the Holy Communion of the Body and Blood of our Saviour Christ, when it shall be publickly administered in the church; that so doing, they may, in case of sudden visitation, have the less cause to be disquieted for lack of the same.}


% \begin{leftbar}
\pilcrow{The Curate shall also instruct the people concerning the Communion of the Sick, as occasion shall require, that they may not be in ignorance that men can receive the Holy Sacrament in their homes, if they be unable, for any just cause, to come to the church.}
% \end{leftbar}

\pilcrow{But if the sick person be not able to come to the Church, and yet is desirous to receive the Communion in his house; then he must give timely notice to the Priest, signifying also, as far as he may, whether there be some to communicate with him; as is much to be desired.}

\smallskip

\pilcrow{When the consecrated Bread and Wine are taken from the church to the sick person, before the Priest administers the Holy Sacrament, he shall use at least the parts of the \emph{Order of Communion} on \emph{pg.~\pageref{reservedSacrament}} here named: the \emph{General Confession} and \emph{Absolution}, 
% (which may be in the shorter form), 
and the prayer \emph{We do not presume, \etc}, except when extreme sickness shall otherwise require: and after the delivery of the Sacrament of Christ’s Body and Blood with the appointed words, he shall say the \emph{Lord’s Prayer} and the \emph{Blessing}. And immediately thereafter any of the consecrated Elements that remain over shall be reverently consumed, or else taken back to the church.}


% \stylesec{The Celebration}{of the}{Holy Communion for the Sick}
% \smallskip
\pilcrow{And a convenient place in the sick man’s house, together with all things necessary, having been prepared that the Curate may reverently minister, he shall there celebrate the \emph{Order of Communion}, according to the form in this Book prescribed; save only that he may, at his discretion, begin with the Collect, Epistle, and Gospel here following, or else with those of the Day.}

% \drop{O praise the {\scshape Lord}, all ye heathen; praise him, all ye nations. For his merciful kindness is ever more and more towards us; and the truth of the {\scshape Lord} endureth for ever.  Praise the {\scshape Lord}. Glory be to the Father, and to the Son, and to the Holy Ghost;  As it was in the beginning, is now, and ever shall be, world without end. Amen.}

% \medskip
% \centerline{Lord, have mercy upon us.}
% \centerline{\emph{Christ, have mercy upon us.}}
% \centerline{Lord, have mercy upon us.}
% \medskip
% \V The Lord be with you.  \R And with thy spirit.
% \centerline{Let us pray.}
\subsection{\stylesubsec{}{The Collect.}{}}


% \begin{leftbar}

\drop{Almighty and immortal God, giver of life and health: We beseech thee to hear our prayers for this thy servant, that by thy blessing upon \emph{him} and upon those who minister to \emph{him}, \emph{he} may be restored to health of body and mind, and give thanks to thee in thy holy Church; through Jesus Christ our Lord. \R Amen.}

\centerline{\rubric{Or this.}}
\drop{Almighty, everliving God, Maker of mankind, who dost correct those whom thou dost love, and chastise every one whom thou dost receive: We beseech thee to have mercy upon this thy servant visited with thine hand, and to grant that \emph{he} may take \emph{his} sickness patiently, and recover \emph{his} bodily health, (if it be thy gracious will); and whensoever \emph{his} soul shall depart from the body, it may be without spot presented unto thee; through Jesus Christ our Lord. \R Amen.}

% The following collect is one of the Postcommunions, made singular.
\centerline{\rubric{Or this.}}
\drop{Assist us mercifully, O Lord, in these our supplications and prayers, and dispose the way of thy servant towards the attainment of everlasting salvation; that among all the changes and chances of this mortal life, \emph{he} may ever be defended by thy most gracious and ready help; through Jesus Christ our Lord. \R Amen.}
% \end{leftbar}

% \medskip
% \subsection{\stylesubsec{}{The Epistle.}{Hebrews 12.~5.}}
% \drop{My son, despise not thou the chastening of the Lord, nor faint when thou art rebuked of him. For whom the Lord loveth he chasteneth; and scourgeth every son whom he receiveth.}
% % \begin{leftbar}

%     \smallskip
% \centerline{\rubric{Or this.}}
% \vspace{-10pt}
\subsection{\stylesubsec{}{The Epistle.}{2 Corinthians 1.~3.}}
\drop{Blessed be God, even the Father of our Lord Jesus Christ, the Father of mercies, and the God of all comfort; who comforteth us in all our tribulation, that we may be able to comfort them which are in any trouble, by the comfort wherewith we ourselves are comforted of God. For as the sufferings of Christ abound in us, so our consolation also aboundeth by Christ.}
% \end{leftbar}

\medskip
% \subsection{\stylesubsec{}{The Gospel.}{St.~John 5.~24.}}
% \drop{Verily, verily I say unto you, He that heareth my word, and believeth on him that sent me, hath everlasting life, and shall not come into condemnation; but is passed from death unto life.}

% % \begin{leftbar}
%     \smallskip
% \centerline{\rubric{Or this.}}
% \vspace{-10pt}
\subsection{\stylesubsec{}{The Gospel.}{St.~John 10.~14, 15; 27–30.}}
\drop{I am the good shepherd; and I know mine own, and mine own know me, even as the Father knoweth me, and I know the Father; and I lay down my life for the sheep. My sheep hear my voice, and I know them, and they follow me: and I give unto them eternal life; and they shall never perish, and no one shall pluck them out of my hand. My Father, which hath given them unto me, is greater than all; and no one is able to pluck them out of the Father’s hand. I and the Father are one.}
% \end{leftbar}

\pilcrow{After which the Priest shall proceed according to the form before prescribed for the Order of Communion.}
%, beginning at  these words \emph{Ye that do truly,} \etc}
% the {\emph Offertory}, \emph{pg.~\pageref{offertory}}}



% \drop{I will offer to thee the sacrifice of thanksgiving, and will call upon the Name of the {\scshape Lord}.}

% \V The Lord be with you. \R And with thy spirit.

% \V Lift up your hearts.  \R We lift them up unto the Lord.

% \V Let us give thanks unto our Lord God. \R It is meet and right so to do.


% \centerline{\rubric{Then shall the Priest turn to the Lord’s Table, and say,}}

% \drop{It is very meet, right, and our bounden duty, that we should at all times, and in all places, give thanks unto thee, O Lord, Holy Father, Almighty, Everlasting God.  Therefore with Angels and Archangels, and with all the company of heaven, we laud and magnify thy glorious Name; evermore praising thee, and saying,}
% \smallskip

% \drop{Holy, holy, holy, Lord God of hosts, heaven and earth are full of thy glory: Glory be to thee, O Lord most High. \grecross\ Blessed is he that cometh in the Name of the Lord;
% Hosanna in the highest.}

% \bigskip

% \pilcrow{When the Priest, standing before the Table, hath so ordered the Bread and Wine, that he may with the more readiness and decency break the Bread before the people, and take the Cup into his hands, he shall say the Prayer of Consecration, as followeth.}
% \drop{Almighty God, our heavenly Father, who of thy tender mercy didst give thine only Son Jesus Christ to suffer death upon the Cross for our redemption; who made there (by his one oblation of himself once offered) a full, perfect, and sufficient sacrifice, oblation, and satisfaction, for the sins of the whole world; and did institute, and in his holy Gospel command us to continue, a perpetual memory of that his precious death, until his coming again;

% Hear us, O merciful Father, we most humbly beseech thee; and grant that we receiving these thy creatures of bread and wine, according to thy Son our Saviour Jesus Christ’s holy institution, in remembrance of his death and passion, may be partakers of his most blessed \grealtcross\ Body and \grealtcross\ Blood: 

% Who, in the same night that he was betrayed, \footnote{\rubric{Here the Priest is to take the Paten unto his hands:}}took Bread; and, when he had given thanks, \footnote{\rubric{And here to break the Bread:}}he brake it, and gave it to his disciples, saying, Take, eat, \footnote{\rubric{And here to lay his hand upon all the Bread.}}{\scshape this is my Body which is given for you}: Do this in remembrance of me. Likewise after supper he \footnote{\rubric{Here he is to take the Cup into his hand:}}took the Cup; and, when he had given thanks, he gave it to them, saying, Drink ye all of this; \footnote{\rubric{And here to lay his hand upon every vessel (be it Chalice or Flagon) in which there is any Wine to be consecrated.}}{\scshape for this is my Blood of the New Testament, which is shed for you and for many for the remission of sins}: Do this, as oft as ye shall drink it, in remembrance of me.}

% Wherefore, O Lord and heavenly Father, we thy humble servants, having in remembrance the precious death of thy dear Son, his mighty resurrection and glorious ascension, looking also for his coming again, do render unto thee most hearty thanks for the innumerable benefits which he hath procured unto us; and we entirely desire thy fatherly goodness mercifully to accept this our sacrifice of praise and thanksgiving; most humbly beseeching thee to grant, that by the merits and death of thy Son Jesus Christ, and through faith in his blood, we and all thy whole Church may obtain remission of our sins, and all other benefits of his passion.

% And here we offer and present unto thee, O Lord, ourselves, our souls and bodies, to be a reasonable, holy, and lively sacrifice unto thee; 
% and we pray thee of thine almighty goodness to send upon us, and upon these thy gifts, thy holy and blessed Spirit, who is the Sanctifier and the Giver of life; humbly beseeching thee, that all we, who are partakers of this holy Communion, may be fulfilled with thy grace and heavenly \grecross\ benediction. 

% And although we be unworthy, through our manifold sins, to offer unto thee any sacrifice, yet we beseech thee to accept this our bounden duty and service; not weighing our merits, but pardoning our offences;

% Through Jesus Christ our Lord; by whom, and with whom, in the unity of the Holy Ghost, all honour and glory be unto thee, O Father Almighty, world without end. \R Amen.

% \smallskip
% {\centering\footnotesize\rubric{Here shall the people join with the Priest in the Lord’s Prayer, the Priest first saying,}\par}
% As our Saviour Christ hath commanded and taught us we are bold to say,
% \drop{Our Father, which art in heaven, Hallowed be thy Name. Thy kingdom come. Thy will be done, in earth as it is in heaven. Give us this day our daily bread. And forgive us our trespasses, As we forgive them that trespass against us. And lead us not into temptation; But deliver us from evil.  For thine is the kingdom, The power, and the glory, For ever and ever. Amen.}

% \bigskip
% \pilcrow{Then shall the Priest say to them that come to receive the holy Communion,}
% \drop{Draw near with faith, and take this Holy Sacrament to your comfort; and make your humble confession to Almighty God, meekly kneeling upon your knees.}

% \smallskip
% \rubric{Then shall be said by the Minister and people together,}
% \drop{We confess to God Almighty, the Father, the Son, and the Holy Ghost, that we have sinned in thought, word, and deed, through our own grievous fault.  Wherefore we pray God to have mercy upon us.}

% \medskip
% {\centering\footnotesize\rubric{Then shall the Priest standing up, and turning himself to the people, pronounce this Absolution.}\par}
% \drop{Almighty God have mercy upon you, forgive you all your sins, and deliver you from all evil, confirm and strengthen you in all goodness, and bring you to everlasting life; through Jesus Christ our Lord. \R Amen.}

% \centerline{\pilcrow{Then shall the Priest say,}}
% Hear what comfortable words our Saviour Christ saith unto all that truly turn to him.
% \drop{Come unto me all that travail and are heavy laden, and I will refresh you.}\scripture{St.~Matthew xj.~28}

% So God loved the world, that he gave his only-begotten Son, to the end that all that believe in him should not perish, but have everlasting life.\scripture{St.~John iij.~16}

% \centerline{Hear also what Saint Paul saith.}

% This is a true saying, and worthy of all men to be received, That Christ Jesus came into the world to save sinners.\scripture{1 Timothy i.~15.}

% \centerline{Hear also what Saint John saith.}

% If any man sin, we have an Advocate with the Father, Jesus Christ the righteous; and he is the propitiation for our sins.\scripture{1 St.~John ij.~1.}

% \medskip

% {\centering\footnotesize\rubric{Then shall the Priest, kneeling down at the Lord’s Table, say in the name of all them that shall receive the Communion this Prayer following.}\par}
% \drop{We do not presume to come to this thy Table, O merciful Lord, trusting in our own righteousness, but in thy manifold and great mercies. We are not worthy so much as to gather up the crumbs under thy Table. But thou art the same Lord, whose property is always to have mercy: Grant us therefore, gracious Lord, so to eat the flesh of thy dear Son Jesus Christ, and to drink his blood, that our sinful bodies may be made clean by his body, and our souls washed through his most precious blood, and that we may evermore dwell in him, and he in us. Amen.}

\medskip

\pilcrow{At the time of the distribution of the Holy Sacrament, the priest shall first receive the Communion himself, and after minister unto them that are appointed to communicate with the sick, and last of all to the sick person.}
% \medskip

% {\centering\footnotesize\rubric{And, when he delivereth the Bread to any one, he shall say,}\par}
% \drop{The Body of our Lord Jesus Christ, which was given for thee, preserve thy body and soul unto everlasting life.}

% {\centering\footnotesize\rubric{And the Minister that delivereth the Cup to any one shall say,}\par}
% \drop{The Blood of our Lord Jesus Christ, which was shed for thee, preserve thy body and soul unto everlasting life.}

% \medskip

% {\footnotesize\rubric{When all have communicated, the Minister shall return to the Lord’s Table, and reverently place upon it what remaineth of the consecrated Elements, covering the same with a fair linen cloth.}\par}


% After shall be said as followeth.

% \drop{ALMIGHTY and everliving God, we most heartily thank thee, for that thou dost vouchsafe to feed us, who have duly received these holy mysteries, with the spiritual food of the most precious Body and Blood of thy Son our Saviour Jesus Christ; and dost assure us thereby of thy favour and goodness towards us; and that we are very members incorporate in the mystical body of thy Son, which is the blessed company of all faithful people; and are also heirs through hope of thy everlasting kingdom, by the merits of the most precious death and passion of thy dear Son. And we most humbly beseech thee , O heavenly Father, so to assist us with thy grace, that we may continue in that holy fellowship, and do all such good works as thou hast prepared for us to walk in; through Jesus Christ our Lord, to whom, with thee and the Holy Ghost, be all honour and glory, world without end. Amen.}
% \medskip

% {\centering\footnotesize\rubric{Then the Priest shall let them depart with this Blessing.}\par}
% \drop{The peace of God, which passeth all understanding, keep your hearts and minds in the knowledge and love of God, and of his son Jesus Christ our Lord: and the blessing of God Almighty, the Father, \grealtcross\ the Son, and the Holy Ghost, be amongst you and remain with you always. \R Amen.}

% \bigskip

% \begin{leftbar}
% \pilcrow{In case of extreme necessity the Priest may beg in with the Consecration and, immediately after the delivery of the Holy Sacrament to the sick person, end with the Blessing.}
% \end{leftbar}


\medskip

\pilcrow{The Priest shall instruct the people that if any man, by reason of great sickness, or any other just impediment, be not able at any time to receive the Sacrament of Christ’s Body and Blood, yet if he do truly repent him of his sins, and stedfastly believe that Jesus Christ both suffered death upon the Cross for him, and shed his Blood for his redemption, earnestly remembering the benefits he hath thereby, and giving him hearty thanks therefore, he doth eat and drink the Body and Blood of our Saviour Christ profitably to his Soul’s health, although he do not receive the Sacrament with his mouth.}

% When the sick person is visited, and receiveth the holy Communion all at one time, then the Priest, for more expedition, shall cut off the form of the Visitation at the Psalm [In thee, O Lord, have I put my trust \etc] and go straight to the Communion.

% \pilcrow{In the time of the plague, sweat, or such other like contagious times of sickness or diseases, when none of the Parish or neighbors can be gotten to communicate with the sick in their houses, for fear of the infection, upon special request of the diseased, the Minister may only communicate with him.}

\bigskip


\centerline{\rule{0.5\textwidth}{0.5pt}}
\medskip
\pilcrow{When it is desirable to administer both kinds together, the words of administration shall be said thus}

\smallskip

\drop{The Body of our Lord Jesus Christ, which was given for thee, and his Blood which was shed for thee, preserve thy body and soul unto everlasting life.}

\smallskip

\centeredrubric{Take this in remembrance that Christ died for thee, and feed on him in thy heart by faith with thanksgiving.}


\medskip

\pilcrow{{\scshape Note}, that the same order shall be observed, with the permission of the Bishop, when it is deemed necessary, through grave danger of infection, to administer both kinds together to certain communicants at the open Communion.}


\section[Special Prayers]{Special Prayers to be Used as Occasion may Serve}
\subseccaption{}{A Litany for the Sick or Dying.}

\drop{O God the Father,}

\qquad\emph{Have mercy.}

O God the Son,

\qquad\emph{Have mercy.}

O God the Holy Ghost,

\qquad\emph{Have mercy.}

O Holy Trinity, one God,

\qquad\emph{Have mercy.}

Remember not, Lord, our offences.

\qquad\emph{Spare us, Good Lord.}

From all evil and sin,

\qquad\emph{Good Lord, deliver \emph{him.}}

From the assaults of the devil,

\qquad\emph{Good Lord, deliver \emph{him.}}

From thy wrath, and from everlasting damnation,

\qquad\emph{Good Lord, deliver \emph{him.}}

In the hour of death,

\qquad\emph{Good Lord, deliver \emph{him.}}

In the day of judgement,

\qquad\emph{Good Lord, deliver \emph{him.}}

By the mystery of thine Incarnation,

\qquad\emph{Save \emph{him}, O Lord.}

By thy Cross and Passion,

\qquad\emph{Save \emph{him.}, O Lord.}

By thy Resurrection and final Triumph,

\qquad\emph{Save \emph{him}, O Lord.}

That it may please thee to grant \emph{him} relief in pain;

\qquad\emph{We beseech thee to hear us.}

To give \emph{him} such health as is agreeable to thy will;

\qquad\emph{We beseech thee to hear us.}

That it may please thee to deliver \emph{his} soul;

\qquad\emph{We beseech thee to hear us.}

To cleanse \emph{him} from \emph{his} sin;

\qquad\emph{We beseech thee to hear us.}
    
That it may please thee to receive \emph{him} to thyself;

\qquad\emph{We beseech thee to hear us.}

To set \emph{him} in a place of light and peace;

\qquad\emph{We beseech thee to hear us.}

To number \emph{him} with thy saints and thine elect;

\qquad\emph{We beseech thee to hear us.}

Son of God;

\qquad\emph{We beseech thee to hear us.}

O Lamb of God;

\qquad\emph{Have mercy upon us.}

O Lamb of God;

\qquad\emph{Grant \emph{him} thy peace.}


\centerline{\rule{0.5\textwidth}{0.5pt}}


\centerline{\pilcrow{The following Prayers may be used as occasion requires.}}
\subseccaption{}{For Healing.}
\drop{O God, who by the might of thy command canst drive away from men’s bodies all sickness and infirmity: Be present in thy goodness with this thy servant, that \emph{his} weakness being banished, and \emph{his} health restored, \emph{he} may live to glorify thy holy Name; through our Lord Jesus Christ. \R Amen.}


\subseccaption{}{For a Sick Child.}
\drop{O Lord Jesus Christ, who didst with joy receive and bless the children brought to thee: Give thy blessing to this thy child; and in thine own time deliver \emph{him} from \emph{his} bodily pain, that \emph{he} may live to serve thee all \emph{his} days. \R Amen.}


\subseccaption{}{For one troubled in Conscience.}
\drop{O blessed Lord, the Father of mercies and the God of all comfort; We beseech thee, look down in pity and compassion on thy servant, whose soul is full of trouble: give \emph{him} a right understanding of \emph{himself}, and also of thy will for \emph{him}, that \emph{he} may neither cast away \emph{his} confidence in thee, nor place it anywhere but in thee; deliver \emph{him} from the fear of evil; lift up the light of thy countenance upon \emph{him}, and give \emph{him} thine everlasting peace; through the merits and mediation of Jesus Christ our Lord. \R Amen.}

%Am1928
\subseccaption{}{For a Person under Affliction.}
\drop{O merciful God, and heavenly Father, who hast taught us in thy holy Word that thou dost not willingly afflict or grieve the children of men; Look with pity, we beseech thee, upon the sorrows of thy servant for whom our prayers are offered. Remember \emph{him}, O Lord, in mercy; endue \emph{his} soul with patience; comfort \emph{him} with a sense of thy goodness; lift up thy countenance upon \emph{him}, and give \emph{him} peace; through Jesus Christ our Lord. \R Amen.}



\subseccaption{}{For a Convalescent.}
\drop{O Lord, whose compassions fail not, and whose mercies are new every morning: We give thee hearty thanks that it hath pleased thee to give to this our \emph{brother} both relief from pain and hope of renewed health; continue, we beseech thee, in \emph{him} the good work that thou hast begun; that, daily increasing in bodily strength, and humbly rejoicing in thy goodness, \emph{he} may so order \emph{his} life and conversation as always to think and do such things as shall please thee; through Jesus Christ our Lord. \R Amen.}


\subseccaption{}{For a Dying Child.}
\drop{O Lord Jesu Christ, the only-begotten Son of God, who for our sakes didst become a babe in Bethlehem: We commit unto thy loving care this child whom thou art calling to thyself. Send thy holy angel to lead \emph{him} gently to those heavenly habitations where the souls of them that sleep in thee have perpetual peace and joy, and fold \emph{him} in the everlasting arms of thine unfailing love; who livest and reignest with the Father and the Holy Ghost, one God world without end. \R Amen.}


\subseccaption{}{Commendatory Prayers.}
\drop{Thou knowest, Lord, the secrets of our hearts; shut not thy merciful ears to our prayer; but spare us, Lord most holy, O God most mighty, O holy and merciful Saviour, thou most worthy Judge eternal, suffer us not at our last hour, for any pains of death, to fall from thee. \R Amen.}

\smallskip

\drop{Unto thee, O Lord, we commend the soul of thy servant \emph{N.}, that, dying to the world, \emph{he} may live to thee; and whatsoever sins \emph{he} has committed through the frailty of earthly life, we beseech thee to do away by thy most loving and merciful forgiveness; through Jesus Christ our Lord. \R Amen.}

\smallskip

\drop{O Almighty God, with whom do live the spirits of just men made perfect, after they are delivered from their earthly prisons: We humbly commend the soul of this thy servant, our dear \emph{brother}, into thy hands, as into the hands of a faithful Creator, and most merciful Saviour; most humbly beseeching thee, that it may be precious in thy sight. Wash it, we pray thee, in the blood of that immaculate Lamb that was slain to take away the sins of the world; that whatsoever defilements it may have contracted in the midst of this miserable and naughty world, through the lusts of the flesh, or the wiles of Satan, being purged and done away, it may be presented pure and without spot before thee; through the merits of Jesus Christ thine only Son our Lord. \R Amen.}

\smallskip

\subseccaption{}{At the Point of Death.}

\drop{Go forth upon thy journey from this world, O Christian soul,}

In the Name of God the Almighty Father who created thee. \R Amen.

In the Name of Jesus Christ who suffered for thee. \R Amen.

In the Name of the Holy Ghost who strengtheneth thee. \R Amen.

In communion with the blessed Saints, and aided by Angels and Archangels,  and all the armies of the heavenly host. \R Amen.

May thy portion this day be in peace, and thy dwelling in the heavenly Jerusalem. \R Amen.

\bigskip
{\footnotesize
{\scshape Note}.— The following prayers and passages of Holy Scripture are suitable for use with the sick person: The Collect in the Communion of the Sick and the Collects appointed for the first, second and fourth Sundays in Advent, the third, fourth, and Sixth Sundays after Epiphany, Ash Wednesday, the second Sunday in Lent, the Sunday next before Easter, the fourth Sunday after Easter, Ascension Day, the Sunday after Ascension, Trinity Sunday, the fourth, sixth, seventh, twelfth, fifteenth, eighteenth, and twenty-first Sundays after Trinity, the Transfiguration, St.~Michael and All Angels, St.~Luke the Evangelist, and All Saints’ Day.

\newcommand{\numberedSuggestion}[3]{{\addfontfeatures{Numbers={Monospaced}}#1.\enspace}{\emph{#2}:\enskip}#3\par}

\numberedSuggestion{1}{Confidence in God}{Psalms 27, 46, 91, 121; Proverbs 3.~11–26; Isaiah 26.~1–9; 40.~1–11; 40.~25 to end; Lamentations 3.~22–41; St.~Matthew 6.~24 to end; Romans 8.~31 to end.}
\numberedSuggestion{2}{Answer to Prayer}{Psalms 30, 34.}
\numberedSuggestion{3}{Prayer for Divine Aid}{Psalms 43, 86, 143; St.~James 5.~10 to end.}
\numberedSuggestion{4}{Penitence}{Psalms 51, 130.}
\numberedSuggestion{5}{Praise and Thanksgiving}{Psalms 103, 146; Isaiah 12.}
\numberedSuggestion{6}{God’s dealing with Man through Affliction}{Job 33.~14–30; Hebrews 12.~1–11.}
\numberedSuggestion{7}{Christ our Example in Suffering}{Isaiah 53; St.~Matthew 26.~36–46; St.~Luke 23.~27–49.}
\numberedSuggestion{8}{God’s call to Repentance and Faith}{Isaiah 55.}
\numberedSuggestion{9}{The Beatitudes}{St.~Matthew 5.~1–12.}
\numberedSuggestion{10}{Watchfulness}{St.~Luke 12.~32–40.}
\numberedSuggestion{11}{Christ the Good Shepherd}{Psalm 23; St.~John 10.~1–18.}
\numberedSuggestion{12}{The Resurrection}{St.~John 20.~1–18; 20.~19 to end; 2 Corinthians 4.~13—5.~9.}
\numberedSuggestion{13}{Redemption}{Romans 5.~1–11; 8.~18 to end; 1 St.~John 1.~1–9.}
\numberedSuggestion{14}{Christian Love}{1 Corinthians 13.}
\numberedSuggestion{15}{Growth in Grace}{Ephesians 3.~13 to end; 6.~10–20; Philippians 3.~7–14.}
\numberedSuggestion{16}{Patience in Suffering}{St.~James 5.~10 to end.}
\numberedSuggestion{17}{God’s Love to Men}{1 St.~John 3.~1–7; 4.~9 to end.}
\numberedSuggestion{18}{The Life of the World to come}{Revelation 7.~9 to end; 21.~1–7; 21.~22 to end; 22.~1–5.}
\numberedSuggestion{19}{Our Lord’s last Discourse before his Passion}{St.~John 14, 15, 16, 17.}
\numberedSuggestion{20}{Christian Hope on the Approach of Death}{Deuteronomy 33.~27; Psalm 16.~9 to end; Psalm 23; St.~John 3.~16; 2 Corinthians 4.~16—5.~1; Revelation 21.~4–7.}
}

\fleuron

% split in two in 1928: for sick, and for dying.
% \subseccaption{}{A Prayer for a sick Child.}
% \drop{O Almighty God, and merciful Father, to whom alone belong the issues of life and death: Look down from heaven, we humbly beseech thee, with the eyes of mercy upon this child now lying upon the bed of sickness: Visit him, O Lord, with thy salvation; deliver him in thy good appointed time from his bodily pain, and save his soul for thy mercies’ sake: That, if it shall be thy pleasure to prolong his days here on earth, he may live to thee, and be an instrument of thy glory, by serving thee faithfully, and doing good in his generation; or else receive him into those heavenly habitations, where the souls of them that sleep in the Lord Jesus enjoy perpetual rest and felicity. Grant this, O Lord, for thy mercies’ sake, in the same thy Son our Lord Jesus Christ, who liveth and reigneth with thee and the Holy Ghost, ever one God, world without end. \R Amen.}

% \subseccaption{}{A Prayer for a sick person, when there appeareth small hope of recovery.}
% \drop{O Father of mercies, and God of all comfort, our only help in time of need: We fly unto thee for succour in behalf of this thy servant, here lying under thy hand in great weakness of body. Look graciously upon him, 0 Lord; and the more the outward man decayeth, strengthen him, we beseech thee, so much the more continually with thy grace and Holy Spirit in the inner man. Give him unfeigned repentance for all the errors of his life past, and stedfast faith in thy Son Jesus; that his sins may be done away by thy mercy, and his pardon sealed in heaven, before he go hence, and be no more seen. We know, 0 Lord, that there is no word impossible with thee; and that, if thou wilt, thou canst even yet raise him up, and grant him a longer continuance amongst us: Yet, forasmuch as in all appearance the time of his dissolution draweth near, so fit and prepare him, we beseech thee, against the hour of death, that after his departure hence in peace, and in thy favour, his soul may be received into thine everlasting kingdom, through the merits and mediation of Jesus Christ, thine only Son, our Lord and Saviour. \R Amen.}

%in the "Commendatory Prayers", slightly shortened.
% \subseccaption{}{A commendatory Prayer for a sick person at the point of departure.}
% \drop{O Almighty God, with whom do live the spirits of just men made perfect, after they are delivered from their earthly prisons: We humbly commend the soul of this thy servant, our dear brother, into thy hands, as into the hands of a faithful Creator, and most merciful Saviour; most humbly beseeching thee, that it may be precious in thy sight. Wash it, we pray thee, in the blood of that immaculate Lamb, that was slain to take away the sins of the world; that whatsoever defilements it may have contracted in the midst of this miserable and naughty world, through the lusts of the flesh, or the wiles of Satan, being purged and done away, it may be presented pure and without spot before thee. And teach us who survive, in this and other like daily spectacles of mortality, to see how frail and un, certain our own condition is; and so to number our days, that we may seriously apply our hearts to that holy and heavenly wisdom, whilst we live here, which may in the end bring us to life everlasting, through the merits of Jesus Christ thine only Son our Lord. \R Amen.}

% \subseccaption{}{A Prayer for persons troubled in mind or in conscience.}
% \drop{O blessed Lord, the Father of mercies, and the God of all comforts: We beseech thee, took down in pity and compassion upon this thy afflicted servant. Thou writest bitter things against him, and makest him to possess his former iniquities; thy wrath lieth hard upon him, and his soul is full of trouble: But, 0 merciful God, who hast written thy holy Word for our learning, that we, through patience and comfort of thy holy Scriptures, might have hope; give him a right understanding of himself, and of thy threats and promises; that he may neither cast away his confidence in thee, nor place it any where but in thee. Give him strength against all his temptations, and heal all his distempers. Break not the bruised reed, nor quench the smoking flax. Shut not up thy tender mercies in displeasure; but make him to hear of joy and gladness, that the bones which thou hast broken may rejoice. Deliver him from fear of the enemy, and lift up the light of thy countenance upon him, and give him peace, through the merits and mediation of Jesus Christ our Lord. \R Amen.}

% \begin{leftbar} %scottish 1912
% \subseccaption{}{A Prayer for the recovery of a sick person.}
% \drop{Almighty and immortal God, giver of life and health; We beseech thee to hear our prayers for thy servant N, for whom we implore thy mercy, that by thy blessing upon him and upon those who minister to him of thy healing gifts, he may be restored, if it be thy gracious will, to health of body and mind, and give thanks to thee in thy holy Church; through Jesus Christ our Lord. \R Amen.}
% \end{leftbar}

% If any question arise as to the manner of doing anything that is here enjoined or permitted, it shall be referred to the Bishop for his decision.

\chapter[The Burial of the Dead]{\stylechapter{The Order for}{The Burial of the Dead}{}}

\pilcrow{Here is to be noted, that the Office ensuing is not to be used for any that die unbaptized, or excommunicate, or have laid violent hands upon themselves.}

% Canada:1918
\pilcrow{Note also, That when this Office is not to be used, the Priest may at the grave read the Sentences beginning \emph{Man that is born,} followed by the Lesser Litany, the Lord’s Prayer, one or more Collects from this Book at his discretion, and \emph{The grace of the Lord \etc}}

\section{The Procession}

\pilcrow{The Minister and Clerks, meeting the Corpse at the entrance of the Church-yard, and going before it, either into the Church, or towards the Grave, shall say, or sing [one or more of the following Sentences; together with one or more of the Penitential Psalms \emph{(6, 32, 38, 51, 102, 130, 143)} if need so require.],}
\drop{I am the resurrection and the life, saith the Lord: he that believeth in me, though he were dead, yet shall he live: and whosoever liveth and believeth in me shall never die.\scripture{St.~John xi.~25, 26.}}

\drop{I know that my Redeemer liveth, and that he shalt stand at the latter day upon the earth. And though after my skin 
worms destroy this body, %KJV
% hath been thus destroyed, %adapted from rsv
% earth: and though this body be destroyed, yet shall I see God: %am1928
yet in my flesh shall I see God: whom I shall see for myself, and mine eyes shall behold, and not another. \scripture{Job xix.~25, 26, 27.}}


\drop{We brought nothing into this world, and it is certain we can carry nothing out. The {\scshape Lord} gave, and the {\scshape Lord} hath taken away; blessed be the Name of the {\scshape Lord}.\scripture{1 Timothy.~vi.~7. Job i.~21.}}

\section{The Service in church}
\pilcrow{After they are come into the Church, shall be sung or said one or more of these Psalms following. Note, that at the end of each of the Psalms the \emph{Gloria Patri} shall be omitted.} %1954


%1928 proposed
% \pilcrow{Before and after any psalm or group of psalms may be said or sung the Anthem following,}

% \drop{O Saviour of the world, who by thy Cross and precious Blood hast redeemed us, Save us and help us, we humbly beseech thee, O Lord.}
{\red\scshape Anthem.} O Saviour of the world, who by thy Cross and precious Blood hast redeemed us, \star\ Save us, and help us, we humbly beseech thee, O Lord.

%1929 
% After they are come into the church, shall be sung or said one or more of these Psalms following.


% Anthem. O Saviour of the world, who by thy Cross and precious Blood hast redeemed us : save us, and help us, we humbly beseech thee, O Lord.

%(1918)(1926)(1928e)(1929)(1954)
\subsection[{Psalm 23}]{\stylesubsec{Psalm 23.}{Dominus regit me.}{}}
\drop{The {\scshape Lord} is my shepherd; \star\ therefore can I lack nothing.}

2\enspace He shall feed me in a green pasture, \star\ and lead me forth beside the waters of comfort.

3\enspace He shall convert my soul, \star\ and bring me forth in the paths of righteousness, for his Name’s sake.

4\enspace Yea, though I walk through the valley of the shadow of death, I will fear no evil; \star\ for thou art with me; thy rod and thy staff comfort me.

5\enspace Thou shalt prepare a table before me against them that trouble me; \star\ thou hast anointed my head with oil, and my cup shall be full.

6\enspace But thy loving-kindness and mercy shall follow me all the days of my life; \star\ and I will dwell in the house of the Lord for ever.

Rest eternal grant unto them, O Lord \star\ and let light perpetual shine upon them.


% Psalm 39.  (1918)(1662)(1926)(1928a)(1929)
\subsection[{Psalm 39}]{\stylesubsec{Psalm 39.}{Dixi, custodiam.}{}}
\drop{I said, I will take heed to my ways, \star\ that I offend not in my tongue.}

2\enspace I will keep my mouth as it were with a bridle, \star\ while the ungodly is in my sight.

3\enspace I held my tongue, and spake nothing: \star\ I kept silence, yea, even from good words; but it was pain and grief to me.

4\enspace My heart was hot within me: and while I was thus musing the fire kindled, \star\ and at the last I spake with my tongue:

5\enspace {\scshape Lord}, let me know mine end, and the number of my days; \star\ that I may be certified how long I have to live.

6\enspace Behold, thou hast made my days as it were a span long, \star\ and mine age is even as nothing in respect of thee; and verily every man living is altogether vanity.

7\enspace For man walketh in a vain shadow, and disquieteth himself in vain; \star\ he heapeth up riches, and cannot tell who shall gather them.

8\enspace And now, Lord, what is my hope? \star\ truly my hope is even in thee.

9\enspace Deliver me from all mine offences; \star\ and make me not a rebuke unto the foolish.

10\enspace I became dumb, and opened not my mouth; \star\ for it was thy doing.

11\enspace Take thy plague away from me: \star\ I am even consumed by the means of thy heavy hand.

12\enspace When thou with rebukes dost chasten man for sin, thou makest his beauty to consume away, like as it were a moth fretting a garment: \star\ every man therefore is but vanity.

13\enspace Hear my prayer, O {\scshape Lord}, and with thine ears consider my calling; \star\ hold not thy peace at my tears;

14\enspace For I am a stranger with thee, and a sojourner, \star\ as all my fathers were.

15\enspace O spare me a little, that I may recover my strength, \star\ before I go hence, and be no more seen.

Rest eternal grant unto them, O Lord \star\ and let light perpetual shine upon them.

% Psalm 90.  (1918)(1662)(1926)(1928a)(1929)(1954)
\subsection[{Psalm 90}]{\stylesubsec{Psalm 90.}{Domine, refugium.}{}}
\drop{Lord, thou hast been our refuge, \star\ from one generation to another.}

2\enspace Before the mountains were brought forth, or ever the earth and the world were made, \star\ thou art God from everlasting, and world without end.

3\enspace Thou turnest man to destruction; \star\ again thou sayest, Come again, ye children of men.

4\enspace For a thousand years in thy sight are but as yesterday, \star\ seeing that is past as a watch in the night.
5\enspace As soon as thou scatterest them they are even as a sleep; \star\ and fade away suddenly like the grass.

6\enspace In the morning it is green, and groweth up; \star\ but in the evening it is cut down, dried up, and withered.

7\enspace For we consume away in thy displeasure, \star\ and are afraid at thy wrathful indignation.

8\enspace Thou hast set our misdeeds before thee; \star\ and our secret sins in the light of thy countenance.

9\enspace For when thou art angry all our days are gone: \star\ we bring our years to an end, as it were a tale that is told.

10\enspace The days of our age are threescore years and ten; and though men be so strong that they come to fourscore years, \star\ yet is their strength then but labour and sorrow; so soon passeth it away, and we are gone.

11\enspace But who regardeth the power of thy wrath? \star\ for even thereafter as a man feareth, so is thy displeasure.

12\enspace So teach us to number our days, \star\ that we may apply our hearts unto wisdom.

13\enspace Turn thee again, O {\scshape Lord}, at the last, \star\ and be gracious unto thy servants.

14\enspace O satisfy us with thy mercy, and that soon: \star\ so shall we rejoice and be glad all the days of our life.

15\enspace Comfort us again now after the time that thou hast plagued us; \star\ and for the years wherein we have suffered adversity.

16\enspace Shew thy servants thy work, \star\ and their children thy glory.

17\enspace And the glorious majesty of the {\scshape Lord} our God be upon us: \star\ prosper thou the work of our hands upon us, O prosper thou our handywork.

Rest eternal grant unto them, O Lord \star\ and let light perpetual shine upon them.


% Psalm 130. (1928a)(1928e)(1929)(1954)
\subsection[{Psalm 130}]{\stylesubsec{Psalm 130.}{De profundis.}{}}

\drop{Out of the deep have I called unto thee, O {\scshape Lord}; \star\ Lord, hear my voice.}

2\enspace O let thine ears consider well \star\ the voice of my complaint.

3\enspace If thou, {\scshape Lord}, wilt be extreme to mark what is done amiss, \star\ O Lord, who may abide it?

4\enspace For there is mercy with thee; \star\ therefore shalt thou be feared.

5\enspace I look for the {\scshape Lord}; my soul doth wait for him; \star\ in his word is my trust.

6\enspace My soul fleeth unto the Lord \star\ before the morning watch, I say, before the morning watch.

7\enspace O Israel, trust in the {\scshape Lord}, for with the {\scshape Lord} there is mercy, \star\ and with him is plenteous redemption.

8\enspace And he shall redeem Ísrael \star\ from all his sins.

Rest eternal grant unto them, O Lord \star\ and let light perpetual shine upon them.


\rubric{Anthem.} O Saviour of the world, who by thy Cross and precious Blood hast redeemed us, \star\ Save us, and help us, we humbly beseech thee, O Lord.


\medskip
\pilcrow{Then shall follow the Lesson, taken out of the fifteenth Chapter of the former Epistle of Saint Paul to the Corinthians.}
\centerline{1 Corinthians 15.~20.}
\drop{Now is Christ risen from the dead, and become the first-fruits of them that slept. For since by man came death, by man came also the resurrection of the dead. For as in Adam all die, even so in Christ shall all be made alive. But every man in his own order: Christ the firstfruits; afterward they that are Christ’s, at his coming. Then cometh the end, when he shall have delivered up the kingdom to God, even the Father; when he shall have put down all rule, and all authority, and power. For he must reign, till he hath put all enemies under his feet. The last enemy that shall be destroyed is death.
\begin{leftbar}
For he hath put all things under his feet. But when he saith, all things are put under him, it is manifest that he is excepted, which did put all things under him. And when all things shall be subdued unto him, then shall the Son also himself be subject unto him that put all things under him, that God may be all in all. Else what shall they do which are baptized for the dead, if the dead rise not at all? Why are they then baptized for the dead? and why stand we in jeopardy every hour? I protest by your rejoicing, which I have in Christ Jesus our Lord, I die daily. If after the manner of men I have fought with beasts at Ephesus, what advantageth it me, if the dead rise not? Let us eat and drink, for to-morrow we die. Be not deceived: evil communications corrupt good manners. Awake to righteousness, and sin not: for some have not the knowledge of God. I speak this to your shame.
\end{leftbar}

But some man will say, How are the dead raised up? and with what body do they come? Thou fool, that which thou sowest is not quickened, except it die. And that which thou sowest, thou sowest not that body that shall be, but bare grain, it may chance of wheat, or of some other grain: But God giveth it a body, as it hath pleased him, and to every seed his own body. All flesh is not the same flesh; but there is one kind of flesh of men, another flesh of beasts, another of fishes, and another of birds. There are also celestial bodies, and bodies terrestrial; but the glory of the celestial is one, and the glory of the terrestrial is another. There is one glory of the sun, and another glory of the moon, and another glory of the stars; for one star differeth from another star in glory. So also is the resurrection of the dead: It is sown in corruption; it is raised in incorruption: It is sown in dishonour; it is raised in glory: It is sown in weakness; it is raised in power: It is sown a natural body; it is raised a spiritual body. There is a natural body, and there is a spiritual body. And so it is written, The first man Adam was made a living soul; the last Adam was made a quickening spirit. Howbeit, that was not first which is spiritual, but that which is natural; and afterward that which is spiritual. The first man is of the earth, earthy: the second man is the Lord from heaven. As is the earthy, such are they that are earthy: and as is the heavenly, such are they also that are heavenly. And as we have borne the image of the earthy, we shall also bear the image of the heavenly. 

Now this I say, brethren, that flesh and blood cannot inherit the kingdom of God; neither doth corruption inherit incorruption. Behold, I shew you a mystery: We shall not all sleep, but we shall all be changed, in a moment, in the twinkling of an eye, at the last trump, (for the trumpet shall sound,) and the dead shall be raised incorruptible, and we shall be changed. For this corruptible must put on incorruption, and this mortal must put on immortality. So when this corruptible shall have put on incorruption, and this mortal shall have put on immortality; then shall be brought to pass the saying that is written, Death is swallowed up in victory. O death, where is thy sting? O grave, where is thy victory? The sting of death is sin, and the strength of sin is the law. But thanks be to God, which giveth us the victory through our Lord Jesus Christ. Therefore, my beloved brethren, be ye stedfast, unmoveable, always abounding in the work of the Lord, forasmuch as ye know that your labour is not in vain in the Lord.}

% \begin{leftbar}
\centerline{\rubric{Or one of the following Lessons:}}
\centerline{2 Corinthians 4.~16-end. and 5.~1-10.}
\drop{For which cause we faint not; but though our outward man perish, yet the inward man is renewed day by day. For our light affliction, which is but for a moment, worketh for us a far more exceeding and eternal weight of glory; While we look not at the things which are seen, but at the things which are not seen: for the things which are seen are temporal; but the things which are not seen are eternal. For we know that if our earthly house of this tabernacle were dissolved, we have a building of God, an house not made with hands, eternal in the heavens. For in this we groan, earnestly desiring to be clothed upon with our house which is from heaven: If so be that being clothed we shall not be found naked. For we that are in this tabernacle do groan, being burdened: not for that we would be unclothed, but clothed upon, that mortality might be swallowed up of life. Now he that hath wrought us for the selfsame thing is God, who also hath given unto us the earnest of the Spirit. Therefore we are always confident, knowing that, whilst we are at home in the body, we are absent from the Lord: (For we walk by faith, not by sight:) We are confident, I say, and willing rather to be absent from the body, and to be present with the Lord. Wherefore we labour, that, whether present or absent, we may be accepted of him. For we must all appear before the judgment seat of Christ; that every one may receive the things done in his body, according to that he hath done, whether it be good or bad.}


\medskip
\centerline{Revelation 7.~9.}%-17  Proofed 12-17-21
\drop{After this I beheld, and, lo, a great multitude, which no man could number, of all nations, and kindreds, and people, and tongues, stood before the throne, and before the Lamb, clothed with white robes, and palms in their hands; and cried with a loud voice, saying, Salvation to our God which sitteth upon the throne, and unto the Lamb. And all the angels stood round about the throne, and about the elders, and the four beasts, and fell before the throne on their faces, and worshipped God, saying, Amen; Blessing, and glory, and wisdom, and thanksgiving, and honour, and power, and might, be unto our God for ever and ever. Amen. And one of the elders answered, saying unto me, What are these which are arrayed in white robes? and whence came they? And I said unto him, Sir, thou knowest. And he said to me, These are they which came out of great tribulation, and have washed their robes, and made them white in the blood of the Lamb. Therefore are they before the throne of God, and serve him day and night in his temple: and he that sitteth on the throne shall dwell among them. They shall hunger no more, neither thirst any more; neither shall the sun light on them, nor any heat. For the Lamb which is in the midst of the throne shall feed them, and shall lead them unto living fountains of waters: and God shall wipe away all tears from their eyes.}


\medskip
\centerline{Revelation 21.~1-7.}
\drop{And I saw a new heaven and a new earth: for the first heaven and the first earth were passed away; and there was no more sea. And I John saw the holy city, new Jerusalem, coming down from God out of heaven, prepared as a bride adorned for her husband. And I heard a great voice out of heaven saying, Behold, the tabernacle of God is with men, and he will dwell with them, and they shall be his people, and God himself shall be with them, and be their God. And God shall wipe away all tears from their eyes; and there shall be no more death, neither sorrow, nor crying, neither shall there be any more pain: for the former things are passed away. And he that sat upon the throne said, Behold, I make all things new. And he said unto me, Write: for these words are true and faithful. And he said unto me, It is done. I am Alpha and Omega, the beginning and the end. I will give unto him that is athirst of the fountain of the water of life freely. He that overcometh shall inherit all things; and I will be his God, and he shall be my son.}
% \end{leftbar}


% Scott has 3 others  more like gospels.


\medskip

\centerline{\rubric{Then the Minister shall say,}}
\centerline{Lord, have mercy upon us.}
\centerline{\emph{Christ, have mercy upon us.}}
\centerline{Lord, have mercy upon us.}

\medskip
\ourFather

\smallskip

\V Enter not into judgement with thy servant, O Lord; \R For in thy sight shall no man living be justified.

\V Grant unto \emph{him} eternal rest; \R And let perpetual light shine upon \emph{him}.

\V We believe verily to see the goodness of the Lord: \R In the land of the living.

\V O Lord, hear our prayer; \R And let our cry come unto thee.

{\centering\pilcrow{THen shall be said one or more of the following Prayers, the Minister first saying,}}

\centerline{Let us pray.}
\drop{Almighty God, with whom do live the spirits of them that depart hence in the Lord, and with whom the souls of the faithful, after they are delivered from the burden of the flesh, are in joy and felicity:
% We give thee hearty thanks, for that it hath pleased thee to deliver this our brother out of the miseries of this sinful world; beseeching thee, that it may please thee, of thy gracious goodness, shortly to accomplish the number of thine elect, and to hasten thy kingdom; that we, with all those that are departed in the true faith of thy holy Name, may have our perfect consummation and bliss, both in body and soul, in thy eternal and everlasting glory; through Jesus Christ our Lord. \R Amen.}
%1549
% \drop{O Lord, with whom do live the spirits of them that be dead; and in whom the souls of them that be elected, after they be delivered from the burden of the flesh, be in joy and felicity: 
Grant unto this thy servant, that the sins which \emph{he} committed in this world be not imputed unto \emph{him}, but that \emph{he}, escaping the gates of hell and the pains of eternal darkness, may ever dwell in the region of light, with Abraham, Isaac, and Jacob, in the place where is no weeping, sorrow, nor heaviness; and when that dreadful day of the general resurrection shall come, make \emph{him} to rise also with the just and righteous, and receive this body again to glory, then made pure and incorruptible: set \emph{him} on the right hand of thy Son Jesus Christ, among thy holy and elect, that then \emph{he} may hear with them these most sweet and comfortable words: Come, ye blessed children of my Father, receive the kingdom prepared for you from the beginning of the world: Grant this, we beseech thee, O merciful Father, through Jesus Christ, our Mediator and Redeemer. \R Amen.}

%This collect from the old mass, moved back with its original ending.
% The Collect
% \drop{O merciful God, the Father of our Lord Jesus Christ, who is the resurrection and the life; ... Resurrection in the last day,}

% we may be found acceptable in thy sight; and receive that blessing, which thy well-beloved Son shall then pronounce to all that love and fear thee, saying, 

% This from the original prayer above:
% Come, ye blessed children of my Father, receive the kingdom prepared for you from the beginning of the world: Grant this, we beseech thee, O merciful Father, through Jesus Christ, our Mediator and Redeemer. \R Amen.

% \begin{leftbar}
% 1923:
% BLESSED Lord, who art the Father of mercies and the God of all consolation: We beseech thee of thy great goodness, to comfort those who by the death of this our brother are sorely bereaved; and teach us so to number our days that, while we live here, we may seriously apply our hearts to that holy and heavenly wisdom, which may in the end bring us to life everlasting, through the merits of Jesus Christ, thine only Son our Lord. Amen.

\smallskip
% 1928
\subsection{\stylesubsec{}{The Collects.}{}}
\drop{O Father of all, we pray to thee for those whom we love, but see no longer. Grant them thy peace; let light perpetual shine upon them; and in thy loving wisdom and almighty power work in them the good purpose of thy perfect will; through Jesus Christ our Lord. \R Amen.}

\smallskip

\drop{Almighty God, Father of all mercies and giver of all comfort: Deal graciously, we pray thee, with those who mourn, that casting every care on thee, they may know the consolation of thy love; through Jesus Christ our Lord. \R Amen.}

\smallskip

\drop{O heavenly Father, who in thy Son Jesus Christ, hast given us a true faith, and a sure hope: Help us, we pray thee, to live as those who believe and trust in the Communion of Saints, the forgiveness of sins, and the resurrection to life everlasting, and strengthen this faith and hope in us all the days of our life: through the love of thy Son, Jesus Christ our Saviour. \R Amen.}

\smallskip

{\centering\footnotesize\rubric{Here may follow the Collect of \emph{All Saints’ Day}, or that of the \emph{Twelfth Sunday after Trinity}, or others from the \emph{Prayers upon Several Occasions.}}\par}
% \end{leftbar}

\medskip
\centerline{\rubric{And then shall be said,}}
\smallskip

\medskip
\theGrace

\smallskip
%Scottish
\drop{May the souls of the faithful departed through the mercy of God rest in peace. \R Amen.}


\section{The Communion}
\pilcrow{When there is a special celebration of the Holy Communion on the day of the Burial, the Priest shall use the Collect appointed in this Order, or the Collect of Easter Even, and for the Epistle}

\subsection{\stylesubsec{}{The Collect.}{}} %Moved back from above.
\drop{O merciful God, the Father of our Lord Jesu Christ, who is the resurrection and the life; in whom whosoever believeth shall live, though he die; and whosoever liveth, and believeth in him, shall not die eternally; who also hath taught us by his holy Apostle Paul, not to be sorry, as men without hope, for them that sleep in him: We meekly beseech thee, O Father, to raise us from the death of sin unto the life of righteousness; that, when we shall depart this life, we may sleep in him, as our hope is this our \emph{brother} doth: and at the general resurrection in the last day both we, and this our \emph{brother} departed, receiving again our bodies, and rising again in thy most gracious favour, may with all thine elect Saints obtain eternal joy.
Grant this, O Lord God, by the means of our Advocate Jesus Christ: which with thee and the Holy Ghost, liveth and reigneth one God for ever. \R Amen.}

\subsection{\stylesubsec{}{The Epistle.}{1 Thessalonians 4.~13}}
\drop{I would not have you to be ignorant, brethren, concerning them which are asleep, that ye sorrow not, even as others which have no hope. For if we believe that Jesus died and rose again, even so them also which sleep in Jesus will God bring with him. For this we say unto you by the word of the Lord, that we which are alive and remain unto the coming of the Lord shall not precede them which are asleep. For the Lord himself shall descend from heaven with a shout, with the voice of the archangel, and with the trump of God: and the dead in Christ shall rise first: then we which are alive and remain shall be caught up together with them in the clouds, to meet the Lord in the air: and so shall we ever be with the Lord. Wherefore comfort one another with these words.}

\begin{leftbar}
\centerline{\rubric{Or this.}}
\centerline{2 Corinthians 4.~16, \emph{and part of chapter 5.}}
‘Though our outward man. . . swallowed up of life’
\end{leftbar}


\subsection{\stylesubsec{}{The Gospel.}{John 6.~37.}}
\drop{Jesus said, All that the Father giveth me shall come to me; and him that cometh to me I will in no wise cast out. For I came down from heaven, not to do mine own will, but the will of him that sent me. And this is the Father's will which hath sent me, that of all which he hath given me I should lose nothing, but should raise it up again at the last day. And this is the will of him that sent me, that every one which seeth the Son, and believeth on him, may have everlasting life: and I will raise him up at the last day.}

\begin{leftbar}
\centerline{\rubric{Or this.}}
\centerline{St.~John 5.~24}
‘Jesus said, Verily, verily, . . . the resurrection of judgement’.
\end{leftbar}

\medskip

%scottish 1929
{\centering\footnotesize\rubric{At Holy Communion in connexion with burials or at memorials of the departed, if the Agnus Dei be sung or said for ‘\emph{have mercy upon us}’, and ‘\emph{grant us thy peace}’, substitute ‘\emph{grant them rest}’, and ‘\emph{grant them rest eternal}’.}\par}




\section{The Burial}
\begin{leftbar}
%1928 proposed
\pilcrow{If the ground be not consecrated, the Priest on coming to the grave may say the prayer following.}

\drop{O God, the Father of our Lord Jesus Christ, vouchsafe, we beseech thee, to \grealtcross\ bless this grave to be the peaceful resting-place of the body of thy servant; through the same thy blessed Son, who is the resurrection and the life, and who liveth and reigneth with thee and the Holy Ghost; one God, world without end. \R Amen.}

Or

scottish, 1929
BENEDICTION OF A GRAVE IN UNCONSECRATED GROUND (also south african)

\pilcrow{When the Priest and people shall have come to the place the Priest shall say,}

\centerline{Let us pray.}
\drop{O Lord Jesu Christ, who wast laid in the new tomb of Joseph, and didst thereby sanctify the grave to be a bed of hope to thy people: Vouchsafe, we beseech thee, to \grealtcross\ bless, \grealtcross\ hallow, and \grealtcross\ consecrate this grave, that it may be a resting-place, peaceful and secure, for the body of thy servant which we are about to commit to thy gracious keeping; who art the resurrection and the life, and who livest and reignest with the Father and the Holy Ghost, one God, world without end. \R Amen.}
\end{leftbar}


\pilcrow{When they come to the Grave, while the Corpse is made ready to be laid into the earth, the Priest shall say, or the Priest and Clerks shall sing:}
\drop{Man that is born of a woman hath but a short time to live, and is full of misery. He cometh up, and is cut down, like a flower; he fleeth as it were a shadow, and never continueth in one stay.}

\drop{In the midst of life we are in death: of whom may we seek for succour, but of thee, O Lord, who for our sins art justly displeased?}

Yet, O Lord God most holy, O Lord most mighty, O holy and most merciful Saviour, deliver us not into the bitter pains of eternal death.

Thou knowest, Lord, the secrets of our hearts; shut not thy merciful ears to our prayer; but spare us, Lord most holy, O God most mighty, O holy and merciful Saviour, thou most worthy judge eternal, suffer us not, at our last hour, for any pains of death, to fall from thee.

\medskip

\pilcrow{Then, while the earth shall be cast upon the Body by some standing by, the Priest shall say,}

% 1549
\drop{I commend thy soul to God the Father Almighty, and thy
% 1662
% \drop{Forasmuch as it hath pleased Almighty God of his great mercy to take unto himself the soul of our dear brother here departed, we therefore commit his}
body to the ground; earth to earth, ashes to ashes, dust to dust; in sure and certain hope of the Resurrection to eternal life, through our Lord Jesus Christ; who shall change our vile body, that it may be like unto his glorious body, according to the mighty working, whereby he is able to subdue all things to himself.}

\pilcrow{When this Order is used at the cremation of the body, in place of the words ‘\emph{commit his body to the ground, earth to earth, ashes to ashes, dust to dust}’ shall be said the words ‘\emph{commit his body to be consumed by fire}’: and in this case it shall suffice to say one or more of the \emph{\scshape Prayers} at the burial of the ashes.}

\rubric{When this Order is used at the burial of the body after cremation, in place of the words ‘\emph{commit his body to the ground, earth to earth, ashes to ashes, dust to dust}’ shall be said the words ‘\emph{commit his ashes to the ground, earth to earth, dust to dust}’, or ‘\emph{commit his ashes to their resting-place}’.}

\centerline{\pilcrow{Then shall be said or sung,}}
\drop{I heard a voice from heaven, saying unto me, Write, From henceforth blessed are the dead which die in the Lord: even so saith the Spirit: for they rest from their labours.}


\centerline{Let us pray.}
\drop{We commend into thy hands of mercy, most merciful Father, the soul of this, our \emph{brother} departed, \emph{N.} And \emph{his} body we commit to the earth, beseeching thine infinite goodness to give us grace to live in thy fear and love, and to die in thy favour: that when the judgment shall come, which thou hast committed to thy well-beloved Son, both this our \emph{brother}, and we, may be found acceptable in thy sight, and receive that blessing, which thy well-beloved Son shall then pronounce to all that love and fear thee, saying, Come, ye blessed children of my Father, receive the kingdom prepared for you from the beginning of the world: Grant this, O merciful Father, for the honour of Jesu Christ, our only Savior, Mediator, and Advocate. \R Amen.}

\centerline{\rubric{This Prayer shall also be added.}}
\drop{Almighty God, we give thee hearty thanks for this thy \emph{servant}, whom thou hast delivered from the miseries of this wretched world, from the body of death and all temptation, and, as we trust, hast brought \emph{his} soul, which \emph{he} committed into thy holy hands, into sure consolation and rest: Grant, we beseech thee, that at the day of judgment \emph{his} soul, and all the souls of thy elect, departed out of this life, may with us, and we with them, fully receive thy promises, and be made perfect altogether, through the glorious resurrection of thy Son Jesus Christ our Lord. \R Amen.}

\medskip
1923/1928
\nowUntoTheKing

%scottish
\begin{leftbar}    
    \theGrace
\end{leftbar}

\fleuron

% \centerline{\rule{0.5\textwidth}{0.5pt}}

\chapter[The Burial of a Child]{\stylechapter{An Order which may be used for}{The Burial of a Child}{}}


% \section{An Order which may be Used for the Burial of A [Baptized] Child}
\pilcrow{The Minister and Clerks meeting the body at the entrance of the church-yard, and going before it either into the church or towards the grave, shall say or sing,}

\section{The Procession}
\drop{I am the resurrection and the life, saith the Lord: he that believeth in me, though he were dead, yet shall he live: and whosoever liveth and believeth in me shall never die.\scripture{St.~John xi.~25, 26.}}

\drop{I know that my Redeemer liveth, and that he shalt stand at the latter day upon the earth. 
% And though after my skin worms destroy this body, %KJV
% hath been thus destroyed, %adapted from rsv
% earth: and though this body be destroyed, yet shall I see God: %am1928
% yet in my flesh shall I see God: 
Whom I shall see for myself, and mine eyes shall behold, and not another. \scripture{Job 19.~25, 27.}}

\drop{We brought nothing into this world, and it is certain we can carry nothing out. The {\scshape Lord} gave, and the {\scshape Lord} hath taken away; blessed be the Name of the {\scshape Lord}.\scripture{1 Timothy.~6.~7. Job 1.~21.}}

\drop{He shall feed his flock like a shepherd: he shall gather the lambs with his arm, and carry them in his bosom. \scripture{Isaiah 40.~11}}


\section{The Service in Church}

\centerline{\pilcrow{After they are come into the church shall be read this Psalm,}}

\subsection[{Psalm 23}]{\stylesubsec{Psalm 23.}{Dominus regit me.}{}}
\drop{The {\scshape Lord} is my shepherd; \star\ therefore can I lack nothing.}

2\enspace He shall feed me in a green pasture, \star\ and lead me forth beside the waters of comfort.

3\enspace He shall convert my soul, \star\ and bring me forth in the paths of righteousness, for his Name’s sake.

4\enspace Yea, though I walk through the valley of the shadow of death, I will fear no evil; \star\ for thou art with me; thy rod and thy staff comfort me.

5\enspace Thou shalt prepare a table before me against them that trouble me; \star\ thou hast anointed my head with oil, and my cup shall be full.

6\enspace But thy loving-kindness and mercy shall follow me all the days of my life; \star\ and I will dwell in the house of the Lord for ever.

Glory be to the Father, and to the Son,\ \star\ and to the Holy Ghost;

As it was in the beginning, is now, and ever shall be,\ \star\ world without end. Amen.

\medskip
\centerline{\pilcrow{Then shall follow this Lesson.}}

\centerline{St.~Mark 10.~13.}
\drop{They brought young children to him, that he should touch them: and his disciples rebuked those that brought them. But when Jesus saw it, he was much displeased, and said unto them, Suffer the little children to come unto me, and forbid them not; for of such is the kingdom of God. Verily I say unto you, Whosoever shall not receive the kingdom of God as a little child, he shall not enter therein. And he took them up in his arms, put his hands upon them, and blessed them.}
 
\medskip

\centerline{\pilcrow{Then the Minister shall say,}}
\centerline{Lord, have mercy upon us.}
\centerline{\emph{Christ, have mercy upon us.}}
\centerline{Lord, have mercy upon us.}

\medskip
\ourFather

\smallskip
\centerline{\pilcrow{The following Versicles and Responses may then be said:}}

\V Grant unto \emph{him} eternal rest; \R And let perpetual light shine upon \emph{him}.

\V We believe verily to see the goodness of the Lord; \R In the land of the living.

\V O Lord, hear our prayer; \R And let our cry come unto thee.

\pilcrow{Then shall be said one or both of the following prayers, the Minister first saying,}

\centerline{Let us pray.}
\drop{O Lord Jesu Christ, who didst take little children into thine arms and bless them:

[1928]Open thou our eyes, we beseech thee, to perceive that it is of thy goodness that thou hast taken this thy

[SA]Grant that in perfect confidence we may commit this 

child into the everlasting arms of thine infinite love; who livest and reignest with the Father and the Holy Spirit, ever one God, world without end. \R Amen.}

\smallskip
\drop{O God, whose ways are hidden and thy works most wonderful, who makest nothing in vain and lovest all that thou hast made: Comfort thou thy servants, whose hearts are sore smitten and oppressed; and grant that they may so love and serve thee in this life, that together with this thy child they may obtain the fulness of thy promises in the world to come; through Jesus Christ our Lord. \R Amen.}

\medskip
\theGrace
 
\section{The Burial}
\pilcrow{When they come to the grave, while the body is made ready to be laid into the earth, the Minister shall say, or the Minister and Clerks shall sing:}

\drop{Man that is born of a woman hath but a short time to live. He cometh up, and is cut down, like a flower; he fleeth as it were a shadow, and never continueth in one stay.}

% (these from SA)
\begin{leftbar}
While the child was yet alive I fasted and wept: for I said, Who can tell whether God will be gracious to me, that the child may live? But now he is dead, wherefore should I fast? Can I bring him back again? I shall go to him, but he shall not return to me.

A voice was heard in Ramah, lamentation, and bitter weeping; Rachel weeping for her children refused to be comforted for her children, because they were not.  Thus saith the Lord; Refrain thy voice from weeping, and thine eyes from tears: for thy work shall be rewarded, saith the Lord; and they shall come again from the land of the enemy.  And there is hope in thine end, saith the Lord, that thy children shall come again to their own border
\end{leftbar}


\medskip

\pilcrow{Then, while the earth shall be cast upon the body by some standing by, the Minister shall say,}
\drop{Forasmuch as it hath pleased Almighty God of his great mercy to take unto himself the soul of this child here departed, we therefore commit \emph{his} body to the ground; earth to earth, ashes to ashes, dust to dust; in sure and certain hope of the resurrection to eternal life, through our Lord Jesus Christ; who shall change the body of our low estate, that it may be like unto his glorious body, according to the mighty working, whereby he is able to subdue all things to himself.}

\centerline{\rubric{Or this.}}
\drop{We commend unto thy hands of mercy, most merciful Father, the soul of this thy child; and we commit \emph{his} body to the ground; earth to earth, ashes to ashes, dust to dust.  And we beseech thine infinite goodness to give us grace to live in thy fear and love, and to die in thy favour, that when the judgement shall come which thou hast comitted to thy well-beloved Son, both this child and we may be found acceptable in thy sight.  Grant this, O merciful Father, for the sake of Jesus Christ, our only Saviour, Mediator, and Advocate. \R Amen.}


\smallskip
\centerline{\pilcrow{Then shall be said or sung,}}

\drop{They shall hunger no more, neither thirst any more; neither shall the sun light on them, nor any heat. For the Lamb which is in the midst of the throne shall feed them, and shall lead them unto living fountains of waters: and God shall wipe away all tears from their eyes.}

\centerline{\pilcrow{Here shall be added by the Minister,}}

\nowUntoTheKing



\centerline{\rule{0.5\textwidth}{0.5pt}}
American 1928:
\drop{It is to be noted that this Office is appropriate to be used only for the faithful departed in Christ, provided that in any other case the Minister may, at his discretion, use such part of this Office, or such devotions taken from other parts of this Book, as may be fitting.}

 
\section{An Order which may be Used when the Prayer Book Service may not Be Used}
\pilcrow{The Priest, meeting the Corpse at the entrance to the Church-yard, and going before it towards the Grave, shall say,}

\pilcrow{The Minister shall await the body at the grave. When the body has been brought to the graveside, he shall say:}
 (Indian)
\subsection[{Psalm 130}]{\stylesubsec{Psalm 130.}{De profundis.}{}}

\drop{Out of the deep have I called unto thee, O {\scshape Lord}; \star\ Lord, hear my voice.}

2\enspace O let thine ears consider well \star\ the voice of my complaint.

3\enspace If thou, {\scshape Lord}, wilt be extreme to mark what is done amiss, \star\ O Lord, who may abide it?

4\enspace For there is mercy with thee; \star\ therefore shalt thou be feared.

5\enspace I look for the {\scshape Lord}; my soul doth wait for him; \star\ in his word is my trust.

6\enspace My soul fleeth unto the Lord \star\ before the morning watch, I say, before the morning watch.

7\enspace O Israel, trust in the {\scshape Lord}, for with the {\scshape Lord} there is mercy, \star\ and with him is plenteous redemption.

8\enspace And he shall redeem Israel \star\ from all his sins.

\medskip
\centerline{\pilcrow{When they come to the Grave shall be said,}}

% Domine, refugium. Psalm 90. 1-12.

% LORD, thou hast been our refuge : from one generation to another.
%     Before the mountains were brought forth, or ever the earth and the world were made : thou art God from everlasting, and world without end.
%     Thou turnest man to destruction : again thou sayest, Come again, ye children of men.
%     For a thousand years in thy sight are but as yesterday: seeing that is past as a watch in the night.
%     As soon as thou scatterest them, they are even as a sleep: and fade away suddenly like the grass.
%     In the morning it is green, and groweth up : but in the evening it is cut down, dried up, and withered.
%     For we consume away in thy displeasure: and are afraid at thy wrathful indignation.
%     Thou hast set our misdeeds before thee : and our secret sins in the light of thy countenance.
%     For when thou art angry all our days are gone : we bring our years to an end, as it were a tale that is told.
%     The days of our age are threescore years and ten; and though men be so strong, that they come to fourscore years : yet is their strength then but labour and sorrow; so soon passeth it away, and we are gone.
%     But who regardeth the power of thy wrath : for even thereafter as a man feareth, so is thy displeasure.
%     So teach us to number our days : that we may apply our hearts unto wisdom.

%     SA and Indian skip here.

\centerline{\pilcrow{Then may be read,}}
\centerline{St.~John 5.~25.}
\drop{Jesus said, Verily, verily, I say unto you, The hour is coming, and now is, when the dead shall hear the voice of the Son of God: and they that hear shall live. For as the Father hath life in himself; so hath he given to the Son to have life in himself; and hath given him authority to execute judgement also, because he is the Son of man. Marvel not at this: for the hour is coming, in the which all that are in the graves shall hear his voice, and shall come forth; they that hath done good, unto the resurrection of life; and they that hath done evil, unto the resurrection of judgement.}

% SA, Indian: Anthems here.
\drop{Man that is born of a woman hath but a short time to live, and is full of misery. He cometh up, and is cut down, like a flower; he fleeth as it were a shadow, and never continueth in one stay.}

\drop{In the midst of life we are in death: of whom may we seek for succour, but of thee, O Lord, who for our sins art justly displeased?}

Yet, O Lord God most holy, O Lord most mighty, O holy and most merciful Saviour, deliver us not into the bitter pains of eternal death.

Thou knowest, Lord, the secrets of our hearts; shut not thy merciful ears to our prayer; but spare us, Lord most holy, O God most mighty, O holy and merciful Saviour, thou most worthy judge eternal, suffer us not, at our last hour, for any pains of death, to fall from thee.

\medskip
\pilcrow{When the Corpse has been laid in the Grave the Priest shall say,}
SA, : 
\drop{We commit the body of our dear \emph{brother} to the ground; earth to earth, ashes to ashes, dust to dust; and we commend \emph{his} soul to the just and merciful judgement of him who alone hath perfect understanding, even Jesus Christ our Lord. \R Amen.}



\centerline{\rubric{Then the Minister shall say,}}
\centerline{Lord, have mercy upon us.}
\centerline{\emph{Christ, have mercy upon us.}}
\centerline{Lord, have mercy upon us.}

\medskip
\ourFather


Ant. Remember not, Lord, our offenses.. (from litany)

\V O Lord, deal not with us after our sins. \R Neither reward us after our iniquities.

\drop{We humbly beseech thee, O Father, mercifully to look upon our infirmities; and for the glory of thy Name turn from us all those evils that we most righteously have deserved; and grant, that in all our troubles we may put our whole trust and confidence in thy mercy, and evermore serve thee in holiness and pureness of living, to thy honour and glory; through our only Media tor and Advocate, Jesus Christ our Lord. Amen.}


Collect: Almighty God, give us grace that we may

\drop{Almighty God, the fountain of all wisdom, who knowest our necessities before we ask, and our ignorance in asking: We beseech thee to have compassion upon our infirmities; and those things, which for our unworthiness we dare not, and for our blindness we cannot ask, vouchsafe to give us, for the worthiness of thy Son Jesus Christ our Lord. \R Amen.}

SA: 
\drop{Almighty God, Father of all mercies and giver of all comfort: Deal graciously, we pray thee, with those who mourn, that casting every care on thee, they may know the consolation of thy love; through Jesus Christ our Lord. \R Amen.}

but indian: 
\drop{O God, whose ways are hidden and thy works past understanding, who makest nothing in vain and lovest all that thou hast made: Deal graciously, we pray thee, with those who mourn because of this bereavement, that casting every care on thee, they may know the consolation of thy love; through Jesus Christ our Saviour. \R Amen.}

Indian: Fountain of all wisdom
Sa: \drop{O Saviour of the world, who by thy Cross and precious Blood hast redeemed us, Save us and help us, we humbly beseech thee, O Lord.}

\medskip
\centerline{\rubric{And then shall be said,}}
\theGrace

 

 
\centerline{\rule{0.5\textwidth}{0.5pt}}

 
\section{An Order for the Burial of an Unbaptized Child}
\pilcrow{On the way to the Grave the following sentences may be said,}

\drop{God made not death: neither delighteth he when the living perish. He created man for incorruption: and made him an image of his own proper being.\scripture{Wisdom of Solomon 1.~13, 2.~23}}

\drop{Despair not then, seeing that thou art far from being able to love his creature more than he. For as his majesty is, so also is his mercy. \scripture{2 Esdras 8.~47; Ecclesiasticus 2.~18}}

\drop{He shall feed his flock like a shepherd, he shall gather the lambs with his arm, and carry them in his bosom. \scripture{Isaiah 40.~11}}

\drop{The Lord gave, and the Lord hath taken away; blessed be the name of the Lord.\scripture{Job 1.~21}}
(SA omits first)
(Indian is only the 4th and 3rd)

\medskip
\centerline{\pilcrow{When they come to the Grave shall be said,}}
% SA: Psalm 121


\subsection[{Psalm 23}]{\stylesubsec{Psalm 23.}{Dominus regit me.}{}}
\drop{The {\scshape Lord} is my shepherd; \star\ therefore can I lack nothing.}

2\enspace He shall feed me in a green pasture, \star\ and lead me forth beside the waters of comfort.

3\enspace He shall convert my soul, \star\ and bring me forth in the paths of righteousness, for his Name’s sake.

4\enspace Yea, though I walk through the valley of the shadow of death, I will fear no evil; \star\ for thou art with me; thy rod and thy staff comfort me.

5\enspace Thou shalt prepare a table before me against them that trouble me; \star\ thou hast anointed my head with oil, and my cup shall be full.

6\enspace But thy loving-kindness and mercy shall follow me all the days of my life; \star\ and I will dwell in the house of the Lord for ever.

(SA: Psalm 121)

\medskip
\centerline{\pilcrow{Then may be read one of the following,}}

Indian: Jeremiah 31:15
Thus saith the Lord: A voice is heard in Ramah...again to their own border.
Or baruch.


\centerline{St.~Matthew 18.~10.}
\drop{Take heed that ye despise not one of these little ones; for I say unto you, That in heaven their angels do always behold the face of my Father which is in heaven. For the Son of man is come to save that which was lost. How think ye? if a man have an hundred sheep, and one of them be gone astray, doth he not leave the ninety and nine, and goeth into the mountains, and seeketh that which is gone astray? And if so be that he find it, verily I say unto you, he rejoiceth more of that sheep, than of the ninety and nine which went not astray. Even so it is not the will of your Father which is in heaven, that one of these little ones should perish.}


\centerline{Baruch 4.~19.}
\drop{Go your way, my children, go your way: for I am left desolate. I have put off the clothing of peace, and put upon me the sack-cloth of my prayer: I will cry unto the Everlasting in my days. Be of good cheer, my children, cry unto the Lord, and he shall deliver you from the power and hand of the enemies. For my hope is in the Everlasting, that he will save you; and joy is come unto me from the Holy One, because of the mercy which shall soon come unto you from the Everlasting our Saviour. For I sent you out with mourning and weeping: but God will give you to me again with joy and gladness for ever.}

\medskip
\centerline{\pilcrow{As the body is being laid in the Grave shall be said,}}
\drop{Unto God’s loving mercy we commit this child, that he may grant \emph{him} a share in the unsearchable riches of the redemption wrought by his Son, our Lord and Saviour Jesus Christ. \R Amen.}


\medskip

\centerline{\rubric{Then the Minister shall say,}}
\centerline{Lord, have mercy upon us.}
\centerline{\emph{Christ, have mercy upon us.}}
\centerline{Lord, have mercy upon us.}

\medskip
\ourFather

\medskip

\centerline{Let us pray.}
\drop{O God, whose ways are hidden and thy works most wonderful, who makest nothing in vain and lovest all that thou hast made: Comfort thou thy servants, whose hearts are sore smitten and oppressed; and grant that they may so love and serve thee in this life, that together with this thy child they may obtain the fulness of thy promises in the world to come; through Jesus Christ our Lord. \R Amen.}

Indian: Numerous collects.
\medskip
\centerline{\rubric{And then shall be said,}}
\theGrace

\chapter[The Churching of Women]{\stylechapter{The Thanksgiving of Women after Child-Birth\\ {\small commonly called}}{The Churching of Women}{}}

\pilcrow{The woman, at the usual time after her delivery, shall come into the Church decently apparelled, and there shall kneel down in some convenient place, as hath been accustomed, or as the Ordinary shall direct: And then the Priest shall say unto her,}

\drop{Forasmuch as it hath pleased Almighty God of his goodness to give you safe deliverance, and hath preserved you in the great danger of child-birth; you shall therefore give hearty thanks unto God, and say,}


\bigskip
% 1549: Levavi oculos. Psalm cxxi., 1559

% 1928 
\pilcrow{Then shall be said by both of them the following Psalm, the woman still kneeling.}

%The following psalm has some verses changed from the psalter.
%(Then shall the Priest say the 116th Psalm.)

\subsection[{Psalm 116}]{\stylesubsec{Psalm 116.}{Dilexi, quoniam.}{}}
\drop{I am wéll pléased \star\  that the {\scshape Lord} hath heard the  vóice of mý prayer;}

2\enspace That he hath inclined his éar  unto mé; \star\  therefore will I call upon him as  lóng as Í live.

3\enspace The snares of death compassed me  róund abóut, \star\  and the pains of hell gat  hóld upón me.

4\enspace I \emph{found} trouble and heaviness, and I \emph{called} upon the Náme  of the {\scshape Lórd}; \star\  O {\scshape Lord}, I beseech thee, delíver mý soul.

5\enspace Gracious is the  {\scshape Lórd}, and ríghteous; \star\ yea, our Gód is mérciful.

6\enspace The {\scshape Lord} presérveth the símple: \star\ I was in misery, ánd he hélped me.

7\enspace Turn again then unto thy rést,  O my sóul; \star\ for the {\scshape Lórd} hath rewárded thee.

8\enspace And why? thou hast delivered my  sóul from déath, \star\ mine eyes from tears, and my féet from fálling.

9\enspace I will wálk before the {\scshape Lórd} \star\  in the lánd  of the líving.

10\enspace I believed, and therefore will I speak; but Í  was sore tróubled: \star\  I said in my haste, All mén are líars.

11\enspace What reward shall I gíve unto the {\scshape Lórd} \star\  for all the benefits that hé hath done únto me?

12\enspace I will receive the cúp of salvátion, \star\  and call upon the Náme of thé {\scshape Lord}.

13\enspace I will pay my vows now in the presence of áll his péople: \star\   in the courts of the {\scshape Lord}’s house, even in the midst of thee, O Jerúsalem. Práise the {\scshape Lord}.

Glory be to the Father, and to the Son, \star\  and to the Holy Ghost;

As it was in the beginning, is now, and ever shall be, \star\  world without end. Amen.

\medskip
\centerline{\rubric{Or,}}
\subsection[{Psalm 127}]{\stylesubsec{Psalm 127.}{Nisi Dominus.}{}}
\drop{Except the {\scshape Lord} búild the hóuse,  \star\  their labour is but lóst that búild it.}

2\enspace Except the {\scshape Lord} kéep the cíty,  \star\  the watchman wáketh bút in vain.

3\enspace It is but lost labour that ye haste to rise up early, and so late take rest, and eat the bréad of cárefulness;  \star\  for so he giveth hís belóved sleep.

4\enspace Lo, children and the frúit of the wómb,  \star\  are an heritage and gift that cómeth óf the {\scshape Lord}.

5\enspace Like as the arrows in the hánd of the gíant, * even so are the yöung chíldren.

6\enspace Happy is the man that hath his quíver fúll of them; * they shall not be ashamed when they speak with their én·emies ín the gate.

Glory be to the Father, and to the Son, \star\  and to the Holy Ghost;

As it was in the beginning, is now, and ever shall be, \star\  world without end. Amen.

\medskip
\centerline{\pilcrow{Then the Priest shall say,}}


\centerline{Lord, have mercy upon us.}
\centerline{\emph{Christ, have mercy upon us.}}
\centerline{Lord, have mercy upon us.}

\smallskip
\ourFather

\V O Lord, save this woman thy servant; \R Who putteth her trust in thee.

\V Be thou to her a strong tower;  \R From the face of her enemy.

\V Lord, hear our prayer.  \R And let our cry come unto thee.

\centerline{Let us pray.}
\drop{O Almighty God, we give thee humble thanks for that thou hast vouchsafed to deliver this woman thy servant from the great pain and peril of child-birth; Grant, we beseech thee, most merciful Father, that she, through thy help, may both faithfully live, and walk according to thy will, in this life present; and also may be partaker of everlasting glory in the life to come; through Jesus Christ our Lord. \R Amen.}

% 1923 adds only the following; 1928 adds the following two.
\begin{leftbar}
\bigskip
\pilcrow{Then, if there be no Communion at the time of the Churching, shall the Priest say to the Woman,}
\drop{Unto God’s gracious mercy and protection we commit thee. The {\scshape Lord} bless thee, and keep thee. The {\scshape Lord} make his face to shine upon thee, and be gracious unto thee. The {\scshape Lord} lift up his countenance upon thee, and give thee peace, both now and evermore. \R Amen.}




\bigskip
\pilcrow{Prayers which may be used at the discretion of the Priest before the Blessing.}
\drop{O God, our heavenly Father, we thank thee and praise thy glorious name, that thou hast been pleased to bless this woman thy servant, and to bestow upon her the gift of a child: Grant, we beseech thee, most merciful Father, that she [with her husband] may diligently lead her child in the way of righteousness, to their own great blessing and the glory of thy name; through Jesus Christ our Lord. \R Amen.}

\medskip
\drop{O God, whose ways are hidden and thy works most wonderful, who makest nothing in vain, and lovest all that thou hast made: Comfort this thy servant whose heart is sore sitten and oppressed; and grant that she may so love and serve thee in this life, that she may obtain the fulness of thy promises in the world to come; through Jesus Christ our Lord. \R Amen.}
\end{leftbar}

\pilcrow{The woman, that cometh to give her thanks, must offer accustomed offerings; and, if there be a Communion, it is convenient that she receive the Holy Communion.}


\fleuron

\chapter[A Commination]{\stylechapter{}{A Commination}{or Denouncing of God’s Anger and Judgements against Sinners\\ }}
{\centering\footnotesize\emph{With certain Prayers, to be used on the first Day of Lent, and at other times}\par}
%, as the Ordinary shall appoint

% "General Sentence" - warning of excommunication, 4 times a year.


\medskip
\pilcrow{After Morning Prayer, the Litany ended according to the accustomed manner, the Priest shall, in the Reading-Pew or Pulpit, say,}

\drop{Brethren, in the Primitive Church there was a godly discipline, that, at the beginning of Lent, such persons as stood convicted of notorious sin were put to open penance, and punished in this world, that their souls might be saved in the day of the Lord; and that others, admonished by their example, might be the more afraid to offend.}

Wherefore, lest by disuse of the said discipline God’s judgement upon sin be lightly regarded, it is thought good, that at this time (in the presence of you all) it should be declared that God will surely judge them that transgress his holy Commandments; and that ye, imploring his mercy, should answer \emph{Amen} in token that ye assent and submit to his righteous condemnation: To the intent that, being admonished of the great indignation of God against sinners, ye may the rather be moved to earnest and true repentance; and may walk more warily in these dangerous days; fleeing from such vices, for which ye affirm with your own mouths the judgement of God to be due.

% \drop{Cursed is the man that maketh any carved or molten image, to worship it.}

% \centerline{\rubric{And the people shall answer and say,}  Amen.}

% Cursed is he that curseth his father or mother. \R Amen.

% Cursed is he that removeth his neighbour’s landmark. \R Amen.

% Cursed is he that maketh the blind to go out of his way. \R Amen.

% Cursed is he that perverteth the judgement of the stranger, the fatherless, and widow. \R Amen.

% Cursed is he that smiteth his neighbour secretly. \R Amen.

% Cursed is he that lieth with his neighbour’s wife. \R Amen.

% Cursed is he that taketh reward to slay the innocent. \R Amen.

% Cursed is he that putteth his trust in man, and taketh man for his defence, and in his heart goeth from the Lord. \R Amen.

% Cursed are the unmerciful, fornicators, and adulterers, covetous persons, idolaters, slanderers, drunkards, and extortioners. \R Amen.

% \medskip
% \centerline{\rubric{Or,}}
% \medskip

\drop{The Lord our God is one Lord: them that serve other gods, God shall judge.}

\centerline{\pilcrow{And the people shall answer and say,}}
\centerline{\R Amen.  Lord have mercy upon us.}

II. Idolaters and all them that worship God’s creatures, God shall judge;

\centerline{\R Amen.  Lord have mercy upon us.}

III. Blasphemers and all them that take God’s name in vain, God shall judge; 

\centerline{\R Amen.  Lord have mercy upon us.}

IV. The Lord’s day is holy; them that profane it, God shall judge;

\centerline{\R Amen.  Lord have mercy upon us.}

V. Him that honoureth not his father or his mother, and them that are lawless or seditious, God shall judge;

\centerline{\R Amen.  Lord have mercy upon us.}

VI. Murderers and all them that are malicious or cruel, God shall judge;

\centerline{\R Amen.  Lord have mercy upon us.}

VII. Adulterers and fornicators and all unclean persons, God shall judge;

\centerline{\R Amen.  Lord have mercy upon us.}

VIII. Robbers and thieves and them that defraud, God shall judge;

\centerline{\R Amen.  Lord have mercy upon us.}

IX. False witnesses and all evil speakers, liars and slanderers, God shall judge;

\centerline{\R Amen.  Lord have mercy upon us.}

X. Covetous persons and extortioners and them that grind the faces of the poor, God shall judge;

\R Amen. Lord, have mercy upon us, and lay not these sins to our charge.


\centerline{\rubric{Minister.}}

% \drop{Now seeing that all they are accursed (as the prophet David beareth witness) who do err and go astray from the commandments of God; let us (remembering the dreadful judgement hanging over our heads, and always ready to fall upon us) return unto our Lord God, with all contrition and meekness of heart; bewailing and lamenting our sinful life, acknowledging and confessing our offences, and seeking to bring forth worthy fruits of penance. For now is the axe put unto the root of the trees, so that every tree that bringeth not forth good fruit is hewn down, and cast into the fire. It is a fearful thing to fall into the hands of the living God: he shall pour down rain upon the sinners, snares, fire and brimstone, storm and tempest; this shall be their portion to drink. For lo, the Lord is come out of his place to visit the wickedness of such as dwell upon the earth. But who may abide the day of his coming? Who shall be able to endure when he appeareth? His fan is in his hand, and he will purge his floor, and gather his wheat into the bam; but he will burn the chaff with unquenchable fire. The day of the Lord cometh as a thief in the night: and when men shall say, Peace, and all things are safe, then shall sudden destruction come upon them, as sorrow cometh upon a woman travailing with child, and they shall not escape. Then shall appear the wrath of God in the day of vengeance, which obstinate sinners, through the stubbornness of their heart, have heaped unto them, selves; which despised the goodness, patience, and long, sufferance of God, when he calleth them continually to repentance. Then shall they call upon me, (saith the Lord,) but I will not hear; they shall seek me early, but they shall not find me; and that, because they hated knowledge, and received not the fear of the Lord, but abhorred my counsel, and despised my correction. Then shall it be too late to knock when the door shall be shut; and too late to cry for mercy when it is the time of justice. O terrible voice of most just judgement, which shall be pronounced upon them, when it shall be said unto them, Go, ye cursed, into the fire everlasting, which is prepared for the devil and his angels. Therefore, brethren, take we heed betime, while the day of salvation lasteth; for the night cometh, when none can work. But let us, while we have the light, believe in the light, and walk as children of the light; that we be not cast into utter darkness, where is weeping and gnashing of teeth. Let us not abuse the goodness of God, who calleth us mercifully to amendment, and of his endless pity promiseth us forgiveness of that which is past, if with a perfect and true heart we return unto him. For though our sins be as red as scarlet, they shall be made white as snow; and though they be like purple, yet they shall be made white as wool. Turn ye (saith the Lord) from all your wickedness, and your sin shall not be your destruction: Cast away from you all your ungodliness that ye have done: Make you new hearts, and a new spirit: Wherefore will ye die, O ye house of Israel, seeing that I have no pleasure in the death of him that dieth, saith the Lord God? Tom ye then, and ye shall live. Although we have sinned, yet have we an Advocate with the Father, Jesus Christ the righteous; and he is the propitiation for our sins. For he was wounded for our offences, and smitten for our wickedness. Let us therefore return unto him, who is the merciful receiver of all true penitent sinners; assuring ourselves that he is ready to receive us, and most willing to pardon us, if we come unto him with faithful repentance; if we submit ourselves unto him, and from henceforth walk in his ways; if we will take his easy yoke, and light burden upon us, to follow him in lowliness, patience, and charity, and be ordered by the governance of his Holy Spirit; seeking always his glory, and serving him duly in our vocation with thanksgiving: This if we do, Christ will deliver us from the curse of the law, and from the extreme malediction which shall light upon them that shall be set on the left hand; and he will set us on his right hand, and give us the gracious benediction of his Father, commanding us to take possession of his glorious kingdom: Unto which he vouchsafe to bring us all, for his infinite mercy. Amen.}

% \centerline{Here shall follow The Prayers}
%from the proposed 1928:
% \section{An Alternative Commination}
% 


\drop{Now seeing that all they are condemned who do err and go astray from the commandments of God; let us (remembering the dreadful judgement hanging over our heads, and always ready to fall upon us) return unto our Lord God, with all contrition and meekness of heart; bewailing and lamenting our sinful life, acknowledging and confessing our offences, and seeking to bring forth worthy fruits of penance.  For it is a fearful thing to fall into the hands of the living God and to hear the terrible voice of his most just judgement which shall be pronounced upon obstinate sinners when it shall be said unto them, Go, ye cursed, into the fire everlasting, which is prepared for the devil and his angels.}
Therefore, brethren, take we heed betime, while the day of salvation lasteth.
% \begin{leftbar}
% For now is the axe put unto the root of the trees, so that every tree that bringeth not forth good fruit is hewn down, and cast into the fire. 
% It is a fearful thing to fall into the hands of the living God: he shall pour down rain upon the sinners, snares, fire and brimstone, storm and tempest; this shall be their portion to drink. For lo, the Lord is come out of his place to visit the wickedness of such as dwell upon the earth. But who may abide the day of his coming? Who shall be able to endure when he appeareth? His fan is in his hand, and he will purge his floor, and gather his wheat into the barn; but he will burn the chaff with unquenchable fire. The day of the Lord cometh as a thief in the night: and when men shall say, Peace, and all things are safe, then shall sudden destruction come upon them, as sorrow cometh upon a woman travailing with child, and they shall not escape. Then shall appear the wrath of God in the day of vengeance, which obstinate sinners, through the stubbornness of their heart, have heaped unto themselves; which despised the goodness, patience, and long-sufferance of God, when he calleth them continually to repentance. Then shall they call upon me, (saith the Lord,) but I will not hear; they shall seek me early, but they shall not find me; and that, because they hated knowledge, and received not the fear of the Lord, but abhorred my counsel, and despised my correction. Then shall it be too late to knock when the door shall be shut; and too late to cry for mercy when it is the time of justice. O terrible voice of most just judgement, which shall be pronounced upon them, when it shall be said unto them, Go, ye cursed, into the fire everlasting, which is prepared for the devil and his angels.
% Therefore, brethren, take we heed betime, while the day of salvation lasteth; for the night cometh, when none can work. But let us, while we have the light, believe in the light, and walk as children of the light; that we be not cast into utter darkness, where is weeping and gnashing of teeth. Let us not abuse the goodness of God, who calleth us mercifully to amendment, and of his endless pity promiseth us forgiveness of that which is past, if with a perfect and true heart we return unto him. For though our sins be as red as scarlet, they shall be made white as snow; and though they be like purple, yet they shall be made white as wool. Turn ye (saith the Lord) from all your wickedness, and your sin shall not be your destruction: Cast away from you all your ungodliness that ye have done: Make you new hearts, and a new spirit: Wherefore will ye die, O ye house of Israel, seeing that I have no pleasure in the death of him that dieth, saith the Lord God? Turn ye then, and ye shall live. 
% \end{leftbar}
Although we have sinned, yet have we an Advocate with the Father, Jesus Christ the righteous; and he is the propitiation for our sins. For he was wounded for our offences, and smitten for our wickedness. Let us therefore return unto him, who is the merciful receiver of all true penitent sinners; assuring ourselves that he is ready to receive us, and most willing to pardon us, if we come unto him with faithful repentance; if we submit ourselves unto him, and from henceforth walk in his ways; if we will take his easy yoke, and light burden upon us, to follow him in lowliness, patience, and charity, and be ordered by the governance of his Holy Spirit; seeking always his glory, and serving him duly in our vocation with thanksgiving. This if we do, Christ will deliver us from the extreme malediction which shall light upon them that shall be set on the left hand; and he will set us on his right hand, and give us the gracious benediction of his Father, commanding us to take possession of his glorious kingdom: Unto which he vouchsafe to bring us all, for his infinite mercy. Amen.


\medskip

% The Supplement
% to the Indian Book of Common Prayer
\begin{leftbar}
\stylesec{The Form for the Blessing of Ashes}{on}{Ash Wednesday}
% \section{The Blessing of Ashes}
% \pilcrow{Before the Lord’s Supper ashes prepared from the palms blessed the previous Palm Sunday, or other suitable ashes, may be blessed as follows:}

\pilcrow{Ashes prepared from the palms blessed the previous Palm Sunday, or other suitable ashes, shall be placed in a vessel near the holy Table; and the Priest, standing at the Epistle side, shall say,}
\smallskip
\V The Lord be with you.  \R And with thy spirit.

\centerline{Let us pray.}
\drop{O God, our faithful Creator, who wouldest not the death of a sinner, but rather that he should turn from his wickedness, and live: Look with mercy upon the frailty of our mortal nature; and of thy goodness vouchsafe to \grealtcross\ bless these ashes which are now to be set upon our heads as a token of humility and of sorrow for our sins. We acknowledge that we are but dust and ashes, and that, by reason of our offences, unto dust we shall return; yet we beseech thy mercy to grant the forgiveness of all our sins and the pardon which thou hast promised to all who truly repent and believe in thy Son; who with thee and the Holy Spirit, liveth and reigneth, one God, world without end. \R Amen.}

\centerline{\rubric{Here may the ashes be sprinkled and censed.}}

\bigskip
\pilcrow{Then shall the Priest put ashes on his own forehead, or if there be another Priest present, he shall put the ashes on the officiant’s forehead; after which the people shall kneel at the Communion rail and the Priest shall put the ashes on their foreheads. During the imposition \emph{Psalm 25} may be said or sung, or some suitable Lenten hymn may be sung.}

\medskip

\centerline{\rubric{The Priest shall say to each person, as the ashes are imposed:}}

Remember, O man, that dust thou art, and unto dust shalt thou return.

\centerline{\rubric{Or}}

Remember that thou art a sinner, and repent.
\end{leftbar}

\medskip

\section{The Prayers}
\pilcrow{Then shall they all kneel upon their knees, and the Priest and Clerks kneeling (in the place where they are accustomed to say the Litany) shall say this Psalm.}

\subsection{\stylesubsec{Psalm 51.}{Miserere mei, Deus.}{}}
\drop{Have mercy upon me, O God, after thy great goodness;\ \star\ according to the multitude of thy mercies do away mine offences.}

2\enspace Wash me throughly from my wickedness,\ \star\ and cleanse me from my sin.

3\enspace For I knowledge my faults,\ \star\ and my sin is ever before me.

4\enspace Against thee only have I sinned, and done this evil in thy sight;\ \star\ that thou mightest be justified in thy saying, and clear when thou art judged.

5\enspace Behold, I was shapen in wickedness,\ \star\ and in sin hath my mother conceived me.

6\enspace But lo, thou requirest truth in the inward parts,\ \star\ and shalt make me to understand wisdom secretly.

7\enspace Thou shalt purge me with hyssop, and i shall be clean;\ \star\ thou shalt wash me, and I shall be whiter than snow.

8\enspace Thou shalt make me hear of joy and gladness,\ \star\ that the bones which thou hast broken may rejoice.

9\enspace Turn thy face from my sins,\ \star\ and put out all my misdeeds.

10\enspace Make me a clean heart, O God,\ \star\ and renew a right spirit within me.

11\enspace Cast me not away from thy presence,\ \star\ and take not thy holy Spirit from me.

12\enspace O give me the comfort of thy help again,\ \star\ and stablish me with thý free Spirit.

13\enspace Then shall I teach thy ways únto the wicked,\ \star\ and sinners shall be converted únto thee.

14\enspace Deliver me from blood-guiltiness, O God, thou that art the God of my health;\ \star\ and my tongue shall sing of thy righteousness.

15\enspace Thou shalt open my lips, O Lord,\ \star\ and my mouth shall shew thy praise.

16\enspace For thou desirest no sacrifice, else would I give it thee;\ \star\ but thou delightest not in burnt-offerings.

17\enspace The sacrifice of God is a troubled spirit:\ \star\ a broken and contrite heart, O God, shalt thou not despise.

18\enspace O be favourable and gracious únto Sion;\ \star\ build thou the walls of Hierúsalem.

19\enspace Then shalt thou be pleased with the sacrifice of righteousness, with the burnt-offerings and oblations;\ \star\ then shall they offer young bullocks upon thine altar.

Glory be to the Father, and to the Son,\ \star\ and to the Holy Ghost;

As it was in the beginning, is now, and ever shall be,\ \star\ world without end. Amen.

\medskip

\centerline{Lord, have mercy upon us.}
\centerline{\emph{Christ, have mercy upon us.}}
\centerline{Lord, have mercy upon us.}

\smallskip
\ourFather


\V O Lord, save thy servants; \R That put their trust in thee.

\V Send unto them help from above; \R And evermore mightily defend them.

\V Help us, O God our Saviour. \R And for the glory of thy Name deliver us; be merciful to us sinners, for thy name’s sake.

\V O Lord, hear our prayer; \R And let our cry come unto thee.

\centerline{Let us pray.}
\drop{O Lord, we beseech thee, mercifully hear our prayers, and spare all those who confess their sins unto thee; that they, whose consciences by sin are accused, by thy merciful pardon may be \grecross\ absolved; through Christ our Lord. \R Amen.}

\smallskip

\drop{O most mighty God, and merciful Father, who hast compassion upon all men, and hatest nothing that thou hast made; who wouldest not the death of a sinner, but that he should rather turn from his sin, and be saved: Mercifully forgive us our trespasses; receive and comfort us, who are grieved and wearied with the burden of our sins. Thy property is always to have mercy; to thee only it appertaineth to forgive sins. Spare us therefore, good Lord, spare thy people, whom thou hast redeemed; enter not into judgement with thy servants, who are vile earth, and miserable sinners; but so turn thine anger from us, who meekly acknowledge our vileness, and truly repent us of our faults, and so make haste to help us in this world, that we may ever live with thee in the world to come; through Jesus Christ our Lord. \R Amen.}


\medskip

\pilcrow{Then shall the people say this Anthem that followeth, after the Minister.}
\drop{Turn thou us, O good Lord, and so shall we be turned. Be favourable, O Lord, \Be favourable to thy people, \ Who turn to thee in weeping, fasting, and praying. For thou art a merciful God, \ Full of compassion, long-suffering, and of great pity. Thou sparest when we deserve punishment, \ And in thy wrath thinkest upon mercy. Spare thy people, good Lord, spare them, \ And let not thine heritage be brought to confusion. Hear us, O Lord, for thy mercy is great, \ And after the multitude of thy mercies look upon us; \ Through the merits and mediation of thy blessed Son, Jesus Christ our Lord. Amen.}

\medskip

\centerline{\pilcrow{Then the Minister alone shall say,}}
\drop{The {\scshape Lord} \cross bless us, and keep us: the {\scshape Lord} make his face to shine upon us, and be gracious unto us: the {\scshape Lord} lift up the light of his countenance upon us, and give us peace, now and for evermore. \R Amen.}

\fleuron
% \chapter{Blessings}

\section{The order for the Conjuring of Water}

\smallskip
\pilcrow{The Priest prepares the salt as follows,}

\drop{I exorcize thee, creature of salt, by the living \grealtcross\ God, by the holy \grealtcross\ God, by the omnipotent \grealtcross\  God, that thou mayest be purified from all evil influence, in the Name of Him who is Lord of Angels and of men, who filleth the whole earth with his majesty and glory.  \R Amen.}

\drop{We pray thee, O God, in thy boundless lovingkindness to stretch forth the right hand of thy power upon this creature of salt which we \grealtcross\ bless and \grealtcross\ hallow in thy holy Name.  Grant that this salt may make for health of mind and body to all who partake thereof, and that there may be banished from the place where it is used every power of adversity and every illusion or artifice of evil; through Christ our Lord.  \R Amen.}

\smallskip

\pilcrow{The Priest prepares the water as follows,}

\drop{I exorcize thee, creature of water, by the living \grealtcross\ God, by the holy \grealtcross\ God, by the omnipotent \grealtcross\  God, that thou mayest be purified from all evil influence, in the Name of Him who is Lord of Angels and of men, who filleth the whole earth with his majesty and glory.  \R Amen.}

\drop{O God, who for the helping and safeguarding of men dost hallow the water set apart for the service of thy holy Church, send forth thy light and thy power upon this element of water which we \grealtcross\ bless and \grealtcross\ hallow in thy holy Name.  Grant that whosoever uses this water in faithfulness of spirit may be strengthened in all goodness, and that everything sprinkled with it may be made holy and pure and guarded from all assaults of evil; through Christ our Lord.  \R Amen.}

\smallskip

\pilcrow{The Priest casts the salt thrice into the water crosswise, as he says the following,}
\drop{Let salt and water mingle together in the Name of the \grealtcross\ Father, and of the \grealtcross\  Son, and of the Holy \grealtcross\ Ghost.  \R Amen.}


\smallskip

\V The Lord be with you.  \R And with thy spirit.

\drop{O God, the giver of invincible strength and King of irresistible power, whose splendour shines throughout the whole of creation: We pray thee to look upon this thy creature of salt and water, to pour down upon it the radiance of thy \grealtcross\ blessing and to \grealtcross\ hallow it with the dew of thy lovingkindness, that wherever it shall be sprinkled and thy holy Name shall be invoked in prayer, every noble aspiration may be strengthened, every good resolve made firm, and the fellowship of the Holy Spirit vouchsafed to us who place our trust in thee; thou who with the Son livest and reignest in the unity of the same Holy Spirit, God throughout all ages of ages.  \R Amen.}

\pilcrow{The Altar, clergy, and people are then sprinkled, while the following is sung.}

Anthem. Thou shalt purge me, \star\ O Lord, with hyssop, and I shall be clean: thou shalt wash me, and I shall be whiter than snow.

\centerline{\rubric{But from Easter until Whitsunday,}}

Anthem.
\begin{leftbar}
    
\V O Lord, shew thy mercy upon us.  \R And grant us thy salvation. (Alleluya.)


\V Glory be to the Father, and to the Son, and to the Holy Ghost; As it was in the beginning, is now, and ever shall be, world without end. Amen.

\star\ Thou shalt wash me, and I shall be whiter than snow.

\pilcrow{This Anthem is said at the sprinkling of holy water on all Sundays throughout the year, except from Easter to the Feast of the Holy Trinity.  It shall be said even on Passion Sunday and Palm Sunday with \emph{Glory be to the father \etc}}

\pilcrow{From Easter to the Feast of the Holy Trinity the following Anthem should be said at the sprinkling of holy water, the precentor commencing the Anthem.}
\drop{I saw water issuing out of the temple on the right-hand side, alleluya.  And all to whom that water came were made whole, and shall say, Alleluya, alleluya.}

\rubric{Ps.} O give thanks unto the {\scshape Lord}, for he is grácious; because his mercy endureth for ever.

\rubric{Ant.} I saw water, \etc

\V Glory be to the Father, and to the Son, and to the Holy Ghost;As it was in the beginning, is now, and ever shall be, world without end. Amen.

\end{leftbar}


\centerline{Let us pray.}

\drop{Graciously hear us, O Lord, Holy Father, Almighty, everlasting God; and vouchsafe to send thy holy Angel from Heaven to keep, cherish, protect, visit, and defend all who are assembled in this holy habitation.  Through Christ our Saviour.  \R Amen.}

\fleuron


\newcounter{psalmnumber}
\newcommand{\daynumber}{Day 1: M}

\makepagestyle{psalter}
\makeevenhead{psalter}{\daynumber}{\scshape The Psalms}{Psalm \thepsalmnumber}
\makeoddhead{psalter}{Psalm \thepsalmnumber}{\scshape The Psalms}{\daynumber}
\pagestyle{psalter}

\newcommand{\psalterday}[2]{
    \section{Day #1. #2 Prayer}
    \directlua{
        if "#2" == "Morning" then
            tex.print([[\noexpand\renewcommand{\noexpand\daynumber}{Day #1: M.}]])
        else 
        tex.print([[\noexpand\renewcommand{\noexpand\daynumber}{Day #1: E.}]])
        end
    }
}
\newcommand{\psalm}[2]{
    \subsection[{Psalm #1}]{\stylesubsec{Psalm #1.}{#2}{}}
    \setcounter{psalmnumber}{#1}
}


%
\chapter[The Psalter]{The Psalter, or Psalms of David}
\centerline{\emph{after the translation of the Great Bible}}
\centerline{\emph{pointed as they are to be sung or said in churches.}}
\medskip


\directlua{printPsalter()}




TODO

1. The Order has no constitution except its members.

As it was said: \emph{Others he saved, himself he cannot save.}

2. It recommends nevertheless that its members shall make a formal act of union with it and of recognition of their own nature.

As it was said: \emph{Am I my brothers keeper?}

3. Its concern is the practice of the apprehension of the Co-inherence both as a natural and a supernatural principle.

As it was said: \emph{Let us make man in our image.}

4. It is therefore, \emmph{per necessitatem}, Christian.

As it was said: \emph{And who ever says that there was when this was not, let him be anathema.}

5.. It recommends therefor the study, on the contemplative side, of the Co-inherence of the Holy and Blessed Trinity, of the Two natures in the single person, of the Mother and Son, of the communicated Eucharist, and of the whole catholic Church.

As it was said: \emph{figlia et tuo figlio.}

And on the the active side, of methods of exchange, in the Sate, in all forms of love, and in all natural things, such as child-birth.

As it was said: \emph{Bear ye one another's burdens.}

6. It concludes in the Divine Substitution of Messias* all forms of exchange and substitution, and it invokes this Act as the root of all.

As it was said: \emph{He must become, as it were, a double man.}

7. The Order will associate itself primarily with four feasts: the Feast of the Annunciation, the Feast of the Blessed Trinity, the Feast of the Transfiguration, and the Commemoration of All Souls.

As it was said: \emph{Another will be in me and I in him.}





Appendix1
Blessing of anything
Blessing of Holy Water

Appendix 2

Personal prayers (1928 bcp)


Grace before Meat.

\drop{Bless, O Father, thy gifts to our use and us to thy service; for Christ’s sake. \R Amen.}

\drop{Give us grateful hearts, our Father, for all thy mercies, and make us mindful of the needs of others; through Jesus Christ our Lord. \R Amen.}
 

The Lord's Prayer 7+1 times a day.


Itinerary
before bathing (asperges, baptism?)
before washing (wash before praying, and eating/drinking)
Psalm 26: 6-12 in the Hebrew
\drop{I will wash my hands in innocency, O {\scshape Lord}; * and so will I go to thine altar.}

7 That I may shew the voice of thanksgiving, * and tell of all thy wondrous works.

8 {\scshape Lord}, I have loved the habitation of thy house, * and the place where thine honour dwelleth.

9 O shut not up my soul with the sinners, * nor my life with the blood-thirsty.

10 In whose hands is wickedness, * and their right hand is full of gifts.

11 But as for me, I will walk innocently: * O deliver me, and be merciful unto me.

12 My foot standeth right: * I will praise the {\scshape Lord} in the congregations.

before praying
Lucernaria/exorcism-blessing of flame

before mass
\drop{Come, Holy Ghost, our souls inspire,}\\
And lighten with celestial fire.\\
Thou the anointing Spirit art,\\
Who dost thy sevenfold gifts impart.

Thy blessed unction from above,\\
Is comfort, life, and fire of love.\\
Enable with perpetual light\\
The dulness of our blinded sight.

Anoint and cheer our soiled face\\
With the abundance of thy grace.\\
Keep far our foes, give peace at home;\\
Where thou art guide, no ill can come.

Teach us to know the Father, Son,\\
And thee, of both, to be but One;\\
That, through the ages all along,\\
This may be our endless song:

Praise to thy eternal merit,\\
Father, Son, and Holy Spirit.



130 for the dead.


Thanksgivings
after meals
after a journey
after mass

Appendix 3
Order of the coinherence

Fix Hierusalem -> Jerusalem; Iacob-Jacob. There being no textual evidence of the Formers I can find.


Ant. What ǐs this * that he sâith, ȧ lǐttlê while? allėlûya: wé cǎnnȯt têll whȧt hě sâith, állêluya.



Chapter: Big all-caps
Section: Regular all caps (Te deum, venite)
Subheading: Regular (Scriptural citation)
subsubsection (italic)


Chapter:   
The order for (caps, 1em?)
MORNING PRAYER (caps, 2em?)
daily throughout the year (caps, 1em?)

Section
;
a supplication
;
The First Sunday in Advent
;
The Nativity of Our Lord, or the 1
Birthday of Christ 1 
commonly called 2
Christmas day 1
[December 25] 2
;
The Sunday called  (it) 3
Septuagesima (all caps)1
or the third Sunday before Lent. 3
;
Day 9. Evening Prayer

subsection
VENITE< EXULTEMUS DOMINO (caps, 1em?) 1
Psalm 95. (.75em?) 2

;
Psalm 47 (ALL CAPS); Omnes gentes, plaudite. (it)


Anglican Breviary
The Song of Zacharias
_Benedictus Dominus, Deus Israel._
Luke I.68.

Diurnal
The Song of the Blessed Virgin (SC)
_Magnificat_

subsubsection
2 Corinthians 13. (it, 1em)
; 
The Collect
;
The Epistle. (it) Romans 13.8 (normal)

Baskerville:



_Te Deum Laduamus_
_Beedictus_ S. LUKE I. 68
_Jubilate DEO. PSAL. 100.

_the Second Collect, for Peace_


_The_ Nativity _of our Lord, or the_ Birth-Day _of_ CHRIST, _commonl called_ Christmas-day.



Psalms:
_The_ 1. _Day.     Psalms   |     Psalms  _The_ 2. _Day._
PSAL. 8. _Domine, Dominus noster

M O R N I N G  P R A Y E R
PSAL. 9. _Confitebor tibi.

Current:
Day 12 : Mn.   THE PSALMS   Ps. 64 | Ps. 65  THE PSALMS  Day 12 : Ev.


Appendix:

Service of catechising

"the forme of" or "the order for the conjuring of water"?


Canticles for the weekdays? Maybe with the extra benedicite *there?* (or not.)

Sources:
Collect "For Reconciliation with the Jews" on Good Friday from the Prayer Book Society of Canada; Authorized for use \emph{ad libitum} by the General Synod of the Anglican Church of Canada.


% \chapter{The Kalendar (1662)}

Keep the normal kalendar (this isn't appealing, after all, to the 1662-only folks.)
Keep the saints days, but not the extras, this isn't the real kalendar, and the saints
are only anchors here.
Add in Rev. instead of Acts in december?

\directlua{printKalendar2()} 
% {\tiny\begin{longtabu} to \linewidth 
%   {@{} c @{\hspace{.5em}} c @{\hspace{1em}}X[3,l]
%   |@{\hspace{.3em}}X[1,r]@{\hspace{.3em}}|@{\hspace{.3em}}X[1,r]@{\hspace{.3em}}||@{\hspace{.3em}}X[1,r]@{\hspace{.3em}}|@{\hspace{.3em}}X[1,r]@{\hspace{.3em}}|}

%    &          &          & \multicolumn{2}{c}{\scshape Mattins} & \multicolumn{2}{c}{\scshape Evensong} \\
%    &          &          & 1st        & 2nd        & 1st        & 2nd  \\ 
%    &          &          & Lesson     & Lesson     & Lesson     & Lesson \\ \hline
% 1  & {\red A} & \dub{Circumcision}\dotfill & & & &\\ %of our Lord.
% 2  & b        & \dotfill & Gen.~1     & Matt.~1    & Gen.~2     & Rom.~1 \\
% 3  & c        & \dotfill & \dotfill 3 & \dotfill 2 & \dotfill 4 & \dotfill 2\\
% 4  & d        & \dotfill & \dotfill 5 & \dotfill 3 & \dotfill 6 & \dotfill 3\\
% 5  & e        & \dotfill & \dotfill 7 & \dotfill 4 & \dotfill 8 & \dotfill 4\\
% 6  & f        & \dub{\red Epiphany}\dotfill & & & &\\ %of our Lord}.
% 7  & g        & \dotfill & \dotfill 9 & \dotfill 5 & \dotfill 12 & \dotfill 5 \\ %(Keys of LXX.) 
% 8  & {\red A} & S.~Lucian, B.M.\dotfill&\dotfill 13 & \dotfill 6 &  \dotfill 14 &    \dotfill 6\\
% 9  & b        & \dotfill & \dotfill 15 & \dotfill 7 & \dotfill 16 & \dotfill    7 \\
% 10 & c        & \dotfill & \dotfill 17 & \dotfill 8 & \dotfill 18 &  \dotfill     8 \\
% 11 & d        & \dotfill & \dotfill 19 & \dotfill 9 &  \dotfill 20 & \dotfill    9 \\
% 12 & e        & \dotfill & \dotfill 21 & \dotfill 10 & \dotfill 22 & \dotfill   10\\
% 13 & f        & \mem{S.~Hilary, B.C.}\dotfill&   \dotfill   23 &     \dotfill   11 &   \dotfill   24 &   \dotfill   11\\
% 14 & g        & \dotfill &   \dotfill   25 &   \dotfill     12 & \dotfill     26 &   \dotfill   12\\


% \end{longtabu}}

justus.anglican.org/resources/bcp/copyrights.html:
While it is commonly understood that all books published prior to 1923 are in the public domain, copyright notice or no, it is not so commonly known that many texts published after that date are also in the public domain. Under U. S. copyright law, any book published between 1923 and 1977 with no copyright notice is in the public domain. Also, if the book was published in the US with a copyright notice between 1923 and 1963 but that copyright was not renewed, it is in the public domain.

% \chapter{The Hymnal}

\def\hymn{\obeylines\hymna}
\def\hymnverses{\obeylines\hymna}
\newcommand{\hymna}[1]{
    \directlua{printhymn([[#1]])}
}



\section{Sunday, during Summer}
¶ At Mattins\hfill Nocte surgentes

\hymnverses{
Now, from the slumbers of the night arising,
Chaunt we the holy psalmody of David,
Hymns to our Master, with our best endeavour,
Sweetly intoning.

2. So may our Monarch pitifully hear us,
That we may merit with his Saints to enter
Mansions eternal, therewithal possessing
Joy beatific.

3. This he vouchsafe us, God for ever blessed,
Father eternal, Son, and Holy Spirit,
Whose is the glory, which through all creation
Ever resoundeth. Amen.
}

\rubric{Sundays} \V The Lord is high above all péople.  \R And his glory above the héavens.

\rubric{Weekdays} \V Let thy merciful kindness, O Lord, be upon us.  \R As we do put our trust in thee.

\medskip
¶ At Lauds\hfill Ecce jam noctis

\hymnverses{
Lo! the dim shadows of the night are waning;
Lightsome and blushing, dawn of day returneth;
Fervent in spirit, to the mighty Father
Pray we devoutly.

2. So shall our Maker, of his great compassion,
Banish all sickness, kindly health bestowing;
And may he grant us, of a Father’s goodness,
Mansions in heaven.


3. This he vouchsafe us, God for ever blessed,
Father eternal, Son, and holy Spirit,
Whose is the glory, which through all creation
Ever resoundeth. Amen.
}

\rubric{Sundays} \V The Lord is Kíng.  \R He hath put on glorious apparel, allelúya.

\rubric{Weekdays} \V Have I not thought upon thee when I was waking?  \R Because thou hast been my helper.

¶ At Evensong\hfill Lucis Creator optime

\hymnverses{
O blest Creator of the light,  
Who mak’st the day with radiance bright,  
And o’er the forming world didst call  
The light from chaos first of all:

2. Whose wisdom join’d in meet array
The morn and eve, and named them Day:
Night comes with all its darkling fears,
Regard thy people’s prayers and tears.

3. Lest sunk in sin, and ’whelm’d with strife,
They lose the gift of endless life:
While thinking but the thoughts of time,
They weave new chains of woe and crime.

4. But grant them grace that they may strain
The heav’nly gate and prize to gain:
Each harmful lure aside to cast.
And purge away each error past.

5. O Father, that we ask be done,
Through Jesus Christ, thine only Son:
Who, with the Holy Ghost and thee,
Shall live and reign eternally. Amen.
}

\V Let my prayer be set forth, O Lord.  \R In thy sight as the incense.


\rubric{Monday.} My soul doth magnify \star\ the Lord.

\rubric{Tuesday.} My spirit hath rejoiced \star\ in God my Saviour.

\rubric{Wednesday.} O Lord my God, \star\ thou hast regarded my lowliness.

\rubric{Thursday.} He hath put down \star\ the mighty from their seat; and hath exalted the humble and meek that confess his Christ.

\rubric{Friday.} God hath holpen \star\ his servant Israel, as he promised Abraham and his seed; and he hath exalted the humble for ever.


¶ At Compline.\hfill Te lucis ante terminum

\hymnverses{
To thee, before the close of day,  
Creator of the world, we pray,  
That, with thy wonted favour, thou  
Wouldst be our guard and keeper now.

2. From all ill dreams defend our eyes,
From nightly fears and fantasies:
Tread under foot our ghostly foe,
That no pollution we may know.

3. O Father, that we ask be done.
Through Jesus Christ, thine only Son:
Who, with the holy Ghost and thee,
Shall live and reign eternally. Amen.
}

\V Keep us, O Lórd.  \R As the apple of an eye, hide us under the shadow of thy wíngs.

\bigskip
\section{A Virgin}
¶ At First Evensong \& Mattins.\hfill Virginis Proles

\hymnverses{
Child of a Virgin,   Maker of thy Mother,
Born of a Maiden,  as of Maid conceived,
While we a Virgin’s   triumphs are rehearsing,
Hear our petition.

*2. She, thine own maiden, double blessing winneth,
Striving to vanquish all her nature’s weakness.
E’en by that weakness o’er a world of bloodshed
Victory gaining.

*3. Death and its terrors undismay’d beholding,
Death’s cruel handmaid, torture, she despiseth;
Shedding her life-blood, meet is she to enter
Holiest heaven.

4. God ever-loving, as for us she pleadeth,
Pity our failings, all our sins forgiving:
Thus shall re-echo pure and heart-felt praises
Unto thine honour.

5. Praise to the Father, to the Sole-begotten,
And the blest Spirit, with the twain co-equal,
One only Godhead, who throughout the ages
Reigneth for ever. Amen.
}

¶ Note that if the Virgin be not a Martyr, verses 2 and 3 are ommitted.
¶ For Many Virgins, the hymn and \V is {Ihesu, corona virginum}, as at Lauds.

\rubric{Evensong.} \V Full of grace are thy líps. \R Therefore God hath blessed thee for éver. .

\rubric{Mattins.} \V God shall give her the help of his cóuntenance.  \R God is in the midst of her, therefore shall she not be remóved. .

\bigskip


¶ At Lauds \& Second Evensong.\hfill Ihesu corona virginum

\hymnverses{
Jesu, the Virgins’ Crown, do thou
Accept us as in prayer we bow:
Born of that Virgin, whom alone
The Mother and the Maid we own.

2. Among the lilies thou dost feed,
By Virgin quires accompanied—
With glory deck’d, the spotless brides
Whose bridal gifts thy love provides.

3. They, wheresoe’er thy footsteps bend,
With hymns and praises still attend:
In blessed troops they follow thee,
With dance, and song, and melody.

4. We pray thee therefore to bestow
Upon our senses here below
Thy grace, that so we may endure
From taint of all corruption pure.

Ordinary Doxology
5. All laud to God the Father be,
All praise, eternal Son, to thee:
All glory, as is ever meet,
To God the holy Paraclete. Amen.
}

\V The virgins that be her fellows shall bear her company.  \R And shall be brought unto thee.

¶ For Many Virgins, the hymn and \V is {Rex gloriose martyrum}, as at Lauds for Many Martyrs.




\section{The First Sunday in Advent}

Office. Ad te levavi.
\lettrine{U}{nto} thee, O Lord, lift I up my soul; O my God, in thee have I trusted, let me not • be confóunded; * neither let mine enemies triumph over me; for all they that look for thee shall • not be ashamed. \rubric{Ps.} Shew me • thý ways, Ó Lord * and • teach me thý paths.

\rubric{Grail.} For all they • that lóok for thee * shall not be a•shámed, Ó Lord.  \V Make known to me • thy wáys, O Lord * and • téach me thy paths.  \rubric{Alleluya.} \V Shew us thy mercy, • O Lord * and grant us • thy salvátion.   \rubric{Offertory.} Unto thee, O Lord, lift I up my soul; O my God, in thee have I trusted, let me • not be confóunded * neither let mine enemies triumph over me; for all they that look for thee shall • not be ashamed.  \rubric{Communion.} The Lord shall shew • lóving-kíndness * and our land • shall give her increase.

\section{The Second Sunday of Advent}
Office. Populus Sion.
\lettrine{O}{ people} of Sion, behold, the Lord is nigh at hand to re•déem the nátions: * and in the gladness of your heart the Lord shall cause his glori•ous voice to be heard.  \rubric{Ps.} Hear, O thou • Shépherd of Ísrael: * thou that leadest • Jóseph like a sheep.

\rubric{Grail.} Out of Sion hath God • appéared: * in • pérfect béauty.  \V Gather my saints toge•ther únto me: * those that have made a covenant with • mé with sácrifice.  \rubric{Alleluya.} \V For the powers of heaven shall be • shaken: * and then shall they see the Son of Man coming in a cloud with power • and great glóry. \rubric{Offertory.} Wilt not thou turn again, O God, and quicken us; that thy peo•ple may rejóice in thee: * shew us thy mercy, O Lord; and grant • us thy salvation.  \rubric{Communion.} Jerusalem, haste • thee, and stánd on high: * and behold the joy and gladness, which cometh unto thee • from God thy Saviour.


\section{The Third Sunday of Advent}
Office. Gaudete.
\lettrine{R}{ejoice} ye in the Lord alway, and again I say, rejoice ye;   Let your moderation be known unto all men; the • Lord is át hand. * Be careful for nothing, nor troubled; but in all things, by prayer and supplication, with thanksgiving, let your requests be • made known unto God.  \rubric{Ps.} And the peace of God, which passeth all • únderstánding * shall • keep your héarts and minds.

\rubric{Grail.} Shew thyself, O Lord, thou that sittest upon • the Chérubim: * stir • úp thy stréngth and come.  \V Hear, O thou Shepherd • of Ísrael: * thou that leadest • Jóseph líke a sheep.  \rubric{Alleluya.} \V Stir up thy strength, • O Lord: * and • come and hélp us. \rubric{Offertory.} Lord, thou art become gracious unto thy land; thou hast turned away the capti•vity of Jácob; * thou hast forgiven the of•fence of thy people.  \rubric{Communion.} Say to them that are • of a féarful heart: * Be strong, fear not; behold, your God • will come and save you.


\section{The Fourth Sunday of Advent}
¶ Should the Fourth Sunday fall on {Christmas Eve}, the {Office, Grail, Alleluya, Offertory,} and {Communion} will be those of {Christmas Eve}, below.
Office. Memento nostri.

\lettrine{R}{emember} us, O Lord, with the favour that thou bearest unto thy people; O visit us with • thy salvátion: * that we, beholding the felicity of thy chosen, may rejoice in the gladness of thy people, and may glory with • thine inheritance. \rubric{Ps.} We have sinned • wíth our fáthers: * we have done amiss, • and dealt wíckedly.

\rubric{Grail.} The Lord is nigh unto all them that call • upón him; * yea, all such as call up•ón him fáithfully. \V My mouth shall speak the praise • of thé Lord; * and let all flesh give thanks un•tó his hóly name. \rubric{Alleluia.} \V Come, O Lord, and tar•ry not: * forgive the misdeeds • of thy péople. \rubric{Offertory.} Be strong, fear not; behold, our God will • come with a récompense: * – he • will come and save us. \rubric{Communion.} Behold, a Virgin shall con•ceive and béar a Son, * and his name shall be call•ed Emmanuel.


\section{Christmas Eve}
Office. Hodie scietis.
\lettrine{T}{o-day} shall ye know that the Lord will come to deliver you * and at sunrise shall ye behold his glory. \rubric{Ps.} The earth is the Lord’s, and all that is therein * the compass of the world, and they that dwell therein.

\rubric{Grail.} To-day shall ye know that the Lord will come to deliver you; * and at sunrise shall ye behold his glory. \V Hear, O thou Shepherd of Israel, thou that leadest Joseph like a sheep: * show thyself also, thou that sittest upon the Cherubim, before Ephraim, Benjamin, and Manasses. \rubric{Alleluya.} \V On the morrow the iniquity of the earth shall be blotted out: * and the Saviour of the world shall reign over us. \rubric{Offertory.} Lift up your heads, O ye gates, and be ye lift up, ye everlasting doors: * and the King of Glory shall come in. \rubric{Communion.} The glory of the Lord shall be revealed: * and all flesh shall see the salvation of our God.

\section{The Fifteenth Sunday after Trinity}
\centerline{Office. \emph{Inclina, Domine.}}
\drop{Bów dówn, O Lórd, thine éar to • me, and héar me: \star\ O my Gód, sáve thy sérvant, that trústeth in thée; have mércy upón me, O Lórd, for I have cálled • dáily upón thee. \rubric{Ps.} Cómfort the • sóul of thy sérvant: \star\ for únto thée, O Lórd, do I • líft up my sóul.}

\rubric{Grail.} It is bétter to • trüst in the Lórd: \star\ than to pút any • cónfidence in mán. \V It is bétter to • trüst in the Lórd: \star\ than to pút any cónfi•dence in prínces. \rubric{Allelúya.} \V My héart is réady, O Gód, my héart is • réady: \star\ I will síng, yéa, I will práise thee, with the bést • mémber thát I have. \rubric{Offertory.} I wáited pátiently for the Lórd, and he • inclíned únto me: \star\ hé heard my cálling, and hath pút a néw sóng in my móuth, éven a thanksgív•ing únto our Gód. \rubric{Communion.} Whóso éateth my Flésh, and • drínketh my Blóod: \star\ dwélleth in mé, and I • in hím, sáith the Lórd.

\end{document}
