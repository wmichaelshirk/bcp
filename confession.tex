\chapter[Confession]{\stylechapter{The Order for \\ the Reconciliation of a Penitent\\ {\small commonly called}}{Confession}{}}

\pilcrow{Note. Such as shall be satisfied with a general Confession should not be offended with them that do use, to their further satisfying, confession to the Priest; and those also which think needful or convenient, for the quietness of their own consciences, particularly to open their sins to the Priest, should not be offended with them that are satisfied with their humble confession to God, and the general Confession to the Church. But in all things everyone should follow and keep the rule of charity, and be satisfied with his own conscience, not judging other men’s minds and consciences, whereas he hath no warrant of God’s word to the same.} %1929 Scottish; slightly altered from 1549
\medskip
%SA
\pilcrow{[Every Priest in his exercising of this ministry of reconciliation, comitted by Christ to his Church, is solemnly bound to observe secrecy concerning all those matters which are thus confessed before him.]}
\bigskip 

%SA
\pilcrow{[At the time appointed the penitent shall kneel down in some convenient place [in the Church] and the Priest shall say unto him,]}

\medskip
Anglican Service Book: 

\centerline{\pilcrow{The Penitent, kneeling, begins,}}
\centerline{Bless me, for I have sinned.}

\subsubsection{The Priest gives the blessing,}
%AFPB (without first [])
%SA ( "thou mayest rightly and truly confess thy sins")
\drop{The Lord be in thy heart and upon thy lips, that so thou mayest worthily [and rightly] confess all thy sins, [\grealtcross\ in the Name of the Father, and of the Son, and of the Holy Ghost. Amen.]}

\medskip
\pilcrow{The Penitent then makes \emph{his} confession, saying,}
%SA
\pilcrow{[Then shall the penitent make confession of his sins, in this form or the like:]}
%1928 Proposed
\drop{I confess to God Almighty, the Father, the Son, and the Holy Ghost, that I have sinned in thought, word, and deed, through my own grievous fault; wherefore I pray God to have mercy on me. And especially I have sinned in these ways....}

% \rubric{The penitent then states the specific sins he can remember, and should end with the following:}

%SA
[For these and all my other sins which I cannot now remember, I am heartily [very, truly] sorry, firmly purpose amendment, and humbly ask pardon of God, and of you penance, counsel, and absolution.  Wherefore I pray God to have mercy upon me, and you to pray for me to the Lord our God.]

% Anglican Service Book.
For these and all other sins which I cannot now remember, I am truly sorry. I firmly intend amendment of life, and I humbly beg forgiveness of God and his Church, and ask thee for penance, counsel, and absolution.
\medskip
%AFPB
\pilcrow{After the uninterrupted confession, the Priest may find it helpful to question the penitent, so that advice about possible reparation, or restitution, or how to face the future more successfully may be given.}

\smallskip
%AFPB
\pilcrow{Then some form of penance is given. This is \emph{not} a \emph{penalty} but some useful act which aids the penitent to make outward embodiment of his contrite purpose.}

%SA
\pilcrow{[After which confession the Priest shall give such counsel and penance as may be convenient, and, if he is assured of his repentance, he shall absolve the penitent after this sort:]}
%    Here the Priest may offer counsel, direction, and comfort.
\medskip
\pilcrow{The Priest then pronounces this absolution:}

\drop{Our Lord Jesus Christ, who hath left power to his Church to absolve all sinners who truly repent and believe in him, of his great mercy forgive thee thine offences: And by his authority committed to me, I absolve thee from all thy sins, \grealtcross\  In the Name of the Father, and of the Son, and of the Holy Ghost. \R Amen.}


\medskip
Anglican Service Book:
\smallskip
The Lord hath put away all thy sins. \R Thanks be to God.

\subsubsection{The Priest concludes,}

\centerline{Go (\rubric{or,} Abide) in peace, and pray for me, a sinner.}

\medskip
%SA
\centerline{\rubric{Then shall the Priest dismiss the penitent with a Blessing.}}
AFPB

\drop{The blessing of God Almighty, \grealtcross\ the Father, the Son, and the Holy Ghost, be upon thee and remain with thee always. \R Amen.}

Go (\rubric{or,} Abide) in peace; the Lord hath put away all thy sins.


\fleuron

% Book of Common Prayer (1662) Absolution
% Book of Common Prayer (proposed, 1923, Visitation of the Sick) Confession
% The Armed Forces Prayer Book (1951) - Rubrics & confession (parts)
% Book of Common Prayer (1979) - traditionalized.
% Anglican Service Book (some tweaks)
% http://justus.anglican.org/resources/bcp/1928/AFPB_baptism&confession.htm


% ¶ Then let him tell his sins, which being ended the Priest shall say—God Almighty have mercy, and The Almighty and merciful Lord, as in the Ordinary of the Mass.

% {Almighty God have mercy upon thee, forgive thee thy sins, and bring thee to everlasting life.  Amen.}
% {May the Almighty and Merciful Lord grant thee pardon, absolution, and remission of thy sins.  Amen.}

% The Passion of our Lord Jesus Christ, the merits of the Blessed Virgin Mary, and of all the Saints, whatsoever good thou hast done, or evil thou hast endured, be to the for the remission of sins, the increase of grace, and the reward of eternal life.  Amen.

% ¶ Here let him enjoin the Penance, saying—
% And for a special Penance thou shalt say or do this or that.
% ¶ Then let him absolve him, and say—






