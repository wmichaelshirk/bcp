

\chapter[Of Confession]{The Order for the Reconciliation of a Penitent}
\subsubsection{commonly called Confession}

\pilcrow{The Penitent, kneeling, begins,}
Bless me, for I have sinned.

\subsubsection{The Priest gives the blessing,}
\drop{The Lord be in thy heart and upon thy lips, that so thou mayest worthily and rightly confess all thy sins, in the Name of the Father, and of the Son, and of the Holy Ghost. Amen.}

\medskip
\pilcrow{The Penitent then makes \emph{his} confession, saying,}

\drop{I confess to God Almighty, the Father, the Son, and the Holy Ghost, that I have sinned in thought, word, and deed, through my own grievous fault.} % Wherefore I pray God to have mercy upon me.}

And especially I have sinned in these ways . . .

% \rubric{The penitent then states the specific sins he can remember, and should end with the following:}

For these and all other sins which I cannot now remember, I am truly sorry. I pray God to have mercy on me. I firmly intend amendment of life, and I humbly beg forgiveness of God and his Church, and ask thee for penance, counsel, and absolution.
\medskip
\pilcrow{After the confession, the Priest may find it helpful to question the penitent, so that advice about possible reparation, or restitution, or how to face the future more successfully may be given.}

\pilcrow{Then some form of penance is given. This is not a penalty but some useful act which aids the penitent to make outward embodiment of his contrite purpose.}

%    Here the Priest may offer counsel, direction, and comfort.
\medskip
\pilcrow{The Priest then pronounces this absolution:}

\drop{Our Lord Jesus Christ, who hath left power to his Church to absolve all sinners who truly repent and believe in him, of his great mercy forgive thee thine offences: And by his authority committed to me, I absolve thee from all thy sins, \grealtcross\  In the Name of the Father, and of the Son, and of the Holy Ghost. \R Amen.}

\smallskip
The Lord hath put away all thy sins. \R Thanks be to God.

\subsubsection{The Priest concludes,}

Go in peace, and pray for me, a sinner.

\fleuron

% Book of Common Prayer (1662) Absolution
% Book of Common Prayer (proposed, 1923, Visitation of the Sick) Confession
% The Armed Forces Prayer Book (1951) - Rubrics & confession (parts)
% Book of Common Prayer (1979) - traditionalized.
% Anglican Service Book (some tweaks)



% ¶ Then let him tell his sins, which being ended the Priest shall say—God Almighty have mercy, and The Almighty and merciful Lord, as in the Ordinary of the Mass.

% {Almighty God have mercy upon thee, forgive thee thy sins, and bring thee to everlasting life.  Amen.}
% {May the Almighty and Merciful Lord grant thee pardon, absolution, and remission of thy sins.  Amen.}

% The Passion of our Lord Jesus Christ, the merits of the Blessed Virgin Mary, and of all the Saints, whatsoever good thou hast done, or evil thou hast endured, be to the for the remission of sins, the increase of grace, and the reward of eternal life.  Amen.

% ¶ Here let him enjoin the Penance, saying—
% And for a special Penance thou shalt say or do this or that.
% ¶ Then let him absolve him, and say—






