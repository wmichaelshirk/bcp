\chapter{Preface}
\drop{It is a most invaluable part of that blessed liberty wherewith Christ hath made us free, that in his worship different forms and usages may without offence be allowed, provided the substance of the Faith be kept entire; and that, in every Church, what cannot be clearly determined to belong to Doctrine must be referred to Discipline and therefore, by common consent and authority, may be altered, abridged, enlarged, amended, or otherwise disposed of, as may seem most convenient for the edification of the people, “according to the various exigencies of times and occasions.”}

The wisdom of our fathers under the good hand of God gave to the Church of England the Book of Common Prayer in English speech. It is, and we believe that it will always be, one of the great books of the world. Nothing save the English version of the Holy Scriptures is enwoven so closely in the language and the deepest thoughts of English speaking people.

% There was never any thing by the wit of man so well devised, or so sure established, which in continuance of time hath not been corrupted: As, among other things, it may plainly appear by the Common Prayers in the Church, commonly called Divine Service. The first original and ground whereof if a man would search out by the ancient Fathers, he shall find, that the same was not ordained but of a good purpose, and for a great advancement of godliness. 






The \emph{Church of England}, to which the Protestant Episcopal Church in these States is indebted, under God, for her first foundation, hath, in the Preface of her Book of Common Prayer, laid it down as a rule, that “The particular forms of Divine Worship, and the Rites and Ceremonies appointed to be used therein, being things in their own nature indifferent and alterable, and so acknowledged, it is but reasonable that upon weighty and important considerations, according to the various exigencies of times and occasions, such changes and alterations should be made therein; as to those who are in places of authority should, from time to time, seem either necessary or expedient.”

% The same Church hath not only in her Preface, but likewise in her Articles and Homilies declared the necessity and expediency of occasional alterations and amendments in her Forms of Public Worship; and we find accordingly, that, seeking to “keep the happy mean between too much stiffness in refusing, and too much easiness in admitting variations in things once advisedly established”, she hath, in the reign of several Princes since the first compiling of her Liturgy in the time of Edward the Sixth upon just and weighty considerations her thereunto moving yielded to make such alterations in some particulars, as in their respective times were thought convenient; yet so as that the main body and essential parts of the same (as well in the chiefest materials as in the frame and order thereof) have still been continued firm and unshaken.

% Her general aim in these different reviews and alterations hath been, as she farther declares in her said Preface "to do that which according to her best understanding, might most tend to the preservation of peace and unity in the Church; the procuring of reverence, and the exciting of piety and devotion in the worship of God; and finally the cutting off occasion, from them that seek occasion. Of cavil or quarrel against her Liturgy." And although, according to her judgment, there be not "any thing in it contrary to the Word of God or to sound doctrine, or which a godly man may not with a good conscience use and submit unto, or which is not fairly defensible, if allowed such just and favourable construction as in common equity ought to be allowed to all human writings;" yet upon the principles already laid down, it cannot but be supposed that further alterations would in time be found expedient. Accordingly, a commission for a review was issued in the year 1689: but this great and good work miscarried at that time; and the Civil Authority has not since thought proper to revive it by any new commission.

% But when in the course of Divine Providence, these American States became independent with respect to civil government, their ecclesiastical independence was necessarily included; and the different religious denominations of Christians in these States were left at full and equal liberty to model and organize their respective Churches, and forms of worship, and discipline, in such manner as they might judge most convenient for their future prosperity.%; consistently with the constitution and laws of their country.
% The attention of this Church was in the first place drawn to those alterations in the Liturgy which became necessary in the prayers for our Civil Rulers, in Consequence of the Revolution. And the principal care herein was to make them conformable to what ought to be the proper end of all such prayers, namely, that "Rulers may have grace, wisdom, and understanding to execute justice, and to maintain truth;" and that the people "may lead quiet and peaceable lives, in all godliness and honesty."

% But while these alterations were in review before the Convention, they could not but, with gratitude to God, embrace the happy occasion which was offered to them (uninfluenced and unrestrained by any worldly authority whatsoever) to take a further review of the Public Service, and to establish such other alterations and amendments therein as might be deemed expedient.

It seems unnecessary to enumerate all the different alterations and amendments. They will appear, and it is to be hoped, the reasons of them also, upon a comparison of this with the Book of Common Prayer of the Church of England. % In which it will also appear that this Church is far from intending to depart from the Church of England in any essential point of doctrine, discipline, or worship; or further than local circumstances require.








And now, this important work being brought to a conclusion, it is hoped the whole will be received and examined by every true member of our Church, and every sincere Christian, with a meek, candid, and charitable frame of mind; without prejudice or prepossessions; seriously considering what Christianity is, and what the truths of the Gospel are; and earnestly beseeching Almighty God to accompany with his blessing every endeavour for promulgating them to mankind in the clearest, plainest, most affecting and majestic manner, for the sake of Jesus Christ, our blessed Lord and Saviour.  

In all things we have set before our eyes the duty of faithfulness to the teaching of Scripture and the godly and decent order of the ancient Fathers, and we pray that by God's blessing upon our work those who use this book may be enabled to keep the unity of the Spirit in the bond of peace.

