\newcommand \Sol[1]{{\versiculus☉}~in~{\versiculus#1}}
\newcommand \dub[1]{{\footnotesize\sffamily #1}}		% Double Feasts
\newcommand \mem[1]{\emph{#1}} % Memorials

%\renewcommand \dub[1]{{\sffamily #1}}		% Double Feasts
%\renewcommand \Rul[1]{{\scshape #1}}		% Ruled Feasts
%\markright{Kalendar}\part{Kalendar}

\chapter{The Kalendar}
\directlua{printKalendar(false)}


\newpage

\section*{A Table to Find Easter-Day}
{\begin{multicols}{2}
{\tiny
\noindent\begin{tabular} { @{}c@{\hspace{.15cm}} c@{\hspace{.2cm}} r@{\hspace{.2cm}} l@{\hspace{.2cm}} }
Golden & Month & Day & Sunday \\
Number &    &   & Letter \\
\hline
xiv & March & 22 & d \\
iij & ” & 23 & e \\
    & ” & 24 & f \\
xj  & ” & 25 & g \\
    & ” & 26 & A \\
xix & ” & 27 & b \\
viij & ” & 28 & c \\
    & ” & 29 & d \\
xvj & ” & 30 & e \\
v   & ” & 31 & f \\
    & April & 1 & g \\
xiij & ” & 2 & A\\
ij & ” & 3 & b\\
  & ” & 4 & c \\
x &	”	&5&	d\\
&”	&6&	e\\
xviij	&”&	7&	f\\
vij	&”&	8&	g\\
&”	&9&	A\\
xv	&”&	10&	b\\
iv	&”&	11&	c\\
&”	&12&	d\\
xij	&”&	13&	e\\
j	&”&	14&	f\\
&”	&15&	g\\
ix	&”&	16&	A\\
xvij	&”&	17&	b\\
vj	&”&	18&	c\\
&”	&19&	d\\
&”	&20&	e\\
&”	&21&	f\\
&”	&22&	g\\
&”	&23&	A\\
&”	&24&	b\\
&”	&25&	c\\

\end{tabular}}
%\vfill
%\columnbreak
\footnotesize
\vspace{6pt}

This Table contains so much of the Calendar as is necessary for the determining of Easter; to find which, look for the Golden Number of the year in the first Column of the Table, against which stands the day of the Paschal Full Moon; then look in the third Column for the Sunday Letter, next after the day of the Full Moon, and the day of the Month standing against that Sunday Letter is Easter Day. If the Full Moon happens upon a Sunday, then (according to the first rule) the next Sunday after is Easter-Day.

To find the Golden Number, or Prime, add one to the Year of our Lord, and then divide by 19; the remainder, if any, is the Golden Number; but if nothing remaineth, then 19 is the Golden Number.

To find the Dominical or Sunday Letter, according to the Calendar, until the Year 2099 inclusive, add to the Year of our Lord its Fourth Part, omitting Fractions; and also the Number 6: Divide the sum by 7; and if there is no remainder, the A is the Sunday Letter: But if any number remaineth, then the Letter standing against that number in the small annexed Table is the Sunday Letter.
\vspace{6pt}

{\centering\setlength{\extrarowheight}{2pt}
{\tiny\begin{tabular} { | c c c c c c c | }
\hline
0 & 1 & 2 & 3 & 4 & 5 & 6 \\
\hline
A & G & F & E & D & C & B \\
\hline
\end{tabular}}\par}

\vspace{2pt}
For the next Century, that is, from the year 2100 till the year 2199 inclusive, add to the current year its fourth part, and also the number 5, and then divide by 7, and proceed as in the last Rule.

Note, that in all Bissextile or Leap-Years, the Letter found as above will be the Sunday Letter, from the intercalated day exclusive to the end of the year.

The Golden Numbers in the foregoing Calendar will point out the Days of the Paschal Full Moons from the Year 1900, to the Year 2199 inclusive.
\vspace{12pt}
\end{multicols}}
