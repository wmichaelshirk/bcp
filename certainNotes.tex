% Primarily from "A Prayer Book Revised", 1913
\chapter[Certain Notes]{\stylechapter{}{Certain Notes}{for the more Plain Explanation and Decent Ministration of things Contained in this Book}}

% ! The changeable portions of the Mass for [[Sundays]] and \Dub{Greater Feasts} will be found in the Propers through the Year.  At Mattins and Evensong, the same Collect is used as at Mass and proper Lessons are appointed, but the Invitatory and Hymns are found in the Hymnal in the Season or class of Saint.
% On \Dub{Principle Feasts}, there is furthermore a proper Preface for the Mass.

% On Lesser Feasts, the common Mass, Invitatory, and Hymns for the class of Saint will generally serve, unless propers be provided.
% The Mass of a Sunday shall serve throughout the week, unless another be appointed.

\drop{The word ‘Minister’ in this Book includes bishops, priests, and deacons. When the word ‘Bishop’ is used, none but a bishop may say the words there appointed; when the word ‘Priest’, then may none but a bishop or priest use the words; when the word ‘Deacon’ is used, then shall the words appointed to the deacon be said by one who is in that office, or by a bishop or priest executing that office for the occasion, or by the priest himself when there is no other minister.}

A Clerk is any person appointed to lead in the singing, or to serve the minister and to lead in the responses; the clerk may also read the Lessons and the Epistle.

When one service follows upon another, opportunity shall be given for people to come and go between the services, whether by the singing of a hymn or by a pause. And none shall go out of church during any service or sermon except in case of necessity.

A sermon shall be preached every Sunday at the time appointed. On Sundays and Holy-days in general, a lecture or sermon on a catechetical topic may be delivered after the Second Lesson at Evensong, or the Priest, or one chosen by him for this purpose as Catechist, may instruct the young people of his parish.

The \emph{Gloria} is always to be added to the Psalms, and to the Canticles specified in the rubric, except from Morning Prayer on Maundy Thursday until Evening Prayer on Easter Even; and also it is omitted at all Funeral and Memorial services.

When any minister or reader says a prayer or other form together with the people, he that reads shall say alone the opening words (as, \emph{Our Father}, \emph{I believe in God}, \emph{Glory be to God on high}, and in other places as far as the comma); and the clerks and people shall take up the following words with him.

The full ending of a Collect may be used on any occasion, whether it be printed or not; except that when more than two Collects are said together, without any intermediate bidding, the first and the last shall have the full ending (and the people shall say \emph{Amen}), and the intermediate Collects shall have no ending. The normal full ending is, \emph{Through Jesus Christ thy Son our Lord, who liveth and reigneth with thee in the unity of the Holy Ghost, ever one God, world without end; \emph{or, if our Lord has been already mentioned in the Collect,} Through the same thy Son Jesus Christ, \etc; \emph{or if the Holy Spirit has been already mentioned,} who liveth and reigneth with thee and the same Spirit \etc}


When Anthems are appointed, they are to be sung in full before the Psalm, and to be repeated at the end of the \emph{Gloria} (or of the Psalm itself, when there is no \emph{Gloria} said); but in any Procession the Anthem may be repeated after each verse, if necessity require.

Saying is to be taken to include singing; and words that are appointed to be sung may be said, if need be. But words which are directed to be said in a humble voice should be said without any musical note or inflection.

% When it is desired to use music composed for them, other authorized liturgical texts may be used in place of the corresponding texts in this book.

To avoid a continual repetition of rubrics, let it here also be said that a minister who is reading the service is not included in a general direction to kneel. He stands to read, unless it be expressly stated that he is to kneel down. All others present kneel during prayers, unless it be otherwise stated, except any who are helping the priest in his ministration.

Whenever any passage from the Scripture is read, he that reads shall stand and turn towards the people, who may sit; except that when the Liturgical Gospel is read, they also shall stand, and turn towards the minister who reads. And whenever the priest speaks to the people, as in absolutions and benedictions, he shall turn to them. All are to stand when Canticles are sung; but during the singing of the Psalms it is lawful to sit.

% If more specific details on the ceremonial of the services are desired, {A Directory of Ceremonial} by René Vilatte Press is recommended.

And since there must of necessity be many things not mentioned in these Notes, we may well, for the rest, observe that golden rule of the venerable Council of Nicæa, “Let ancient customs prevail,” till reason plainly requires the contrary.


\fleuron
